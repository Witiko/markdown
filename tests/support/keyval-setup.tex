renderers = {%
  attributeIdentifier = {%
    \TYPE{attributeIdentifier:   #1}},
  attributeClassName = {%
    \TYPE{attributeClassName:    #1}},
  attributeKeyValue = {%
    \TYPE{BEGIN attributeKeyValue}%
    \TYPE{- key:                 #1}%
    \TYPE{- value:               #2}%
    \TYPE{END attributeKeyValue}},
  fencedCodeAttributeContextBegin = {%
    \TYPE{BEGIN fencedCodeAttributeContext}},
  fencedCodeAttributeContextEnd = {%
    \TYPE{END fencedCodeAttributeContext}},
  fencedDivAttributeContextBegin = {%
    \TYPE{BEGIN fencedDivAttributeContext}},
  fencedDivAttributeContextEnd = {%
    \TYPE{END fencedDivAttributeContext}},
  bracketedSpanAttributeContextBegin = {%
    \TYPE{BEGIN bracketedSpanAttributeContext}},
  bracketedSpanAttributeContextEnd = {%
    \TYPE{END bracketedSpanAttributeContext}%
    \GOBBLE},
  linkAttributeContextBegin = {%
    \TYPE{BEGIN linkAttributeContext}},
  linkAttributeContextEnd = {%
    \TYPE{END linkAttributeContext}%
    \GOBBLE},
  imageAttributeContextBegin = {%
    \TYPE{BEGIN imageAttributeContext}},
  imageAttributeContextEnd = {%
    \TYPE{END imageAttributeContext}%
    \GOBBLE},
  codeSpanAttributeContextBegin = {%
    \TYPE{BEGIN codeSpanAttributeContext}},
  codeSpanAttributeContextEnd = {%
    \TYPE{END codeSpanAttributeContext}%
    \GOBBLE},
  tableAttributeContextBegin = {%
    \TYPE{BEGIN tableAttributeContext}},
  tableAttributeContextEnd = {%
    \TYPE{END tableAttributeContext}%
    \GOBBLE},
  documentBegin = {%
    \TYPE{BEGIN document}},
  documentEnd = {%
    \TYPE{END document}},
  interblockSeparator = {%
    \TYPE{interblockSeparator}%
    \GOBBLE},
  paragraphSeparator = {%
    \TYPE{paragraphSeparator}%
    \GOBBLE},
  softLineBreak = {%
    \TYPE{softLineBreak}%
    \GOBBLE},
  hardLineBreak = {%
    \TYPE{hardLineBreak}%
    \GOBBLE},
  ellipsis = {%
    \TYPE{ellipsis}%
    \GOBBLE},
  sectionBegin = {%
    \TYPE{BEGIN section}},
  sectionEnd = {%
    \TYPE{END section}},
  headerAttributeContextBegin = {%
    \TYPE{BEGIN headerAttributeContext}},
  headerAttributeContextEnd = {%
    \TYPE{END headerAttributeContext}},
  nbsp = {%
    \TYPE{nbsp}%
    \GOBBLE},
  leftBrace = {%
    \TYPE{leftBrace}%
    \GOBBLE},
  rightBrace = {%
    \TYPE{rightBrace}%
    \GOBBLE},
  dollarSign = {%
    \TYPE{dollarSign}%
    \GOBBLE},
  percentSign = {%
    \TYPE{percentSign}%
    \GOBBLE},
  ampersand = {%
    \TYPE{ampersand}%
    \GOBBLE},
  underscore = {%
    \TYPE{underscore}%
    \GOBBLE},
  hash = {%
    \TYPE{hash}%
    \GOBBLE},
  circumflex = {%
    \TYPE{circumflex}%
    \GOBBLE},
  backslash = {%
    \TYPE{backslash}%
    \GOBBLE},
  tilde = {%
    \TYPE{tilde}%
    \GOBBLE},
  pipe = {%
    \TYPE{pipe}%
    \GOBBLE},
  codeSpan = {%
    \TYPE{codeSpan:              #1}},
  link = {%
    \TYPE{BEGIN link}%
    \TYPE{- label:               #1}%
    \TYPE{- URI:                 #3}%
    \TYPE{- title:               #4}%
    \TYPE{END link}},
  image = {%
    \TYPE{BEGIN image}%
    \TYPE{- label:               #1}%
    \TYPE{- URI:                 #3}%
    \TYPE{- title:               #4}%
    \TYPE{END image}},
  contentBlock = {%
    \TYPE{BEGIN contentBlock}%
    \TYPE{- suffix:              #1}%
    \TYPE{- URI:                 #3}%
    \TYPE{- title:               #4}%
    \TYPE{END contentBlock}},
  contentBlockOnlineImage = {%
    \TYPE{BEGIN contentBlockOnlineImage}%
    \TYPE{- suffix:              #1}%
    \TYPE{- URI:                 #3}%
    \TYPE{- title:               #4}%
    \TYPE{END contentBlockOnlineImage}},
  contentBlockCode = {%
    \TYPE{BEGIN contentBlockCode}%
    \TYPE{- suffix:              #1}%
    \TYPE{- language:            #2}%
    \TYPE{- URI:                 #4}%
    \TYPE{- title:               #5}%
    \TYPE{END contentBlockCode}},
  ulBegin = {%
    \TYPE{ulBegin}},
  ulBeginTight = {%
    \TYPE{ulBeginTight}},
  ulItem = {%
    \TYPE{ulItem}},
  ulItemEnd = {%
    \TYPE{ulItemEnd}},
  ulEnd = {%
    \TYPE{ulEnd}},
  ulEndTight = {%
    \TYPE{ulEndTight}},
  olBegin = {%
    \TYPE{olBegin}},
  olBeginTight = {%
    \TYPE{olBeginTight}},
  fancyOlBegin = {%
    \TYPE{BEGIN fancyOlBegin}%
    \TYPE{- numstyle:            #1}%
    \TYPE{- numdelim:            #2}%
    \TYPE{END fancyOlBegin}},
  fancyOlBeginTight = {%
    \TYPE{BEGIN fancyOlBeginTight}%
    \TYPE{- numstyle:            #1}%
    \TYPE{- numdelim:            #2}%
    \TYPE{END fancyOlBeginTight}},
  olItem = {%
    \TYPE{olItem}},
  olItemEnd = {%
    \TYPE{olItemEnd}},
  olItemWithNumber = {%
    \TYPE{olItemWithNumber:      #1}},
  fancyOlItem = {%
    \TYPE{fancyOlItem}},
  fancyOlItemEnd = {%
    \TYPE{fancyOlItemEnd}},
  fancyOlItemWithNumber = {%
    \TYPE{fancyOlItemWithNumber: #1}},
  olEnd = {%
    \TYPE{olEnd}},
  olEndTight = {%
    \TYPE{olEndTight}},
  fancyOlEnd = {%
    \TYPE{fancyOlEnd}},
  fancyOlEndTight = {%
    \TYPE{fancyOlEndTight}},
  dlBegin = {%
    \TYPE{dlBegin}},
  dlBeginTight = {%
    \TYPE{dlBeginTight}},
  dlItem = {%
    \TYPE{dlItem:                #1}},
  dlItemEnd = {%
    \TYPE{dlItemEnd}},
  dlDefinitionBegin = {%
    \TYPE{dlDefinitionBegin}},
  dlDefinitionEnd = {%
    \TYPE{dlDefinitionEnd}},
  dlEnd = {%
    \TYPE{dlEnd}},
  dlEndTight = {%
    \TYPE{dlEndTight}},
  emphasis = {%
    \TYPE{emphasis:              #1}},
  strongEmphasis = {%
    \TYPE{strongEmphasis:        #1}},
  blockQuoteBegin = {%
    \TYPE{blockQuoteBegin}},
  blockQuoteEnd = {%
    \TYPE{blockQuoteEnd}},
  lineBlockBegin = {%
    \TYPE{lineBlockBegin}},
  lineBlockEnd = {%
    \TYPE{lineBlockEnd}},
  inputVerbatim = {%
    \TYPE{inputVerbatim:         #1}},
  inputFencedCode = {%
    \TYPE{BEGIN fencedCode}%
    \TYPE{- src:                 #1}%
    \TYPE{- infostring:          #3}%
    \TYPE{END fencedCode}},
  headingOne = {%
    \TYPE{headingOne:            #1}},
  headingTwo = {%
    \TYPE{headingTwo:            #1}},
  headingThree = {%
    \TYPE{headingThree:          #1}},
  headingFour = {%
    \TYPE{headingFour:           #1}},
  headingFive = {%
    \TYPE{headingFive:           #1}},
  headingSix = {%
    \TYPE{headingSix:            #1}},
  thematicBreak = {%
    \TYPE{thematicBreak}},
  note = {%
    \TYPE{note:              #1}},
  cite = {%
    \CITATIONS{#1}},
  textCite = {%
    \TEXTCITATIONS{#1}},
  table = {%
    \TABLE{#1}{#2}{#3}},
  inlineHtmlComment = {%
    \TYPE{inlineHtmlComment:     #1}},
  inlineHtmlTag = {%
    \TYPE{inlineHtmlTag:         #1}},
  inputBlockHtmlElement = {%
    \TYPE{inputBlockHtmlElement: #1}},
  tickedBox = {%
    \TYPE{tickedBox}%
    \GOBBLE},
  halfTickedBox = {%
    \TYPE{halfTickedBox}%
    \GOBBLE},
  untickedBox = {%
    \TYPE{untickedBox}%
    \GOBBLE},
  jekyllDataBoolean = {%
    \TYPE{BEGIN jekyllDataBoolean}%
    \TYPE{- key:                 #1}%
    \TYPE{- value:               #2}%
    \TYPE{END jekyllDataBoolean}},
  jekyllDataEmpty = {%
    \TYPE{jekyllDataEmpty:       #1}},%
  jekyllDataNumber = {%
    \TYPE{BEGIN jekyllDataNumber}%
    \TYPE{- key:                 #1}%
    \TYPE{- value:               #2}%
    \TYPE{END jekyllDataNumber}},
  jekyllDataString = {%
    \TYPE{BEGIN jekyllDataString}%
    \TYPE{- key:                 #1}%
    \TYPE{- value:               #2}%
    \TYPE{END jekyllDataString}},
  jekyllDataBegin = {%
    \TYPE{jekyllDataBegin}},
  jekyllDataEnd = {%
    \TYPE{jekyllDataEnd}},
  jekyllDataSequenceBegin = {%
    \TYPE{BEGIN jekyllDataSequenceBegin}%
    \TYPE{- key:                 #1}%
    \TYPE{- length:              #2}%
    \TYPE{END jekyllDataSequenceBegin}},
  jekyllDataSequenceEnd = {%
    \TYPE{jekyllDataSequenceEnd}},
  jekyllDataMappingBegin = {%
    \TYPE{BEGIN jekyllDataMappingBegin}%
    \TYPE{- key:                 #1}%
    \TYPE{- length:              #2}%
    \TYPE{END jekyllDataMappingBegin}},
  jekyllDataMappingEnd = {%
    \TYPE{jekyllDataMappingEnd}},
  mark = {%
    \TYPE{mark:                  #1}},
  strikeThrough = {%
    \TYPE{strikeThrough:         #1}},
  superscript = {%
    \TYPE{superscript:           #1}},
  subscript = {%
    \TYPE{subscript:             #1}},
  displayMath = {%
    \TYPE{displayMath:           #1}},
  inlineMath = {%
    \TYPE{inlineMath:            #1}},
  inputRawInline = {%
    \TYPE{BEGIN rawInline}%
    \TYPE{- src:                 #1}%
    \TYPE{- raw attribute:       #2}%
    \TYPE{END rawInline}},
  inputRawBlock = {%
    \TYPE{BEGIN rawBlock}%
    \TYPE{- src:                 #1}%
    \TYPE{- raw attribute:       #2}%
    \TYPE{END rawBlock}},
  replacementCharacter = {%
    \TYPE{replacementCharacter}%
    \GOBBLE},
}%
