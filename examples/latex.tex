\documentclass{book}
\usepackage{ifxetex,ifluatex}
\ifxetex
  \usepackage{polyglossia}
  \setmainlanguage{english}
  \usepackage{fontspec}
\else\ifluatex
  \usepackage{polyglossia}
  \setmainlanguage{english}
  \usepackage{fontspec}
\else
  \usepackage[english]{babel}
  \usepackage[utf8]{inputenc}
  \usepackage[T1]{fontenc}
  \usepackage{lmodern}
\fi\fi
\usepackage{booktabs}
\usepackage[
  contentBlocks,
  debugExtensions,
  definitionLists,
  fancy_lists,
  fencedCode,
  hashEnumerators,
  inlineNotes,
  jekyllData,
  lineBlocks,
  notes,
  pipeTables,
  rawAttribute,
  smartEllipses,
  strikeThrough,
  subscripts,
  superscripts,
  tableCaptions,
  taskLists,
]{markdown}
\begin{markdown*}{hybrid}
---
title:  An Example *Markdown* Document
author: Vít Novotný
date:   \today
---
\end{markdown*}
\begin{document}
% Typeset the document `example.md` by letting the Markdown package handle
% the conversion internally.
\markdownInput{./example.md}

% Typeset the document `example.tex` that we prepared separately using the
% Lua command-line interface of the Markdown package and that contains a
% plain TeX representation of the document `example.md`.
\catcode`\%=12\relax
\catcode`\#=12\relax
\InputIfFileExists{./example.tex}{}{}
\catcode`\%=14\relax
\catcode`\#=6\relax

\begin{markdown}
Here are some non-ASCII characters: *ěščřžýáíé*.
\end{markdown}

\begin{markdown*}{html, hybrid}
Here is some <b>HTML code</b> mixed *with Markdown*. In \TeX, the HTML code
will be silently ignored, whereas in \TeX 4ht, the HTML code will be passed
through to the output:

<table border="1">
  <tr>
    <td>Emil</td>
    <td>Tobias</td>
    <td>Linus</td>
  </tr>
  <tr>
    <td>16</td>
    <td>14</td>
    <td>10</td>
  </tr>
</table>
\end{markdown*}
\end{document}
