\enableregime[utf]
\usetypescript[modern][ec]
\setupbodyfont[10pt,rm]
\setupexternalfigures[location={local,global,default}]

% Load the Markdown module.
\usemodule[t][markdown]

% Set options of the Markdown module.
\def\markdownOptionHashEnumerators{true}
\def\markdownOptionDefinitionLists{true}
\def\markdownOptionSmartEllipses{true}
\def\markdownOptionFootnotes{true}
\def\markdownOptionInlineFootnotes{true}
\def\markdownOptionFencedCode{true}
\def\markdownOptionContentBlocks{true}
\def\markdownOptionPipeTables{true}
\def\markdownOptionTableCaptions{true}
\def\markdownOptionTaskLists{true}

% Set renderers of the Markdown module.
\definetyping
  [latex]

\setuptyping
  [latex]
  [option=TEX]

\starttext

% Typeset the document `example.md` by letting the Markdown package handle
% the conversion internally.
\markdownInput{./example.md}

% Typeset the document `example.tex` that we prepared separately using the
% Lua command-line interface and that contains a plain TeX representation
% of the document `example.md`.
\catcode`\%=12\relax
\doiffileelse{./example.tex}{\input example}{}
\catcode`\%=14\relax

% Typeset inline markdown text.
\startmarkdown

Here are some non-ASCII characters: *ěščřžýáíé*
and ConTeXt special characters: *|*.

Here is a hard line break that we inserted directly from the TeX source
by typing two spaces at the end of a line.  
This is stretching TeX's abilities and is only supported in ConTeXt MkIV
and later.

\stopmarkdown

\stoptext
