\enableregime[utf]
\usetypescript[modern][ec]
\setupbodyfont[10pt,rm]
\setupexternalfigures[location={local,global,default}]

% Load the Markdown module.
\startluacode
local kpse = require("kpse")
kpse.set_program_name("luatex")
\stopluacode
\usemodule[t][markdown]

% Set options of the Markdown module.
\setupmarkdown
  [
    contentBlocks = yes,
    debugExtensions = yes,
    definitionLists = yes,
    fancy_lists = yes,
    fencedCode = yes,
    hashEnumerators = yes,
    inlineNotes = yes,
    lineBlocks = yes,
    notes = yes,
    pipeTables = yes,
    rawAttribute = yes,
    smartEllipses = yes,
    strikethrough = yes,
    subscripts = yes,
    superscripts = yes,
    tableCaptions = yes,
    taskLists = yes,
    texMathDollars = yes,
    texMathDoubleBackslash = yes,
    texMathSingleBackslash = yes,
  ]

% Set renderers of the Markdown module.
\definehighlight
  [markdownRendererEmphasis]
  [style=\em]

\definehighlight
  [markdownRendererStrongEmphasis]
  [style=bold]

\definehighlight
  [markdownRendererCodeSpan]
  [style=mono]

\definetyping
  [latex]

\setuptyping
  [latex]
  [option=TEX]

\starttext

% Set and typeset the document metadata using a YAML metadata block.
\def\doctitle#1{\gdef\@title{#1}}
\def\author#1{\gdef\@author{#1}}
\def\date#1{\gdef\@date{#1}}

\setupyaml
  [
    jekyllDataRenderers = {
      title = {\doctitle{#1}},
      author = {\author{#1}},
      date = {\date{#1}},
    },
    renderers = {
      jekyllDataEnd = {%
        \startalignment[center]
          \blank[force,2*big]
            {\tfd \@title}
          \blank[3*medium]
            {\tfa \@author}
          \blank[2*medium]
            {\tfa \@date}
          \blank[3*medium]
        \stopalignment
      },
    }
  ]

\startyaml

title:  An Example *Markdown* Document
author: Vít Starý Novotný
date:   `\currentdate`{=tex}

\stopyaml

% Typeset the document `example.md` by letting the Markdown package handle
% the conversion internally. Optionally, we can specify additional options
% between the square brackets similarly to the command `\setupmarkdown`.
% Unlike `\setupmarkdown`, the options will only apply for this document.
\inputmarkdown[smart_ellipses = yes]{./example.md}

% Typeset the document `example.tex` that we prepared separately using the
% Lua command-line interface and that contains a plain TeX representation
% of the document `example.md`.
\catcode`\%=12\relax
\catcode`\#=12\relax
\doiffileelse{./example.tex}{\input example}{}
\catcode`\%=14\relax
\catcode`\#=6\relax

% Besides inputting external files, Markdown text can we written directly
% into a LaTeX document. Markdown text and LaTeX code can be freely combined.
\startmarkdown

This is a paragraph of *Markdown text* with inline `\LaTeX`{=tex} code.

``` {=tex}
This is a paragraph of \LaTeX{} code with inline \markinline{*Markdown text*}.
```

\stopmarkdown

\stoptext
