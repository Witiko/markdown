% \iffalse
%<*driver>
\documentclass{ltxdockit}
\usepackage[american]{babel}
\usepackage{amsmath,btxdockit,doc,fancyvrb,graphicx,hologo,microtype,minted}

% Set up the style.
\usepackage{xcolor}
\definecolor{spot}{rgb}{0,0.2,0.6}
\usepackage{fontspec}
\defaultfontfeatures[\rmfamily,\sffamily,\ttfamily]{}
\emergencystretch=1em
\fvset{gobble=0,frame=single}
\setcounter{secnumdepth}{4}
\addtokomafont{title}{\sffamily}
\addtokomafont{paragraph}{\spotcolor}
\addtokomafont{section}{\spotcolor}
\addtokomafont{subsection}{\spotcolor}
\addtokomafont{subsubsection}{\spotcolor}
\addtokomafont{descriptionlabel}{\spotcolor}
\setkomafont{caption}{\bfseries\sffamily\spotcolor}
\setkomafont{captionlabel}{\bfseries\sffamily\spotcolor}
\hypersetup{citecolor=spot}
\let\oldCodelineNo\theCodelineNo
\def\theCodelineNo{\textcolor[gray]{0.5}{\oldCodelineNo}}
\makeatletter\renewcommand\fps@figure{htbp}\makeatother
\setkeys{Gin}{
  width = \columnwidth,
  height = 0.65\paperheight,
  keepaspectratio,
}

% Define some markup.
\let\pkg\relax % A package name
\def\inline#1{% Inline code
  \textcolor{spot}{\text{\texttt{#1}}}}
\newcommand\acro[1]{% An acronym
  \textsc{\MakeLowercase{#1}}}

% Set up index.
\DisableCrossrefs
\usepackage{makeidx}
\usepackage[columns=1, totoc]{idxlayout}
\makeindex

\newcommand\mdef[1]{% A TeX macro definition
  \index{#1@\cs{#1}|textit}%
  \phantomsection\label{macro:#1}\textcolor{spot}{\cs{#1}}}
\newcommand\mref[1]{% A TeX macro reference
  \index{#1@\cs{#1}}%
  \hyperref[macro:#1]{\textcolor{spot}{\cs{#1}}}}

\newcommand\envmdef[1]{% A LaTeX environment definition
  \index{#1@\texttt{#1}|textit}%
  \phantomsection\label{environment:#1}\inline{#1}}
\newcommand\envmref[1]{% A LaTeX environment reference
  \index{#1@\texttt{#1}}%
  \hyperref[environment:#1]{\inline{#1}}}

\newcommand\luamdef[1]{% A Lua object / method definition
  \index{#1@\texttt{#1}|textit}%
  \phantomsection\label{lua:#1}\inline{#1}}
\newcommand\luamref[1]{% A Lua object / method reference
  \index{#1@\texttt{#1}}%
  \hyperref[lua:#1]{\inline{#1}}}

\newcommand\Optitem[2][]{\penalty -1000\relax % An option item definition
  \index{#2@\texttt{#2}|textit}%
  \phantomsection\label{opt:#2}\optitem[#1]{#2}}
\newcommand\Valitem[2][]{\penalty -1000\relax % A value item definition
  \index{#2@\texttt{#2}}%
  \phantomsection\label{opt:#2}\valitem[#1]{#2}}
\newcommand\Opt[1]{% An option / value item reference
  \index{#1@\texttt{#1}}%
  \hyperref[opt:#1]{\inline{#1}}}

% Set up markdown.
\usepackage[
  citations,
  definitionLists,
  fencedCode,
  notes,
  headerAttributes,
  html,
  hybrid,
  inlineNotes,
  jekyllData,
  relativeReferences,
  stripPercentSigns,
  underscores = false,
]{markdown}
\markdownSetup{
  import = {
    witiko/dot,
    witiko/graphicx/http,
    witiko/markdown/techdoc = {
      options as lua-options
    },
  },
}

% Set up Unicode characters.
\usepackage{newunicodechar}
\newunicodechar{☒}{\markdownRendererTickedBox}
\newunicodechar{⌛}{\markdownRendererHalfTickedBox}
\newunicodechar{☐}{\markdownRendererUntickedBox}
\usepackage{emoji}
\makeatletter
\@ifpackagelater{emoji}{2020/03/16}{
  \newunicodechar{😉}{\emoji{winking-face}}
}{
  \newunicodechar{😉}{;-)}
}
\makeatother

% Set up the catcodes.
\catcode`\_=12 % We won't be typesetting much math and Lua contains lots of `_`.
\catcode`\^^B=8 % When we occasionally need subscripts, we will use `^^B` (STX).

% Set up the title page.
\begin{markdown}
---
title:    A Markdown Interpreter for \TeX
url:      https://github.com/witiko/markdown
authors:  [Vít Starý Novotný, Andrej Genčur]
email:    witiko@mail.muni.cz
revision: \markdownVersion
date:     \markdownLastModified
---
\end{markdown}
\CodelineIndex

% Set up the figures.
\usepackage{tikz}
\usepackage{mathabx,pgf-umlsd,tikz}
\usetikzlibrary{automata,arrows.meta,positioning,shapes.geometric}

% Set up the bibliography.
\usepackage[
  backend=biber,
  style=numeric,
  sorting=none,
  autolang=other,
  sortlocale=auto]{biblatex}
\addbibresource{markdown.bib}

\begin{document}
\DocInput{markdown.dtx}
\printbibliography[heading=bibintoc]
\printindex
\end{document}
%</driver>
%<*manual-css>

html {
  font-size: 100%;
  overflow-y: scroll;
  -webkit-text-size-adjust: 100%;
  -ms-text-size-adjust: 100%;
}

body {
  color: #444;
  font-family: Georgia, Palatino, 'Palatino Linotype', Times, 'Times New Roman', serif;
  font-size: 12px;
  line-height: 1.7;
  padding: 1em;
  margin: auto;
  max-width: 42em;
  background: #fefefe;
}

a {
  color: #0645ad;
}

a:visited {
  color: #0b0080;
}

a:hover {
  color: #06e;
}

a:active {
  color: #faa700;
}

a:focus {
  outline: thin dotted;
}

*::-moz-selection {
  background: rgba(255, 255, 0, 0.3);
  color: #000;
}

*::selection {
  background: rgba(255, 255, 0, 0.3);
  color: #000;
}

a::-moz-selection {
  background: rgba(255, 255, 0, 0.3);
  color: #0645ad;
}

a::selection {
  background: rgba(255, 255, 0, 0.3);
  color: #0645ad;
}

p {
  margin: 1em 0;
}

img {
  max-width: 100%;
}

h1, h2, h3, h4, h5, h6 {
  color: #111;
  line-height: 125%;
  margin-top: 2em;
  font-weight: normal;
}

h4, h5, h6 {
  font-weight: bold;
}

h1 {
  font-size: 2.5em;
}

h2 {
  font-size: 2em;
}

h3 {
  font-size: 1.5em;
}

h4 {
  font-size: 1.2em;
}

h5 {
  font-size: 1em;
}

h6 {
  font-size: 0.9em;
}

blockquote {
  color: #666666;
  margin: 0;
  padding-left: 3em;
  border-left: 0.5em #EEE solid;
}

hr {
  display: block;
  height: 2px;
  border: 0;
  border-top: 1px solid #aaa;
  border-bottom: 1px solid #eee;
  margin: 1em 0;
  padding: 0;
}

pre, code, kbd, samp {
  color: #000;
  font-family: monospace, monospace;
  _font-family: 'courier new', monospace;
  font-size: 0.98em;
}

pre {
  white-space: pre;
  white-space: pre-wrap;
  word-wrap: break-word;
  counter-reset: line;
}

.sourceCode.linenos > span {
  counter-increment: line;
}

.sourceCode.linenos > span:before{
  content: counter(line) "  ";
}

b, strong {
  font-weight: bold;
}

dfn {
  font-style: italic;
}

ins {
  background: #ff9;
  color: #000;
  text-decoration: none;
}

mark {
  background: #ff0;
  color: #000;
  font-style: italic;
  font-weight: bold;
}

sub, sup {
  font-size: 75%;
  line-height: 0;
  position: relative;
  vertical-align: baseline;
}

sup {
  top: -0.5em;
}

sub {
  bottom: -0.25em;
}

ul, ol {
  margin: 1em 0;
  padding: 0 0 0 2em;
}

li p:last-child {
  margin-bottom: 0;
}

ul ul, ol ol {
  margin: .3em 0;
}

dl {
  margin-bottom: 1em;
}

dt {
  font-weight: bold;
  margin-bottom: .8em;
}

dd {
  margin: 0 0 .8em 2em;
}

dd:last-child {
  margin-bottom: 0;
}

img {
  border: 0;
  -ms-interpolation-mode: bicubic;
  vertical-align: middle;
}

figure {
  display: block;
  text-align: center;
  margin: 1em 0;
}

figure img {
  border: none;
  margin: 0 auto;
}

figcaption {
  font-size: 0.8em;
  font-style: italic;
  margin: 0 0 .8em;
}

table {
  margin-bottom: 2em;
  border-bottom: 1px solid #ddd;
  border-right: 1px solid #ddd;
  border-spacing: 0;
  border-collapse: collapse;
}

table th {
  padding: .2em 1em;
  background-color: #eee;
  border-top: 1px solid #ddd;
  border-left: 1px solid #ddd;
}

table td {
  padding: .2em 1em;
  border-top: 1px solid #ddd;
  border-left: 1px solid #ddd;
  vertical-align: top;
}

.author {
  font-size: 1.2em;
  text-align: center;
}

@media only screen and (min-width: 480px) {
  body {
    font-size: 14px;
  }
}
@media only screen and (min-width: 768px) {
  body {
    font-size: 16px;
  }
}
@media print {
  * {
    background: transparent !important;
    color: black !important;
    filter: none !important;
    -ms-filter: none !important;
  }

  body {
    font-size: 12pt;
    max-width: 100%;
  }

  a, a:visited {
    text-decoration: underline;
  }

  hr {
    height: 1px;
    border: 0;
    border-bottom: 1px solid black;
  }

  a[href]:after {
    content: " (" attr(href) ")";
  }

  abbr[title]:after {
    content: " (" attr(title) ")";
  }

  .ir a:after, a[href^="javascript:"]:after, a[href^="#"]:after {
    content: "";
  }

  pre, blockquote {
    border: 1px solid #999;
    padding-right: 1em;
    page-break-inside: avoid;
  }

  tr, img {
    page-break-inside: avoid;
  }

  img {
    max-width: 100% !important;
  }

  @page :left {
    margin: 15mm 20mm 15mm 10mm;
}

  @page :right {
    margin: 15mm 10mm 15mm 20mm;
}

  p, h2, h3 {
    orphans: 3;
    widows: 3;
  }

  h2, h3 {
    page-break-after: avoid;
  }
}

.tex sub, .latex sub, .latex sup {
  text-transform: uppercase;
}

.tex sub, .latex sub {
  vertical-align: -0.1ex;
  margin-left: -0.1667em;
  margin-right: -0.125em;
}

.tex, .latex, .tex sub, .latex sub {
  font-size: 1em;
}

.latex sup {
  font-size: 0.85em;
  vertical-align: -0.35em;
  margin-left: -0.36em;
  margin-right: -0.15em;
}

abbr {
  text-transform: lowercase;
  font-variant: small-caps;
}

%</manual-css>
%<*techdoc-block-diagram>
\begingroup
\ifx\du\undefined\newlength{\du}\fi
\setlength{\du}{15\unitlength}
\begin{tikzpicture}
\pgftransformxscale{1.000000}
\pgftransformyscale{-1.23800}
\definecolor{dialinecolor}{rgb}{0.000000, 0.000000, 0.000000}
\pgfsetstrokecolor{dialinecolor}
\definecolor{dialinecolor}{rgb}{1.000000, 1.000000, 1.000000}
\pgfsetfillcolor{dialinecolor}
\definecolor{dialinecolor}{rgb}{1.000000, 1.000000, 1.000000}
\pgfsetfillcolor{dialinecolor}
\fill (4.282853\du,-3.750000\du)--(4.282853\du,-1.850000\du)--(11.971656\du,-1.850000\du)--(11.971656\du,-3.750000\du)--cycle;
\pgfsetlinewidth{0.040000\du}
\pgfsetdash{}{0pt}
\pgfsetdash{}{0pt}
\pgfsetmiterjoin
\definecolor{dialinecolor}{rgb}{0.000000, 0.000000, 0.000000}
\pgfsetstrokecolor{dialinecolor}
\draw (4.282853\du,-3.750000\du)--(4.282853\du,-1.850000\du)--(11.971656\du,-1.850000\du)--(11.971656\du,-3.750000\du)--cycle;
\definecolor{dialinecolor}{rgb}{0.000000, 0.000000, 0.000000}
\pgfsetstrokecolor{dialinecolor}
\node at (8.127254\du,-2.705000\du){\LaTeX{} layer};
\definecolor{dialinecolor}{rgb}{1.000000, 1.000000, 1.000000}
\pgfsetfillcolor{dialinecolor}
\fill (-5.055135\du,-3.788555\du)--(-5.055135\du,-1.888555\du)--(2.894865\du,-1.888555\du)--(2.894865\du,-3.788555\du)--cycle;
\pgfsetlinewidth{0.040000\du}
\pgfsetdash{}{0pt}
\pgfsetdash{}{0pt}
\pgfsetmiterjoin
\definecolor{dialinecolor}{rgb}{0.000000, 0.000000, 0.000000}
\pgfsetstrokecolor{dialinecolor}
\draw (-5.055135\du,-3.788555\du)--(-5.055135\du,-1.888555\du)--(2.894865\du,-1.888555\du)--(2.894865\du,-3.788555\du)--cycle;
\definecolor{dialinecolor}{rgb}{0.000000, 0.000000, 0.000000}
\pgfsetstrokecolor{dialinecolor}
\node at (-1.080135\du,-2.743555\du){\Hologo{ConTeXt} layer};
\definecolor{dialinecolor}{rgb}{1.000000, 1.000000, 1.000000}
\pgfsetfillcolor{dialinecolor}
\fill (-8.200000\du,-0.224996\du)--(-8.200000\du,1.675004\du)--(11.870000\du,1.675004\du)--(11.870000\du,-0.224996\du)--cycle;
\pgfsetlinewidth{0.040000\du}
\pgfsetdash{}{0pt}
\pgfsetdash{}{0pt}
\pgfsetmiterjoin
\definecolor{dialinecolor}{rgb}{0.000000, 0.000000, 0.000000}
\pgfsetstrokecolor{dialinecolor}
\draw (-8.200000\du,-0.224996\du)--(-8.200000\du,1.675004\du)--(11.870000\du,1.675004\du)--(11.870000\du,-0.224996\du)--cycle;
\definecolor{dialinecolor}{rgb}{0.000000, 0.000000, 0.000000}
\pgfsetstrokecolor{dialinecolor}
\node at (1.835000\du,0.820004\du){Plain \TeX{} layer};
\definecolor{dialinecolor}{rgb}{1.000000, 1.000000, 1.000000}
\pgfsetfillcolor{dialinecolor}
\fill (-11.150000\du,3.325006\du)--(-11.150000\du,5.225006\du)--(11.800000\du,5.225006\du)--(11.800000\du,3.325006\du)--cycle;
\pgfsetlinewidth{0.040000\du}
\pgfsetdash{}{0pt}
\pgfsetdash{}{0pt}
\pgfsetmiterjoin
\definecolor{dialinecolor}{rgb}{0.000000, 0.000000, 0.000000}
\pgfsetstrokecolor{dialinecolor}
\draw (-11.150000\du,3.325006\du)--(-11.150000\du,5.225006\du)--(11.800000\du,5.225006\du)--(11.800000\du,3.325006\du)--cycle;
\definecolor{dialinecolor}{rgb}{0.000000, 0.000000, 0.000000}
\pgfsetstrokecolor{dialinecolor}
\node at (0.325000\du,4.370006\du){Lua layer};
\pgfsetlinewidth{0.040000\du}
\pgfsetdash{}{0pt}
\pgfsetdash{}{0pt}
\pgfsetbuttcap
{
\definecolor{dialinecolor}{rgb}{0.000000, 0.000000, 0.000000}
\pgfsetfillcolor{dialinecolor}
\pgfsetarrowsstart{stealth}
\pgfsetarrowsend{stealth}
\definecolor{dialinecolor}{rgb}{0.000000, 0.000000, 0.000000}
\pgfsetstrokecolor{dialinecolor}
\draw (-9.906400\du,-5.181570\du)--(-9.937836\du,3.024193\du);
}
\pgfsetlinewidth{0.040000\du}
\pgfsetdash{}{0pt}
\pgfsetdash{}{0pt}
\pgfsetbuttcap
{
\definecolor{dialinecolor}{rgb}{0.000000, 0.000000, 0.000000}
\pgfsetfillcolor{dialinecolor}
\pgfsetarrowsstart{stealth}
\pgfsetarrowsend{stealth}
\definecolor{dialinecolor}{rgb}{0.000000, 0.000000, 0.000000}
\pgfsetstrokecolor{dialinecolor}
\draw (-6.919693\du,-5.181570\du)--(-6.938734\du,-0.450800\du);
}
\pgfsetlinewidth{0.040000\du}
\pgfsetdash{}{0pt}
\pgfsetdash{}{0pt}
\pgfsetbuttcap
{
\definecolor{dialinecolor}{rgb}{0.000000, 0.000000, 0.000000}
\pgfsetfillcolor{dialinecolor}
\pgfsetarrowsstart{stealth}
\pgfsetarrowsend{stealth}
\definecolor{dialinecolor}{rgb}{0.000000, 0.000000, 0.000000}
\pgfsetstrokecolor{dialinecolor}
\draw (-1.063746\du,-5.175791\du)--(-1.063746\du,-3.975793\du);
}
\definecolor{dialinecolor}{rgb}{1.000000, 1.000000, 1.000000}
\pgfsetfillcolor{dialinecolor}
\fill (-10.972500\du,-7.267480\du)--(-10.972500\du,-5.367480\du)--(11.977500\du,-5.367480\du)--(11.977500\du,-7.267480\du)--cycle;
\pgfsetlinewidth{0.040000\du}
\pgfsetdash{}{0pt}
\pgfsetdash{}{0pt}
\pgfsetmiterjoin
\definecolor{dialinecolor}{rgb}{0.000000, 0.000000, 0.000000}
\pgfsetstrokecolor{dialinecolor}
\draw (-10.972500\du,-7.267480\du)--(-10.972500\du,-5.367480\du)--(11.977500\du,-5.367480\du)--(11.977500\du,-7.267480\du)--cycle;
\definecolor{dialinecolor}{rgb}{0.000000, 0.000000, 0.000000}
\pgfsetstrokecolor{dialinecolor}
\node at (0.502500\du,-6.322480\du){User code};
\pgfsetlinewidth{0.040000\du}
\pgfsetdash{}{0pt}
\pgfsetdash{}{0pt}
\pgfsetbuttcap
{
\definecolor{dialinecolor}{rgb}{0.000000, 0.000000, 0.000000}
\pgfsetfillcolor{dialinecolor}
\pgfsetarrowsstart{stealth}
\pgfsetarrowsend{stealth}
\definecolor{dialinecolor}{rgb}{0.000000, 0.000000, 0.000000}
\pgfsetstrokecolor{dialinecolor}
\draw (8.149986\du,-5.159541\du)--(8.149986\du,-3.959543\du);
}
\pgfsetlinewidth{0.040000\du}
\pgfsetdash{}{0pt}
\pgfsetdash{}{0pt}
\pgfsetbuttcap
{
\definecolor{dialinecolor}{rgb}{0.000000, 0.000000, 0.000000}
\pgfsetfillcolor{dialinecolor}
\pgfsetarrowsstart{stealth}
\pgfsetarrowsend{stealth}
\definecolor{dialinecolor}{rgb}{0.000000, 0.000000, 0.000000}
\pgfsetstrokecolor{dialinecolor}
\draw (-1.074995\du,-1.634548\du)--(-1.074995\du,-0.434550\du);
}
\pgfsetlinewidth{0.040000\du}
\pgfsetdash{}{0pt}
\pgfsetdash{}{0pt}
\pgfsetbuttcap
{
\definecolor{dialinecolor}{rgb}{0.000000, 0.000000, 0.000000}
\pgfsetfillcolor{dialinecolor}
\pgfsetarrowsstart{stealth}
\pgfsetarrowsend{stealth}
\definecolor{dialinecolor}{rgb}{0.000000, 0.000000, 0.000000}
\pgfsetstrokecolor{dialinecolor}
\draw (8.138736\du,-1.618298\du)--(8.138736\du,-0.418300\du);
}
\pgfsetlinewidth{0.040000\du}
\pgfsetdash{}{0pt}
\pgfsetdash{}{0pt}
\pgfsetbuttcap
{
\definecolor{dialinecolor}{rgb}{0.000000, 0.000000, 0.000000}
\pgfsetfillcolor{dialinecolor}
\pgfsetarrowsstart{stealth}
\pgfsetarrowsend{stealth}
\definecolor{dialinecolor}{rgb}{0.000000, 0.000000, 0.000000}
\pgfsetstrokecolor{dialinecolor}
\draw (1.689638\du,1.881695\du)--(1.689638\du,3.081693\du);
}
\end{tikzpicture}
\endgroup
%</techdoc-block-diagram>
%<*techdoc-bibliography>
@online{starynovotny24,
  author    = {Starý Novotný, Vít and Enrico Gregorio and Max Chernoff and P. Spratte, Jonathan},
  title     = {Convert control sequence with a variable number of undelimited parameters into a token list},
  url       = {https://tex.stackexchange.com/q/716362/70941},
  urldate   = {2024-04-28},
}
@book{tantau21,
  author    = {Till Tantau and Joseph Wright and Vedran Miletić},
  title     = {The Beamer class},
  date      = {2021-02-10},
  url       = {https://mirrors.ctan.org/macros/latex/contrib/beamer/doc/beameruserguide.pdf},
  urldate   = {2021-02-11}}
@online{sotkov17,
  author    = {Sotkov, Anton},
  title     = {File transclusion syntax for Markdown},
  date      = {2017-01-19},
  url       = {https://github.com/iainc/Markdown-Content-Blocks},
  urldate   = {2018-01-08}}
@book{luatex21,
  author    = {{Lua\TeX{} development team}},
  title     = {Lua\TeX{} reference manual},
  date      = {2021-07-23},
  note      = {Version 1.10 (stable)},
  url       = {https://www.pragma-ade.com/general/manuals/luatex.pdf},
  urldate   = {2022-09-30}}
@book{latex17,
  author    = {Braams, Johannes and Carlisle, David and Jeffrey, Alan and
               Lamport, Leslie and Mittelbach, Frank and Rowley, Chris and
               Schöpf, Rainer},
  title     = {The \Hologo{LaTeX2e} Sources},
  date      = {2017-04-15},
  url       = {https://mirrors.ctan.org/macros/latex/base/source2e.pdf},
  urldate   = {2018-01-08}}
@book{mittelbach17,
  author    = {Mittelbach, Frank},
  title     = {The \texttt{doc} and \texttt{shortvrb} Packages},
  date      = {2017-04-15},
  url       = {https://mirrors.ctan.org/macros/latex/base/doc.pdf},
  urldate   = {2018-02-19}}
@book{poore17,
  author    = {Poore, Geoffrey M.},
  title     = {The \texttt{minted} Package},
  subtitle  = {Highlighted source code in \LaTeX},
  date      = {2017-07-19},
  version   = {v2.5},
  url       = {https://mirrors.ctan.org/macros/latex/contrib/minted/minted.pdf},
  urldate   = {2020-09-01}}
@online{macfarlane22,
  title     = {Pandoc},
  subtitle  = {a universal document converter},
  author    = {John MacFarlane},
  year      = {2022},
  url       = {https://pandoc.org/},
  urldate   = {2022-10-05}}
@online{novotny15,
  author    = {Novotný, Vít},
  year      = {2015},
  title     = {TeXový interpret jazyka Markdown (markdown.sty)},
  location  = {Brno, Czech Republic},
  publisher = {Masaryk University},
  url       = {https://www.muni.cz/en/research/projects/32984},
  urldate   = {2018-02-19}}
@book{ierusalimschy13,
  author    = {Ierusalimschy, Roberto},
  year      = {2013},
  title     = {Programming in Lua},
  edition   = {3},
  isbn      = {978-85-903798-5-0},
  pagetotal = {xviii, 347},
  location  = {Rio de Janeiro},
  publisher = {PUC-Rio}}
@book{knuth86a,
  author    = {Knuth, Donald Ervin},
  year      = {1986},
  title     = {The \TeX{}book},
  edition   = {3},
  isbn      = {0-201-13447-0},
  pagetotal = {ix, 479},
  series    = {Computers \& Typesetting},
  volume    = {A},
  location  = {Reading, MA},
  publisher = {Addison-Wesley}}
@book{knuth86b,
  author    = {Knuth, Donald Ervin},
  year      = {1986},
  title     = {\TeX: The Program},
  isbn      = {978-0-201-13437-7},
  pagetotal = {xvi, 594},
  series    = {Computers \& Typesetting},
  volume    = {B},
  location  = {Reading, MA},
  publisher = {Addison-Wesley}}
@book{eijkhout92,
  author    = {Victor Eijkhout},
  title     = {\TeX{} by Topic},
  subtitle  = {A \TeX nician's Reference},
  isbn      = {978-0-201-56882-0},
  pagetotal = {307},
  date      = {1992-02-01},
  location  = {Wokingham, England},
  publisher = {Addison-Wesley}}
@inproceedings{sharif10,
  author    = {Sharif, Bonita and Maletic, Jonathan I.},
  booktitle = {2010 IEEE 18th International Conference on Program Comprehension},
  title     = {An Eye Tracking Study on camelCase and under\_score Identifier Styles},
  year      = {2010},
  pages     = {196-205},
  doi       = {10.1109/ICPC.2010.41}}
%</techdoc-bibliography>
%<@@=markdown>
%<*themes-witiko-markdown-techdoc>
\ProvidesPackage{markdownthemewitiko_markdown_techdoc}[2022/12/13]
\RequirePackage{etoolbox}
\gdef\ltd@title@author{\@gobble}%
\markdownSetup{
  renderers = {
    head*Four = {\paragraph{#1}\leavevmode},
  },
  rendererPrototypes = {
    codeSpan = {\inline{#1}},
    jekyllData(End) = {%
      \AfterEndPreamble{%
        \hypersetup{pdfauthor=\ltd@title@author}%
        \printtitlepage
        \tableofcontents
        {\def\addcontentsline##1##2##3{}\listoffigures}%
      }%
    },
  },
  jekyllDataRenderers = {
    /authors/* = {%
      \expandafter\gdef\expandafter\ltd@title@author\expandafter{\ltd@title@author, #1}%
    },
    title = {%
      \gdef\ltd@title@title{#1}%
      \gdef\ltd@title@subtitle{}%
      \hypersetup{pdftitle={#1}}%
    },
    date = {\gdef\ltd@title@date{#1}},
    email = {\gdef\ltd@title@email{#1}},
    revision = {\gdef\ltd@title@revision{#1}},
    url = {\gdef\ltd@title@url{#1}},
  }
}
\ExplSyntaxOn
\markdownSetup{
  rendererPrototypes = {
    headerAttributeContextEnd = {
      \seq_map_inline:Nn
        \l_@@_header_identifiers_seq
        { \label { sec:##1 } }
      \seq_clear:N
        \l_@@_header_identifiers_seq
    },
  },
}
\ExplSyntaxOff
\renewcommand\markdownLaTeXRendererDirectOrIndirectLink[4]{%
  #1\footnote{See \url{#3}.}}
\RequirePackage{varioref}
\vrefwarning
\markdownSetupSnippet{options}{
  rendererPrototypes = {
    dlBegin = {\begin{optionlist}},
    dlItem = {
      #1
      \begingroup
      \markdownSetup{
        renderers = {
          dlBegin = {
            \begingroup
            \markdownSetup{
              renderers = {
                dlItem = {\item[####1]},
                dlItemEnd = {}}}
            \begin{valuelist}
          },
          dlEnd = {
            \end{valuelist}
            \endgroup
          },
        },
      }%
    },
    dlItemEnd = {\endgroup},
    dlEnd = {\end{optionlist}},
  }
}
%</themes-witiko-markdown-techdoc>
%<*manual>

---
title:  Markdown Package User Manual
author: Vít Starý Novotný, Andrej Genčur
date:   \markdownVersion{} \markdownLastModified{}
---

<link href="https://afeld.github.io/emoji-css/emoji.css" rel="stylesheet" />

% \fi
% \par
% \begin{markdown}

Introduction
============
The [Markdown package][pkg] converts [CommonMark][] markup to \TeX{} commands. The
functionality is provided both as a Lua module and as plain \TeX{}, \LaTeX{}, and
\Hologo{ConTeXt} macro packages that can be used to directly typeset \TeX{} documents
containing markdown markup. Unlike other converters, the Markdown package
does not require any external programs, and makes it easy to redefine how each
and every markdown element is rendered.  Creative abuse of the markdown
syntax is encouraged. 😉

 [commonmark]: https://commonmark.org/
               (CommonMark: A strongly defined, highly compatible specification of Markdown)
 [pkg]:        https://ctan.org/pkg/markdown
               (CTAN: Package markdown)

% This document is a technical documentation for the \pkg{Markdown} package. It
% consists of three sections. This section introduces the package and outlines
% its prerequisites. Section <#sec:interfaces> describes the interfaces
% exposed by the package. Section <#sec:implementation> describes the
% implementation of the package. The technical documentation contains only a
% limited number of tutorials and code examples. You can find more of these in
% the [user manual.][manual]
%
%  [manual]: http://mirrors.ctan.org/macros/generic/markdown/markdown.html
%
% \end{markdown}
% \iffalse

This document is a user manual for the [Markdown package][pkg]. It provides
tutorials and code examples. For an in-depth description of the package
requirements, interfaces, and implementation, please refer to the [technical
documentation][techdoc].

 [techdoc]: http://mirrors.ctan.org/macros/generic/markdown/markdown.pdf
            (A Markdown Interpreter for TeX)

%</manual>
%<*lua,lua-cli>
% \fi
%  \begin{macrocode}
local metadata = {
    version   = "(((VERSION)))",
    comment   = "A module for the conversion from markdown to plain TeX",
    author    = "John MacFarlane, Hans Hagen, Vít Starý Novotný, Andrej Genčur",
    copyright = {"2009-2016 John MacFarlane, Hans Hagen",
                 "2016-2024 Vít Starý Novotný, Andrej Genčur"},
    license   = "LPPL 1.3c"
}

%    \end{macrocode}
% \iffalse
%</lua,lua-cli>
%<*lua>
% \fi
%  \begin{macrocode}
if not modules then modules = { } end
modules['markdown'] = metadata
%    \end{macrocode}
% \iffalse
%</lua>
%<*manual>

Requirements
------------

The package requires either [our official Docker image][docker], which contains
the latest development version of the Markdown package, or a \TeX{}
distribution: [\TeX{} Live][tl] ≥ 2022 is known to work with the current
version of the Markdown package and so are recent versions of [Mik\TeX{}][mik].
If you are using an older, incomplete, or atypical \TeX{} distribution, please
consult the [technical documentation][techdoc] for a detailed list of
requirements.

 [docker]: https://hub.docker.com/r/witiko/markdown/tags (witiko/markdown - Docker Image)
 [tl]: https://www.tug.org/texlive/ (TeX Live - TeX Users Group)
 [mik]: https://miktex.org/ (Home - MiKTeXorg)

Installation
------------

If the Markdown package is not included in your \TeX{} distribution, you will
need to install it.

From [Releases][], download [an archive `markdown.zip` for this version of the
Markdown package (\markdownShortVersion{})][this-release] or a different version
that you wish to install. Then, unzip the archive. If you downloaded an archive
for a different version of the Markdown package, you should now locate a file
named `markdown.html` with the user manual for that version, open it, and
follow the installation steps in it rather than the steps from this manual.

 [releases]: https://github.com/witiko/markdown/releases (Releases - witiko/markdown)
 [this-release]: https://github.com/witiko/markdown/releases/download/\markdownShortVersion{}/markdown.zip (Release \markdownShortVersion{} - witiko/markdown)

Alternatively, download the package from the repository using Git, enter the
directory named `markdown` and run the `make base` command using GNU Make:
``` sh
git clone https://github.com/witiko/markdown
cd markdown
make base
``````

Either of the two abovelisted approaches should produce the following files:

* `markdown.lua`: The Lua module
* `libraries/markdown-tinyyaml.lua`: An external library for reading \acro{yaml}
* `markdown-cli.lua`: The Lua command-line interface
* `markdown.tex`: The plain \TeX{} macro package
* `markdown.sty`: The \LaTeX{} package
* `markdownthemewitiko_dot.sty`: The `witiko/dot` \LaTeX{} theme
* `markdownthemewitiko_graphicx_http.sty`: The `witiko/graphicx/http` \LaTeX{} theme
* `markdownthemewitiko_tilde.tex`: The `witiko/tilde` theme
* `markdownthemewitiko_markdown_defaults.tex`,
  `markdownthemewitiko_markdown_defaults.sty`, and
  `t-markdownthemewitiko_markdown_defaults.tex`: The `witiko/markdown/defaults`
  theme
* `t-markdown.tex`: The \Hologo{ConTeXt} module

### Local Installation

To perform a local installation, place the above files into your \TeX{}
directory structure. This is generally where the individual files should be
placed:

* `⟨TEXMF⟩/tex/luatex/markdown/markdown.lua`
* `⟨TEXMF⟩/tex/luatex/markdown/markdown-tinyyaml.lua`
* `⟨TEXMF⟩/scripts/markdown/markdown-cli.lua`
* `⟨TEXMF⟩/tex/generic/markdown/markdown.tex`
* `⟨TEXMF⟩/tex/generic/markdown/markdownthemewitiko_tilde.tex`
* `⟨TEXMF⟩/tex/generic/markdown/markdownthemewitiko_markdown_defaults.tex`
* `⟨TEXMF⟩/tex/latex/markdown/markdown.sty`
* `⟨TEXMF⟩/tex/latex/markdown/markdownthemewitiko_dot.sty`
* `⟨TEXMF⟩/tex/latex/markdown/markdownthemewitiko_graphicx_http.sty`
* `⟨TEXMF⟩/tex/latex/markdown/markdownthemewitiko_markdown_defaults.sty`
* `⟨TEXMF⟩/tex/context/third/markdown/t-markdown.tex`
* `⟨TEXMF⟩/tex/context/third/markdown/t-markdownthemewitiko_markdown_defaults.tex`

where `⟨TEXMF⟩` corresponds to a root of your \TeX{} distribution, such as
`/usr/share/texmf` and `~/texmf` on UN\*X systems or
`C:\Users\`\meta{Your username}`\texmf` on Windows systems. When in doubt,
consult the manual of your \TeX{} distribution.

### Portable Installation

Alternatively, you can also store the above files in the same folder as your
\TeX{} document and distribute them together. This way your document can be
portably typeset on legacy \TeX{} distributions.

This is where the individual files should be placed:

* `./markdown.lua`
* `./markdown-tinyyaml.lua`
* `./markdown-cli.lua`
* `./markdown/markdown.tex`
* `./markdown.sty`
* `./t-markdown.tex`
* `./markdownthemewitiko_dot.sty`
* `./markdownthemewitiko_graphicx_http.sty`
* `./markdownthemewitiko_tilde.tex`
* `./markdownthemewitiko_markdown_defaults.tex`
* `./markdownthemewitiko_markdown_defaults.sty`
* `./t-markdownthemewitiko_markdown_defaults.tex`

The file `markdown.tex` *must* be placed in a directory named `markdown`.

%</manual>
%<*lua>
% \fi
% \par
% \begin{markdown}
%
% Requirements
%-------------
%
% This section gives an overview of all resources required by the package.
%
%### Lua Requirements {#lua-prerequisites}
%
% The Lua part of the package requires that the following Lua modules are
% available from within the Lua\TeX{} engine (though not necessarily in the
% LuaMeta\TeX{} engine).
%
% \pkg{LPeg${}\geq{}$0.10}
%
%:    A pattern-matching library for the writing of recursive descent parsers
%     via the Parsing Expression Grammars (\acro{peg}s). It is used by the
%     \pkg{Lunamark} library to parse the markdown input.
%     \pkg{LPeg${}\geq{}$0.10} is included in Lua\TeX${}\geq{}$0.72.0 (\TeX
%     Live${}\geq{}2013$).
%
% \end{markdown}
%  \begin{macrocode}
local lpeg = require("lpeg")
%    \end{macrocode}
% \par
% \begin{markdown}
%
% \pkg{Selene Unicode}
%
%:    A library that provides support for the processing of wide strings. It is
%     used by the \pkg{Lunamark} library to cast image, link, and note tags
%     to the lower case. \pkg{Selene Unicode} is included in all releases of
%     Lua\TeX{} (\TeX Live${}\geq{}2008$).
%
% \end{markdown}
%  \begin{macrocode}
local unicode = require("unicode")
%    \end{macrocode}
% \par
% \begin{markdown}
%
% \pkg{MD5}
%
%:    A library that provides \acro{md5} crypto functions. It is used by the
%     \pkg{Lunamark} library to compute the digest of the input for caching
%     purposes. \pkg{MD5} is included in all releases of Lua\TeX{} (\TeX
%     Live${}\geq{}2008$).
%
% \end{markdown}
%  \begin{macrocode}
local md5 = require("md5");
%    \end{macrocode}
% \begin{markdown}
%
% \pkg{Kpathsea}
%
%:    A package that implements the loading of third-party Lua libraries
%     and looking up files in the \TeX{} directory structure.
%
% \end{markdown}
%  \begin{macrocode}
(function()
%    \end{macrocode}
% \begin{markdown}
%
% If \pkg{Kpathsea} has not been loaded before or if Lua\TeX{} has not yet
% been initialized, configure \pkg{Kpathsea} on top of loading it. Since
% \Hologo{ConTeXt} MkIV provides a `kpse` global that acts as a
% stub for \pkg{Kpathsea} and the \pkg{lua-uni-case} library expects that
% `kpse` is a reference to the full \pkg{Kpathsea} library, we load
% \pkg{Kpathsea} to the `kpse` global.
%
% \end{markdown}
%  \begin{macrocode}
  local should_initialize = package.loaded.kpse == nil
                       or tex.initialize ~= nil
  kpse = require("kpse")
  if should_initialize then
    kpse.set_program_name("luatex")
  end
end)()
%    \end{macrocode}
% \par
% \begin{markdown}
%
% All the abovelisted modules are statically linked into the current version of
% the Lua\TeX{} engine~[@luatex21, Section 4.3]. Beside these, we also include
% the following third-party Lua libraries:
%
% \pkg{lua-uni-algos}
%
%:    A package that implements Unicode case-folding in \TeX{} Live${}\geq{}2020$.
%
% \end{markdown}
% \iffalse
%</lua>
%<*depends>
% \fi
%  \begin{macrocode}
hard lua-uni-algos
%    \end{macrocode}
% \iffalse
%</depends>
%<*lua>
% \fi
%  \begin{macrocode}
local uni_algos = require("lua-uni-algos")
%    \end{macrocode}
% \par
% \begin{markdown}
%
% \pkg{api7/lua-tinyyaml}
%
%:    A library that provides a regex-based recursive descent \acro{yaml}
%     (subset) parser that is used to read \acro{yaml} metadata when the
%     \Opt{jekyllData} option is enabled. We carry a copy of the library
%     in file `markdown-tinyyaml.lua` distributed together with the Markdown
%     package. <!-- TODO: Stop carrying the copy of the library in TeX Live
%     2023. -->
%
% \end{markdown}
% \iffalse
%</lua>
%<*depends>
% \fi
%  \begin{macrocode}
# hard lua-tinyyaml  # TODO: Uncomment after TeX Live 2022 has been deprecated.
%    \end{macrocode}
% \iffalse
%</depends>
%<*tex>
% \fi
% \par
% \begin{markdown}
%
%### Plain \TeX{} Requirements {#tex-prerequisites}
%
% The plain \TeX{} part of the package requires that the plain \TeX{}
% format (or its superset) is loaded, all the Lua prerequisites (see
% Section <#sec:lua-prerequisites>), and the following packages:
%
% \pkg{expl3}
%
%:    A package that enables the expl3 language from the \LaTeX3 kernel in
%     \TeX{} Live${}\leq{}2019$. It is used to implement reflection
%     capabilities that allow us to enumerate and inspect high-level concepts
%     such as options, renderers, and renderer prototypes.
%
% \end{markdown}
% \iffalse
%</tex>
%<*depends>
% \fi
%  \begin{macrocode}
hard l3kernel
%    \end{macrocode}
% \iffalse
%</depends>
%<*context>
% \fi
%  \begin{macrocode}
\unprotect
%    \end{macrocode}
% \iffalse
%</context>
%<*context,tex>
% \fi
%  \begin{macrocode}
\ifx\ExplSyntaxOn\undefined
  \input expl3-generic
\fi
%    \end{macrocode}
% \iffalse
%</context,tex>
%<*tex>
% \fi
% \begin{markdown}
%
% \pkg{lt3luabridge}
%
%:    A package that allows us to execute Lua code with LuaTeX as well as
%     with other TeX engines that provide the *shell escape* capability,
%     which allows them to execute code with the system's shell.
%
% \end{markdown}
% \iffalse
%</tex>
%<*depends>
% \fi
%  \begin{macrocode}
hard lt3luabridge
%    \end{macrocode}
% \iffalse
%</depends>
%<*tex>
% \fi
% \begin{markdown}
%
% The plain \TeX{} part of the package also requires the following Lua module:
%
% \pkg{Lua File System}
%
%:    A library that provides access to the filesystem via \acro{os}-specific
%     syscalls. It is used by the plain \TeX{} code to create the cache
%     directory specified by the \Opt{cacheDir} option before interfacing with
%     the \pkg{Lunamark} library. \pkg{Lua File System} is included in all
%     releases of Lua\TeX{} (\TeX Live${}\geq{}2008$).
%
%     The plain \TeX{} code makes use of the `isdir` method that was added
%     to the \pkg{Lua File System} library by the Lua\TeX{} engine
%     developers~[@luatex21, Section 4.2.4].
%
% The \pkg{Lua File System} module is statically linked into the Lua\TeX{}
% engine~[@luatex21, Section 4.3].
%
% Unless you convert markdown documents to \TeX{} manually using the Lua
% command-line interface (see Section <#sec:lua-cli-interface>), the plain
% \TeX{} part of the package will require that either the Lua\TeX{}
% `\directlua` primitive or the shell access file stream 18 is available in
% your \TeX{} engine. If only the shell access file stream is available in your
% \TeX{} engine (as is the case with \hologo{pdfTeX} and \Hologo{XeTeX}), then
% unless your \TeX{} engine is globally configured to enable shell access, you
% will need to provide the `-shell-escape` parameter to your engine when
% typesetting a document.
%
% \end{markdown}
% \iffalse
%</tex>
%<*latex>
% \fi
% \par
% \begin{markdown}
%
%### \LaTeX{} Requirements {#latex-prerequisites}
%
% The \LaTeX{} part of the package requires that the \Hologo{LaTeX2e} format is
% loaded,
% \end{markdown}
% \iffalse
%<*themes-witiko-dot,latex-themes-witiko-graphicx-http>
% \fi
%  \begin{macrocode}
\NeedsTeXFormat{LaTeX2e}
\RequirePackage{expl3}
%    \end{macrocode}
% \iffalse
%</themes-witiko-dot,latex-themes-witiko-graphicx-http>
%</latex>
%<*depends>
% \fi
% \begin{markdown}
% a \TeX{} engine that extends \Hologo{eTeX}, and all the plain \TeX{}
% prerequisites (see Section <#sec:tex-prerequisites>):
%
% The following packages are soft prerequisites. They are only used to provide
% default token renderer prototypes (see sections
% <#sec:texrendererprototypes> and
% <#sec:latex-token-renderer-prototypes>) or \LaTeX{} themes (see Section
% <#sec:latexthemes>) and will not be loaded if the option `plain` has been
% enabled (see Section <#sec:plain>):
%
% \pkg{url}
%
%:    A package that provides the `\url` macro for the typesetting of links.
%
% \end{markdown}
%  \begin{macrocode}
soft url
%    \end{macrocode}
% \begin{markdown}
%
% \pkg{graphicx}
%
%:    A package that provides the `\includegraphics` macro for the typesetting
%     of images.Furthermore, it also provides a key-value interface that is
%     used in the default renderer prototypes for image attribute contexts.
%
% \end{markdown}
%  \begin{macrocode}
soft graphics
%    \end{macrocode}
% \begin{markdown}
%
% \pkg{paralist}
%
%:    A package that provides the `compactitem`, `compactenum`, and
%     `compactdesc` macros for the typesetting of tight bulleted lists,
%     ordered lists, and definition lists as well as the rendering of
%     fancy lists.
%
% \end{markdown}
%  \begin{macrocode}
soft paralist
%    \end{macrocode}
% \begin{markdown}
%
% \pkg{ifthen}
%
%:    A package that provides a concise syntax for the inspection of macro
%     values. It is used in the `witiko/dot` \LaTeX{} theme (see Section
%     <#sec:latexthemes>).
%
% \end{markdown}
%  \begin{macrocode}
soft latex
soft epstopdf-pkg  # required by `latex`
%    \end{macrocode}
% \begin{markdown}
%
% \pkg{fancyvrb}
%
%:    A package that provides the `\VerbatimInput` macros for the verbatim
%     inclusion of files containing code.
%
% \end{markdown}
%  \begin{macrocode}
soft fancyvrb
%    \end{macrocode}
% \begin{markdown}
%
% \pkg{csvsimple}
%
%:    A package that provides the `\csvautotabular` macro for typesetting
%     \acro{csv} files in the default renderer prototypes for iA\\,Writer
%     content blocks.
%
% \end{markdown}
%  \begin{macrocode}
soft csvsimple
soft pgf  # required by `csvsimple`, which loads `pgfkeys.sty`
soft tools  # required by `csvsimple`, which loads `shellesc.sty`
%    \end{macrocode}
% \begin{markdown}
%
% \pkg{gobble}
%
%:    A package that provides the `\@gobblethree` \TeX{} command that
%     is used in the default renderer prototype for citations. The package
%     is included in \TeX Live${}\geq{}2016$.
%
% \end{markdown}
%  \begin{macrocode}
soft gobble
%    \end{macrocode}
% \begin{markdown}
%
% \pkg{amsmath} and \pkg{amssymb}
%
%:    Packages that provide symbols used for drawing ticked and unticked
%     boxes.
%
% \end{markdown}
%  \begin{macrocode}
soft amsmath
soft amsfonts
%    \end{macrocode}
% \begin{markdown}
%
% \pkg{catchfile}
%
%:    A package that catches the contents of a file and puts it in a macro. It
%     is used in the `witiko/graphicx/http` \LaTeX{} theme, see Section
%     <#sec:latexthemes>.
%
% \end{markdown}
%  \begin{macrocode}
soft catchfile
%    \end{macrocode}
% \begin{markdown}
%
% \pkg{grffile}
%
%:    A package that extends the name processing of the \pkg{graphics} package
%     to support a larger range of file names in $2006\leq{}$\TeX{}
%     Live${}\leq{}2019$.  Since \TeX{} Live${}\geq{}2020$, the functionality
%     of the package has been integrated in the \LaTeXe{} kernel. It is used in
%     the `witiko/dot` and `witiko/graphicx/http` \LaTeX{} themes, see Section
%     <#sec:latexthemes>.
%
% \end{markdown}
%  \begin{macrocode}
soft grffile
%    \end{macrocode}
% \begin{markdown}
%
% \pkg{etoolbox}
%
%:    A package that is used to polyfill the general hook management system in
%     the default renderer prototypes for \acro{yaml} metadata, see Section
%     <#sec:latex-yaml-metadata>, and also in the default renderer prototype
%     for identifier attributes.
%
% \end{markdown}
%  \begin{macrocode}
soft etoolbox
%    \end{macrocode}
% \begin{markdown}
%
% \pkg{soulutf8}
%
%:    A package that is used in the default renderer prototype for
%     strike-throughs and marked text.
%     <!-- TODO: 1,$s/soulutf8/soul/g in TeX Live 2023. -->
%
% \end{markdown}
%  \begin{macrocode}
soft soul
%    \end{macrocode}
% \begin{markdown}
%
% \pkg{ltxcmds}
%
%:    A package that is used to detect whether the \pkg{minted} and
%     \pkg{listings} packages are loaded in the default renderer prototype
%     for fenced code blocks.
%
% \end{markdown}
%  \begin{macrocode}
soft ltxcmds
%    \end{macrocode}
% \begin{markdown}
%
% \pkg{verse}
%
%:    A package that is used in the default renderer prototypes for
%     line blocks.
%
% \end{markdown}
%  \begin{macrocode}
soft verse
%    \end{macrocode}
% \begin{markdown}
%
% \end{markdown}
% \iffalse
%</depends>
%<*context>
% \fi
% \par
% \begin{markdown}
%
%### \Hologo{ConTeXt} Prerequisites
% The \Hologo{ConTeXt} part of the package requires that either the Mark II or
% the Mark IV format is loaded, all the plain \TeX{} prerequisites (see
% Section <#sec:tex-prerequisites>), and the following \Hologo{ConTeXt}
% modules:
%
% \pkg{m-database}
%
%:    A module that provides the default token renderer prototype for
%     iA\\,Writer content blocks with the \acro{csv} filename extension (see
%     Section <#sec:texrendererprototypes>).
%
% Feedback
%---------
% Please use the \pkg{Markdown} project page on
% [GitHub](https://github.com/witiko/markdown/issues) to report bugs and submit
% feature requests. If you do not want to report a bug or request a feature but
% are simply in need of assistance, you might want to consider posting your
% question to the [\TeX-\LaTeX{} Stack Exchange.](https://tex.stackexchange.com)
% community question answering web site under the `markdown` tag.
%
% Acknowledgements
%-----------------
% The Lunamark Lua module provides speedy markdown parsing for the package. I
% would like to thank John Macfarlane, the creator of Lunamark, for releasing
% Lunamark under a permissive license, which enabled its use in the Markdown
% package.
%
% Extensive user documentation for the Markdown package was kindly written by
% Lian Tze Lim and published by Overleaf.
%
% Funding by the Faculty of Informatics at the Masaryk~University
% in~Brno~[@novotny15] is gratefully acknowledged.
%
% Support for content slicing (Lua options \Opt{shiftHeadings} and \Opt{slice})
% and pipe tables (Lua options \Opt{pipeTables} and \Opt{tableCaptions}) was
% graciously sponsored by David Vins and Omedym.
%
% The \TeX{} implementation of the package draws inspiration from several
% sources including the source code of \Hologo{LaTeX2e}, the \pkg{minted}
% package by Geoffrey M. Poore, which likewise tackles the issue of
% interfacing with an external interpreter from \TeX{}, the \pkg{filecontents}
% package by Scott Pakin and others.
%
% Interfaces {#interfaces}
%============
%
% This part of the documentation describes the interfaces exposed by the package
% along with usage notes and examples. It is aimed at the user of the package.
%
% Since neither \TeX{} nor Lua provide interfaces as a language construct, the
% separation to interfaces and implementations is a *gentlemen's agreement*. It
% serves as a means of structuring this documentation and as a promise to the
% user that if they only access the package through the interface, the future
% minor versions of the package should remain backwards compatible.
%
% Figure <#fig:block-diagram> shows the high-level structure of the Markdown
% package: The translation from markdown to \TeX{} *token renderers* is exposed
% by the Lua layer. The plain \TeX{} layer exposes the conversion capabilities
% of Lua as \TeX{} macros. The \LaTeX{} and \Hologo{ConTeXt} layers provide
% syntactic sugar on top of plain \TeX{} macros. The user can interface with
% any and all layers.
%
% \end{markdown}
% \begin{figure}
% \centering
% \input markdown-figure-block-diagram
% \caption{A block diagram of the \pkg{Markdown} package}
% \label{fig:block-diagram}
% \end{figure}
% \iffalse
%</context>
%<*manual>

First Document
--------------

In this section, we will take the necessary steps to typeset our first markdown
document in \TeX{}. This will serve as our first hands-on experience with the
package and also as a reassurance that the package has been correctly installed.

If you are using [our official Docker image][docker], you need to prefix all
commands in this section with `docker run --rm -v "$PWD"/workdir:/workdir -w
/workdir witiko/markdown`. For example, instead of `luatex document.tex`, you
would execute the following command:
``` sh
docker run --rm -v "$PWD"/workdir:/workdir -w /workdir witiko/markdown \
  luatex document.tex
``````

### Using Lua

Using a text editor, create a text document named `document.tex` with the
following content:
``` tex
\input markdown
\begingroup
\catcode`\%=12
\catcode`\#=12
\input hello
\endgroup
\bye
```````

#### Using the Lua Module

Using a text editor, create a text document named `hello.lua` with the
following content:
``` lua
#!/usr/bin/env texlua
local kpse = require("kpse")
kpse.set_program_name("luatex")
local markdown = require("markdown")
local convert = markdown.new()
print(convert("Hello *world*!"))
```````
Next, invoke LuaTeX from the terminal:
``` sh
texlua hello.lua > hello.tex
luatex document.tex
``````
A PDF document named `document.pdf` should be produced and contain the text
“Hello *world*!” Invoking pdfTeX should have the same effect:
``` sh
texlua hello.lua > hello.tex
pdftex document.tex
``````

#### Using the Lua Command-Line Interface

Using a text editor, create a text document named `hello.md` with the
following content:
``` md
Hello *world*!
``````
Next, invoke LuaTeX from the terminal:
``` sh
texlua ⟨CLI pathname⟩ -- hello.md hello.tex
luatex document.tex
``````
where \meta{CLI pathname} corresponds to the location of the Lua CLI script file,
such as `~/texmf/scripts/markdown/markdown-cli.lua` on UN\*X systems or
`C:\Users\`\meta{Your username}`\texmf\scripts\markdown\markdown-cli.lua` on Windows
systems. Use the command `kpsewhich -a markdown-cli.lua` to locate the Lua CLI
script file using [Kpathsea][].

 [Kpathsea]: https://tug.org/kpathsea/ (Kpathsea - TeX Users Group)

A PDF document named `document.pdf` should be produced and contain the text “Hello
*world*!” Invoking pdfTeX should have the same effect:
``` sh
texlua ⟨CLI pathname⟩ -- hello.md hello.tex
pdftex document.tex
``````

### Using Plain \TeX{}

Using a text editor, create a text document named `document.tex` with the
following content:
``` tex
\input markdown
\markdownBegin
Hello *world*!
\markdownEnd
\bye
```````
Next, invoke LuaTeX from the terminal:
``` sh
luatex document.tex
``````
A PDF document named `document.pdf` should be produced and contain the text
“Hello *world*!”

Instead of LuaTeX, you may also use pdfTeX:
``` sh
pdftex --shell-escape document.tex
```````
This should also produce a PDF document named `document.pdf` with the same content.

***

Instead of writing your markdown document between `\markdownBegin` and
`\markdownEnd`, you can also include markdown documents using the
`\markdownInput` macro, similarly to how you might use the `\input` TeX
primitive to include \TeX{} documents.

Using a text editor, create a text document named `hello.md` with the
following content:
``` md
Hello *world*!
``````
Create also a text document named `document.tex` with the following content:
``` tex
\input markdown
\markdownInput{hello.md}
\bye
```````
Next, invoke LuaTeX or pdfTeX from the terminal like in the previous example.
A PDF document named `document.pdf` should be produced and contain the text
“Hello *world*!”

### Using \LaTeX{}

Using a text editor, create a text document named `document.tex` with the
following content:
``` tex
\documentclass{article}
\usepackage{markdown}
\begin{document}
\begin{markdown}
Hello *world*!
\end{markdown}
\end{document}
```````
Next, invoke LuaTeX from the terminal:
``` sh
lualatex document.tex
``````
A PDF document named `document.pdf` should be produced and contain the text “Hello
*world*!”

Instead of LuaTeX, you may also use pdfTeX:
``` sh
pdflatex --shell-escape document.tex
``````
This should also produce a PDF document named `document.pdf` with the same content.

***

Instead of writing your markdown document between `\begin{markdown}` and
`\end{markdown}`, you can also include markdown documents using the
`\markdownInput` macro, similarly to how you might use the `\input` TeX
primitive to include \LaTeX{} documents.

Using a text editor, create a text document named `hello.md` with the
following content:
``` md
Hello *world*!
``````
Create also a text document named `document.tex` with the following content:
``` tex
\documentclass{article}
\usepackage{markdown}
\begin{document}
\begin{markdown}
\markdownInput{hello.md}
\end{markdown}
\end{document}
```````
Next, invoke LuaTeX or pdfTeX from the terminal like in the previous example.
A PDF document named `document.pdf` should be produced and contain the text
“Hello *world*!”

***

As the next step, try typesetting the example documents distributed along with
the Markdown package:
``` sh
git clone https://github.com/witiko/markdown
cd markdown/examples
lualatex latex-luatex.tex
``````
A PDF document named `latex-luatex.pdf` should be produced. Open the text
documents `latex-luatex.tex` and `example.md` in a text editor to see how the
example documents are structured. Try changing the documents and typesetting
them as follows:
``` sh
lualatex latex-luatex.tex
``````
to see the effect of your changes.

### Using \Hologo{ConTeXt}

Using a text editor, create a text document named `document.tex` with the
following content:
``` tex
\usemodule[t][markdown]
\starttext
\startmarkdown
Hello *world*!
\stopmarkdown
\stoptext
```````
Next, invoke LuaTeX from the terminal:
``` sh
context --luatex document.tex
``````
A PDF document named `document.pdf` should be produced and contain the text “Hello
*world*!”

***

Instead of writing your markdown document between `\startmarkdown` and
`\stopmarkdown`, you can also include markdown documents using the
`\inputmarkdown` macro, similarly to how you might use the `\input` TeX
primitive to include \Hologo{ConTeXt} documents.

Using a text editor, create a text document named `hello.md` with the
following content:
``` md
Hello *world*!
``````
Create also a text document named `document.tex` with the following content:
``` tex
\usemodule[t][markdown]
\starttext
\inputmarkdown{hello.md}
\stoptext
```````
Next, invoke LuaTeX from the terminal like in the previous example.
A PDF document named `document.pdf` should be produced and contain the text
“Hello *world*!”

***

As the next step, try typesetting the example documents distributed along with
the Markdown package:
``` sh
git clone https://github.com/witiko/markdown
cd markdown/examples
context --luatex context.tex
``````
A PDF document named `context.pdf` should be produced. Open the text documents
`context.tex` and `example.md` in a text editor to see how the example documents
are structured. Try changing the documents and typesetting them as follows:
``` sh
context --luatex context.tex
``````
to see the effect of your changes.

Examples
========

In this section, I will describe the individual parts of the Markdown package.
Each part will be shown by example, leaving the implementation details to the
[technical documentation][techdoc].

/markdown-interfaces.md
/markdown-options.md
/markdown-tokens.md

%</manual>
%<*lua>
% \fi
% \par
% \begin{markdown}
%
% Lua Interface {#luainterface}
%---------------
%
% \end{markdown}
% \iffalse
%</lua>
%<*manual-interfaces>

Interfaces
----------

In this section, I will describe the individual interfaces exposed by the
Markdown package starting with the low-level Lua interfaces and all the way up
to the \LaTeX{} and \Hologo{ConTeXt} interfaces intended for the ordinary
user.

### Lua

The Lua programming language is what drives the conversion from markdown to
\TeX{} in the Markdown package. Based on the [Lunamark][] Lua library by John
MacFarlane, the Lua implementation is largely independent on \TeX{}, and can be
used separately from typesetting a document. Lua provides two interfaces: a Lua
module and a command-line interface (CLI).

 [Lunamark]: https://github.com/jgm/lunamark
             (Lua library for conversion between markup formats)

%</manual-interfaces>
%<*lua>
% \fi
% \begin{markdown}
%
% The Lua interface provides the conversion from \acro{utf}-\oldstylenums8
% encoded markdown to plain \TeX{}. This interface is used by the plain \TeX{}
% implementation (see Section <#sec:teximplementation>) and will be of
% interest to the developers of other packages and Lua modules.
%
% The Lua interface is implemented by the `markdown` Lua module.
%
% \end{markdown}
%  \begin{macrocode}
local M = {metadata = metadata}
%    \end{macrocode}
% \par
% \begin{markdown}
%
%### Conversion from Markdown to Plain \TeX{} {#lua-conversion}
%
% The Lua interface exposes the \luamdef{new}`(options)` function.  This
% function returns a conversion function from markdown to plain \TeX{} according
% to the table `options` that contains options recognized by the Lua interface
% (see Section <#sec:lua-options>). The `options` parameter is optional; when
% unspecified, the behaviour will be the same as if `options` were an empty
% table.
%
% The following example Lua code converts the markdown string `Hello
% *world*!` to a \TeX{} output using the default options and prints the \TeX{}
% output:
% ``` lua
% local md = require("markdown")
% local convert = md.new()
% print(convert("Hello *world*!"))
% ```````
%
%### User-Defined Syntax Extensions {#lua-user-extensions}
%
% For the purpose of user-defined syntax extensions, the Lua interface also
% exposes the \luamdef{reader} object, which performs the lexical and
% syntactic analysis of markdown text and which exposes the
% \luamdef{reader->insert_pattern} and \luamdef{reader->add_special_character}
% methods for extending the \acro{peg} grammar of markdown.
%
% The read-only \luamdef{walkable_syntax} hash table stores those rules of the
% \acro{peg} grammar of markdown that can be represented as an ordered choice
% of terminal symbols. These rules can be modified by user-defined syntax
% extensions.
%
% \end{markdown}
%  \begin{macrocode}
local walkable_syntax = {
  Block = {
    "Blockquote",
    "Verbatim",
    "ThematicBreak",
    "BulletList",
    "OrderedList",
    "DisplayHtml",
    "Heading",
  },
  BlockOrParagraph = {
    "Block",
    "Paragraph",
    "Plain",
  },
  Inline = {
    "Str",
    "Space",
    "Endline",
    "EndlineBreak",
    "LinkAndEmph",
    "Code",
    "AutoLinkUrl",
    "AutoLinkEmail",
    "AutoLinkRelativeReference",
    "InlineHtml",
    "HtmlEntity",
    "EscapedChar",
    "Smart",
    "Symbol",
  },
}
%    \end{macrocode}
% \par
% \begin{markdown}
%
% The \luamref{reader->insert_pattern} method inserts a \acro{peg} pattern into
% the grammar of markdown. The method receives two mandatory arguments: a
% selector string in the form `"`\meta{left-hand side terminal symbol}
% \meta{`before`, `after`, or `instead of`} \meta{right-hand side terminal
% symbol}`"` and a \acro{peg} pattern to insert, and an optional third argument
% with a name of the \acro{peg} pattern for debugging purposes (see the
% \Opt{debugExtensions} option). The name does not need to be unique and shall
% not be interpreted by the Markdown package; you can treat it as a comment.
%
% For example. if we'd like to insert `pattern` into the grammar between the
% `Inline -> LinkAndEmph` and `Inline -> Code` rules, we would call
% \luamref{reader->insert_pattern} with `"Inline after LinkAndEmph"` (or `"Inline
% before Code"`) and `pattern` as the arguments.
%
% The \luamref{reader->add_special_character} method adds a new character with
% special meaning to the grammar of markdown. The method receives the character
% as its only argument.
%
% \end{markdown}
% \iffalse
%</lua>
%<*manual-interfaces>

#### Lua Module

A Lua module is a software library that can be used from in other programs.
The `markdown` Lua module makes it possible to convert markdown to \TeX{} from
within Lua\TeX{} documents and Lua scripts.

The `markdown` Lua module exposes the `new(`\meta{options}`)` method, which
creates a converter function from markdown to \TeX{}. The properties of the
converter function are specified by the Lua table `options`. The parameter is
optional; when unspecified, the behaviour will be the same as if \meta{options}
were an empty table.

Using a text editor, create a text document named `document.tex` with the
following content:
``` tex
\input markdown
\input lmfonts
\begingroup
\catcode`\%=12
\catcode`\#=12
\input example
\endgroup
\bye
```````
Using a text editor, create a text document named `example.lua` with the
following content:
``` lua
#!/usr/bin/env texlua
local kpse = require("kpse")
kpse.set_program_name("luatex")
local markdown = require("markdown")
local input, convert_nomath, convert_math, paragraph

input = [[$\sqrt{-1}$ *equals* $i$.]]
convert_nomath = markdown.new()
convert_math = markdown.new({texMathDollars = true})
paragraph = [[\par]]

print(
  convert_nomath(input) .. paragraph ..
  convert_math(input)
)
```````
Next, invoke LuaTeX from the terminal:
``` sh
texlua example.lua > example.tex
luatex document.tex
``````
A PDF document named `document.pdf` should be produced and contain the
following text:

> \$\\sqrt{-1}\$ *equals* \$i\$.
>
> <math><msqrt><mo>−</mo><mn>1</mn></msqrt></math>
> *equals*
> <math><mi>i</mi></math>.

Invoking pdfTeX should have the same effect:
``` sh
texlua example.lua > example.tex
pdftex document.tex
``````

***

Rather than use the `texlua` interpreter, we can also access the `markdown` Lua
module directly from our document. Using a text editor, create a text document
named `document.tex` with the following content:
``` tex
\input markdown
\input lmfonts
\directlua{
  local markdown = require("markdown")
  local input, convert_nomath, convert_math, paragraph

  input = [[$\string\sqrt{-1}$ *equals* $i$.]]
  convert_nomath = markdown.new()
  convert_math = markdown.new({texMathDollars = true})
  paragraph = [[\par]]

  tex.sprint(
    convert_nomath(input) .. paragraph ..
    convert_math(input)
  )
}
\bye
```````
Next, invoke LuaTeX from the terminal:
``` sh
luatex document.tex
```````
A PDF document named `document.pdf` should be produced and contain the
following text:

> \$\\sqrt{-1}\$ *equals* \$i\$.
>
> <math><msqrt><mo>−</mo><mn>1</mn></msqrt></math>
> *equals*
> <math><mi>i</mi></math>.

In this case, we cannot use pdfTeX, because pdfTeX does not define the
`\directlua` \TeX{} command.

#### Lua Command-Line Interface

The Lua command-line interface (CLI) of the Markdown package makes the
functionality of the Lua module accessible from the command line.  This makes
it possible to convert documents from markdown to \TeX{} manually without any
knowledge of the Lua programming language.

The Lua command-line interface accepts the same options as the `markdown` Lua
module, but now the options are specified as command-line parameters.

Using a text editor, create a text document named `document.tex` with the
following content:
``` tex
\input markdown
\input lmfonts
\begingroup
\catcode`\%=12
\catcode`\#=12
\input nomath
\par
\input math
\endgroup
\bye
```````
Using a text editor, create a text document named `example.md` with the
following content:
``` md
$\sqrt{-1}$ *equals* $i$.
``````
Next, invoke LuaTeX from the terminal:
``` sh
texlua ⟨CLI pathname⟩ -- example.md nomath.tex
texlua ⟨CLI pathname⟩ tex_math_dollars=true -- example.md math.tex
luatex document.tex
``````
where \meta{CLI pathname} corresponds to the location of the Lua CLI script file,
such as `~/texmf/scripts/markdown/markdown-cli.lua` on UN\*X systems or
`C:\Users\`\meta{Your username}`\texmf\scripts\markdown\markdown-cli.lua` on Windows
systems. Use the command `kpsewhich -a markdown-cli.lua` to locate the Lua CLI
script file using [Kpathsea][].

A PDF document named `document.pdf` should be produced and contain the
following text:

> \$\\sqrt{-1}\$ *equals* \$i\$.
>
> <math><msqrt><mo>−</mo><mn>1</mn></msqrt></math>
> *equals*
> <math><mi>i</mi></math>.

Invoking pdfTeX should have the same effect:
``` sh
texlua ⟨CLI pathname⟩ -- example.md nomath.tex
texlua ⟨CLI pathname⟩ tex_math_dollars=true -- example.md math.tex
pdftex document.tex
``````

%</manual-interfaces>
%<*lua>
% \fi
% \begin{markdown}
%
%### Options {#lua-options}
%
% The Lua interface recognizes the following options. When unspecified, the
% value of a key is taken from the \luamdef{defaultOptions} table.
%
% \end{markdown}
% \iffalse
%</lua>
%<*manual-options>

## Options

In this section, I will describe all the options recognized by the Markdown
package.

%</manual-options>
%<*lua,lua-cli>
% \fi
%  \begin{macrocode}
local defaultOptions = {}
%    \end{macrocode}
% \begingroup
% \markdownSetup{snippet=lua-options}
% \par
% \iffalse
%</lua,lua-cli>
%<*manual-options>

### Lua

Lua options control the conversion from markdown to \TeX{}. They are supported
by all interfaces of the Markdown package starting with the low-level Lua
interfaces and all the way up to the \LaTeX{} and \Hologo{ConTeXt} interfaces.

%</manual-options>
%<*tex>
% \fi
% \begin{markdown}
%
% To enable the enumeration of Lua options, we will maintain the
% \mdef{g_\@\@_lua_options_seq} sequence.
%
% \end{markdown}
%  \begin{macrocode}
\ExplSyntaxOn
\seq_new:N \g_@@_lua_options_seq
%    \end{macrocode}
% \begin{markdown}
%
% To enable the reflection of default Lua options and their types, we will
% maintain the \mdef{g_\@\@_default_lua_options_prop} and
% \mdef{g_\@\@_lua_option_types_prop} property lists, respectively.
%
% \end{markdown}
%  \begin{macrocode}
\prop_new:N \g_@@_lua_option_types_prop
\prop_new:N \g_@@_default_lua_options_prop
\seq_new:N \g_@@_option_layers_seq
\tl_const:Nn \c_@@_option_layer_lua_tl { lua }
\seq_gput_right:NV \g_@@_option_layers_seq \c_@@_option_layer_lua_tl
\cs_new:Nn
  \@@_add_lua_option:nnn
  {
    \@@_add_option:Vnnn
      \c_@@_option_layer_lua_tl
      { #1 }
      { #2 }
      { #3 }
  }
\cs_new:Nn
  \@@_add_option:nnnn
  {
    \seq_gput_right:cn
      { g_@@_ #1 _options_seq }
      { #2 }
    \prop_gput:cnn
      { g_@@_ #1 _option_types_prop }
      { #2 }
      { #3 }
    \prop_gput:cnn
      { g_@@_default_ #1 _options_prop }
      { #2 }
      { #4 }
    \@@_typecheck_option:n
      { #2 }
  }
\cs_generate_variant:Nn
  \@@_add_option:nnnn
  { Vnnn }
\tl_const:Nn \c_@@_option_value_true_tl  { true  }
\tl_const:Nn \c_@@_option_value_false_tl { false }
\cs_new:Nn \@@_typecheck_option:n
  {
    \@@_get_option_type:nN
      { #1 }
      \l_tmpa_tl
    \str_case_e:Vn
      \l_tmpa_tl
      {
        { \c_@@_option_type_boolean_tl }
          {
            \@@_get_option_value:nN
              { #1 }
              \l_tmpa_tl
            \bool_if:nF
              {
                \str_if_eq_p:VV
                  \l_tmpa_tl
                  \c_@@_option_value_true_tl ||
                \str_if_eq_p:VV
                  \l_tmpa_tl
                  \c_@@_option_value_false_tl
              }
              {
                \msg_error:nnnV
                  { markdown }
                  { failed-typecheck-for-boolean-option }
                  { #1 }
                  \l_tmpa_tl
              }
          }
      }
  }
\msg_new:nnn
  { markdown }
  { failed-typecheck-for-boolean-option }
  {
    Option~#1~has~value~#2,~
    but~a~boolean~(true~or~false)~was~expected.
  }
\cs_generate_variant:Nn
  \str_case_e:nn
  { Vn }
\cs_generate_variant:Nn
  \msg_error:nnnn
  { nnnV }
\seq_new:N \g_@@_option_types_seq
\tl_const:Nn \c_@@_option_type_clist_tl { clist }
\seq_gput_right:NV \g_@@_option_types_seq \c_@@_option_type_clist_tl
\tl_const:Nn \c_@@_option_type_counter_tl { counter }
\seq_gput_right:NV \g_@@_option_types_seq \c_@@_option_type_counter_tl
\tl_const:Nn \c_@@_option_type_boolean_tl { boolean }
\seq_gput_right:NV \g_@@_option_types_seq \c_@@_option_type_boolean_tl
\tl_const:Nn \c_@@_option_type_number_tl  { number  }
\seq_gput_right:NV \g_@@_option_types_seq \c_@@_option_type_number_tl
\tl_const:Nn \c_@@_option_type_path_tl    { path    }
\seq_gput_right:NV \g_@@_option_types_seq \c_@@_option_type_path_tl
\tl_const:Nn \c_@@_option_type_slice_tl   { slice   }
\seq_gput_right:NV \g_@@_option_types_seq \c_@@_option_type_slice_tl
\tl_const:Nn \c_@@_option_type_string_tl  { string  }
\seq_gput_right:NV \g_@@_option_types_seq \c_@@_option_type_string_tl
\cs_new:Nn
  \@@_get_option_type:nN
  {
    \bool_set_false:N
      \l_tmpa_bool
    \seq_map_inline:Nn
      \g_@@_option_layers_seq
      {
        \prop_get:cnNT
          { g_@@_ ##1 _option_types_prop }
          { #1 }
          \l_tmpa_tl
          {
            \bool_set_true:N
              \l_tmpa_bool
            \seq_map_break:
          }
      }
    \bool_if:nF
      \l_tmpa_bool
      {
        \msg_error:nnn
          { markdown }
          { undefined-option }
          { #1 }
      }
    \seq_if_in:NVF
      \g_@@_option_types_seq
      \l_tmpa_tl
      {
        \msg_error:nnnV
          { markdown }
          { unknown-option-type }
          { #1 }
          \l_tmpa_tl
      }
    \tl_set_eq:NN
      #2
      \l_tmpa_tl
  }
\msg_new:nnn
  { markdown }
  { unknown-option-type }
  {
    Option~#1~has~unknown~type~#2.
  }
\msg_new:nnn
  { markdown }
  { undefined-option }
  {
    Option~#1~is~undefined.
  }
\cs_new:Nn
  \@@_get_default_option_value:nN
  {
    \bool_set_false:N
      \l_tmpa_bool
    \seq_map_inline:Nn
      \g_@@_option_layers_seq
      {
        \prop_get:cnNT
          { g_@@_default_ ##1 _options_prop }
          { #1 }
          #2
          {
            \bool_set_true:N
              \l_tmpa_bool
            \seq_map_break:
          }
      }
    \bool_if:nF
      \l_tmpa_bool
      {
        \msg_error:nnn
          { markdown }
          { undefined-option }
          { #1 }
      }
  }
\cs_new:Nn
  \@@_get_option_value:nN
  {
    \@@_option_tl_to_csname:nN
      { #1 }
      \l_tmpa_tl
    \cs_if_free:cTF
      { \l_tmpa_tl }
      {
        \@@_get_default_option_value:nN
          { #1 }
          #2
      }
      {
        \@@_get_option_type:nN
          { #1 }
          \l_tmpa_tl
        \str_if_eq:NNTF
          \c_@@_option_type_counter_tl
          \l_tmpa_tl
          {
            \@@_option_tl_to_csname:nN
              { #1 }
              \l_tmpa_tl
            \tl_set:Nx
              #2
              { \the \cs:w \l_tmpa_tl \cs_end: }
          }
          {
            \@@_option_tl_to_csname:nN
              { #1 }
              \l_tmpa_tl
            \tl_set:Nv
              #2
              { \l_tmpa_tl }
          }
      }
  }
\cs_new:Nn \@@_option_tl_to_csname:nN
  {
    \tl_set:Nn
      \l_tmpa_tl
      { \str_uppercase:n { #1 } }
    \tl_set:Nx
      #2
      {
        markdownOption
        \tl_head:f { \l_tmpa_tl }
        \tl_tail:n { #1 }
      }
  }
%    \end{macrocode}
% \par
% \begin{markdown}
%
% To make it easier to support different coding styles in the interface,
% engines, we define the \mdef{\@\@_with_various_cases:nn} function
% that allows us to generate different variants of a string using
% different cases.
%
% \end{markdown}
%  \begin{macrocode}
\cs_new:Nn \@@_with_various_cases:nn
  {
    \seq_clear:N
      \l_tmpa_seq
    \seq_map_inline:Nn
      \g_@@_cases_seq
      {
        \tl_set:Nn
          \l_tmpa_tl
          { #1 }
        \use:c { ##1 }
          \l_tmpa_tl
        \seq_put_right:NV
          \l_tmpa_seq
          \l_tmpa_tl
      }
    \seq_map_inline:Nn
      \l_tmpa_seq
      { #2 }
  }
%    \end{macrocode}
% \par
% \begin{markdown}
%
% To interrupt the \mref{\@\@_with_various_cases:nn} function
% prematurely, use the \mdef{\@\@_with_various_cases_break:} function.
%
% \end{markdown}
%  \begin{macrocode}
\cs_new:Nn \@@_with_various_cases_break:
  {
    \seq_map_break:
  }
%    \end{macrocode}
% \begin{markdown}
%
% By default, camelCase and snake\\\_case are supported.
% Additional cases can be added by adding functions to the
% \mdef{g_\@\@_cases_seq} sequence.
%
% \end{markdown}
%  \begin{macrocode}
\seq_new:N \g_@@_cases_seq
\cs_new:Nn \@@_camel_case:N
  {
    \regex_replace_all:nnN
      { _ ([a-z]) }
      { \c { str_uppercase:n } \cB\{ \1 \cE\} }
      #1
    \tl_set:Nx
      #1
      { #1 }
  }
\seq_gput_right:Nn \g_@@_cases_seq { @@_camel_case:N }
\cs_new:Nn \@@_snake_case:N
  {
    \regex_replace_all:nnN
      { ([a-z])([A-Z]) }
      { \1 _ \c { str_lowercase:n } \cB\{ \2 \cE\} }
      #1
    \tl_set:Nx
      #1
      { #1 }
  }
\seq_gput_right:Nn \g_@@_cases_seq { @@_snake_case:N }
%    \end{macrocode}
% \iffalse
%</tex>
%<*manual-options>
% \fi
% \begin{markdown}
%
%### General Behavior
%
% \end{markdown}
% \par
% \iffalse

#### Option `eagerCache`

`eagerCache` (default value: `false`)

% \fi
% \begin{markdown}
%
% \Optitem[false]{eagerCache}{\opt{true}, \opt{false}}
%
:    true

     :  Converted markdown documents will be cached in \Opt{cacheDir}. This can be
        useful for post-processing the converted documents and for recovering
        historical versions of the documents from the cache.  However, it also
        produces a large number of auxiliary files on the disk and obscures the
        output of the Lua command-line interface when it is used for plumbing.

        This behavior will always be used if the \Opt{finalizeCache} option is
        enabled.

:    false

     :  Converted markdown documents will not be cached. This decreases the number
        of auxiliary files that we produce and makes it easier to use the Lua
        command-line interface for plumbing.

        This behavior will only be used when the \Opt{finalizeCache} option is
        disabled.

% \end{markdown}
% \iffalse

##### Lua CLI Example {.unnumbered}

Using a text editor, create a text document named `hello.md` with the
following content:
``` md
Hello *world*!
``````
Next, invoke LuaTeX from the terminal with the \Opt{eagerCache} option
disabled:
``` sh
texlua ⟨CLI pathname⟩ eagerCache=false -- hello.md hello.tex
```````
where \meta{CLI pathname} corresponds to the location of the Lua CLI script file,
such as `~/texmf/scripts/markdown/markdown-cli.lua` on UN\*X systems or
`C:\Users\`\meta{Your username}`\texmf\scripts\markdown\markdown-cli.lua` on Windows
systems. Use the command `kpsewhich -a markdown-cli.lua` to locate the Lua CLI
script file using [Kpathsea][].

A \TeX{} document named `hello.tex` should be produced and contain the
following code:
``` tex
Hello \markdownRendererEmphasis{world}!\relax
```

***

Invoke LuaTeX from the terminal again, this time with the \Opt{eagerCache}
option enabled:

``` tex
texlua ⟨CLI pathname⟩ eagerCache=true -- hello.md hello.tex
```

A \TeX{} document named `hello.tex` should be produced and contain the
following code:

``` tex
\input ./⟨hash⟩.md.tex\relax
```

Additionally, a \TeX{} document named `⟨hash⟩.md.tex` should be produced and
contain the following code:

``` tex
Hello \markdownRendererEmphasis{world}!\relax
```

%</manual-options>
%<*tex>
% \fi
%  \begin{macrocode}
\@@_add_lua_option:nnn
  { eagerCache }
  { boolean }
  { false }
%    \end{macrocode}
% \iffalse
%</tex>
%<*lua,lua-cli>
% \fi
%  \begin{macrocode}
defaultOptions.eagerCache = false
%    \end{macrocode}
% \par
% \iffalse
%</lua,lua-cli>
%<*manual-options>

#### Option `singletonCache`

`singletonCache` (default value: `true`)

% \fi
% \begin{markdown}
%
% \Optitem[true]{singletonCache}{\opt{true}, \opt{false}}
%
:    true

     :  Conversion functions produced by the function \luamref{new}`(options)`
        will be cached in an LRU cache of size 1 keyed by `options`. This is
        more time- and space-efficient than always producing a new conversion
        function but may expose bugs related to the idempotence of conversion
        functions.

        This has been the default behavior since version 3.0.0 of the Markdown
        package.

:    false

     :  Every call to the function \luamref{new}`(options)` will produce a new
        conversion function that will not be cached. This is slower than
        caching conversion functions and may expose bugs related to memory
        leaks in the creation of conversion functions, see also issue
        [#226][issue-226].

        This was the default behavior until version 3.0.0 of the Markdown package.

 [issue-226]: https://github.com/witiko/markdown/pull/226#issuecomment-1599641634

% \end{markdown}
% \iffalse

##### Lua Module Example {.unnumbered}

Using a text editor, create a text document named `document.tex` with the
following content:
``` tex
\input markdown
\begingroup
\catcode`\%=12
\catcode`\#=12
\directlua{
  local markdown = require("markdown")
  local convert1 = markdown.new({})
  local convert2 = markdown.new({singletonCache=false})
  local convert3 = markdown.new({singletonCache=true})
  local newline = [[^^J^^J]]
  tex.print(tostring(convert1) .. ", ")
  tex.print(tostring(convert2) .. ", ")
  tex.print(tostring(convert3))
}
\endgroup
\bye
```````
Then, invoke LuaTeX from the terminal:
``` sh
luatex document.tex
```````
A PDF document named `document.pdf` should be produced and contain the following text:

> function: 0x1a4a038, function: 0x1a52b18, function: 0x1a4a038

As you can see, the caching causes `convert1` and `convert3` to be the same
conversion function.

%</manual-options>
%<*tex>
% \fi
%  \begin{macrocode}
\@@_add_lua_option:nnn
  { singletonCache }
  { boolean }
  { true }
%    \end{macrocode}
% \iffalse
%</tex>
%<*lua,lua-cli>
% \fi
%  \begin{macrocode}
defaultOptions.singletonCache = true
%    \end{macrocode}
% \iffalse
%</lua,lua-cli>
%<*lua>
% \fi
%  \begin{macrocode}
local singletonCache = {
  convert = nil,
  options = nil,
}
%    \end{macrocode}
% \par
% \iffalse
%</lua>
%<*manual-options>

#### Option `unicodeNormalization`

`unicodeNormalization` (default value: `true`)

% \fi
% \begin{markdown}
%
% \Optitem[true]{unicodeNormalization}{\opt{true}, \opt{false}}
%
:    true

     :  Markdown documents will be normalized using one of the four [Unicode
        normalization forms][unicode-normalization] before conversion. The
        Unicode normalization norm used is determined by option
        \Opt{unicodeNormalizationForm}.

:    false

     :  Markdown documents will not be Unicode-normalized before conversion.

 [unicode-normalization]: https://unicode.org/faq/normalization.html

% \end{markdown}
% \iffalse
%</manual-options>
%<*tex>
% \fi
%  \begin{macrocode}
\@@_add_lua_option:nnn
  { unicodeNormalization }
  { boolean }
  { true }
%    \end{macrocode}
% \iffalse
%</tex>
%<*lua,lua-cli>
% \fi
%  \begin{macrocode}
defaultOptions.unicodeNormalization = true
%    \end{macrocode}
% \par
% \iffalse
%</lua,lua-cli>
%<*manual-options>

#### Option `unicodeNormalizationForm`

`unicodeNormalizationForm` (default value: `nfc`)

% \fi
% \begin{markdown}
%
% \Optitem[nfc]{unicodeNormalizationForm}{\opt{nfc}, \opt{nfd}, \opt{nfkc}, \opt{nfkd}}
%
:    nfc

     :  When option \Opt{unicodeNormalization} has been enabled, markdown documents
        will be normalized using Unicode Normalization Form C (NFC) before
        conversion.

:    nfd

     :  When option \Opt{unicodeNormalization} has been enabled, markdown documents
        will be normalized using Unicode Normalization Form D (NFD) before
        conversion.

:    nfkc

     :  When option \Opt{unicodeNormalization} has been enabled, markdown documents
        will be normalized using Unicode Normalization Form KC (NFKC) before
        conversion.

:    nfkd

     :  When option \Opt{unicodeNormalization} has been enabled, markdown documents
        will be normalized using Unicode Normalization Form KD (NFKD) before
        conversion.

% \end{markdown}
% \iffalse
%</manual-options>
%<*tex>
% \fi
%  \begin{macrocode}
\@@_add_lua_option:nnn
  { unicodeNormalizationForm }
  { string }
  { nfc }
%    \end{macrocode}
% \iffalse
%</tex>
%<*lua,lua-cli>
% \fi
%  \begin{macrocode}
defaultOptions.unicodeNormalizationForm = "nfc"
%    \end{macrocode}
% \par
% \iffalse
%</lua,lua-cli>
%<*manual-options>
% \fi
% \begin{markdown}
%
%### File and Directory Names
%
% \end{markdown}
% \par
% \iffalse

#### Option `cacheDir`

`cacheDir` (default value: `"."`)

% \fi
% \begin{markdown}
%
% \Valitem[.]{cacheDir}{path}
%
:    A path to the directory containing auxiliary cache files. If the last
     segment of the path does not exist, it will be created by the Lua
     command-line and plain \TeX{} implementations. The Lua implementation
     expects that the entire path already exists.

     When iteratively writing and typesetting a markdown document, the cache
     files are going to accumulate over time. You are advised to clean the
     cache directory every now and then, or to set it to a temporary filesystem
     (such as `/tmp` on UN*X systems), which gets periodically emptied.

% \end{markdown}
% \iffalse

##### Lua Module Example {.unnumbered}

Using a text editor, create a text document named `document.tex` with the
following content:
``` tex
\input markdown
\begingroup
\catcode`\%=12
\catcode`\#=12
\directlua{
  local markdown = require("markdown")
  local convert = markdown.new({cacheDir = "cache"})
  local input = "Hello *world*!"
  tex.sprint(convert(input)) }
\endgroup
\bye
```````
Create an empty directory named `cache` next to our text document. Then, invoke
LuaTeX from the terminal:
``` sh
luatex document.tex
```````
A PDF document named `document.pdf` should be produced and contain the text
“Hello *world*!” Several cache files of the Markdown package will also be
produced in the `cache` directory as we requested using the `cacheDir` option.

##### Lua CLI Example {.unnumbered}

Using a text editor, create a text document named `document.tex` with the
following content:
``` tex
\input markdown
\begingroup
\catcode`\%=12
\catcode`\#=12
\input hello
\endgroup
\bye
```````
Using a text editor, create a text document named `hello.md` with the
following content:
``` md
Hello *world*!
``````
Next, invoke LuaTeX from the terminal:
``` sh
texlua ⟨CLI pathname⟩ cacheDir=cache -- hello.md hello.tex
luatex document.tex
```````
where \meta{CLI pathname} corresponds to the location of the Lua CLI script file,
such as `~/texmf/scripts/markdown/markdown-cli.lua` on UN\*X systems or
`C:\Users\`\meta{Your username}`\texmf\scripts\markdown\markdown-cli.lua` on Windows
systems. Use the command `kpsewhich -a markdown-cli.lua` to locate the Lua CLI
script file using [Kpathsea][].

A PDF document named `document.pdf` should be produced and contain the text
“Hello *world*!” A directory named `cache` containing several cache files of
the Markdown package will also be produced as we requested using the `cacheDir`
option.

##### Plain \TeX{} Example {.unnumbered}

Using a text editor, create a text document named `document.tex` with the
following content:
``` tex
\input markdown
\def\markdownOptionCacheDir{cache}
\markdownBegin
Hello *world*!
\markdownEnd
\bye
```````
Next, invoke LuaTeX from the terminal:
``` sh
luatex document.tex
``````
A PDF document named `document.pdf` should be produced and contain the text “Hello
*world*!” A directory named `cache` containing several cache files of the
Markdown package will also be produced as we requested using the `cacheDir`
option.

##### \LaTeX{} Example {.unnumbered}

Using a text editor, create a text document named `document.tex` with the
following content:
``` tex
\documentclass{article}
\usepackage[cacheDir=cache]{markdown}
\begin{document}
\begin{markdown}
Hello *world*!
\end{markdown}
\end{document}
```````
Next, invoke LuaTeX from the terminal:
``` sh
lualatex document.tex
``````
A PDF document named `document.pdf` should be produced and contain the text “Hello
*world*!” A directory named `cache` containing several cache files of the
Markdown package will also be produced as we requested using the `cacheDir`
option.

##### \Hologo{ConTeXt} Example {.unnumbered}

Using a text editor, create a text document named `document.tex` with the
following content:
``` tex
\usemodule[t][markdown]
\setupmarkdown[cacheDir = cache]
\starttext
\startmarkdown
Hello *world*!
\stopmarkdown
\stoptext
```````
Next, invoke LuaTeX from the terminal:
``` sh
context --luatex document.tex
``````
A PDF document named `document.pdf` should be produced and contain the text “Hello
*world*!” A directory named `cache` containing several cache files of the
Markdown package will also be produced as we requested using the `cacheDir`
option.

%</manual-options>
%<*tex>
% \fi
%  \begin{macrocode}
\@@_add_lua_option:nnn
  { cacheDir }
  { path }
  { \markdownOptionOutputDir / _markdown_\jobname }
%    \end{macrocode}
% \iffalse
%</tex>
%<*lua,lua-cli>
% \fi
%  \begin{macrocode}
defaultOptions.cacheDir = "."
%    \end{macrocode}
% \par
% \iffalse
%</lua,lua-cli>
%<*manual-options>

#### Option `contentBlocksLanguageMap`

`contentBlocksLanguageMap` (default value: `"markdown-languages.json"`)

% \fi
% \begin{markdown}
%
% \Valitem[markdown-languages.json]{contentBlocksLanguageMap}{filename}
%
:    The filename of the \acro{JSON} file that maps filename extensions to
     programming language names in the iA\\,Writer content blocks when the
     \Opt{contentBlocks} option is enabled.
%    See Section <#sec:texcontentblockrenderers> for more information.

% \end{markdown}
% \iffalse

##### \LaTeX{} Example {.unnumbered}

Using a text editor, create a text document named `table.csv` with the
following content:
``` csv
Name,Surname,Born
Albert,Einstein,1879
Marie,Curie,1867
Thomas,Edison,1847
```````
Create also a text document named `language-map.json` with the following
content:
``` js
{
  "tex": "LaTeX"
}
``````
Create also a text document named `code.tex` with the following content:
``` tex
This is an example code listing in \LaTeX.
```````
Create also a text document named `part.md` with the following content:
``` md
This is a *transcluded markdown document*.
``````
Create also a text document named `document.tex` with the following content:
``` tex
\documentclass{article}
\usepackage{minted}
\usepackage[contentBlocks]{markdown}
\markdownSetup{
  contentBlocksLanguageMap = {language-map.json},
}
\begin{document}
\begin{markdown}
/table.csv  (An example table)
/code.tex   (An example code listing)
/part.md    (A file transclusion example)
\end{markdown}
\end{document}
```````
Next, invoke LuaTeX from the terminal:
``` sh
lualatex --shell-escape document.tex
``````
A PDF document named `document.pdf` should be produced and contain the
following text:

> | Name   | Surname  | Born |
> | ------ | ---------| ---- |
> | Albert | Einstein | 1879 |
> | Marie  | Curie    | 1867 |
> | Thomas | Edison   | 1847 |
>
> Table 1: An example table
>
> ``` tex
> This is an example code listing in \LaTeX.
> ```````
>
> This is a *transcluded markdown document*.

##### \Hologo{ConTeXt} Example {.unnumbered}

Using a text editor, create a text document named `table.csv` with the
following content:
``` csv
Name,Surname,Born
Albert,Einstein,1879
Marie,Curie,1867
Thomas,Edison,1847
```````
Create also a text document named `language-map.json` with the following
content:
``` js
{
  "tex": "ConTeXt"
}
``````
Create also a text document named `code.tex` with the following content:
``` tex
This is an example code listing in \ConTeXt.
```````
Create also a text document named `part.md` with the following content:
``` md
This is a *transcluded markdown document*.
``````
Create also a text document named `document.tex` with the following content:
``` tex
\usemodule[t][markdown]
\setupmarkdown
  [
    contentBlocks = yes,
    contentBlocksLanguageMap = language-map.json,
  ]
\definetyping [ConTeXt]
\setuptyping  [ConTeXt] [option=TEX]
\starttext
\startmarkdown
/table.csv  (An example table)
/code.tex   (An example code listing)
/part.md    (A file transclusion example)
\stopmarkdown
\stoptext
````````
Next, invoke LuaTeX from the terminal:
``` sh
context --luatex document.tex
`````
A PDF document named `document.pdf` should be produced and contain the
following text:

> | Name   | Surname  | Born |
> | ------ | ---------| ---- |
> | Albert | Einstein | 1879 |
> | Marie  | Curie    | 1867 |
> | Thomas | Edison   | 1847 |
>
> Table 1: An example table
>
> ``` tex
> This is an example code listing in \ConTeXt.
> ```````
>
> This is a *transcluded markdown document*.

%</manual-options>
%<*tex>
% \fi
%  \begin{macrocode}
\@@_add_lua_option:nnn
  { contentBlocksLanguageMap }
  { path }
  { markdown-languages.json }
%    \end{macrocode}
% \iffalse
%</tex>
%<*lua,lua-cli>
% \fi
%  \begin{macrocode}
defaultOptions.contentBlocksLanguageMap = "markdown-languages.json"
%    \end{macrocode}
% \par
% \iffalse
%</lua,lua-cli>
%<*manual-options>

#### Option `debugExtensionsFileName`

`debugExtensionsFileName` (default value: `"debug-extensions.json"`)

% \fi
% \begin{markdown}
%
% \Valitem[debug-extensions.json]{debugExtensionsFileName}{filename}
%
:    The filename of the \acro{JSON} file that will be produced when the
     \Opt{debugExtensions} option is enabled. This file will contain the
     extensible subset of the \acro{peg} grammar of markdown
%    (see the \luamref{walkable_syntax} hash table)
     after built-in syntax extensions
%    (see Section <#sec:lua-built-in-extensions>)
%    \iffalse
     (see options \Opt{citations}, \Opt{contentBlocks}, \Opt{definitionLists},
     etc.)
%    \fi
     and user-defined syntax extensions
%    (see Section <#sec:lua-user-extensions>)
%    \iffalse
     (see option \Opt{extensions})
%    \fi
     have been applied.

% \end{markdown}
% \iffalse
%</manual-options>
%<*tex>
% \fi
%  \begin{macrocode}
\@@_add_lua_option:nnn
  { debugExtensionsFileName }
  { path }
  { \markdownOptionOutputDir / \jobname .debug-extensions.json }
%    \end{macrocode}
% \iffalse
%</tex>
%<*lua,lua-cli>
% \fi
%  \begin{macrocode}
defaultOptions.debugExtensionsFileName = "debug-extensions.json"
%    \end{macrocode}
% \par
% \iffalse
%</lua,lua-cli>
%<*manual-options>

#### Option `frozenCacheFileName`

`frozenCacheFileName` (default value: `"frozenCache.tex"`)

% \fi
% \begin{markdown}
%
% \Valitem[frozenCache.tex]{frozenCacheFileName}{path}
%
:    A path to an output file (frozen cache) that will be created when
     the \Opt{finalizeCache} option is enabled and will contain a mapping
     between an enumeration of markdown documents and their auxiliary cache
     files.

     The frozen cache makes it possible to later typeset a plain \TeX{}
     document that contains markdown documents without invoking Lua using
     the \Opt{frozenCache} plain \TeX{} option. As a result, the
     plain \TeX{} document becomes more portable, but further changes in the
     order and the content of markdown documents will not be reflected.

% \end{markdown}
% \iffalse

##### Lua Module Example {.unnumbered}

Using a text editor, create a text document named `document.tex` with the
following content:
``` tex
\input markdown
\begingroup
\catcode`\%=12
\catcode`\#=12
\directlua{
  local markdown = require("markdown")
  local convert = markdown.new({finalizeCache = true, frozenCacheFileName = "cache.tex"})
  local input = "Hello *world*!"
  tex.sprint(convert(input)) }
\endgroup
\bye
```````
Then, invoke LuaTeX from the terminal:
``` sh
luatex document.tex
```````
A PDF document named `document.pdf` should be produced and contain the text
“Hello *world*!” A frozen cache will also be produced in the `cache.tex`
output file as we requested using the `finalizeCache` and `frozenCacheFileName`
options.

##### Lua CLI Example {.unnumbered}

Using a text editor, create a text document named `document.tex` with the
following content:
``` tex
\input markdown
\begingroup
\catcode`\%=12
\catcode`\#=12
\input hello
\endgroup
\bye
```````
Using a text editor, create a text document named `hello.md` with the
following content:
``` md
Hello *world*!
``````
Next, invoke LuaTeX from the terminal:
``` sh
texlua ⟨CLI pathname⟩ finalizeCache=true frozenCacheFileName=cache.tex -- hello.md hello.tex
luatex document.tex
```````
where \meta{CLI pathname} corresponds to the location of the Lua CLI script file,
such as `~/texmf/scripts/markdown/markdown-cli.lua` on UN\*X systems or
`C:\Users\`\meta{Your username}`\texmf\scripts\markdown\markdown-cli.lua` on Windows
systems. Use the command `kpsewhich -a markdown-cli.lua` to locate the Lua CLI
script file using [Kpathsea][].

A PDF document named `document.pdf` should be produced and contain the text
“Hello *world*!” A frozen cache will also be produced in the `cache.tex`
output file as we requested using the `finalizeCache` and `frozenCacheFileName`
options.

##### Plain \TeX{} Example {.unnumbered}

Using a text editor, create a text document named `document.tex` with the
following content:
``` tex
\input markdown
\def\markdownOptionFinalizeCache{true}
\def\markdownOptionFrozenCacheFileName{cache.tex}
\markdownBegin
Hello *world*!
\markdownEnd
\bye
```````
Next, invoke LuaTeX from the terminal:
``` sh
luatex document.tex
``````
A PDF document named `document.pdf` should be produced and contain the text
“Hello *world*!” A frozen cache will also be produced in the `cache.tex`
output file as we requested using the `finalizeCache` and `frozenCacheFileName`
options.

Next, create a new text document `frozen-document.tex` with the following
content:
``` tex
\input markdown
\def\markdownOptionFrozenCache{true}
\def\markdownOptionFrozenCacheFileName{cache.tex}
\markdownBegin
Hi *world*!
\markdownEnd
\bye
```````
Last, invoke pdfTeX without shell access from the terminal:
``` sh
pdftex -no-shell-escape frozen-document.tex
``````
A PDF document named `frozen-document.pdf` should be produced and contain the
text “Hello *world*!” Since we used the contents of the frozen cache using
the `\markdownOptionFrozenCache` option, we were able to typeset the
document without accessing the shell or invoking Lua, but the change in the
content of the markdown document from “Hello *world*!” to “Hi *world*!” was
not reflected.

##### \LaTeX{} Example {.unnumbered}

Using a text editor, create a text document named `document.tex` with the
following content:
``` tex
\documentclass[finalizecache]{article}
\usepackage[frozenCacheFileName=cache.tex]{markdown}
\begin{document}
\begin{markdown}
Hello *world*!
\end{markdown}
\end{document}
```````
Next, invoke LuaTeX from the terminal:
``` sh
lualatex document.tex
``````
A PDF document named `document.pdf` should be produced and contain the text
“Hello *world*!” A frozen cache will also be produced in the `cache.tex`
output file as we requested using the `finalizecache` and `frozenCacheFileName`
options.

Next, create a new text document `frozen-document.tex` with the following
content:
``` tex
\documentclass[frozencache]{article}
\usepackage[frozenCacheFileName=cache.tex]{markdown}
\begin{document}
\begin{markdown}
Hi *world*!
\end{markdown}
\end{document}
```````
Last, invoke pdfTeX without shell access from the terminal:
``` sh
pdflatex -no-shell-escape frozen-document.tex
``````
A PDF document named `frozen-document.pdf` should be produced and contain the
text “Hello *world*!” Since we used the contents of the frozen cache using
the `frozencache` option, we were able to typeset the document without
accessing the shell or invoking Lua, but the change in the content of the
markdown document from “Hello *world*!” to “Hi *world*!” was not reflected.

##### \Hologo{ConTeXt} Example {.unnumbered}

Using a text editor, create a text document named `document.tex` with the
following content:
``` tex
\usemodule[t][markdown]
\setupmarkdown
  [
    finalizeCache = yes,
    frozenCacheFileName = cache.tex,
  ]
\starttext
\startmarkdown
Hello *world*!
\stopmarkdown
\stoptext
```````
Next, invoke LuaTeX from the terminal:
``` sh
context --luatex document.tex
``````
A PDF document named `document.pdf` should be produced and contain the text
“Hello *world*!” A frozen cache will also be produced in the `cache.tex`
output file as we requested using the `finalizeCache` and `frozenCacheFileName`
options.

Next, create a new text document `frozen-document.tex` with the following
content:
``` tex
\usemodule[t][markdown]
\setupmarkdown
  [
    frozenCache = yes,
    frozenCacheFileName = cache.tex,
  ]
\starttext
\startmarkdown
Hi *world*!
\stopmarkdown
\stoptext
```````
Last, invoke LuaTeX from the terminal:
``` sh
context --luatex frozen-document.tex
``````
A PDF document named `frozen-document.pdf` should be produced and contain the
text “Hello *world*!” Since we used the contents of the frozen cache using
the `\markdownOptionFrozenCache` option, we were able to typeset the document
without accessing the shell or invoking Lua, but the change in the content of
the markdown document from “Hello *world*!” to “Hi *world*!” was not reflected.

%</manual-options>
%<*tex>
% \fi
%  \begin{macrocode}
\@@_add_lua_option:nnn
  { frozenCacheFileName }
  { path }
  { \markdownOptionCacheDir / frozenCache.tex }
%    \end{macrocode}
% \iffalse
%</tex>
%<*lua,lua-cli>
% \fi
%  \begin{macrocode}
defaultOptions.frozenCacheFileName = "frozenCache.tex"
%    \end{macrocode}
% \par
% \iffalse
%</lua,lua-cli>
%<*manual-options>
% \fi
% \begin{markdown}
%
%### Parser Options
%
% \end{markdown}
% \par
% \iffalse

#### Option `autoIdentifiers`

`autoIdentifiers` (default value: `false`)

% \fi
% \begin{markdown}
% \Optitem[false]{autoIdentifiers}{\opt{true}, \opt{false}}
%
:    true

     :  Enable the Pandoc [auto identifiers syntax extension][pandoc-auto-identifiers]:

        ``` md
        The following heading received the identifier `sesame-street`:

        # 123 Sesame Street
        ``````

     false

     :  Disable the Pandoc auto identifiers syntax extension.

See also the option \Opt{gfmAutoIdentifiers}.

  [pandoc-auto-identifiers]: https://pandoc.org/MANUAL.html#extension-auto_identifiers

% \end{markdown}
% \iffalse
%</manual-options>
%<*tex>
% \fi
%  \begin{macrocode}
\@@_add_lua_option:nnn
  { autoIdentifiers }
  { boolean }
  { false }
%    \end{macrocode}
% \iffalse
%</tex>
%<*lua,lua-cli>
% \fi
%  \begin{macrocode}
defaultOptions.autoIdentifiers = false
%    \end{macrocode}
% \par
% \iffalse
%</lua,lua-cli>
%<*manual-options>

#### Option `blankBeforeBlockquote`

`blankBeforeBlockquote` (default value: `false`)

% \fi
% \begin{markdown}
% \Optitem[false]{blankBeforeBlockquote}{\opt{true}, \opt{false}}
%
:    true

     :  Require a blank line between a paragraph and the following blockquote.

     false

     :  Do not require a blank line between a paragraph and the following
        blockquote.

% \end{markdown}
% \iffalse

##### Lua Module Example {.unnumbered}

Using a text editor, create a text document named `document.tex` with the
following content:
``` tex
\input markdown
\input lmfonts
\begingroup
\catcode`\%=12
\catcode`\#=12
\directlua{
  local markdown = require("markdown")
  local newline = [[^^J^^J]]
  local convert, input

  convert = markdown.new()
  input = "A paragraph." .. newline ..
          "> A quote."   .. newline
  tex.sprint(convert(input))

  convert = markdown.new({blankBeforeBlockquote = true})
  input = "A paragraph."   .. newline ..
          "> Not a quote." .. newline
  tex.sprint(convert(input)) }
\endgroup
\bye
```````
Then, invoke LuaTeX from the terminal:
``` sh
luatex document.tex
```````
A PDF document named `document.pdf` should be produced and contain the
following text:

> A paragraph.
>
> > A quote.
>
> A paragraph > Not a quote.

##### Lua CLI Example {.unnumbered}

Using a text editor, create a text document named `document.tex` with the
following content:
``` tex
\input markdown
\input lmfonts
\begingroup
\catcode`\%=12
\catcode`\#=12
\input optionfalse
\input optiontrue
\endgroup
\bye
```````
Using a text editor, create a text document named `content.md` with the
following content:
``` md
A paragraph.
> A quote?
``````
Next, invoke LuaTeX from the terminal:
``` sh
texlua ⟨CLI pathname⟩ -- content.md optionfalse.tex
texlua ⟨CLI pathname⟩ blankBeforeBlockquote=true -- content.md optiontrue.tex
luatex document.tex
```````
where \meta{CLI pathname} corresponds to the location of the Lua CLI script file,
such as `~/texmf/scripts/markdown/markdown-cli.lua` on UN\*X systems or
`C:\Users\`\meta{Your username}`\texmf\scripts\markdown\markdown-cli.lua` on Windows
systems. Use the command `kpsewhich -a markdown-cli.lua` to locate the Lua CLI
script file using [Kpathsea][].

A PDF document named `document.pdf` should be produced and contain the
following text:

> A paragraph.
>
> > A quote?
>
> A paragraph. > A quote?

##### Plain \TeX{} Example {.unnumbered}

Using a text editor, create a text document named `document.tex` with the
following content:
``` tex
\input markdown

\markdownBegin
A paragraph.
> A quote.
\markdownEnd

\def\markdownOptionBlankBeforeBlockquote{true}
\markdownBegin
A paragraph.
> Not a quote.
\markdownEnd

\bye
```````
Next, invoke LuaTeX from the terminal:
``` sh
luatex document.tex
``````
A PDF document named `document.pdf` should be produced and contain the
following text:

> A paragraph.
>
> > A quote.
>
> A paragraph > Not a quote.

##### \LaTeX{} Example {.unnumbered}

Using a text editor, create a text document named `document.tex` with the
following content:
``` tex
\documentclass{article}
\usepackage{markdown}
\begin{document}

\begin{markdown}
A paragraph.
> A quote.
\end{markdown}

\begin{markdown}[blankBeforeBlockquote]
A paragraph.
> Not a quote.
\end{markdown}

\end{document}
```````
Next, invoke LuaTeX from the terminal:
``` sh
lualatex document.tex
``````
A PDF document named `document.pdf` should be produced and contain the
following text:

> A paragraph.
>
> > A quote.
>
> A paragraph > Not a quote.

##### \Hologo{ConTeXt} Example {.unnumbered}

Using a text editor, create a text document named `document.tex` with the
following content:
``` tex
\usemodule[t][markdown]
\starttext

\startmarkdown
A paragraph.
> A quote.
\stopmarkdown

\setupmarkdown[blankBeforeBlockquote = yes]
\startmarkdown
A paragraph.
> Not a quote.
\stopmarkdown

\stoptext
```````
Next, invoke LuaTeX from the terminal:
``` sh
context --luatex document.tex
``````
A PDF document named `document.pdf` should be produced and contain the
following text:

> A paragraph.
>
> > A quote.
>
> A paragraph > Not a quote.

%</manual-options>
%<*tex>
% \fi
%  \begin{macrocode}
\@@_add_lua_option:nnn
  { blankBeforeBlockquote }
  { boolean }
  { false }
%    \end{macrocode}
% \iffalse
%</tex>
%<*lua,lua-cli>
% \fi
%  \begin{macrocode}
defaultOptions.blankBeforeBlockquote = false
%    \end{macrocode}
% \par
% \iffalse
%</lua,lua-cli>
%<*manual-options>

#### Option `blankBeforeCodeFence`

`blankBeforeCodeFence` (default value: `false`)

% \fi
% \begin{markdown}
%
% \Optitem[false]{blankBeforeCodeFence}{\opt{true}, \opt{false}}
%
:    true

     :  Require a blank line between a paragraph and the following fenced code
        block.

     false

     :  Do not require a blank line between a paragraph and the following
        fenced code block.

% \end{markdown}
% \iffalse

##### Lua Module Example {.unnumbered}

Using a text editor, create a text document named `document.tex` with the
following content:
``` tex
\input markdown
\input lmfonts
\begingroup
\catcode`\%=12
\catcode`\#=12
\directlua{
  local markdown = require("markdown")
  local newline = [[^^J^^J]]
  local convert, input

  convert = markdown.new({fencedCode = true})
  input = "A paragraph."   .. newline ..
          "```"            .. newline ..
          "A fenced code." .. newline ..
          "```"            .. newline
  tex.sprint(convert(input))

  convert = markdown.new({
    fencedCode = true, blankBeforeCodeFence = true})
  input = "A paragraph."       .. newline ..
          "```"                .. newline ..
          "Not a fenced code." .. newline ..
          "```"                .. newline
  tex.sprint(convert(input)) }
\endgroup
\bye
```````
Then, invoke LuaTeX from the terminal:
``` sh
luatex document.tex
```````
A PDF document named `document.pdf` should be produced and contain the
following text:

> A paragraph.
>
> ```
> A fenced code.
> ```
>
> A paragraph. ``` Not a fenced code. ```

##### Lua CLI Example {.unnumbered}

Using a text editor, create a text document named `document.tex` with the
following content:
``` tex
\input markdown
\input lmfonts
\begingroup
\catcode`\%=12
\catcode`\#=12
\input optionfalse
\input optiontrue
\endgroup
\bye
```````
Using a text editor, create a text document named `content.md` with the
following content:
```` md
A paragraph.
```
A code fence?
```
```````
Next, invoke LuaTeX from the terminal:
``` sh
texlua ⟨CLI pathname⟩ fencedCode=true -- content.md optionfalse.tex
texlua ⟨CLI pathname⟩ fencedCode=true blankBeforeCodeFence=true  -- content.md optiontrue.tex
luatex document.tex
```````
where \meta{CLI pathname} corresponds to the location of the Lua CLI script file,
such as `~/texmf/scripts/markdown/markdown-cli.lua` on UN\*X systems or
`C:\Users\`\meta{Your username}`\texmf\scripts\markdown\markdown-cli.lua` on Windows
systems. Use the command `kpsewhich -a markdown-cli.lua` to locate the Lua CLI
script file using [Kpathsea][].

A PDF document named `document.pdf` should be produced and contain the
following text:

> A paragraph.
>
> ```
> A code fence?
> ```
>
> A paragraph. ``` A code fence? ```

##### Plain \TeX{} Example {.unnumbered}

Using a text editor, create a text document named `document.tex` with the
following content:
```` tex
\input markdown
\def\markdownOptionFencedCode{true}

\markdownBegin
A paragraph.
```
A fenced code.
```
\markdownEnd

\def\markdownOptionBlankBeforeCodeFence{true}
\markdownBegin
A paragraph.
```
Not a fenced code.
```
\markdownEnd

\bye
````````
Next, invoke LuaTeX from the terminal:
``` sh
luatex document.tex
``````
A PDF document named `document.pdf` should be produced and contain the
following text:

> A paragraph.
>
> ```
> A fenced code.
> ```
>
> A paragraph. ``` Not a fenced code. ```

##### \LaTeX{} Example {.unnumbered}

Using a text editor, create a text document named `document.tex` with the
following content:
```` tex
\documentclass{article}
\usepackage[fencedCode]{markdown}
\begin{document}

\begin{markdown}
A paragraph.
```
A fenced code.
```
\end{markdown}

\begin{markdown}[blankBeforeCodeFence]
A paragraph.
```
Not a fenced code.
```
\end{markdown}

\end{document}
````````
Next, invoke LuaTeX from the terminal:
``` sh
lualatex document.tex
``````
A PDF document named `document.pdf` should be produced and contain the
following text:

> A paragraph.
>
> ```
> A fenced code.
> ```
>
> A paragraph. ``` Not a fenced code. ```

##### \Hologo{ConTeXt} Example {.unnumbered}

Using a text editor, create a text document named `document.tex` with the
following content:
```` tex
\usemodule[t][markdown]
\setupmarkdown[fencedCode = yes]
\starttext

\startmarkdown
A paragraph.
```
A fenced code.
```
\stopmarkdown

\setupmarkdown[blankBeforeCodeFence = yes]
\startmarkdown
A paragraph.
```
Not a fenced code.
```
\stopmarkdown

\stoptext
````````
Next, invoke LuaTeX from the terminal:
``` sh
context --luatex document.tex
``````
A PDF document named `document.pdf` should be produced and contain the
following text:

> A paragraph.
>
> ```
> A fenced code.
> ```
>
> A paragraph. ``` Not a fenced code. ```

%</manual-options>
%<*tex>
% \fi
%  \begin{macrocode}
\@@_add_lua_option:nnn
  { blankBeforeCodeFence }
  { boolean }
  { false }
%    \end{macrocode}
% \iffalse
%</tex>
%<*lua,lua-cli>
% \fi
%  \begin{macrocode}
defaultOptions.blankBeforeCodeFence = false
%    \end{macrocode}
% \par
% \iffalse
%</lua,lua-cli>
%<*manual-options>

#### Option `blankBeforeDivFence`

`blankBeforeDivFence` (default value: `false`)

% \fi
% \begin{markdown}
%
% \Optitem[false]{blankBeforeDivFence}{\opt{true}, \opt{false}}
%
:    true

     :  Require a blank line before the closing fence of a fenced div.

     false

     :  Do not require a blank line before the closing fence of a fenced div.

% \end{markdown}
% \iffalse

##### \LaTeX{} Example {.unnumbered}

Using a text editor, create a text document named `document.tex` with the
following content:
```` tex
\documentclass{article}
\usepackage[fencedDivs]{markdown}
\begin{document}

\begin{markdown}
A paragraph.

::: {.identifier}
A fenced div.
:::
\end{markdown}

\begin{markdown}[blankBeforeDivFence]
A paragraph.

::: {.identifier}
Not a fenced div.
:::
\end{markdown}

\end{document}
````````
Next, invoke LuaTeX from the terminal:
``` sh
lualatex document.tex
``````
A PDF document named `document.pdf` should be produced and contain the
following text:

> A paragraph.
>
> A fenced div.
>
> A paragraph.
>
> ::: {.identifier} Not a fenced div. :::

%</manual-options>
%<*tex>
% \fi
%  \begin{macrocode}
\@@_add_lua_option:nnn
  { blankBeforeDivFence }
  { boolean }
  { false }
%    \end{macrocode}
% \iffalse
%</tex>
%<*lua,lua-cli>
% \fi
%  \begin{macrocode}
defaultOptions.blankBeforeDivFence = false
%    \end{macrocode}
% \par
% \iffalse
%</lua,lua-cli>
%<*manual-options>

#### Option `blankBeforeHeading`

`blankBeforeHeading` (default value: `false`)

% \fi
% \begin{markdown}
%
% \Optitem[false]{blankBeforeHeading}{\opt{true}, \opt{false}}
%
:    true

     :  Require a blank line between a paragraph and the following header.

     false

     :  Do not require a blank line between a paragraph and the following
        header.

% \end{markdown}
% \iffalse

##### Lua Module Example {.unnumbered}

Using a text editor, create a text document named `document.tex` with the
following content:
``` tex
\input markdown
\input lmfonts
\def\markdownRendererHeadingOne#1{{\bf #1}\par}
\begingroup
\catcode`\%=12
\catcode`\#=12
\directlua{
  local markdown = require("markdown")
  local newline = [[^^J^^J]]
  local convert, input

  convert = markdown.new()
  input = "A paragraph." .. newline ..
          "A heading."   .. newline ..
          "=========="   .. newline
  tex.sprint(convert(input))

  convert = markdown.new({blankBeforeHeading = true})
  input = "A paragraph."    .. newline ..
          "Not a heading."  .. newline ..
          "=============="  .. newline
  tex.sprint(convert(input)) }
\endgroup
\bye
```````
Then, invoke LuaTeX from the terminal:
``` sh
luatex document.tex
```````
A PDF document named `document.pdf` should be produced and contain the
following text:

> A paragraph.
>
> A heading.
> ==========
>
> A paragraph. Not a heading. ==============

##### Lua CLI Example {.unnumbered}

Using a text editor, create a text document named `document.tex` with the
following content:
``` tex
\input markdown
\input lmfonts
\begingroup
\catcode`\%=12
\catcode`\#=12
\input optionfalse
\input optiontrue
\endgroup
\bye
```````
Using a text editor, create a text document named `content.md` with the
following content:
``` md
A paragraph.
A heading?
==========
``````
Next, invoke LuaTeX from the terminal:
``` sh
texlua ⟨CLI pathname⟩ -- content.md optionfalse.tex
texlua ⟨CLI pathname⟩ blankBeforeHeading=true  -- content.md optiontrue.tex
luatex document.tex
```````
where \meta{CLI pathname} corresponds to the location of the Lua CLI script file,
such as `~/texmf/scripts/markdown/markdown-cli.lua` on UN\*X systems or
`C:\Users\`\meta{Your username}`\texmf\scripts\markdown\markdown-cli.lua` on Windows
systems. Use the command `kpsewhich -a markdown-cli.lua` to locate the Lua CLI
script file using [Kpathsea][].

A PDF document named `document.pdf` should be produced and contain the
following text:

> A paragraph.
>
> A heading?
> ==========
>
> A paragraph. A heading? ==========

##### Plain \TeX{} Example {.unnumbered}

Using a text editor, create a text document named `document.tex` with the
following content:
``` tex
\input markdown

\markdownBegin
A paragraph.
A heading.
==========
\markdownEnd

\def\markdownOptionBlankBeforeHeading{true}
\markdownBegin
A paragraph.
Not a heading.
==============
\markdownEnd

\bye
```````
Next, invoke LuaTeX from the terminal:
``` sh
luatex document.tex
``````
A PDF document named `document.pdf` should be produced and contain the
following text:

> A paragraph.
>
> A heading.
> ==========
>
> A paragraph. Not a heading. ==============

##### \LaTeX{} Example {.unnumbered}

Using a text editor, create a text document named `document.tex` with the
following content:
``` tex
\documentclass{article}
\usepackage{markdown}
\begin{document}

\begin{markdown}
A paragraph.
A heading.
==========
\end{markdown}

\begin{markdown}[blankBeforeHeading]
A paragraph.
Not a heading.
==============
\end{markdown}

\end{document}
```````
Next, invoke LuaTeX from the terminal:
``` sh
lualatex document.tex
``````
A PDF document named `document.pdf` should be produced and contain the
following text:

> A paragraph.
>
> A heading.
> ==========
>
> A paragraph. Not a heading. ==============

##### \Hologo{ConTeXt} Example {.unnumbered}

Using a text editor, create a text document named `document.tex` with the
following content:
``` tex
\usemodule[t][markdown]
\starttext

\startmarkdown
A paragraph.
A heading.
==========
\stopmarkdown

\setupmarkdown[blankBeforeHeading = yes]
\startmarkdown
A paragraph.
Not a heading.
==============
\stopmarkdown

\stoptext
````````
Next, invoke LuaTeX from the terminal:
``` sh
context --luatex document.tex
`````
A PDF document named `document.pdf` should be produced and contain the
following text:

> A paragraph.
>
> A heading.
> ==========
>
> A paragraph. Not a heading. ==============

%</manual-options>
%<*tex>
% \fi
%  \begin{macrocode}
\@@_add_lua_option:nnn
  { blankBeforeHeading }
  { boolean }
  { false }
%    \end{macrocode}
% \iffalse
%</tex>
%<*lua,lua-cli>
% \fi
%  \begin{macrocode}
defaultOptions.blankBeforeHeading = false
%    \end{macrocode}
% \par
% \iffalse
%</lua,lua-cli>
%<*manual-options>

#### Option `blankBeforeList`

`blankBeforeList` (default value: `false`)

% \fi
% \begin{markdown}
%
% \Optitem[false]{blankBeforeList}{\opt{true}, \opt{false}}
%
:    true

     :  Require a blank line between a paragraph and the following list.

     false

     :  Do not require a blank line between a paragraph and the following
        list.

% \end{markdown}
% \iffalse

##### Lua Module Example {.unnumbered}

Using a text editor, create a text document named `document.tex` with the
following content:
``` tex
\input markdown
\input lmfonts
\begingroup
\catcode`\%=12
\catcode`\#=12
\directlua{
  local markdown = require("markdown")
  local newline = [[^^J^^J]]
  local convert, input

  convert = markdown.new()
  input = "A paragraph."   .. newline ..
          "- a list"       .. newline
  tex.sprint(convert(input))

  convert = markdown.new({
    blankBeforeList = true})
  input = "A paragraph."   .. newline ..
          "- not a list"     .. newline
  tex.sprint(convert(input)) }
\endgroup
\bye
```````
Then, invoke LuaTeX from the terminal:
``` sh
luatex document.tex
```````
A PDF document named `document.pdf` should be produced and contain the
following text:

> A paragraph.
>
> - a list
>
> A paragraph. - not a list

##### Lua CLI Example {.unnumbered}

Using a text editor, create a text document named `document.tex` with the
following content:
``` tex
\input markdown
\input lmfonts
\begingroup
\catcode`\%=12
\catcode`\#=12
\input optionfalse
\input optiontrue
\endgroup
\bye
```````
Using a text editor, create a text document named `content.md` with the
following content:
```` md
A paragraph.
- a list?
```````
Next, invoke LuaTeX from the terminal:
``` sh
texlua ⟨CLI pathname⟩ -- content.md optionfalse.tex
texlua ⟨CLI pathname⟩ blankBeforeList=true  -- content.md optiontrue.tex
luatex document.tex
```````
where \meta{CLI pathname} corresponds to the location of the Lua CLI script file,
such as `~/texmf/scripts/markdown/markdown-cli.lua` on UN\*X systems or
`C:\Users\`\meta{Your username}`\texmf\scripts\markdown\markdown-cli.lua` on Windows
systems. Use the command `kpsewhich -a markdown-cli.lua` to locate the Lua CLI
script file using [Kpathsea][].

A PDF document named `document.pdf` should be produced and contain the
following text:

> A paragraph.
>
> - a list?
>
> A paragraph. - a list?

##### Plain \TeX{} Example {.unnumbered}

Using a text editor, create a text document named `document.tex` with the
following content:
```` tex
\input markdown

\markdownBegin
A paragraph.
- a list
\markdownEnd

\def\markdownOptionBlankBeforeList{true}
\markdownBegin
A paragraph.
- not a list
\markdownEnd

\bye
````````
Next, invoke LuaTeX from the terminal:
``` sh
luatex document.tex
``````
A PDF document named `document.pdf` should be produced and contain the
following text:

> A paragraph.
>
> - a list
>
> A paragraph. - not a list

##### \LaTeX{} Example {.unnumbered}

Using a text editor, create a text document named `document.tex` with the
following content:
```` tex
\documentclass{article}
\usepackage{markdown}
\begin{document}

\begin{markdown}
A paragraph.
- a list
\end{markdown}

\begin{markdown}[blankBeforeList]
A paragraph.
- not a list
\end{markdown}

\end{document}
````````
Next, invoke LuaTeX from the terminal:
``` sh
lualatex document.tex
``````
A PDF document named `document.pdf` should be produced and contain the
following text:

> A paragraph.
>
> - a list
>
> A paragraph. - not a list

##### \Hologo{ConTeXt} Example {.unnumbered}

Using a text editor, create a text document named `document.tex` with the
following content:
```` tex
\usemodule[t][markdown]
\starttext

\startmarkdown
A paragraph.
- a list
\stopmarkdown

\setupmarkdown[blankBeforeList = yes]
\startmarkdown
A paragraph.
- not a list
\stopmarkdown

\stoptext
````````
Next, invoke LuaTeX from the terminal:
``` sh
context document.tex
``````
A PDF document named `document.pdf` should be produced and contain the
following text:

> A paragraph.
>
> - a list
>
> A paragraph. - not a list

%</manual-options>
%<*tex>
% \fi
%  \begin{macrocode}
\@@_add_lua_option:nnn
  { blankBeforeList }
  { boolean }
  { false }
%    \end{macrocode}
% \iffalse
%</tex>
%<*lua,lua-cli>
% \fi
%  \begin{macrocode}
defaultOptions.blankBeforeList = false
%    \end{macrocode}
% \par
% \iffalse
%</lua,lua-cli>
%<*manual-options>

#### Option `bracketedSpans`

`bracketedSpans` (default value: `false`)

% \fi
% \begin{markdown}
%
% \Optitem[false]{bracketedSpans}{\opt{true}, \opt{false}}
%
:    true

     :  Enable the Pandoc [bracketed span syntax extension][pandoc-bracketed-spans]:

        ``` md
        [This is *some text*]{.class key=val}
        ``````

:    false

     :  Disable the Pandoc bracketed span syntax extension.

 [pandoc-bracketed-spans]: https://pandoc.org/MANUAL.html#extension-bracketed_spans

% \end{markdown}
% \iffalse

##### \LaTeX{} Example {.unnumbered}

Using a text editor, create a text document named `document.tex` with the
following content:
```` tex
\documentclass{article}
\usepackage[bracketedSpans]{markdown}
\usepackage{expl3}
\ExplSyntaxOn
\markdownSetup{
  renderers = {
    bracketedSpanAttributeContextBegin = {
      \group_begin:
      \color_group_begin:
      \markdownSetup{
        renderers = {
          attributeKeyValue = {
            \str_if_eq:nnT
              { ##1 }
              { color }
              {
                 \color_select:n { ##2 }
              }
          },
        },
      }
    },
    bracketedSpanAttributeContextEnd = {
      \color_group_end:
      \group_end:
    },
  },
}
\ExplSyntaxOff
\begin{document}
\begin{markdown}
Here is some [colored text]{color=red}.
\end{markdown}
\end{document}
```````
Next, invoke LuaTeX from the terminal:
``` sh
lualatex document.tex
``````
A PDF document named `document.pdf` should be produced and contain the
following text:

> Here is some <span style="color: red">colored text</span>.

%</manual-options>
%<*tex>
% \fi
%  \begin{macrocode}
\@@_add_lua_option:nnn
  { bracketedSpans }
  { boolean }
  { false }
%    \end{macrocode}
% \iffalse
%</tex>
%<*lua,lua-cli>
% \fi
%  \begin{macrocode}
defaultOptions.bracketedSpans = false
%    \end{macrocode}
% \par
% \iffalse
%</lua,lua-cli>
%<*manual-options>

#### Option `breakableBlockquotes`

`breakableBlockquotes` (default value: `true`)

% \fi
% \begin{markdown}
%
% \Optitem[true]{breakableBlockquotes}{\opt{true}, \opt{false}}
%
:    true

     :  A blank line separates block quotes.

     false

     :  Blank lines in the middle of a block quote are ignored.

% \end{markdown}
% \iffalse

##### Lua Module Example {.unnumbered}

Using a text editor, create a text document named `document.tex` with the
following content:
``` tex
\input markdown
\input lmfonts
\def\markdownRendererHeadingOne#1{{\bf #1}\par}
\begingroup
\catcode`\%=12
\catcode`\#=12
\directlua{
  local markdown = require("markdown")
  local newline = [[^^J^^J]]
  local convert, input

  convert = markdown.new({breakableBlockquotes = false})
  input = "> A single"     .. newline .. newline ..
          "> block quote." .. newline
  tex.sprint(convert(input))

  convert = markdown.new()
  input = "> A block quote."       .. newline .. newline ..
          "> Another block quote." .. newline
  tex.sprint(convert(input)) }
\endgroup
\bye
```````
Then, invoke LuaTeX from the terminal:
``` sh
luatex document.tex
```````
A PDF document named `document.pdf` should be produced and contain the
following text:

> > A single block quote.
>
> > A block quote.
>
> > Another block quote.

##### Lua CLI Example {.unnumbered}

Using a text editor, create a text document named `document.tex` with the
following content:
``` tex
\input markdown
\input lmfonts
\begingroup
\catcode`\%=12
\catcode`\#=12
\input optionfalse
\input optiontrue
\endgroup
\bye
```````
Using a text editor, create a text document named `content.md` with the
following content:
``` md
> A single block quote

> or two block quotes?
``````
Next, invoke LuaTeX from the terminal:
``` sh
texlua ⟨CLI pathname⟩ breakableBlockquotes=false -- content.md optionfalse.tex
texlua ⟨CLI pathname⟩  -- content.md optiontrue.tex
luatex document.tex
```````
where \meta{CLI pathname} corresponds to the location of the Lua CLI script file,
such as `~/texmf/scripts/markdown/markdown-cli.lua` on UN\*X systems or
`C:\Users\`\meta{Your username}`\texmf\scripts\markdown\markdown-cli.lua` on Windows
systems. Use the command `kpsewhich -a markdown-cli.lua` to locate the Lua CLI
script file using [Kpathsea][].

A PDF document named `document.pdf` should be produced and contain the
following text:

> > A single block quote or two block quotes?
>
> > A single block quote
>
> > or two block quotes?

##### Plain \TeX{} Example {.unnumbered}

Using a text editor, create a text document named `document.tex` with the
following content:
``` tex
\input markdown

\def\markdownOptionBreakableBlockquotes{false}
\markdownBegin
> A single

> block quote.
\markdownEnd

\def\markdownOptionBreakableBlockquotes{true}
\markdownBegin
> A block quote.

> Another block quote.
\markdownEnd

\bye
```````
Next, invoke LuaTeX from the terminal:
``` sh
luatex document.tex
``````
A PDF document named `document.pdf` should be produced and contain the
following text:

> > A single block quote.
>
> > A block quote.
>
> > Another block quote.

##### \LaTeX{} Example {.unnumbered}

Using a text editor, create a text document named `document.tex` with the
following content:
``` tex
\documentclass{article}
\usepackage{markdown}
\begin{document}

\begin{markdown}[breakableBlockquotes = false]
> A single

> block quote.
\end{markdown}

\begin{markdown}
> A block quote.

> Another block quote.
\end{markdown}

\end{document}
```````
Next, invoke LuaTeX from the terminal:
``` sh
lualatex document.tex
``````
A PDF document named `document.pdf` should be produced and contain the
following text:

> > A single block quote.
>
> > A block quote.
>
> > Another block quote.

##### \Hologo{ConTeXt} Example {.unnumbered}

Using a text editor, create a text document named `document.tex` with the
following content:
``` tex
\usemodule[t][markdown]
\starttext

\setupmarkdown[breakableBlockquotes = no]
\startmarkdown
> A single

> block quote.
\stopmarkdown

\setupmarkdown[breakableBlockquotes = yes]
\startmarkdown
> A block quote.

> Another block quote.
\stopmarkdown

\stoptext
````````
Next, invoke LuaTeX from the terminal:
``` sh
context --luatex document.tex
`````
A PDF document named `document.pdf` should be produced and contain the
following text:

> > A single block quote.
>
> > A block quote.
>
> > Another block quote.

%</manual-options>
%<*tex>
% \fi
%  \begin{macrocode}
\@@_add_lua_option:nnn
  { breakableBlockquotes }
  { boolean }
  { true }
%    \end{macrocode}
% \iffalse
%</tex>
%<*lua,lua-cli>
% \fi
%  \begin{macrocode}
defaultOptions.breakableBlockquotes = true
%    \end{macrocode}
% \par
% \iffalse
%</lua,lua-cli>
%<*manual-options>

#### Option `citationNbsps`

`citationNbsps` (default value: `true`)

% \fi
% \begin{markdown}
%
% \Optitem[false]{citationNbsps}{\opt{true}, \opt{false}}
%
:    true

     :  Replace regular spaces with non-breaking spaces inside the prenotes
        and postnotes of citations produced via the pandoc citation syntax
        extension.

     false

     :  Do not replace regular spaces with non-breaking spaces inside the
        prenotes and postnotes of citations produced via the pandoc citation
        syntax extension.

% \end{markdown}
% \iffalse

##### \LaTeX{} Example {.unnumbered}

Using a text editor, create a text document named `document.bib` with the
following content:
``` bib
@book{knuth:tex,
  author    = "Knuth, Donald Ervin",
  title     = "The \TeX book, volume A of Computers and typesetting",
  publisher = "Addison-Wesley",
  year      = "1984"
}
```````
Using a text editor, create a text document named `document.tex` with the
following content:
``` tex
\documentclass{article}
\usepackage[citations]{markdown}
\begin{document}

\begin{markdown}
The TeXbook [@knuth:tex, p. 123 and 130] is good.
\end{markdown}

\begin{markdown}[citationNbsps = false]
The TeXbook [@knuth:tex, p. 123 and 130] is good.
\end{markdown}

\bibliographystyle{plain}
\bibliography{document.bib}
\end{document}
```````
Next, invoke LuaTeX and BibTeX from the terminal:
``` sh
lualatex document.tex
bibtex document.aux
lualatex document.tex
lualatex document.tex
``````
A PDF document named `document.pdf` should be produced and contain the
following text, where the middot (`·`) denotes a non-breaking space:

> The TeXbook [1, p.·123·and·130] is good.
>
> The TeXbook [1, p. 123 and 130] is good.
>
> ### References
> [1] Donald·Ervin Knuth. _The TeXbook, volume A of Computers and typesetting._
> Addison-Wesley, 1984.

%</manual-options>
%<*tex>
% \fi
%  \begin{macrocode}
\@@_add_lua_option:nnn
  { citationNbsps }
  { boolean }
  { true }
%    \end{macrocode}
% \iffalse
%</tex>
%<*lua,lua-cli>
% \fi
%  \begin{macrocode}
defaultOptions.citationNbsps = true
%    \end{macrocode}
% \par
% \iffalse
%</lua,lua-cli>
%<*manual-options>

#### Option `citations`

`citations` (default value: `false`)

% \fi
% \begin{markdown}
%
% \Optitem[false]{citations}{\opt{true}, \opt{false}}
%
:    true

     :  Enable the Pandoc [citation syntax extension][pandoc-citations]:

        ``` md
        Here is a simple parenthetical citation [@doe99] and here
        is a string of several [see @doe99, pp. 33-35; also
        @smith04, chap. 1].

        A parenthetical citation can have a [prenote @doe99] and
        a [@smith04 postnote]. The name of the author can be
        suppressed by inserting a dash before the name of an
        author as follows [-@smith04].

        Here is a simple text citation @doe99 and here is
        a string of several @doe99 [pp. 33-35; also @smith04,
        chap. 1]. Here is one with the name of the author
        suppressed -@doe99.
        ``````

:    false

     :  Disable the Pandoc citation syntax extension.

 [pandoc-citations]: https://pandoc.org/MANUAL.html#extension-citations

% \end{markdown}
% \iffalse

##### \LaTeX{} Example {.unnumbered}

Using a text editor, create a text document named `document.bib` with the
following content:
``` bib
@book{knuth:tex,
  author    = "Knuth, Donald Ervin",
  title     = "The \TeX book, volume A of Computers and typesetting",
  publisher = "Addison-Wesley",
  year      = "1984"
}
```````
Using a text editor, create a text document named `document.tex` with the
following content:
``` tex
\documentclass{article}
\usepackage[backend=biber]{biblatex}
\addbibresource{document.bib}
\usepackage[citations]{markdown}
\begin{document}

\begin{markdown}
The TeXbook [@knuth:tex, p. 123 and 130] was written by @knuth:tex.
\end{markdown}

\printbibliography
\end{document}
```````
Next, invoke LuaTeX and Biber from the terminal:
``` sh
lualatex document.tex
biber document.bcf
lualatex document.tex
lualatex document.tex
``````
A PDF document named `document.pdf` should be produced and contain the
following text:

> The TeXbook [1, p.·123 and 130] was written by Knuth [1].
>
> ### References
> [1] Donald Ervin Knuth. _The \TeX{}book, volume A of Computers and typesetting._
> Addison-Wesley, 1984.

%</manual-options>
%<*tex>
% \fi
%  \begin{macrocode}
\@@_add_lua_option:nnn
  { citations }
  { boolean }
  { false }
%    \end{macrocode}
% \iffalse
%</tex>
%<*lua,lua-cli>
% \fi
%  \begin{macrocode}
defaultOptions.citations = false
%    \end{macrocode}
% \par
% \iffalse
%</lua,lua-cli>
%<*manual-options>

#### Option `codeSpans`

`codeSpans` (default value: `true`)

% \fi
% \begin{markdown}
%
% \Optitem[true]{codeSpans}{\opt{true}, \opt{false}}
%
:    true

     :  Enable the code span syntax:

        ~~~ md
        Use the `printf()` function.
        ``There is a literal backtick (`) here.``
        ~~~

:    false

     :  Disable the code span syntax. This allows you to easily
        use the quotation mark ligatures in texts that do not contain code
        spans:

        ~~~
        ``This is a quote.''
        ~~~~~~

% \end{markdown}
% \iffalse

##### Lua Module Example {.unnumbered}

Using a text editor, create a text document named `document.tex` with the
following content:
``` tex
\input markdown
\input lmfonts
\begingroup
\catcode`\%=12
\catcode`\#=12
\directlua{
  local markdown = require("markdown")
  local convert = markdown.new()
  local input =
    "``This is a code span.'' " ..
    "``This is no longer a code span.''"
  tex.sprint(convert(input)) }
\par
\directlua{
  local markdown = require("markdown")
  local convert = markdown.new({codeSpans = false})
  local input =
    "``This is a quote.'' " ..
    "``This is another quote.''"
  tex.sprint(convert(input)) }
\endgroup
\bye
```````
Then, invoke LuaTeX from the terminal:
``` sh
luatex document.tex
```````
A PDF document named `document.pdf` should be produced and contain the
following text:

> ``This is a code span.'' ``This is no longer a code span.''
>
> “This is a quote.” “This is another quote.”

##### Lua CLI Example {.unnumbered}

Using a text editor, create a text document named `document.tex` with the
following content:
``` tex
\input markdown
\input lmfonts
\begingroup
\catcode`\%=12
\catcode`\#=12
\input optionfalse
\par
\input optiontrue
\endgroup
\bye
```````
Using a text editor, create a text document named `content.md` with the
following content:
``` md
``Is this a code span?'' ``Or a quote?''
``````
Next, invoke LuaTeX from the terminal:
``` sh
texlua ⟨CLI pathname⟩ codeSpans=false -- content.md optionfalse.tex
texlua ⟨CLI pathname⟩ -- content.md optiontrue.tex
luatex document.tex
```````
where \meta{CLI pathname} corresponds to the location of the Lua CLI script file,
such as `~/texmf/scripts/markdown/markdown-cli.lua` on UN\*X systems or
`C:\Users\`\meta{Your username}`\texmf\scripts\markdown\markdown-cli.lua` on Windows
systems. Use the command `kpsewhich -a markdown-cli.lua` to locate the Lua CLI
script file using [Kpathsea][].

A PDF document named `document.pdf` should be produced and contain the
following text:

> “Is this a code span?” “Or a quote?”
>
> ``Is this a code span?'' ``Or a quote?''

##### Plain \TeX{} Example {.unnumbered}

Using a text editor, create a text document named `document.tex` with the
following content:
``` tex
\input markdown

\markdownBegin
``This is a code span.''
``This is no longer a code span.''
\markdownEnd

\def\markdownOptionCodeSpans{false}
\markdownBegin
``This is a quote.''
``This is another quote.''
\markdownEnd

\bye
```````
Next, invoke LuaTeX from the terminal:
``` sh
luatex document.tex
``````
A PDF document named `document.pdf` should be produced and contain the
following text:

> ``This is a code span.'' ``This is no longer a code span.''
>
> “This is a quote.” “This is another quote.”

##### \LaTeX{} Example {.unnumbered}

Using a text editor, create a text document named `document.tex` with the
following content:
``` tex
\documentclass{article}
\usepackage{markdown}
\begin{document}

\begin{markdown}
``This is a code span.''
``This is no longer a code span.''
\end{markdown}

\begin{markdown}[codeSpans=false]
``This is a quote.''
``This is another quote.''
\end{markdown}

\end{document}
```````
Next, invoke LuaTeX from the terminal:
``` sh
lualatex document.tex
``````
A PDF document named `document.pdf` should be produced and contain the
following text:

> ``This is a code span.'' ``This is no longer a code span.''
>
> “This is a quote.” “This is another quote.”

##### \Hologo{ConTeXt} Example {.unnumbered}

Using a text editor, create a text document named `document.tex` with the
following content:
``` tex
\usemodule[t][markdown]
\starttext

\startmarkdown
``This is a code span.''
``This is no longer a code span.''
\stopmarkdown

\setupmarkdown[codeSpans = no]
\startmarkdown
``This is a quote.''
``This is another quote.''
\stopmarkdown

\stoptext
````````
Next, invoke LuaTeX from the terminal:
``` sh
context --luatex document.tex
`````
A PDF document named `document.pdf` should be produced and contain the
following text:

> ``This is a code span.'' ``This is no longer a code span.''
>
> “This is a quote.” “This is another quote.”

%</manual-options>
%<*tex>
% \fi
%  \begin{macrocode}
\@@_add_lua_option:nnn
  { codeSpans }
  { boolean }
  { true }
%    \end{macrocode}
% \iffalse
%</tex>
%<*lua,lua-cli>
% \fi
%  \begin{macrocode}
defaultOptions.codeSpans = true
%    \end{macrocode}
% \par
% \iffalse
%</lua,lua-cli>
%<*manual-options>

#### Option `contentBlocks`

`contentBlocks` (default value: `false`)

% \fi
% \begin{markdown}
%
% \Optitem[false]{contentBlocks}{\opt{true}, \opt{false}}
%
:    true

     :  Enable the
%       iA\\,Writer content blocks syntax extension~[@sotkov17]:
%       \iffalse
        iA\\,Writer content blocks syntax extension:
%       \fi

        ``` md
        http://example.com/minard.jpg (Napoleon's
          disastrous Russian campaign of 1812)
        /Flowchart.png "Engineering Flowchart"
        /Savings Account.csv 'Recent Transactions'
        /Example.swift
        /Lorem Ipsum.txt
        ``````

:    false

     :  Disable the
        iA\\,Writer content blocks syntax extension.

% \end{markdown}
% \iffalse

##### \LaTeX{} Example {.unnumbered}

Using a text editor, create a text document named `table.csv` with the
following content:
``` csv
Name,Surname,Born
Albert,Einstein,1879
Marie,Curie,1867
Thomas,Edison,1847
```````
Create also a text document named `markdown-languages.json` with the following
content:
``` js
{
  "tex": "LaTeX"
}
``````
Create also a text document named `code.tex` with the following content:
``` tex
This is an example code listing in \LaTeX.
```````
Create also a text document named `part.md` with the following content:
``` md
This is a *transcluded markdown document*.
``````
Create also a text document named `document.tex` with the following content:
``` tex
\documentclass{article}
\usepackage{minted}
\usepackage[contentBlocks]{markdown}
\begin{document}
\begin{markdown}
/table.csv  (An example table)
/code.tex   (An example code listing)
/part.md    (A file transclusion example)
\end{markdown}
\end{document}
```````
Next, invoke LuaTeX from the terminal:
``` sh
lualatex --shell-escape document.tex
``````
A PDF document named `document.pdf` should be produced and contain the
following text:

> | Name   | Surname  | Born |
> | ------ | ---------| ---- |
> | Albert | Einstein | 1879 |
> | Marie  | Curie    | 1867 |
> | Thomas | Edison   | 1847 |
>
> Table 1: An example table
>
> ``` tex
> This is an example code listing in \LaTeX.
> ```````
>
> This is a *transcluded markdown document*.

##### \Hologo{ConTeXt} Example {.unnumbered}

Using a text editor, create a text document named `table.csv` with the
following content:
``` csv
Name,Surname,Born
Albert,Einstein,1879
Marie,Curie,1867
Thomas,Edison,1847
```````
Create also a text document named `markdown-languages.json` with the following
content:
``` js
{
  "tex": "ConTeXt"
}
``````
Create also a text document named `code.tex` with the following content:
``` tex
This is an example code listing in \ConTeXt.
```````
Create also a text document named `part.md` with the following content:
``` md
This is a *transcluded markdown document*.
``````
Create also a text document named `document.tex` with the following content:
``` tex
\usemodule[t][markdown]
\setupmarkdown[contentBlocks = yes]
\definetyping [ConTeXt]
\setuptyping  [ConTeXt] [option=TEX]
\starttext
\startmarkdown
/table.csv  (An example table)
/code.tex   (An example code listing)
/part.md    (A file transclusion example)
\stopmarkdown
\stoptext
````````
Next, invoke LuaTeX from the terminal:
``` sh
context --luatex document.tex
`````
A PDF document named `document.pdf` should be produced and contain the
following text:

> | Name   | Surname  | Born |
> | ------ | ---------| ---- |
> | Albert | Einstein | 1879 |
> | Marie  | Curie    | 1867 |
> | Thomas | Edison   | 1847 |
>
> Table 1: An example table
>
> ``` tex
> This is an example code listing in \ConTeXt.
> ```````
>
> This is a *transcluded markdown document*.

%</manual-options>
%<*tex>
% \fi
%  \begin{macrocode}
\@@_add_lua_option:nnn
  { contentBlocks }
  { boolean }
  { false }
%    \end{macrocode}
% \iffalse
%</tex>
%<*lua,lua-cli>
% \fi
%  \begin{macrocode}
defaultOptions.contentBlocks = false
%    \end{macrocode}
% \par
% \iffalse
%</lua,lua-cli>
%<*manual-options>

#### Option `contentLevel`

`contentLevel` (default value: `block`)

% \fi
% \begin{markdown}
%
% \Optitem[block]{contentLevel}{\opt{block}, \opt{inline}}
%
:    block

     :  Treat content as a sequence of blocks.
        ``` md
        - this is a list
        - it contains two items
        ``````

:    inline

     :  Treat all content as inline content.
        ``` md
        - this is a text
        - not a list
        ``````

% \end{markdown}
% \iffalse

##### Plain \TeX{} Example {.unnumbered}

Using a text editor, create a text document named `document.tex` with the
following content:
``` tex
\input markdown
\def\markdownOptionContentLevel{inline}
\markdownBegin
- this is
- a text
\markdownEnd
\bye
```````
Next, invoke LuaTeX from the terminal:
``` sh
luatex document.tex
``````
A PDF document named `document.pdf` should be produced and contain the
following text:

> \- this is - a text

##### \LaTeX{} Example {.unnumbered}

Using a text editor, create a text document named `document.tex` with the
following content:
``` tex
\documentclass{article}
\usepackage{markdown}
\begin{document}

\begin{markdown}
- this is
- a list
\end{markdown}

\begin{markdown}[contentLevel=inline]
- this is
- a text
\end{markdown}

\end{document}
```````
Next, invoke LuaTeX from the terminal:
``` sh
lualatex document.tex
``````
A PDF document named `document.pdf` should be produced and contain the
following text:

> - this is
> - a list
>
> \- this is - a text

##### \Hologo{ConTeXt} Example {.unnumbered}

Using a text editor, create a text document named `document.tex` with the
following content:
``` tex
\usemodule[t][markdown]
\starttext

\startmarkdown
- this is
- a list
\stopmarkdown

\setupmarkdown[contentLevel = inline]
\startmarkdown
- this is
- a text
\stopmarkdown

\stoptext
````````
Next, invoke LuaTeX from the terminal:
``` sh
context --luatex document.tex
`````
A PDF document named `document.pdf` should be produced and contain the
following text:

> - this is
> - a list
>
> \- this is - a text

%</manual-options>
%<*tex>
% \fi
%  \begin{macrocode}
\@@_add_lua_option:nnn
  { contentLevel }
  { string }
  { block }
%    \end{macrocode}
% \iffalse
%</tex>
%<*lua,lua-cli>
% \fi
%  \begin{macrocode}
defaultOptions.contentLevel = "block"
%    \end{macrocode}
% \par
% \iffalse
%</lua,lua-cli>
%<*manual-options>

#### Option `debugExtensions`

`debugExtensions` (default value: `false`)

% \fi
% \begin{markdown}
%
% \Optitem[false]{debugExtensions}{\opt{true}, \opt{false}}
%
:    true

     :   Produce a \acro{JSON} file that will contain the
         extensible subset of the \acro{peg} grammar of markdown
%        (see the \luamref{walkable_syntax} hash table)
         after built-in syntax extensions
%        (see Section <#sec:lua-built-in-extensions>)
%        \iffalse
         (see options \Opt{citations}, \Opt{contentBlocks},
         \Opt{definitionLists}, etc.)
%        \fi
         and user-defined syntax extensions
%        (see Section <#sec:lua-user-extensions>)
%        \iffalse
         (see option \Opt{extensions})
%        \fi
         have been applied. This helps you to see how the different
         extensions interact. The name of the produced \acro{JSON} file is
         controlled by the \Opt{debugExtensionsFileName} option.

:    false

     :  Do not produce a \acro{JSON} file with the \acro{peg} grammar of
        markdown.

% \end{markdown}
% \iffalse

##### \LaTeX{} Example {.unnumbered}

Using a text editor, create a text document named `strike-through.lua` with the
following content:
``` lua
local strike_through = {
  api_version = 2,
  grammar_version = 4,
  finalize_grammar = function(reader)
    local nonspacechar = lpeg.P(1) - lpeg.S("\t ")
    local doubleslashes = lpeg.P("//")
    local function between(p, starter, ender)
      ender = lpeg.B(nonspacechar) * ender
      return (starter * #nonspacechar
             * lpeg.Ct(p * (p - ender)^0) * ender)
    end

    local read_strike_through = between(
      lpeg.V("Inline"), doubleslashes, doubleslashes
    ) / function(s) return {"\\st{", s, "}"} end

    reader.insert_pattern("Inline after LinkAndEmph", read_strike_through,
                          "StrikeThrough")
    reader.add_special_character("/")
  end
}

return strike_through
```````
Using a text editor, create also a text document named `document.tex` with the
following content:
``` tex
\documentclass{article}
\usepackage{soul}
\usepackage[extension = strike-through.lua,
            debugExtensions]{markdown}
\begin{document}
\begin{markdown}
This is //a lunar roving vehicle// strike-through text.
\end{markdown}
\end{document}
```````
Next, invoke LuaTeX from the terminal:
``` sh
lualatex document.tex
``````
A PDF document named `document.pdf` should be produced and contain the
following text:

> This is ~~a lunar roving vehicle~~ strike-through text.

Furthermore, a JSON document named `document.debug-extensions.json` should also
be produced and contain the following text:

``` json
{
    "Block": [
        "Blockquote",
        "Verbatim",
        "FencedCode (built-in fenced_code syntax extension)",
        "ThematicBreak",
        "BulletList",
        "OrderedList",
        "DisplayHtml",
        "Heading"
    ],
    "BlockOrParagraph": [
        "Block",
        "Paragraph",
        "Plain"
    ],
    "EndlineExceptions": [
        "EndlineExceptions (built-in fenced_code syntax extension)"
    ],
    "Inline": [
        "Str",
        "Space",
        "Endline",
        "EndlineBreak",
        "LinkAndEmph",
        "StrikeThrough (user-defined \"./strike-through.lua\" syntax extension)",
        "Code",
        "AutoLinkUrl",
        "AutoLinkEmail",
        "AutoLinkRelativeReference",
        "InlineHtml",
        "HtmlEntity",
        "EscapedChar",
        "Smart",
        "Symbol"
    ]
}
````````

This output shows us that our user-defined syntax extension has been correctly
inserted to the grammar of markdown.

%</manual-options>
%<*tex>
% \fi
%  \begin{macrocode}
\@@_add_lua_option:nnn
  { debugExtensions }
  { boolean }
  { false }
%    \end{macrocode}
% \iffalse
%</tex>
%<*lua,lua-cli>
% \fi
%  \begin{macrocode}
defaultOptions.debugExtensions = false
%    \end{macrocode}
% \par
% \iffalse
%</lua,lua-cli>
%<*manual-options>

#### Option `definitionLists`

`definitionLists` (default value: `false`)

% \fi
% \begin{markdown}
%
% \Optitem[false]{definitionLists}{\opt{true}, \opt{false}}
%
:    true

     :  Enable the pandoc definition list syntax extension:

        ``` md
        Term 1

        :   Definition 1

        Term 2 with *inline markup*

        :   Definition 2

                { some code, part of Definition 2 }

            Third paragraph of definition 2.
        `````

:    false

     :  Disable the pandoc definition list syntax extension.

% \end{markdown}
% \iffalse

##### \LaTeX{} Example {.unnumbered}

Using a text editor, create a text document named `document.tex` with the
following content:
``` tex
\documentclass{article}
\usepackage[definitionLists]{markdown}
\begin{document}
\begin{markdown}
Term 1

:   Definition 1

Term 2 with *inline markup*

:   Definition 2

        { some code, part of Definition 2 }

    Third paragraph of definition 2.
\end{markdown}
\end{document}
```````
Next, invoke LuaTeX from the terminal:
``` sh
lualatex document.tex
``````
A PDF document named `document.pdf` should be produced and contain the
following text:

> Term 1
>
> :   Definition 1
>
> Term 2 with *inline markup*
>
> :   Definition 2
>
>         { some code, part of Definition 2 }
>
>     Third paragraph of definition 2.

##### \Hologo{ConTeXt} Example {.unnumbered}

Using a text editor, create a text document named `document.tex` with the
following content:
``` tex
\usemodule[t][markdown]
\setupmarkdown[definitionLists = yes]
\starttext
\startmarkdown
Term 1

:   Definition 1

Term 2 with *inline markup*

:   Definition 2

        { some code, part of Definition 2 }

    Third paragraph of definition 2.
\stopmarkdown
\stoptext
````````
Next, invoke LuaTeX from the terminal:
``` sh
context --luatex document.tex
`````
A PDF document named `document.pdf` should be produced and contain the
following text:

> Term 1
>
> :   Definition 1
>
> Term 2 with *inline markup*
>
> :   Definition 2
>
>         { some code, part of Definition 2 }
>
>     Third paragraph of definition 2.

%</manual-options>
%<*tex>
% \fi
%  \begin{macrocode}
\@@_add_lua_option:nnn
  { definitionLists }
  { boolean }
  { false }
%    \end{macrocode}
% \iffalse
%</tex>
%<*lua,lua-cli>
% \fi
%  \begin{macrocode}
defaultOptions.definitionLists = false
%    \end{macrocode}
% \par
% \iffalse
%</lua,lua-cli>
%<*manual-options>

#### Option `ensureJekyllData`

`ensureJekyllData` (default value: `false`)

% \fi
% \begin{markdown}
%
% \Optitem[false]{ensureJekyllData}{\opt{true}, \opt{false}}
%
:    false

     :  When the \Opt{jekyllData} and \Opt{expectJekyllData} options are
        enabled, then a markdown document may begin directly with \acro{yaml}
        metadata and may contain nothing but \acro{yaml} metadata. Otherwise,
        the markdown document is processed as markdown text.

:    true

     :  When the \Opt{jekyllData} and \Opt{expectJekyllData} options are
        enabled, then a markdown document must begin directly with \acro{yaml}
        metadata and must contain nothing but \acro{yaml} metadata. Otherwise,
        an error is produced.

% \end{markdown}
% \iffalse
%</manual-options>
%<*tex>
% \fi
%  \begin{macrocode}
\@@_add_lua_option:nnn
  { ensureJekyllData }
  { boolean }
  { false }
%    \end{macrocode}
% \iffalse
%</tex>
%<*lua,lua-cli>
% \fi
%  \begin{macrocode}
defaultOptions.ensureJekyllData = false
%    \end{macrocode}
% \par
% \iffalse
%</lua,lua-cli>
%<*manual-options>

#### Option `expectJekyllData`

`expectJekyllData` (default value: `false`)

% \fi
% \markdownBegin
%
% \Optitem[false]{expectJekyllData}{\opt{true}, \opt{false}}
%
:    false

     :  When the \Opt{jekyllData} option is enabled, then a markdown document
        may begin with \acro{yaml} metadata if and only if the metadata begin
        with the end-of-directives marker (`---`) and they end with either the
        end-of-directives or the end-of-document marker (`...`):

        ~~~~~ latex
        \documentclass{article}
        \usepackage[jekyllData]{markdown}
        \begin{document}
        \begin{markdown}
        ---
        - this
        - is
        - YAML
        ...
        - followed
        - by
        - Markdown
        \end{markdown}
        \begin{markdown}
        - this
        - is
        - Markdown
        \end{markdown}
        \end{document}
        ~~~~~~~~~~~

:    true

     :  When the \Opt{jekyllData} option is enabled, then a markdown document may
        begin directly with \acro{yaml} metadata and may contain nothing but
        \acro{yaml} metadata.

        ~~~~~ latex
        \documentclass{article}
        \usepackage[jekyllData, expectJekyllData]{markdown}
        \begin{document}
        \begin{markdown}
        - this
        - is
        - YAML
        ...
        - followed
        - by
        - Markdown
        \end{markdown}
        \begin{markdown}
        - this
        - is
        - YAML
        \end{markdown}
        \end{document}
        ~~~~~~~~~~~

% \markdownEnd
% \iffalse

##### \LaTeX{} Example {.unnumbered}

Using a text editor, create a text document named `jane-doe.yml` with the
following content:
``` yaml
name: Jane Doe
age:  99
```
Using a text editor, create also a text document named `document.tex` with the
following content:
``` tex
\documentclass{article}
\usepackage[jekyllData]{markdown}
\markdownSetup{
  jekyllDataRenderers = {
    name = {\gdef\name{#1}},
    code = {\gdef\age{#1}},
  },
  renderers = {
    jekyllDataEnd = {\name{} is \age{} years old.},
  }
}
\begin{document}
\markdownInput[expectJekyllData]{jane-doe.yml}
\end{document}
```````
Next, invoke LuaTeX from the terminal:
``` sh
lualatex document.tex
``````
A PDF document named `document.pdf` should be produced and contain the
following text:

> Jane Doe is 99 years old.

%</manual-options>
%<*tex>
% \fi
%  \begin{macrocode}
\@@_add_lua_option:nnn
  { expectJekyllData }
  { boolean }
  { false }
%    \end{macrocode}
% \iffalse
%</tex>
%<*lua,lua-cli>
% \fi
%  \begin{macrocode}
defaultOptions.expectJekyllData = false
%    \end{macrocode}
% \par
% \iffalse
%</lua,lua-cli>
%<*manual-options>

#### Option `extensions`

`extensions` (default value: `{}`)

% \fi
% \begin{markdown}
%
% \Valitem[\{\}]{extensions}{filenames}
%
:    The filenames of user-defined syntax extensions that will be applied to the
     markdown reader. If the \pkg{kpathsea} library is available, files will be
     searched for not only in the current working directory but also in the
     \TeX{} directory structure.

%    A user-defined syntax extension is a Lua file in the following format:
%
%    ``` lua
%    local strike_through = {
%      api_version = 2,
%      grammar_version = 4,
%      finalize_grammar = function(reader)
%        local nonspacechar = lpeg.P(1) - lpeg.S("\t ")
%        local doubleslashes = lpeg.P("//")
%        local function between(p, starter, ender)
%          ender = lpeg.B(nonspacechar) * ender
%          return (starter * #nonspacechar
%                 * lpeg.Ct(p * (p - ender)^0) * ender)
%        end
%
%        local read_strike_through = between(
%          lpeg.V("Inline"), doubleslashes, doubleslashes
%        ) / function(s) return {"\\st{", s, "}"} end
%
%        reader.insert_pattern("Inline after LinkAndEmph", read_strike_through,
%                              "StrikeThrough")
%        reader.add_special_character("/")
%      end
%    }
%
%    return strike_through
%    ```````
%
%    The `api_version` and `grammar_version` fields specify the version of the
%    user-defined syntax extension API and the markdown grammar for which
%    the extension was written. See the current API and grammar versions
%    below:
%
% \end{markdown}
% \iffalse
%</manual-options>
%<*lua>
% \fi
%  \begin{macrocode}
metadata.user_extension_api_version = 2
metadata.grammar_version = 4
%    \end{macrocode}
% \iffalse
%</lua>
%<*manual-options>
% \fi
% \begin{markdown}
%
% Any changes to the syntax extension API or grammar will cause the
% corresponding current version to be incremented.  After Markdown 3.0.0,
% any changes to the API and the grammar will be either backwards-compatible
% or constitute a breaking change that will cause the major version of the
% Markdown package to increment (to 4.0.0).
%
% The `finalize_grammar` field is a function that finalizes the grammar of
% markdown using the interface of a Lua \luamref{reader} object, such as
% the \luamref{reader->insert_pattern} and
% \luamref{reader->add_special_character} methods,
% see Section <#sec:lua-user-extensions>.
%
% \end{markdown}
% \iffalse

##### \LaTeX{} Example {.unnumbered}

Using a text editor, create a text document named `strike-through.lua` with the
following content:
``` lua
local strike_through = {
  api_version = 2,
  grammar_version = 4,
  finalize_grammar = function(reader)
    local nonspacechar = lpeg.P(1) - lpeg.S("\t ")
    local doubleslashes = lpeg.P("//")
    local function between(p, starter, ender)
      ender = lpeg.B(nonspacechar) * ender
      return (starter * #nonspacechar
             * lpeg.Ct(p * (p - ender)^0) * ender)
    end

    local read_strike_through = between(
      lpeg.V("Inline"), doubleslashes, doubleslashes
    ) / function(s) return {"\\st{", s, "}"} end

    reader.insert_pattern("Inline after LinkAndEmph", read_strike_through,
                          "StrikeThrough")
    reader.add_special_character("/")
  end
}

return strike_through
```````
Using a text editor, create also a text document named `document.tex` with the
following content:
``` tex
\documentclass{article}
\usepackage{soul}
\usepackage[extension = strike-through.lua]{markdown}
\begin{document}
\begin{markdown}
This is //a lunar roving vehicle// strike-through text.
\end{markdown}
\end{document}
```````
Next, invoke LuaTeX from the terminal:
``` sh
lualatex document.tex
``````
A PDF document named `document.pdf` should be produced and contain the
following text:

> This is ~~a lunar roving vehicle~~ strike-through text.

%</manual-options>
%<*tex>
% \fi
%  \begin{macrocode}
\cs_generate_variant:Nn
  \@@_add_lua_option:nnn
  { nnV }
\@@_add_lua_option:nnV
  { extensions }
  { clist }
  \c_empty_clist
%    \end{macrocode}
% \iffalse
%</tex>
%<*lua,lua-cli>
% \fi
%  \begin{macrocode}
defaultOptions.extensions = {}
%    \end{macrocode}
% \par
% \iffalse
%</lua,lua-cli>
%<*manual-options>

#### Option `fancyLists`

`fancyLists` (default value: `false`)

% \fi
% \begin{markdown}
%
% \Optitem[false]{fancyLists}{\opt{true}, \opt{false}}
%
:    true

     :  Enable the Pandoc [fancy list syntax extension][pandoc-fancy-lists]:

        ``` md
        a) first item
        b) second item
        c) third item
        ``````

:    false

     :  Disable the Pandoc fancy list syntax extension.

 [pandoc-fancy-lists]: https://pandoc.org/MANUAL.html#org-fancy-lists

% \end{markdown}
% \iffalse

##### \LaTeX{} Example {.unnumbered}

Using a text editor, create a text document named `document.tex` with the
following content:
```` tex
\documentclass{article}
\usepackage{minted}
\usepackage[fancyLists]{markdown}
\begin{document}
\begin{markdown}
a) first item
b) second item
c) third item
\end{markdown}
\end{document}
```````
Next, invoke LuaTeX from the terminal:
``` sh
lualatex --shell-escape document.tex
``````
A PDF document named `document.pdf` should be produced and contain the
following text:

> a) first item
> b) second item
> c) third item

##### \Hologo{ConTeXt} Example {.unnumbered}

Using a text editor, create a text document named `document.tex` with the
following content:
```` tex
\usemodule[t][markdown]
\setupmarkdown[fancyLists = yes]
\starttext
\startmarkdown
a) first item
b) second item
c) third item
\stopmarkdown
\stoptext
````````
Next, invoke LuaTeX from the terminal:
``` sh
context --luatex document.tex
`````
A PDF document named `document.pdf` should be produced and contain the
following text:

> a) first item
> b) second item
> c) third item

%</manual-options>
%<*tex>
% \fi
%  \begin{macrocode}
\@@_add_lua_option:nnn
  { fancyLists }
  { boolean }
  { false }
%    \end{macrocode}
% \iffalse
%</tex>
%<*lua,lua-cli>
% \fi
%  \begin{macrocode}
defaultOptions.fancyLists = false
%    \end{macrocode}
% \par
% \iffalse
%</lua,lua-cli>
%<*manual-options>

#### Option `fencedCode`

`fencedCode` (default value: `true`)

% \fi
% \begin{markdown}
%
% \Optitem[true]{fencedCode}{\opt{true}, \opt{false}}
%
:    true

     :  Enable the commonmark fenced code block extension:

        ~~~~~~~~ md
        ~~~ js
        if (a > 3) {
            moveShip(5 * gravity, DOWN);
        }
        ~~~~~~

          ``` html
          <pre>
            <code>
              // Some comments
              line 1 of code
              line 2 of code
              line 3 of code
            </code>
          </pre>
          ```
        ~~~~~~~~~~~

:    false

     :  Disable the commonmark fenced code block extension.

% \end{markdown}
% \iffalse

##### \LaTeX{} Example {.unnumbered}

Using a text editor, create a text document named `document.tex` with the
following content:
```` tex
\documentclass{article}
\usepackage{minted}
\usepackage{markdown}
\begin{document}
\begin{markdown}
~~~ js
if (a > 3) {
    moveShip(5 * gravity, DOWN);
}
~~~~~~

  ``` html
  <pre>
    <code>
      // Some comments
      line 1 of code
      line 2 of code
      line 3 of code
    </code>
  </pre>
  ```
\end{markdown}
\end{document}
```````
Next, invoke LuaTeX from the terminal:
``` sh
lualatex --shell-escape document.tex
``````
A PDF document named `document.pdf` should be produced and contain the
following text:

> ``` js
> if (a > 3) {
>     moveShip(5 * gravity, DOWN);
> }
> ```
>
> ``` html
> <pre>
>   <code>
>     // Some comments
>     line 1 of code
>     line 2 of code
>     line 3 of code
>   </code>
> </pre>
> ```

##### \Hologo{ConTeXt} Example {.unnumbered}

Using a text editor, create a text document named `document.tex` with the
following content:
```` tex
\usemodule[t][markdown]
\definetyping [js]
\definetyping [html]
\setuptyping  [html] [option=XML]
\starttext
\startmarkdown
~~~ js
if (a > 3) {
    moveShip(5 * gravity, DOWN);
}
~~~~~~

  ``` html
  <pre>
    <code>
      // Some comments
      line 1 of code
      line 2 of code
      line 3 of code
    </code>
  </pre>
  ```
\stopmarkdown
\stoptext
````````
Next, invoke LuaTeX from the terminal:
``` sh
context --luatex document.tex
`````
A PDF document named `document.pdf` should be produced and contain the
following text:

> ``` js
> if (a > 3) {
>     moveShip(5 * gravity, DOWN);
> }
> ```
>
> ``` html
> <pre>
>   <code>
>     // Some comments
>     line 1 of code
>     line 2 of code
>     line 3 of code
>   </code>
> </pre>
> ```

%</manual-options>
%<*tex>
% \fi
%  \begin{macrocode}
\@@_add_lua_option:nnn
  { fencedCode }
  { boolean }
  { true }
%    \end{macrocode}
% \iffalse
%</tex>
%<*lua,lua-cli>
% \fi
%  \begin{macrocode}
defaultOptions.fencedCode = true
%    \end{macrocode}
% \par
% \iffalse
%</lua,lua-cli>
%<*manual-options>

#### Option `fencedCodeAttributes`

`fencedCodeAttributes` (default value: `false`)

% \fi
% \begin{markdown}
%
% \Optitem[false]{fencedCodeAttributes}{\opt{true}, \opt{false}}
%
:    true

     :  Enable the Pandoc [fenced code attribute syntax extension][pandoc-fenced-code-attributes]:

        ```````` md
        ~~~~ {#mycode .haskell .numberLines startFrom=100}
        qsort []     = []
        qsort (x:xs) = qsort (filter (< x) xs) ++ [x] ++
                       qsort (filter (>= x) xs)
        ~~~~~~~~~~~~~~~~~~~~~~~~~~~~~~~~~~~~~~~~~~~~~~~~~
        ```````````

:    false

     :  Disable the Pandoc fenced code attribute syntax extension.

 [pandoc-fenced-code-attributes]: https://pandoc.org/MANUAL.html#extension-fenced_code_attributes

% \end{markdown}
% \iffalse

##### \LaTeX{} Example {.unnumbered}

Using a text editor, create a text document named `document.tex` with the
following content:
```` tex
\documentclass{article}
\usepackage[fencedCode,
            fencedCodeAttributes]{markdown}
\usepackage{minted}
\markdownSetup{
  renderers = {
    fencedCodeAttributeContextBegin = {%
      \begingroup
      \markdownSetup{
        renderers = {
          attributeKeyValue = {%
            \setminted{{#1} = {#2}}%
          },
        },
      }%
    },
    fencedCodeAttributeContextEnd = {%
      \endgroup
    },
  },
}
\begin{document}
\begin{markdown}
~~~ js {linenos=true}
if (a > 3) {
    moveShip(5 * gravity, DOWN);
}
~~~~~~
\end{markdown}
\end{document}
```````
Next, invoke LuaTeX from the terminal:
``` sh
lualatex --shell-escape document.tex
``````
A PDF document named `document.pdf` should be produced and contain the
following text:

> ``` js {.linenos}
> 1. if (a > 3) {
> 2.     moveShip(5 * gravity, DOWN);
> 3. }
> ``````

%</manual-options>
%<*tex>
% \fi
%  \begin{macrocode}
\@@_add_lua_option:nnn
  { fencedCodeAttributes }
  { boolean }
  { false }
%    \end{macrocode}
% \iffalse
%</tex>
%<*lua,lua-cli>
% \fi
%  \begin{macrocode}
defaultOptions.fencedCodeAttributes = false
%    \end{macrocode}
% \par
% \iffalse
%</lua,lua-cli>
%<*manual-options>

#### Option `fencedDivs` {#fenced-divs}

`fencedDivs` (default value: `false`)

% \fi
% \begin{markdown}
%
% \Optitem[false]{fencedDivs}{\opt{true}, \opt{false}}
%
:    true

     :  Enable the Pandoc [fenced div syntax extension][pandoc-fenced-divs]:

        ``` md
        ::::: {#special .sidebar}
        Here is a paragraph.

        And another.
        :::::
        ``````

:    false

     :  Disable the Pandoc fenced div syntax extension.

 [pandoc-fenced-divs]: https://pandoc.org/MANUAL.html#extension-fenced_divs

% \end{markdown}
% \iffalse

##### \LaTeX{} Example {.unnumbered}

Using a text editor, create a text document named `document.tex` with the
following content:
```` tex
\documentclass{article}
\usepackage[fencedDivs]{markdown}
\begin{document}
\begin{markdown}{slice=special}
Here is a regular paragraph.

::::: {#special}
Here is a special paragraph.
:::::

And here is another regular paragraph.
\end{markdown}
\end{document}
```````
Next, invoke LuaTeX from the terminal:
``` sh
lualatex --shell-escape document.tex
``````
A PDF document named `document.pdf` should be produced and contain the
following text:

> Here is a special paragraph.

%</manual-options>
%<*tex>
% \fi
%  \begin{macrocode}
\@@_add_lua_option:nnn
  { fencedDivs }
  { boolean }
  { false }
%    \end{macrocode}
% \iffalse
%</tex>
%<*lua,lua-cli>
% \fi
%  \begin{macrocode}
defaultOptions.fencedDivs = false
%    \end{macrocode}
% \par
% \iffalse
%</lua,lua-cli>
%<*manual-options>

#### Option `finalizeCache`

`finalizeCache` (default value: `false`)

% \fi
% \begin{markdown}
%
% \Optitem[false]{finalizeCache}{\opt{true}, \opt{false}}
%
:    Whether an output file specified with the \Opt{frozenCacheFileName} option
     (frozen cache) that contains a mapping between an enumeration of markdown
     documents and their auxiliary cache files will be created.

     The frozen cache makes it possible to later typeset a plain \TeX{}
     document that contains markdown documents without invoking Lua using
     the \Opt{frozenCache} plain \TeX{} option. As a result, the
     plain \TeX{} document becomes more portable, but further changes in the
     order and the content of markdown documents will not be reflected.

% \end{markdown}
% \iffalse

##### Plain \TeX{} Example {.unnumbered}

Using a text editor, create a text document named `document.tex` with the
following content:
``` tex
\input markdown
\def\markdownOptionFinalizeCache{true}
\markdownBegin
Hello *world*!
\markdownEnd
\bye
```````
Next, invoke LuaTeX from the terminal:
``` sh
luatex document.tex
``````
A PDF document named `document.pdf` should be produced and contain the text
“Hello *world*!” A frozen cache will also be produced as we requested using the
`finalizeCache` option.

Next, change the content of `document.tex` as follows:
``` tex
\input markdown
\def\markdownOptionFrozenCache{true}
\markdownBegin
Hi *world*!
\markdownEnd
\bye
```````
Last, invoke pdfTeX without shell access from the terminal:
``` sh
pdftex -no-shell-escape document.tex
``````
A new PDF document named `document.pdf` should be produced and contain the
same text “Hello *world*!” Since we used the contents of the frozen cache using
the `\markdownOptionFrozenCache` option, we were able to typeset the document
without accessing the shell or invoking Lua, but the change in the content of
the markdown document from “Hello *world*!” to “Hi *world*!” was not reflected.

##### \LaTeX{} Example {.unnumbered}

Using a text editor, create a text document named `document.tex` with the
following content:
``` tex
\documentclass[finalizecache]{article}
\usepackage{markdown}
\begin{document}
\begin{markdown}
Hello *world*!
\end{markdown}
\end{document}
```````
Next, invoke LuaTeX from the terminal:
``` sh
lualatex document.tex
``````
A PDF document named `document.pdf` should be produced and contain the text
“Hello *world*!” A frozen cache will also be produced as we requested using the
`finalizecache` option.

Next, change the content of `document.tex` as follows:
``` tex
\documentclass[frozencache]{article}
\usepackage{markdown}
\begin{document}
\begin{markdown}
Hi *world*!
\end{markdown}
\end{document}
```````
Last, invoke pdfTeX without shell access from the terminal:
``` sh
pdflatex -no-shell-escape document.tex
``````
A new PDF document named `document.pdf` should be produced and contain the
same text “Hello *world*!” Since we used the contents of the frozen cache using
the `\markdownOptionFrozenCache` option, we were able to typeset the document
without accessing the shell or invoking Lua, but the change in the content of
the markdown document from “Hello *world*!” to “Hi *world*!” was not reflected.

##### \Hologo{ConTeXt} Example {.unnumbered}

Using a text editor, create a text document named `document.tex` with the
following content:
``` tex
\usemodule[t][markdown]
\setupmarkdown[finalizeCache = yes]
\starttext
\startmarkdown
Hello *world*!
\stopmarkdown
\stoptext
```````
Next, invoke LuaTeX from the terminal:
``` sh
context --luatex document.tex
``````
A PDF document named `document.pdf` should be produced and contain the text
“Hello *world*!” A frozen cache will also be produced in the `cache.tex`
output file as we requested using the `finalizeCache` option.

Next, change the content of `document.tex` as follows:
``` tex
\usemodule[t][markdown]
\setupmarkdown[frozenCache = yes]
\starttext
\startmarkdown
Hi *world*!
\stopmarkdown
\stoptext
```````
Last, invoke LuaTeX from the terminal:
``` sh
context --luatex document.tex
``````
A new PDF document named `document.pdf` should be produced and contain the
same text “Hello *world*!” Since we used the contents of the frozen cache using
the `\markdownOptionFrozenCache` option, we were able to typeset the document
without accessing the shell or invoking Lua, but the change in the content of
the markdown document from “Hello *world*!” to “Hi *world*!” was not reflected.

%</manual-options>
%<*tex>
% \fi
%  \begin{macrocode}
\@@_add_lua_option:nnn
  { finalizeCache }
  { boolean }
  { false }
%    \end{macrocode}
% \iffalse
%</tex>
%<*lua,lua-cli>
% \fi
%  \begin{macrocode}
defaultOptions.finalizeCache = false
%    \end{macrocode}
% \par
% \iffalse
%</lua,lua-cli>
%<*manual-options>

#### Option `frozenCacheCounter`

`frozenCacheCounter` (default value: `0`)

% \fi
% \begin{markdown}
%
% \Valitem[0]{frozenCacheCounter}{number}
%
:    The number of the current markdown document that will be stored in
     an output file (frozen cache) when the \Opt{finalizeCache} is enabled.
     When the document number is 0, then a new frozen cache will be created.
     Otherwise, the frozen cache will be appended.

     Each frozen cache entry will define a \TeX{} macro
     `\markdownFrozenCache`\meta{number} that will typeset markdown document
     number \meta{number}.

% \end{markdown}
% \iffalse

##### Lua Module Example {.unnumbered}

Using a text editor, create a text document named `document.tex` with the
following content:
``` tex
\input markdown
\begingroup
\catcode`\%=12
\catcode`\#=12
\directlua{
  local markdown = require("markdown")
  local firstConvert = markdown.new({finalizeCache = true, frozenCacheCounter = 0})
  local firstInput = "Hello"
  local secondConvert = markdown.new({finalizeCache = true, frozenCacheCounter = 1})
  local secondInput = "*world*!"
  tex.sprint(firstConvert(firstInput) .. [[ ]] .. secondConvert(secondInput)) }
\endgroup
\bye
```````
Then, invoke LuaTeX from the terminal:
``` sh
luatex document.tex
```````
A PDF document named `document.pdf` should be produced and contain the text
“Hello *world*!” A frozen cache with two entries will also be produced as we
requested using the `frozenCacheCounter` option.

%</manual-options>
%<*tex>
% \fi
%  \begin{macrocode}
\@@_add_lua_option:nnn
  { frozenCacheCounter }
  { counter }
  { 0 }
%    \end{macrocode}
% \iffalse
%</tex>
%<*lua,lua-cli>
% \fi
%  \begin{macrocode}
defaultOptions.frozenCacheCounter = 0
%    \end{macrocode}
% \par
% \iffalse
%</lua,lua-cli>
%<*manual-options>

#### Option `gfmAutoIdentifiers`

`gfmAutoIdentifiers` (default value: `false`)

% \fi
% \begin{markdown}
% \Optitem[false]{gfmAutoIdentifiers}{\opt{true}, \opt{false}}
%
:    true

     :  Enable the Pandoc [GitHub-flavored auto identifiers syntax extension][pandoc-gfm-auto-identifiers]:

        ``` md
        The following heading received the identifier `123-sesame-street`:

        # 123 Sesame Street
        ``````

     false

     :  Disable the Pandoc GitHub-flavored auto identifiers syntax extension.

See also the option \Opt{autoIdentifiers}.

  [pandoc-gfm-auto-identifiers]: https://pandoc.org/MANUAL.html#extension-gfm_auto_identifiers

% \end{markdown}
% \iffalse
%</manual-options>
%<*tex>
% \fi
%  \begin{macrocode}
\@@_add_lua_option:nnn
  { gfmAutoIdentifiers }
  { boolean }
  { false }
%    \end{macrocode}
% \iffalse
%</tex>
%<*lua,lua-cli>
% \fi
%  \begin{macrocode}
defaultOptions.gfmAutoIdentifiers = false
%    \end{macrocode}
% \par
% \iffalse
%</lua,lua-cli>
%<*manual-options>

#### Option `hashEnumerators`

`hashEnumerators` (default value: `false`)

% \fi
% \begin{markdown}
%
% \Optitem[false]{hashEnumerators}{\opt{true}, \opt{false}}
%
:    true

     :  Enable the use of hash symbols (`#`) as ordered item list
        markers:

        ``` md
        #. Bird
        #. McHale
        #. Parish
        ``````

:    false

     :  Disable the use of hash symbols (`#`) as ordered item list
        markers.

% \end{markdown}
% \iffalse

##### \LaTeX{} Example {.unnumbered}

Using a text editor, create a text document named `document.tex` with the
following content:
``` tex
\documentclass{article}
\usepackage{markdown}
\begin{document}

\begin{markdown}
#. Bird
#. McHale
#. Parish
\end{markdown}

\begin{markdown}[hashEnumerators]
#. Bird
#. McHale
#. Parish
\end{markdown}

\end{document}
```````
Next, invoke LuaTeX from the terminal:
``` sh
lualatex document.tex
``````
A PDF document named `document.pdf` should be produced and contain the
following text:

> . Bird
> ========
> . McHale
> ========
> . Parish
> ========
>
> #. Bird
> #. McHale
> #. Parish

##### \Hologo{ConTeXt} Example {.unnumbered}

Using a text editor, create a text document named `document.tex` with the
following content:
``` tex
\usemodule[t][markdown]
\starttext

\startmarkdown
#. Bird
#. McHale
#. Parish
\stopmarkdown

\setupmarkdown[hashEnumerators = yes]
\startmarkdown
#. Bird
#. McHale
#. Parish
\stopmarkdown

\stoptext
````````
Next, invoke LuaTeX from the terminal:
``` sh
context --luatex document.tex
`````
A PDF document named `document.pdf` should be produced and contain the
following text:

> . Bird
> ========
> . McHale
> ========
> . Parish
> ========
>
> #. Bird
> #. McHale
> #. Parish

%</manual-options>
%<*tex>
% \fi
%  \begin{macrocode}
\@@_add_lua_option:nnn
  { hashEnumerators }
  { boolean }
  { false }
%    \end{macrocode}
% \iffalse
%</tex>
%<*lua,lua-cli>
% \fi
%  \begin{macrocode}
defaultOptions.hashEnumerators = false
%    \end{macrocode}
% \par
% \iffalse
%</lua,lua-cli>
%<*manual-options>

#### Option `headerAttributes` {#header-attributes}

`headerAttributes` (default value: `false`)

% \fi
% \begin{markdown}
%
% \Optitem[false]{headerAttributes}{\opt{true}, \opt{false}}
%
:    true

     :  Enable the assignment of HTML attributes to headings:

        ``` md
        # My first heading {#foo}

        ## My second heading ##    {#bar .baz}

        Yet another heading   {key=value}
        ===================
        ``````

:    false

     :  Disable the assignment of HTML attributes to headings.

% \end{markdown}
% \iffalse
%</manual-options>
%<*tex>
% \fi
%  \begin{macrocode}
\@@_add_lua_option:nnn
  { headerAttributes }
  { boolean }
  { false }
%    \end{macrocode}
% \iffalse
%</tex>
%<*lua,lua-cli>
% \fi
%  \begin{macrocode}
defaultOptions.headerAttributes = false
%    \end{macrocode}
% \par
% \iffalse
%</lua,lua-cli>
%<*manual-options>

#### Option `html`

`html` (default value: `true`)

% \fi
% \begin{markdown}
%
% \Optitem[true]{html}{\opt{true}, \opt{false}}
%
:    true

     :  Enable the recognition of inline HTML tags, block HTML elements,
        HTML comments, HTML instructions, and entities in the input. Inline
        HTML tags, block HTML elements and HTML comments will be rendered, HTML
        instructions will be ignored, and HTML entities will be replaced with
        the corresponding Unicode codepoints.

:    false

     :  Disable the recognition of HTML markup. Any HTML markup in the input
        will be rendered as plain text.

% \end{markdown}
% \iffalse

##### Lua Module Example {.unnumbered}

Using a text editor, create a text document named `document.tex` with the
following content:
``` tex
\input markdown
\input lmfonts
\begingroup
\catcode`\%=12
\catcode`\#=12
\directlua{
  local markdown = require("markdown")
  local convert = markdown.new({html = false})
  local newline = [[^^J^^J]]
  local input =
    "<div>*There is no block tag support.*</div>"        .. newline ..
    "*There is no <inline tag="tag"></inline> support.*" .. newline ..
    "_There is no <!-- comment --> support._"            .. newline ..
    "_There is no <? HTML instruction ?> support._"
  tex.sprint(convert(input)) }
\par
\directlua{
  local markdown = require("markdown")
  local convert = markdown.new()
  local input =
    "<div>*There is block tag support.*</div>"        .. newline ..
    "*There is <inline tag="tag"></inline> support.*" .. newline ..
    "_There is <!-- comment --> support._"            .. newline ..
    "_There is <? HTML instruction ?> support._"
  tex.sprint(convert(input)) }
\endgroup
\bye
```````
Then, invoke LuaTeX from the terminal:
``` sh
luatex document.tex
```````
A PDF document named `document.pdf` should be produced and contain the
following text:

> \<div>There is no block tag support.\</div>
> There is no \<inline tag=”tag”>\</inline> support.
> There is no \<!-- comment --> support.
> There is no <? HTML instruction ?> support.
>
> There is support. There is support. There is support.

##### Lua CLI Example {.unnumbered}

Using a text editor, create a text document named `document.tex` with the
following content:
``` tex
\input markdown
\input lmfonts
\begingroup
\catcode`\%=12
\catcode`\#=12
\input optionfalse
\par
\input optiontrue
\endgroup
\bye
```````
Using a text editor, create a text document named `content.md` with the
following content:
``` html
<div>
*Is there block tag support?*
</div>
*Is there <inline tag="tag"></inline> support?*
_Is there <!-- comment --> support?_
_Is there <? HTML instruction ?> support?_
````````
Next, invoke LuaTeX from the terminal:
``` sh
texlua ⟨CLI pathname⟩ html=false -- content.md optionfalse.tex
texlua ⟨CLI pathname⟩ -- content.md optiontrue.tex
luatex document.tex
```````
where \meta{CLI pathname} corresponds to the location of the Lua CLI script file,
such as `~/texmf/scripts/markdown/markdown-cli.lua` on UN\*X systems or
`C:\Users\`\meta{Your username}`\texmf\scripts\markdown\markdown-cli.lua` on Windows
systems. Use the command `kpsewhich -a markdown-cli.lua` to locate the Lua CLI
script file using [Kpathsea][].

A PDF document named `document.pdf` should be produced and contain the
following text:

> \<div>Is there block tag support?\</div>
> Is there \<inline tag=”tag”>\</inline> support?
> Is there \<!-- comment --> support?
> Is there <? HTML instruction ?> support?
>
> Is there support? Is there support? Is there support?

##### Plain \TeX{} Example {.unnumbered}

Using a text editor, create a text document named `document.tex` with the
following content:
``` tex
\input markdown
\input lmfonts

\def\markdownOptionHtml{false}
\markdownBegin
<div>
*There is no block tag support.*
</div>
*There is no <inline tag="tag"></inline> support.*
_There is no <!-- comment --> support._
_There is no <? HTML instruction ?> support._
\markdownEnd

\def\markdownOptionHtml{true}
\markdownBegin
<div>
*There is block tag support.*
</div>
*There is <inline tag="tag"></inline> support.*
_There is <!-- comment --> support._
_There is <? HTML instruction ?> support._
\markdownEnd

\bye
```````
Next, invoke LuaTeX from the terminal:
``` sh
luatex document.tex
``````
A PDF document named `document.pdf` should be produced and contain the
following text:

> \<div>There is no block tag support.\</div>
> There is no \<inline tag=”tag”>\</inline> support.
> There is no \<!-- comment --> support.
> There is no <? HTML instruction ?> support.
>
> There is support. There is support. There is support.

##### \LaTeX{} Example {.unnumbered}

Using a text editor, create a text document named `document.tex` with the
following content:
``` tex
\documentclass{article}
\usepackage{markdown}
\begin{document}

\begin{markdown}[html = false]
<div>
*There is no block tag support.*
</div>
*There is no <inline tag="tag"></inline> support.*
_There is no <!-- comment --> support._
_There is no <? HTML instruction ?> support._
\end{markdown}

\begin{markdown}
<div>
*There is block tag support.*
</div>
*There is <inline tag="tag"></inline> support.*
_There is <!-- comment --> support._
_There is <? HTML instruction ?> support._
\end{markdown}

\end{document}
```````
Next, invoke LuaTeX from the terminal:
``` sh
lualatex document.tex
``````
A PDF document named `document.pdf` should be produced and contain the
following text:

> \<div>There is no block tag support.\</div>
> There is no \<inline tag=”tag”>\</inline> support.
> There is no \<!-- comment --> support.
> There is no <? HTML instruction ?> support.
>
> There is support. There is support. There is support.

##### \Hologo{ConTeXt} Example {.unnumbered}

Using a text editor, create a text document named `document.tex` with the
following content:
``` tex
\usemodule[t][markdown]
\starttext

\setupmarkdown[html = no]
\startmarkdown
<div>
*There is no block tag support.*
</div>
*There is no <inline tag="tag"></inline> support.*
_There is no <!-- comment --> support._
_There is no <? HTML instruction ?> support._
\stopmarkdown

\setupmarkdown[html = yes]
\startmarkdown
<div>
*There is block tag support.*
</div>
*There is <inline tag="tag"></inline> support.*
_There is <!-- comment --> support._
_There is <? HTML instruction ?> support._
\stopmarkdown

\stoptext
````````
Next, invoke LuaTeX from the terminal:
``` sh
context --luatex document.tex
`````
A PDF document named `document.pdf` should be produced and contain the
following text:

> \<div>There is no block tag support.\</div>
> There is no \<inline tag=”tag”>\</inline> support.
> There is no \<!-- comment --> support.
> There is no <? HTML instruction ?> support.
>
> There is support. There is support. There is support.

%</manual-options>
%<*tex>
% \fi
%  \begin{macrocode}
\@@_add_lua_option:nnn
  { html }
  { boolean }
  { true }
%    \end{macrocode}
% \iffalse
%</tex>
%<*lua,lua-cli>
% \fi
%  \begin{macrocode}
defaultOptions.html = true
%    \end{macrocode}
% \par
% \iffalse
%</lua,lua-cli>
%<*manual-options>

#### Option `hybrid`

`hybrid` (default value: `false`)

% \fi
% \begin{markdown}
%
% \Optitem[false]{hybrid}{\opt{true}, \opt{false}}
%
:    true

     :  Disable the escaping of special plain \TeX{} characters, which makes it
        possible to intersperse your markdown markup with \TeX{} code.  The
        intended usage is in documents prepared manually by a human author.
        In such documents, it can often be desirable to mix \TeX{} and markdown
        markup freely.

:    false

     :  Enable the escaping of special plain \TeX{} characters outside verbatim
        environments, so that they are not interpreted by \TeX{}. This is
        encouraged when typesetting automatically generated content or
        markdown documents that were not prepared with this package in mind.

% \end{markdown}
% \iffalse

##### Lua Module Example {.unnumbered}

Using a text editor, create a text document named `document.tex` with the
following content:
``` tex
\input markdown
\input lmfonts
\begingroup
\catcode`\%=12
\catcode`\#=12
\directlua{
  local markdown = require("markdown")
  local input, convert_safe, convert_unsafe, paragraph

  input = [[$\string\sqrt{-1}$ *equals* $i$.]]
  convert_safe = markdown.new()
  convert_unsafe = markdown.new({hybrid = true})
  paragraph = [[\par]]

  tex.sprint(
    convert_safe(input) .. paragraph ..
    convert_unsafe(input)
  )
}
\endgroup
\bye
```````
Then, invoke LuaTeX from the terminal:
``` sh
luatex document.tex
```````
A PDF document named `document.pdf` should be produced and contain the
following text:

> \$\\sqrt {-1}\$ *equals* \$i\$.
>
> <math><msqrt><mo>−</mo><mn>1</mn></msqrt></math>
> *equals*
> <math><mi>i</mi></math>.

##### Lua CLI Example {.unnumbered}

Using a text editor, create a text document named `document.tex` with the
following content:
``` tex
\input markdown
\input lmfonts
\begingroup
\catcode`\%=12
\catcode`\#=12
\input optionfalse
\par
\input optiontrue
\endgroup
\bye
```````
Using a text editor, create a text document named `content.md` with the
following content:
``` md
$\sqrt{-1}$ *equals* $i$.
``````
Next, invoke LuaTeX from the terminal:
``` sh
texlua ⟨CLI pathname⟩ -- content.md optionfalse.tex
texlua ⟨CLI pathname⟩ hybrid=true -- content.md optiontrue.tex
luatex document.tex
```````
where \meta{CLI pathname} corresponds to the location of the Lua CLI script file,
such as `~/texmf/scripts/markdown/markdown-cli.lua` on UN\*X systems or
`C:\Users\`\meta{Your username}`\texmf\scripts\markdown\markdown-cli.lua` on Windows
systems. Use the command `kpsewhich -a markdown-cli.lua` to locate the Lua CLI
script file using [Kpathsea][].

A PDF document named `document.pdf` should be produced and contain the
following text:

> \$\\sqrt {-1}\$ *equals* \$i\$.
>
> <math><msqrt><mo>−</mo><mn>1</mn></msqrt></math>
> *equals*
> <math><mi>i</mi></math>.

##### Plain \TeX{} Example {.unnumbered}

Using a text editor, create a text document named `document.tex` with the
following content:
``` tex
\input markdown
\input lmfonts

\markdownBegin
$\sqrt{-1}$ *equals* $i$.
\markdownEnd

\def\markdownOptionHybrid{true}
\markdownBegin
$\sqrt{-1}$ *equals* $i$.
\markdownEnd

\bye
```````
Next, invoke LuaTeX from the terminal:
``` sh
luatex document.tex
``````
A PDF document named `document.pdf` should be produced and contain the
following text:

> \$\\sqrt {-1}\$ *equals* \$i\$.
>
> <math><msqrt><mo>−</mo><mn>1</mn></msqrt></math>
> *equals*
> <math><mi>i</mi></math>.

##### \LaTeX{} Example {.unnumbered}

Using a text editor, create a text document named `document.tex` with the
following content:
``` tex
\documentclass{article}
\usepackage{markdown}
\begin{document}

\begin{markdown}
$\sqrt{-1}$ *equals* $i$.
\end{markdown}

\begin{markdown}[hybrid]
$\sqrt{-1}$ *equals* $i$.
\end{markdown}

\end{document}
```````
Next, invoke LuaTeX from the terminal:
``` sh
lualatex document.tex
``````
A PDF document named `document.pdf` should be produced and contain the
following text:

> \$\\sqrt {-1}\$ *equals* \$i\$.
>
> <math><msqrt><mo>−</mo><mn>1</mn></msqrt></math>
> *equals*
> <math><mi>i</mi></math>.

##### \Hologo{ConTeXt} Example {.unnumbered}

Using a text editor, create a text document named `document.tex` with the
following content:
``` tex
\usemodule[t][markdown]
\starttext

\startmarkdown
$\sqrt{-1}$ *equals* $i$.
\stopmarkdown

\setupmarkdown[hybrid = yes]
\startmarkdown
$\sqrt{-1}$ *equals* $i$.
\stopmarkdown

\stoptext
````````
Next, invoke LuaTeX from the terminal:
``` sh
context --luatex document.tex
`````
A PDF document named `document.pdf` should be produced and contain the
following text:

> \$\\sqrt {-1}\$ *equals* \$i\$.
>
> <math><msqrt><mo>−</mo><mn>1</mn></msqrt></math>
> *equals*
> <math><mi>i</mi></math>.

%</manual-options>
%<*tex>
% \fi
%  \begin{macrocode}
\@@_add_lua_option:nnn
  { hybrid }
  { boolean }
  { false }
%    \end{macrocode}
% \iffalse
%</tex>
%<*lua,lua-cli>
% \fi
%  \begin{macrocode}
defaultOptions.hybrid = false
%    \end{macrocode}
% \par
% \iffalse
%</lua,lua-cli>
%<*manual-options>

#### Option `inlineCodeAttributes`

`inlineCodeAttributes` (default value: `false`)

% \fi
% \begin{markdown}
%
% \Optitem[false]{inlineCodeAttributes}{\opt{true}, \opt{false}}
%
:    true

     :  Enable the Pandoc [inline code span attribute extension][pandoc-inline-code-attributes]:

        ``` md
        `<$>`{.haskell}
        ``````

:    false

     :  Enable the Pandoc inline code span attribute extension.

 [pandoc-inline-code-attributes]: https://pandoc.org/MANUAL.html#extension-inline_code_attributes

% \end{markdown}
% \iffalse

##### \LaTeX{} Example {.unnumbered}

Using a text editor, create a text document named `document.tex` with the
following content:
```` tex
\documentclass{article}
\usepackage[inlineCodeAttributes]{markdown}
\usepackage{expl3}
\ExplSyntaxOn
\markdownSetup{
  renderers = {
    codeSpanAttributeContextBegin = {
      \group_begin:
      \color_group_begin:
      \markdownSetup{
        renderers = {
          attributeKeyValue = {
            \str_if_eq:nnT
              { ##1 }
              { color }
              {
                 \color_select:n { ##2 }
              }
          },
        },
      }
    },
    codeSpanAttributeContextEnd = {
      \color_group_end:
      \group_end:
    },
  },
}
\ExplSyntaxOff
\begin{document}
\begin{markdown}
Here is some `colored text`{color=red}.
\end{markdown}
\end{document}
```````
Next, invoke LuaTeX from the terminal:
``` sh
lualatex document.tex
``````
A PDF document named `document.pdf` should be produced and contain the
following text:

> Here is some <span style="color: red">`colored text`</span>.

%</manual-options>
%<*tex>
% \fi
%  \begin{macrocode}
\@@_add_lua_option:nnn
  { inlineCodeAttributes }
  { boolean }
  { false }
%    \end{macrocode}
% \iffalse
%</tex>
%<*lua,lua-cli>
% \fi
%  \begin{macrocode}
defaultOptions.inlineCodeAttributes = false
%    \end{macrocode}
% \par
% \iffalse
%</lua,lua-cli>
%<*manual-options>

#### Option `inlineNotes`

`inlineNotes` (default value: `false`)

% \fi
% \begin{markdown}
%
% \Optitem[false]{inlineNotes}{\opt{true}, \opt{false}}
%
:    true

     :  Enable the Pandoc [inline note syntax extension][pandoc-inline-notes]:

        ``` md
        Here is an inline note.^[Inlines notes are easier to
        write, since you don't have to pick an identifier and
        move down to type the note.]
        ``````

:    false

     :  Disable the Pandoc inline note syntax extension.

 [pandoc-inline-notes]: https://pandoc.org/MANUAL.html#extension-inline_notes

% \end{markdown}
% \iffalse

##### \LaTeX{} Example {.unnumbered}

Using a text editor, create a text document named `document.tex` with the
following content:
``` tex
\documentclass{article}
\usepackage[inlineNotes]{markdown}
\begin{document}
\begin{markdown}
Here is an inline note.^[Inlines notes are easier to
write, since you don't have to pick an identifier and
move down to type the note.]
\end{markdown}
\end{document}
```````
Next, invoke LuaTeX from the terminal:
``` sh
lualatex document.tex
``````
A PDF document named `document.pdf` should be produced and contain the
following text:

> Here is an inline note.^[Inlines notes are easier to
> write, since you don't have to pick an identifier and
> move down to type the note.]

##### \Hologo{ConTeXt} Example {.unnumbered}

Using a text editor, create a text document named `document.tex` with the
following content:
``` tex
\usemodule[t][markdown]
\setupmarkdown[inlineNotes = yes]
\starttext
\startmarkdown
Here is an inline note.^[Inlines notes are easier to
write, since you don't have to pick an identifier and
move down to type the note.]
\stopmarkdown
\stoptext
````````
Next, invoke LuaTeX from the terminal:
``` sh
context --luatex document.tex
`````
A PDF document named `document.pdf` should be produced and contain the
following text:

> Here is an inline note.^[Inlines notes are easier to
> write, since you don't have to pick an identifier and
> move down to type the note.]

%</manual-options>
%<*tex>
% \fi
%  \begin{macrocode}
\@@_add_lua_option:nnn
  { inlineNotes }
  { boolean }
  { false }
%    \end{macrocode}
% \iffalse
%</tex>
%<*lua,lua-cli>
% \fi
%  \begin{macrocode}
defaultOptions.inlineNotes = false
%    \end{macrocode}
% \par
% \iffalse
%</lua,lua-cli>
%<*manual-options>

#### Option `jekyllData`

`jekyllData` (default value: `false`)

% \fi
% \begin{markdown}
%
% \Optitem[false]{jekyllData}{\opt{true}, \opt{false}}
%
:    true

     :  Enable the Pandoc [\acro{yaml} metadata block syntax
        extension][pandoc-yaml-metadata-block] for entering
        metadata in \acro{yaml}:

        ~~~~~~ yaml
        ---
        title:  'This is the title: it contains a colon'
        author:
        - Author One
        - Author Two
        keywords: [nothing, nothingness]
        abstract: |
          This is the abstract.

          It consists of two paragraphs.
        ---
        ~~~~~~~~~~~

:    false

     :  Disable the Pandoc \acro{yaml} metadata block syntax extension
        for entering metadata in \acro{yaml}.

 [pandoc-yaml-metadata-block]: https://pandoc.org/MANUAL.html#extension-yaml_metadata_block

% \end{markdown}
% \iffalse

##### Plain \TeX{} Example {.unnumbered}

Using a text editor, create a text document named `document.tex` with the
following content:
``` tex
\input markdown
\def\markdownOptionJekyllData{true}
\ExplSyntaxOn
\keys_define:nn
  { markdown/jekyllData }
  {
    name .code:n = { \gdef\name{#1} },
    age  .code:n = { \gdef\age{#1}  },
  }
\ExplSyntaxOff
\def\markdownRendererJekyllDataEnd{%
  \name{} is \age{} years old.}
\markdownBegin
---
name: Jane Doe
age:  99
---
\markdownEnd
\bye
```````
Next, invoke LuaTeX from the terminal:
``` sh
luatex document.tex
``````
A PDF document named `document.pdf` should be produced and contain the
following text:

> Jane Doe is 99 years old.

##### \LaTeX{} Example {.unnumbered}

Using a text editor, create a text document named `document.tex` with the
following content:
``` tex
\documentclass{article}
\usepackage[jekyllData]{markdown}
\markdownSetup{
  jekyllDataRenderers = {
    name = {\gdef\name{#1}},
    code = {\gdef\age{#1}},
  },
  renderers = {
    jekyllDataEnd = {\name{} is \age{} years old.},
  }
}
\begin{document}

\begin{markdown}
---
name: Jane Doe
age:  99
---
\end{markdown}

\end{document}
```````
Next, invoke LuaTeX from the terminal:
``` sh
lualatex document.tex
``````
A PDF document named `document.pdf` should be produced and contain the
following text:

> Jane Doe is 99 years old.

##### \Hologo{ConTeXt} Example {.unnumbered}

Using a text editor, create a text document named `document.tex` with the
following content:
``` tex
\usemodule[t][markdown]
\setupmarkdown[jekyllData = yes]
\ExplSyntaxOn
\keys_define:nn
  { markdown/jekyllData }
  {
    name .code:n = { \gdef\name{#1} },
    age  .code:n = { \gdef\age{#1}  },
  }
\ExplSyntaxOff
\def\markdownRendererJekyllDataEnd{%
  \name{} is \age{} years old.}
\starttext

\startmarkdown
---
name: Jane Doe
age:  99
---
\stopmarkdown

\stoptext
```````
Next, invoke LuaTeX from the terminal:
``` sh
context --luatex document.tex
``````
A PDF document named `document.pdf` should be produced and contain the
following text:

> Jane Doe is 99 years old.

%</manual-options>
%<*tex>
% \fi
%  \begin{macrocode}
\@@_add_lua_option:nnn
  { jekyllData }
  { boolean }
  { false }
%    \end{macrocode}
% \iffalse
%</tex>
%<*lua,lua-cli>
% \fi
%  \begin{macrocode}
defaultOptions.jekyllData = false
%    \end{macrocode}
% \par
% \iffalse
%</lua,lua-cli>
%<*manual-options>

#### Option `linkAttributes`

`linkAttributes` (default value: `false`)

% \fi
% \begin{markdown}
%
% \Optitem[false]{linkAttributes}{\opt{true}, \opt{false}}
%
:    true

     :  Enable the Pandoc [link and image attribute syntax
        extension][pandoc-link-attributes]:

        ``` md
        An inline ![image](foo.jpg){#id .class width=30 height=20px}
        and a reference ![image][ref] with attributes.

        [ref]: foo.jpg "optional title" {#id .class key=val key2=val2}
        ``````

:    false

     :  Enable the Pandoc link and image attribute syntax extension.

 [pandoc-link-attributes]: https://pandoc.org/MANUAL.html#extension-link_attributes

% \end{markdown}
% \iffalse

##### \LaTeX{} Example {.unnumbered}

Using a text editor, create a text document named `document.tex` with the
following content:
```` tex
\documentclass{article}
\usepackage[linkAttributes]{markdown}
\usepackage{expl3, graphicx}
\ExplSyntaxOn
\markdownSetup{
  renderers = {
    imageAttributeContextBegin = {
      \group_begin:
      \markdownSetup{
        renderers = {
          attributeKeyValue = {
            \setkeys
              { Gin }
              { { ##1 } = { ##2 } }
          },
        },
      }
    },
    imageAttributeContextEnd = {
      \group_end:
    },
  },
}
\ExplSyntaxOff
\begin{document}
\begin{markdown}
Here is an example image:

 ![example image](example-image){width=5cm height=4cm}
\end{markdown}
\end{document}
```````
Next, invoke LuaTeX from the terminal:
``` sh
lualatex document.tex
``````
A PDF document named `document.pdf` should be produced and contain an example
image (from [Martin Scharrer's mwe package][mwe]) displayed at size 5cm × 4cm.

 [mwe]: https://ctan.org/pkg/mwe (mwe – Packages and image files for MWEs)

%</manual-options>
%<*tex>
% \fi
%  \begin{macrocode}
\@@_add_lua_option:nnn
  { linkAttributes }
  { boolean }
  { false }
%    \end{macrocode}
% \iffalse
%</tex>
%<*lua,lua-cli>
% \fi
%  \begin{macrocode}
defaultOptions.linkAttributes = false
%    \end{macrocode}
% \par
% \iffalse
%</lua,lua-cli>
%<*manual-options>

#### Option `lineBlocks`

`lineBlocks` (default value: `false`)

% \fi
% \begin{markdown}
%
% \Optitem[false]{lineBlocks}{\opt{true}, \opt{false}}
%
:    true

     :  Enable the Pandoc [line block syntax
        extension][pandoc-line-blocks]:

        ``` md
        | this is a line block that
        | spans multiple
        | even
          discontinuous
        | lines
        ``````

:    false

     :  Disable the Pandoc line block syntax extension.

 [pandoc-line-blocks]: https://pandoc.org/MANUAL.html#extension-line_blocks

% \end{markdown}
% \iffalse

##### Plain \TeX{} Example {.unnumbered}

Using a text editor, create a text document named `document.tex` with the
following content:
``` tex
\input markdown
\def\markdownOptionLineBlocks{true}
\markdownBegin
| I would spread the cloths under your feet:
| But I, being poor, have only my dreams;
| I have spread my dreams under your feet;
| Tread softly because you tread on my dreams.
\markdownEnd
\bye
```````
Next, invoke LuaTeX from the terminal:
``` sh
luatex document.tex
``````
A PDF document named `document.pdf` should be produced and contain the
following text:

> | I would spread the cloths under your feet:
> | But I, being poor, have only my dreams;
> | I have spread my dreams under your feet;
> | Tread softly because you tread on my dreams.

##### \LaTeX{} Example {.unnumbered}

Using a text editor, create a text document named `document.tex` with the
following content:
``` tex
\documentclass{article}
\usepackage[lineBlocks]{markdown}
\begin{document}
\begin{markdown}
| I would spread the cloths under your feet:
| But I, being poor, have only my dreams;
| I have spread my dreams under your feet;
| Tread softly because you tread on my dreams.
\end{markdown}
\end{document}
```````
Next, invoke LuaTeX from the terminal:
``` sh
lualatex document.tex
``````
A PDF document named `document.pdf` should be produced and contain the
following text:

> | I would spread the cloths under your feet:
> | But I, being poor, have only my dreams;
> | I have spread my dreams under your feet;
> | Tread softly because you tread on my dreams.

##### \Hologo{ConTeXt} Example {.unnumbered}

Using a text editor, create a text document named `document.tex` with the
following content:
``` tex
\usemodule[t][markdown]
\setupmarkdown[lineBlocks = yes]
\starttext
\startmarkdown
| I would spread the cloths under your feet:
| But I, being poor, have only my dreams;
| I have spread my dreams under your feet;
| Tread softly because you tread on my dreams.
\stopmarkdown
\stoptext
```````
Next, invoke LuaTeX from the terminal:
``` sh
context --luatex document.tex
``````
A PDF document named `document.pdf` should be produced and contain the
following text:

> | I would spread the cloths under your feet:
> | But I, being poor, have only my dreams;
> | I have spread my dreams under your feet;
> | Tread softly because you tread on my dreams.

%</manual-options>
%<*tex>
% \fi
%  \begin{macrocode}
\@@_add_lua_option:nnn
  { lineBlocks }
  { boolean }
  { false }
%    \end{macrocode}
% \iffalse
%</tex>
%<*lua,lua-cli>
% \fi
%  \begin{macrocode}
defaultOptions.lineBlocks = false
%    \end{macrocode}
% \par
% \iffalse
%</lua,lua-cli>
%<*manual-options>

#### Option `mark`

`mark` (default value: `false`)

% \fi
% \begin{markdown}
%
% \Optitem[false]{mark}{\opt{true}, \opt{false}}
%
:    true

     :  Enable the Pandoc [mark syntax extension][pandoc-mark]:

        ``` md
        This ==is highlighted text.==
        ``````

:    false

     :  Disable the Pandoc mark syntax extension.

 [pandoc-mark]: https://pandoc.org/MANUAL.html#extension-mark

% \end{markdown}
% \iffalse

##### \LaTeX{} Example {.unnumbered}

Using a text editor, create a text document named `document.tex` with the
following content:
``` tex
\documentclass{article}
\usepackage[mark]{markdown}
\begin{document}
\begin{markdown}

This ==is highlighted text.==

\end{markdown}
\end{document}
```````
Next, invoke LuaTeX from the terminal:
``` sh
lualatex document.tex
``````
A PDF document named `document.pdf` should be produced and contain the
following text:

> This <mark>is highlighted text.</mark>

%</manual-options>
%<*tex>
% \fi
%  \begin{macrocode}
\@@_add_lua_option:nnn
  { mark }
  { boolean }
  { false }
%    \end{macrocode}
% \iffalse
%</tex>
%<*lua,lua-cli>
% \fi
%  \begin{macrocode}
defaultOptions.mark = false
%    \end{macrocode}
% \par
% \iffalse
%</lua,lua-cli>
%<*manual-options>

#### Option `notes`

`notes` (default value: `false`)

% \fi
% \begin{markdown}
%
% \Optitem[false]{notes}{\opt{true}, \opt{false}}
%
:    true

     :  Enable the Pandoc [note syntax extension][pandoc-footnotes]:

        ``` md
        Here is a note reference,[^1] and another.[^longnote]

        [^1]: Here is the note.

        [^longnote]: Here's one with multiple blocks.

            Subsequent paragraphs are indented to show that they
        belong to the previous note.

                { some.code }

            The whole paragraph can be indented, or just the
            first line.  In this way, multi-paragraph notes
            work like multi-paragraph list items.

        This paragraph won't be part of the note, because it
        isn't indented.
        ``````

:    false

     :  Disable the Pandoc note syntax extension.

 [pandoc-footnotes]: https://pandoc.org/MANUAL.html#extension-footnotes

% \end{markdown}
% \iffalse

##### \LaTeX{} Example {.unnumbered}

Using a text editor, create a text document named `document.tex` with the
following content:
``` tex
\documentclass{article}
\usepackage[notes]{markdown}
\begin{document}
\begin{markdown}
Here is a note reference,[^1] and another.[^longnote]

[^1]: Here is the note.

[^longnote]: Here's one with multiple blocks.

    Subsequent paragraphs are indented to show that they
belong to the previous note.

        { some.code }

    The whole paragraph can be indented, or just the
    first line.  In this way, multi-paragraph notes
    work like multi-paragraph list items.

This paragraph won't be part of the note, because it
isn't indented.
\end{markdown}
\end{document}
```````
Next, invoke LuaTeX from the terminal:
``` sh
lualatex document.tex
``````
A PDF document named `document.pdf` should be produced and contain the
following text:

> Here is a note reference,[^1] and another.[^longnote]
>
>     Subsequent paragraphs are indented to show that they
> belong to the previous note.
>
>         { some.code }
>
>     The whole paragraph can be indented, or just the
>     first line.  In this way, multi-paragraph notes
>     work like multi-paragraph list items.
>
> This paragraph won't be part of the note, because it
> isn't indented.

##### \Hologo{ConTeXt} Example {.unnumbered}

Using a text editor, create a text document named `document.tex` with the
following content:
``` tex
\usemodule[t][markdown]
\setupmarkdown[notes = yes]
\starttext
\startmarkdown
Here is a note reference,[^1] and another.[^longnote]

[^1]: Here is the note.

[^longnote]: Here's one with multiple blocks.

    Subsequent paragraphs are indented to show that they
belong to the previous note.

        { some.code }

    The whole paragraph can be indented, or just the
    first line.  In this way, multi-paragraph notes
    work like multi-paragraph list items.

This paragraph won't be part of the note, because it
isn't indented.
\stopmarkdown
\stoptext
````````
Next, invoke LuaTeX from the terminal:
``` sh
context --luatex document.tex
`````
A PDF document named `document.pdf` should be produced and contain the
following text:

> Here is a note reference,[^1] and another.[^longnote]
>
>     Subsequent paragraphs are indented to show that they
> belong to the previous note.
>
>         { some.code }
>
>     The whole paragraph can be indented, or just the
>     first line.  In this way, multi-paragraph notes
>     work like multi-paragraph list items.
>
> This paragraph won't be part of the note, because it
> isn't indented.

%</manual-options>
%<*tex>
% \fi
%  \begin{macrocode}
\@@_add_lua_option:nnn
  { notes }
  { boolean }
  { false }
%    \end{macrocode}
% \iffalse
%</tex>
%<*lua,lua-cli>
% \fi
%  \begin{macrocode}
defaultOptions.notes = false
%    \end{macrocode}
% \par
% \iffalse
%</lua,lua-cli>
%<*manual-options>

#### Option `pipeTables` {#pipe-tables}

`pipeTables` (default value: `false`)

% \fi
% \begin{markdown}
%
% \Optitem[false]{pipeTables}{\opt{true}, \opt{false}}
%
:    true

     :  Enable the \acro{PHP} Markdown pipe table syntax extension:

        ``` md
        | Right | Left | Default | Center |
        |------:|:-----|---------|:------:|
        |   12  |  12  |    12   |    12  |
        |  123  |  123 |   123   |   123  |
        |    1  |    1 |     1   |     1  |
        ``````

:    false

     :  Disable the \acro{PHP} Markdown pipe table syntax extension.

% \end{markdown}
% \iffalse

##### \LaTeX{} Example {.unnumbered}

Using a text editor, create a text document named `document.tex` with the
following content:
``` tex
\documentclass{article}
\usepackage[pipeTables]{markdown}
\begin{document}
\begin{markdown}
| Right | Left | Default | Center |
|------:|:-----|---------|:------:|
|   12  |  12  |    12   |    12  |
|  123  |  123 |   123   |   123  |
|    1  |    1 |     1   |     1  |
\end{markdown}
\end{document}
```````
Next, invoke LuaTeX from the terminal:
``` sh
lualatex document.tex
``````
A PDF document named `document.pdf` should be produced and contain the
following text:

> | Right | Left | Default | Center |
> |------:|:-----|---------|:------:|
> |   12  |  12  |    12   |    12  |
> |  123  |  123 |   123   |   123  |
> |    1  |    1 |     1   |     1  |

##### \Hologo{ConTeXt} Example {.unnumbered}

Using a text editor, create a text document named `document.tex` with the
following content:
``` tex
\usemodule[t][markdown]
\setupmarkdown[pipeTables = yes]
\starttext
\startmarkdown
| Right | Left | Default | Center |
|------:|:-----|---------|:------:|
|   12  |  12  |    12   |    12  |
|  123  |  123 |   123   |   123  |
|    1  |    1 |     1   |     1  |
\stopmarkdown
\stoptext
````````
Next, invoke LuaTeX from the terminal:
``` sh
context --luatex document.tex
`````
A PDF document named `document.pdf` should be produced and contain the
following text:

> | Right | Left | Default | Center |
> |------:|:-----|---------|:------:|
> |   12  |  12  |    12   |    12  |
> |  123  |  123 |   123   |   123  |
> |    1  |    1 |     1   |     1  |

%</manual-options>
%<*tex>
% \fi
%  \begin{macrocode}
\@@_add_lua_option:nnn
  { pipeTables }
  { boolean }
  { false }
%    \end{macrocode}
% \iffalse
%</tex>
%<*lua,lua-cli>
% \fi
%  \begin{macrocode}
defaultOptions.pipeTables = false
%    \end{macrocode}
% \par
% \iffalse
%</lua,lua-cli>
%<*manual-options>

#### Option `preserveTabs`

`preserveTabs` (default value: `true`)

% \fi
% \begin{markdown}
%
% \Optitem[true]{preserveTabs}{\opt{true}, \opt{false}}
%
:    true

     :  Preserve tabs in code block and fenced code blocks.

:    false

     :  Convert any tabs in the input to spaces.

% \end{markdown}
% \iffalse
%</manual-options>
%<*tex>
% \fi
%  \begin{macrocode}
\@@_add_lua_option:nnn
  { preserveTabs }
  { boolean }
  { true }
%    \end{macrocode}
% \iffalse
%</tex>
%<*lua,lua-cli>
% \fi
%  \begin{macrocode}
defaultOptions.preserveTabs = true
%    \end{macrocode}
% \par
% \iffalse
%</lua,lua-cli>
%<*manual-options>

#### Option `rawAttribute`

`rawAttribute` (default value: `false`)

% \fi
% \begin{markdown}
%
% \Optitem[false]{rawAttribute}{\opt{true}, \opt{false}}
%
:    true

     :  Enable the Pandoc [raw attribute syntax
        extension][pandoc-raw-attribute]:

        ``` md
        `$H_2 O$`{=tex} is a liquid.
        ```

        To enable raw blocks, the \Opt{fencedCode} option must also
        be enabled:

        ~~~~~~~~ md
        Here is a mathematical formula:
        ``` {=tex}
        \[distance[i] =
            \begin{dcases}
                a & b \\
                c & d
            \end{dcases}
        \]
        ```
        ~~~~~~~~~~~

        The \Opt{rawAttribute} option is a good alternative to the \Opt{hybrid}
        option. Unlike the \Opt{hybrid} option, which affects the entire
        document, the \Opt{rawAttribute} option allows you to isolate the parts
        of your documents that use TeX:

:    false

     :  Disable the Pandoc raw attribute syntax extension.

 [pandoc-raw-attribute]: https://pandoc.org/MANUAL.html#extension-raw_attribute

% \end{markdown}
% \iffalse

##### \LaTeX{} Example {.unnumbered}

Using a text editor, create a text document named `document.tex` with the
following content:
```` tex
\documentclass{article}
\usepackage[rawAttribute, fencedCode]{markdown}
\usepackage{expl3}
\begin{document}
\begin{markdown}
`$H_2 O$`{=tex} is a liquid.

``` {=html}
<p>Here is some HTML content that will be ignored.</p>
```
\end{markdown}
\end{document}
````````
Next, invoke LuaTeX from the terminal:
``` sh
lualatex document.tex
``````
A PDF document named `document.pdf` should be produced and contain the
following text:

> H~2~O is a liquid.

%</manual-options>
%<*tex>
% \fi
%  \begin{macrocode}
\@@_add_lua_option:nnn
  { rawAttribute }
  { boolean }
  { false }
%    \end{macrocode}
% \iffalse
%</tex>
%<*lua,lua-cli>
% \fi
%  \begin{macrocode}
defaultOptions.rawAttribute = false
%    \end{macrocode}
% \par
% \iffalse
%</lua,lua-cli>
%<*manual-options>

#### Option `relativeReferences`

`relativeReferences` (default value: `false`)

% \fi
% \begin{markdown}
%
% \Optitem[false]{relativeReferences}{\opt{true}, \opt{false}}
%
:    true

     :  Enable [relative references][rfc3986] in autolinks:

        ``` md
        I conclude in Section <#conclusion>.

        Conclusion {#conclusion}
        ==========
        In this paper, we have discovered that most
        grandmas would rather eat dinner with their
        grandchildren than get eaten. Begone, wolf!
        ```

:    false

    :   Disable relative references in autolinks.

 [rfc3986]: https://datatracker.ietf.org/doc/html/rfc3986#section-4.2

% \end{markdown}
% \iffalse

##### \LaTeX{} Example {.unnumbered}

Using a text editor, create a text document named `document.tex` with the
following content:
``` tex
\documentclass{article}
\usepackage[headerAttributes, relativeReferences]{markdown}
\begin{document}
\begin{markdown}

I conclude in Section <#conclusion>.

Conclusion {#conclusion}
==========
In this paper, we have discovered that most
grandmas would rather eat dinner with their
grandchildren than get eaten. Begone, wolf!

\end{markdown}
\end{document}
```````
Next, invoke LuaTeX twice from the terminal:
``` sh
lualatex document.tex
lualatex document.tex
``````
A PDF document named `document.pdf` should be produced and contain the
following text:

> I conclude in Section 1.
>
> # 1. Conclusion
>
> In this paper, we have discovered that most grandmas would rather eat dinner
> with their grandchildren than get eaten. Begone, wolf!

%</manual-options>
%<*tex>
% \fi
%  \begin{macrocode}
\@@_add_lua_option:nnn
  { relativeReferences }
  { boolean }
  { false }
%    \end{macrocode}
% \iffalse
%</tex>
%<*lua,lua-cli>
% \fi
%  \begin{macrocode}
defaultOptions.relativeReferences = false
%    \end{macrocode}
% \par
% \iffalse
%</lua,lua-cli>
%<*manual-options>

#### Option `shiftHeadings`

`shiftHeadings` (default value: `0`)

% \fi
% \begin{markdown}
%
% \Valitem[0]{shiftHeadings}{shift amount}
%
:    All headings will be shifted by \meta{shift amount}, which can be both
     positive and negative. Headings will not be shifted beyond level 6 or
     below level 1. Instead, those headings will be shifted to level 6, when
     \meta{shift amount} is positive, and to level 1, when \meta{shift amount}
     is negative.

% \end{markdown}
% \iffalse

##### Plain \TeX{} Example {.unnumbered}

Using a text editor, create a text document named `example.md` with the
following content:

``` md
## A section
```

Using a text editor, create a text document named `document.tex` with the
following content:
``` tex
\input markdown

\font\normal=cmr10\normal
\font\big=cmr10 at 12pt
\def\markdownRendererHeadingTwo#1{{\big #1\par}}
\font\bigger=cmr10 scaled 1440
\def\markdownRendererHeadingOne#1{{\bigger #1\par}}

\def\markdownOptionShiftHeadings{-1}
\markdownInput{example.md}

\def\markdownOptionShiftHeadings{0}
\markdownInput{example.md}

\def\markdownOptionShiftHeadings{+1}
\markdownInput{example.md}

\bye
```````
Next, invoke LuaTeX from the terminal:
``` sh
luatex document.tex
``````
A PDF document named `document.pdf` should be produced and contain the
following text:

> # A section
> ## A section
> ### A section

##### \LaTeX{} Example {.unnumbered}

Using a text editor, create a text document named `document.tex` with the
following content:
``` tex
\documentclass{article}
\usepackage{markdown}
\begin{filecontents}[overwrite,nosearch,noheader]{example.md}
## A section
\end{filecontents}
\begin{document}
\markdownInput[shiftHeadings=-1]{example.md}
\markdownInput{example.md}
\markdownInput[shiftHeadings=+1]{example.md}
\end{document}
```````
Next, invoke LuaTeX from the terminal:
``` sh
lualatex document.tex
``````
A PDF document named `document.pdf` should be produced and contain the
following text:

> # A section
> ## A section
> ### A section

##### \Hologo{ConTeXt} Example {.unnumbered}

Using a text editor, create a text document named `example.md` with the
following content:

``` md
## A section
```

Using a text editor, create a text document named `document.tex` with the
following content:
``` tex
\usemodule[t][markdown]
\starttext

\inputmarkdown[shiftHeadings = -1]{example.md}
\inputmarkdown[shiftHeadings =  0]{example.md}
\inputmarkdown[shiftHeadings = +1]{example.md}

\stoptext
````````
Next, invoke LuaTeX from the terminal:
``` sh
context --luatex document.tex
`````
A PDF document named `document.pdf` should be produced and contain the
following text:

> # A section
> ## A section
> ### A section

%</manual-options>
%<*tex>
% \fi
%  \begin{macrocode}
\@@_add_lua_option:nnn
  { shiftHeadings }
  { number }
  { 0 }
%    \end{macrocode}
% \iffalse
%</tex>
%<*lua,lua-cli>
% \fi
%  \begin{macrocode}
defaultOptions.shiftHeadings = 0
%    \end{macrocode}
% \par
% \iffalse
%</lua,lua-cli>
%<*manual-options>

#### Option `slice`

`slice` (default value: `^ $`)

% \fi
% \begin{markdown}
%
% \Valitem[\textasciicircum{} \\$]{slice}{the beginning and the end of a slice}
%
:    Two space-separated selectors that specify the slice of a document that
     will be processed, whereas the remainder of the document will be ignored.
     The following selectors are recognized:

     - The circumflex (`^`) selects the beginning of a document.
     - The dollar sign (`$`) selects the end of a document.
     - `^`\meta{identifier} selects the beginning of
%      a section (see the \Opt{headerAttributes} option)
%      \iffalse
       a [section](#header-attributes)
%      \fi
%      or a fenced div (see the \Opt{fencedDivs} option) with the \acro{HTML}
%      attribute `#`\meta{identifier}.
%      \iffalse
       or a [fenced div](#fenced-divs)
       with the \acro{HTML} attribute `#`\meta{identifier}.
%      \fi
     - `$`\meta{identifier} selects the end of a section with the \acro{HTML}
       attribute `#`\meta{identifier}.
     - \meta{identifier} corresponds to `^`\meta{identifier} for the first
       selector and to `$`\meta{identifier} for the second selector.

     Specifying only a single selector, \meta{identifier}, is equivalent
     to specifying the two selectors \meta{identifier} \meta{identifier},
     which is equivalent to `^`\meta{identifier} `$`\meta{identifier}, i.e.
     the entire section with the \acro{HTML} attribute `#`\meta{identifier} will be
     selected.

% \end{markdown}
% \iffalse

##### Plain \TeX{} Example {.unnumbered}

Using a text editor, create a text document named `hamlet.md` with the
following content:

``` md
# The Tragedy of Hamlet
Shakespeare's longest play.

## Act III {#act-3}
Hamlet kills Polonius.

## Act V   {#act-5}
Hamlet dies.

## Act I   {#act-1}
Hamlet talks to ghost.
```

Using a text editor, create a text document named `document.tex` with the
following content:
``` tex
\input markdown
\def\markdownOptionHeaderAttributes{true}

\font\normal=cmr10\normal
\font\big=cmr10 at 12pt
\def\markdownRendererHeadingTwo#1{{\big #1\par}}
\font\bigger=cmr10 scaled 1440
\def\markdownRendererHeadingOne#1{{\bigger #1\par}}

\def\markdownOptionSlice{^ ^act-3}
\markdownInput{hamlet.md}

\def\markdownOptionSlice{act-1}
\markdownInput{hamlet.md}

\def\markdownOptionSlice{act-3 act-5}
\markdownInput{hamlet.md}

\bye
```````
Next, invoke LuaTeX from the terminal:
``` sh
luatex document.tex
``````
A PDF document named `document.pdf` should be produced and contain the
following text:

> # The Tragedy of Hamlet
> Shakespeare's longest play.
>
> ## Act I
> Hamlet talks to ghost.
>
> ## Act III
> Hamlet kills Polonius.
>
> ## Act V
> Hamlet dies.

##### \LaTeX{} Example {.unnumbered}

Using a text editor, create a text document named `document.tex` with the
following content:
``` tex
\documentclass{article}
\usepackage[headerAttributes]{markdown}
\begin{filecontents}[overwrite,nosearch,noheader]{hamlet.md}
# The Tragedy of Hamlet
Shakespeare's longest play.

## Act III {#act-3}
Hamlet kills Polonius.

## Act V   {#act-5}
Hamlet dies.

## Act I   {#act-1}
Hamlet talks to ghost.
\end{filecontents}
\begin{document}
\markdownInput[slice=^ ^act-3]{hamlet.md}
\markdownInput[slice=act-1]{hamlet.md}
\markdownInput[slice=act-3 act-5]{hamlet.md}
\end{document}
```````
Next, invoke LuaTeX from the terminal:
``` sh
lualatex document.tex
``````
A PDF document named `document.pdf` should be produced and contain the
following text:

> # The Tragedy of Hamlet
> Shakespeare's longest play.
>
> ## Act I
> Hamlet talks to ghost.
>
> ## Act III
> Hamlet kills Polonius.
>
> ## Act V
> Hamlet dies.

##### \Hologo{ConTeXt} Example {.unnumbered}

Using a text editor, create a text document named `hamlet.md` with the
following content:

``` md
# The Tragedy of Hamlet
Shakespeare's longest play.

## Act III {#act-3}
Hamlet kills Polonius.

## Act V   {#act-5}
Hamlet dies.

## Act I   {#act-1}
Hamlet talks to ghost.
```

Using a text editor, create a text document named `document.tex` with the
following content:
``` tex
\usemodule[t][markdown]
\setupmarkdown[headerAttributes = yes]
\starttext

\inputmarkdown[slice = ^ ^act-3]{example.md}
\inputmarkdown[slice = act-1]{example.md}
\inputmarkdown[slice = act-3 act-5]{example.md}

\stoptext
````````
Next, invoke LuaTeX from the terminal:
``` sh
context --luatex document.tex
`````
A PDF document named `document.pdf` should be produced and contain the
following text:

> # The Tragedy of Hamlet
> Shakespeare's longest play.
>
> ## Act I
> Hamlet talks to ghost.
>
> ## Act III
> Hamlet kills Polonius.
>
> ## Act V
> Hamlet dies.

%</manual-options>
%<*tex>
% \fi
%  \begin{macrocode}
\@@_add_lua_option:nnn
  { slice }
  { slice }
  { ^~$ }
%    \end{macrocode}
% \iffalse
%</tex>
%<*lua,lua-cli>
% \fi
%  \begin{macrocode}
defaultOptions.slice = "^ $"
%    \end{macrocode}
% \par
% \iffalse
%</lua,lua-cli>
%<*manual-options>

#### Option `smartEllipses`

`smartEllipses` (default value: `false`)

% \fi
% \begin{markdown}
%
% \Optitem[false]{smartEllipses}{\opt{true}, \opt{false}}
%
:    true

     :   Convert any ellipses in the input to the
         \mref{markdownRendererEllipsis} \TeX{} macro.

:    false

     :  Preserve all ellipses in the input.

% \end{markdown}
% \iffalse

##### Lua Module Example {.unnumbered}

Using a text editor, create a text document named `document.tex` with the
following content:
``` tex
\input markdown
\input lmfonts
\def\markdownRendererEllipsis{. . .}
\begingroup
\catcode`\%=12
\catcode`\#=12
\directlua{
  local markdown = require("markdown")
  local convert = markdown.new()
  local input = "These are just three regular dots ..."
  tex.sprint(convert(input)) }
\par
\directlua{
  local markdown = require("markdown")
  local convert = markdown.new({smartEllipses = true})
  local input = "... and this is a victorian ellipsis."
  tex.sprint(convert(input)) }
\endgroup
\bye
```````
Then, invoke LuaTeX from the terminal:
``` sh
luatex document.tex
```````
A PDF document named `document.pdf` should be produced and contain the
following text:

> These are just three regular dots ...
>
> . . . and this is a victorian ellipsis.

##### Lua CLI Example {.unnumbered}

Using a text editor, create a text document named `document.tex` with the
following content:
``` tex
\input markdown
\input lmfonts
\def\markdownRendererEllipsis{. . .}
\begingroup
\catcode`\%=12
\catcode`\#=12
\input optionfalse
\par
\input optiontrue
\endgroup
\bye
```````
Using a text editor, create a text document named `content.md` with the
following content:
``` md
Are these just three regular dots, a victorian ellipsis, or ... ?
``````
Next, invoke LuaTeX from the terminal:
``` sh
texlua ⟨CLI pathname⟩ -- content.md optionfalse.tex
texlua ⟨CLI pathname⟩ smartEllipses=true -- content.md optiontrue.tex
luatex document.tex
```````
where \meta{CLI pathname} corresponds to the location of the Lua CLI script file,
such as `~/texmf/scripts/markdown/markdown-cli.lua` on UN\*X systems or
`C:\Users\`\meta{Your username}`\texmf\scripts\markdown\markdown-cli.lua` on Windows
systems. Use the command `kpsewhich -a markdown-cli.lua` to locate the Lua CLI
script file using [Kpathsea][].

A PDF document named `document.pdf` should be produced and contain the
following text:

> Are these just three regular dots, a victorian ellipsis, or ... ?
>
> Are these just three regular dots, a victorian ellipsis, or . . . ?

##### Plain \TeX{} Example {.unnumbered}

Using a text editor, create a text document named `document.tex` with the
following content:
``` tex
\input markdown
\def\markdownRendererEllipsis{. . .}

\markdownBegin
These are just three regular dots ...
\markdownEnd

\def\markdownOptionSmartEllipses{true}
\markdownBegin
... and this is a victorian ellipsis.
\markdownEnd

\bye
```````
Next, invoke LuaTeX from the terminal:
``` sh
luatex document.tex
``````
A PDF document named `document.pdf` should be produced and contain the
following text:

> These are just three regular dots ...
>
> . . . and this is a victorian ellipsis.

##### \LaTeX{} Example {.unnumbered}

Using a text editor, create a text document named `document.tex` with the
following content:
``` tex
\documentclass{article}
\usepackage{markdown}
\markdownSetup{
  renderers = {
    ellipsis = {. . .}
  }
}
\begin{document}

\begin{markdown}
These are just three regular dots ...
\end{markdown}

\begin{markdown}[smartEllipses]
... and this is a victorian ellipsis.
\end{markdown}

\end{document}
```````
Next, invoke LuaTeX from the terminal:
``` sh
lualatex document.tex
``````
A PDF document named `document.pdf` should be produced and contain the
following text:

> These are just three regular dots ...
>
> . . . and this is a victorian ellipsis.

##### \Hologo{ConTeXt} Example {.unnumbered}

Using a text editor, create a text document named `document.tex` with the
following content:
``` tex
\usemodule[t][markdown]
\def\markdownRendererEllipsis{. . .}
\starttext

\startmarkdown
These are just three regular dots ...
\stopmarkdown

\setupmarkdown[smartEllipses = yes]
\startmarkdown
... and this is a victorian ellipsis.
\stopmarkdown

\stoptext
````````
Next, invoke LuaTeX from the terminal:
``` sh
context --luatex document.tex
`````
A PDF document named `document.pdf` should be produced and contain the
following text:

> These are just three regular dots ...
>
> . . . and this is a victorian ellipsis.

%</manual-options>
%<*tex>
% \fi
%  \begin{macrocode}
\@@_add_lua_option:nnn
  { smartEllipses }
  { boolean }
  { false }
%    \end{macrocode}
% \iffalse
%</tex>
%<*lua,lua-cli>
% \fi
%  \begin{macrocode}
defaultOptions.smartEllipses = false
%    \end{macrocode}
% \par
% \iffalse
%</lua,lua-cli>
%<*manual-options>

#### Option `startNumber`

`startNumber` (default value: `true`)

% \fi
% \begin{markdown}
%
% \Optitem[true]{startNumber}{\opt{true}, \opt{false}}
%
:    true

     :   Make the number in the first item of an ordered lists significant. The
         item numbers will be passed to the
         \mref{markdownRendererOlItemWithNumber} \TeX{} macro.

:    false

     :   Ignore the numbers in the ordered list items. Each item will only
         produce a \mref{markdownRendererOlItem} \TeX{} macro.

% \end{markdown}
% \iffalse

##### \LaTeX{} Example {.unnumbered}

Using a text editor, create a text document named `document.tex` with the
following content:
``` tex
\documentclass{article}
\usepackage{markdown}
\begin{document}

\begin{markdown}
The following list respects the numbers specified in the markup:

3. third item
4. fourth item
5. fifth item
\end{markdown}

\begin{markdown}[startNumber=false]
The following list does not respect the numbers specified in the
markup:

3. third item
4. fourth item
5. fifth item
\end{markdown}

\end{document}
```````
Next, invoke LuaTeX from the terminal:
``` sh
lualatex document.tex
``````
A PDF document named `document.pdf` should be produced and contain the
following text:

> The following list respects the numbers specified in the markup:
>
> 3. third item
> 4. fourth item
> 5. fifth item
>
> The following list does not respect the numbers specified in the markup:
>
> 1. third item
> 2. fourth item
> 3. fifth item

##### \Hologo{ConTeXt} Example {.unnumbered}

Using a text editor, create a text document named `document.tex` with the
following content:
``` tex
\usemodule[t][markdown]
\starttext

\startmarkdown
The following list respects the numbers specified in the markup:

3. third item
4. fourth item
5. fifth item
\stopmarkdown

\setupmarkdown[startNumber = no]
\startmarkdown
The following list respects the numbers specified in the markup:

3. third item
4. fourth item
5. fifth item
\stopmarkdown
\stoptext
````````
Next, invoke LuaTeX from the terminal:
``` sh
context --luatex document.tex
`````
A PDF document named `document.pdf` should be produced and contain the
following text:

> The following list respects the numbers specified in the markup:
>
> 3. third item
> 4. fourth item
> 5. fifth item
>
> The following list does not respect the numbers specified in the markup:
>
> 1. third item
> 2. fourth item
> 3. fifth item

%</manual-options>
%<*tex>
% \fi
%  \begin{macrocode}
\@@_add_lua_option:nnn
  { startNumber }
  { boolean }
  { true }
%    \end{macrocode}
% \iffalse
%</tex>
%<*lua,lua-cli>
% \fi
%  \begin{macrocode}
defaultOptions.startNumber = true
%    \end{macrocode}
% \par
% \iffalse
%</lua,lua-cli>
%<*manual-options>

#### Option `strikeThrough`

`strikeThrough` (default value: `false`)

% \fi
% \begin{markdown}
%
% \Optitem[false]{strikeThrough}{\opt{true}, \opt{false}}
%
:    true

     :  Enable the Pandoc [strike-through syntax
        extension][pandoc-strikeout]:

        ``` md
        This ~~is deleted text.~~
        ``````

:    false

     :  Disable the Pandoc strike-through syntax extension.

 [pandoc-strikeout]: https://pandoc.org/MANUAL.html#extension-strikeout

% \end{markdown}
% \iffalse

##### Plain \TeX{} Example {.unnumbered}

Using a text editor, create a text document named `document.tex` with the
following content:
``` tex
\input markdown
\def\markdownOptionStrikeThrough{true}
\input soulutf8.sty
\def\markdownRendererStrikeThrough#1{\st{#1}}
\markdownBegin
This is ~~a lunar roving vehicle~~ strike-through text.
\markdownEnd
\bye
```````
Next, invoke LuaTeX from the terminal:
``` sh
luatex document.tex
``````
A PDF document named `document.pdf` should be produced and contain the
following text:

> This is ~~a lunar roving vehicle~~ strike-through text.

##### \LaTeX{} Example {.unnumbered}

Using a text editor, create a text document named `document.tex` with the
following content:
``` tex
\documentclass{article}
\usepackage[strikeThrough]{markdown}
\usepackage{soulutf8}
\markdownSetup{
  renderers = {
    strikeThrough = {\st{#1}},
  },
}
\begin{document}
\begin{markdown}
This is ~~a lunar roving vehicle~~ strike-through text.
\end{markdown}
\end{document}
```````
Next, invoke LuaTeX from the terminal:
``` sh
lualatex document.tex
``````
A PDF document named `document.pdf` should be produced and contain the
following text:

> This is ~~a lunar roving vehicle~~ strike-through text.

##### \Hologo{ConTeXt} Example {.unnumbered}

Using a text editor, create a text document named `document.tex` with the
following content:
``` tex
\usemodule[t][markdown]
\setupmarkdown[strikeThrough = yes]
\def\markdownRendererStrikeThrough#1{\overstrikes{#1}}
\starttext
\startmarkdown
This is ~~a lunar roving vehicle~~ strike-through text.
\stopmarkdown
\stoptext
```````
Next, invoke LuaTeX from the terminal:
``` sh
context --luatex document.tex
``````
A PDF document named `document.pdf` should be produced and contain the
following text:

> This is ~~a lunar roving vehicle~~ strike-through text.

%</manual-options>
%<*tex>
% \fi
%  \begin{macrocode}
\@@_add_lua_option:nnn
  { strikeThrough }
  { boolean }
  { false }
%    \end{macrocode}
% \iffalse
%</tex>
%<*lua,lua-cli>
% \fi
%  \begin{macrocode}
defaultOptions.strikeThrough = false
%    \end{macrocode}
% \par
% \iffalse
%</lua,lua-cli>
%<*manual-options>

#### Option `stripIndent`

`stripIndent` (default value: `false`)

% \fi
% \markdownBegin
%
% \Optitem[false]{stripIndent}{\opt{true}, \opt{false}}
%
:    true

     :   Strip the minimal indentation of non-blank lines from all
         lines in a markdown document. Requires that the
         \Opt{preserveTabs} Lua option is disabled:

         ``` tex
         \documentclass{article}
         \usepackage[stripIndent]{markdown}
         \begin{document}
             \begin{markdown}
                 Hello *world*!
             \end{markdown}
         \end{document}
         ```````

:    false

     :   Do not strip any indentation from the lines in a markdown
         document.

% \markdownEnd
% \iffalse

##### Plain \TeX{} Example {.unnumbered}

Using a text editor, create a text document named `document.tex` with the
following content:
``` tex
\input markdown
\def\markdownOptionStripIndent{true}
\markdownBegin
    Hello *world*!
\markdownEnd
\bye
```````
Next, invoke LuaTeX from the terminal:
``` sh
luatex document.tex
``````
A PDF document named `document.pdf` should be produced and contain the text
“Hello *world*!”

##### \LaTeX{} Example {.unnumbered}

Using a text editor, create a text document named `document.tex` with the
following content:
``` tex
\documentclass{article}
\usepackage[stripIndent]{markdown}
\begin{document}
    \begin{markdown}
        Hello *world*!
    \end{markdown}
\end{document}
```````
Next, invoke LuaTeX from the terminal:
``` sh
lualatex document.tex
``````
A PDF document named `document.pdf` should be produced and contain the
text “Hello *world*!”

##### \Hologo{ConTeXt} Example {.unnumbered}

Using a text editor, create a text document named `document.tex` with the
following content:
``` tex
\usemodule[t][markdown]
\setupmarkdown[stripIndent = yes]
\starttext
    \startmarkdown
        Hello *world*!
    \stopmarkdown
\stoptext
````````
Next, invoke LuaTeX from the terminal:
``` sh
context --luatex document.tex
`````
A PDF document named `document.pdf` should be produced and contain the
text “Hello *world*!”

%</manual-options>
%<*tex>
% \fi
%  \begin{macrocode}
\@@_add_lua_option:nnn
  { stripIndent }
  { boolean }
  { false }
%    \end{macrocode}
% \iffalse
%</tex>
%<*lua,lua-cli>
% \fi
%  \begin{macrocode}
defaultOptions.stripIndent = false
%    \end{macrocode}
% \par
% \iffalse
%</lua,lua-cli>
%<*manual-options>

#### Option `subscripts`

`subscripts` (default value: `false`)

% \fi
% \begin{markdown}
%
% \Optitem[false]{subscripts}{\opt{true}, \opt{false}}
%
:    true

     :  Enable the Pandoc [subscript syntax
        extension][pandoc-subscript]:

        ``` md
        H~2~O is a liquid.
        ``````

:    false

     :  Disable the Pandoc subscript syntax extension.

 [pandoc-subscript]: https://pandoc.org/MANUAL.html#extension-superscript-subscript

% \end{markdown}
% \iffalse

##### \LaTeX{} Example {.unnumbered}

Using a text editor, create a text document named `document.tex` with the
following content:
``` tex
\documentclass{article}
\usepackage[subscripts]{markdown}
\begin{document}
\begin{markdown}
H~2~O is a liquid.
\end{markdown}
\end{document}
```````
Next, invoke LuaTeX from the terminal:
``` sh
lualatex document.tex
``````
A PDF document named `document.pdf` should be produced and contain the
following text:

> H~2~O is a liquid.

##### \Hologo{ConTeXt} Example {.unnumbered}

Using a text editor, create a text document named `document.tex` with the
following content:
``` tex
\usemodule[t][markdown]
\setupmarkdown[subscripts = yes]
\starttext
\startmarkdown
H~2~O is a liquid.
\stopmarkdown
\stoptext
```````
Next, invoke LuaTeX from the terminal:
``` sh
context --luatex document.tex
``````
A PDF document named `document.pdf` should be produced and contain the
following text:

> H~2~O is a liquid.

%</manual-options>
%<*tex>
% \fi
%  \begin{macrocode}
\@@_add_lua_option:nnn
  { subscripts }
  { boolean }
  { false }
%    \end{macrocode}
% \iffalse
%</tex>
%<*lua,lua-cli>
% \fi
%  \begin{macrocode}
defaultOptions.subscripts = false
%    \end{macrocode}
% \par
% \iffalse
%</lua,lua-cli>
%<*manual-options>

#### Option `superscripts`

`superscripts` (default value: `false`)

% \fi
% \begin{markdown}
%
% \Optitem[false]{superscripts}{\opt{true}, \opt{false}}
%
:    true

     :  Enable the Pandoc [superscript syntax
        extension][pandoc-superscript]:

        ``` md
        2^10^ is 1024.
        ``````

:    false

     :  Disable the Pandoc superscript syntax extension.

 [pandoc-superscript]: https://pandoc.org/MANUAL.html#extension-superscript-subscript

% \end{markdown}
% \iffalse

##### \LaTeX{} Example {.unnumbered}

Using a text editor, create a text document named `document.tex` with the
following content:
``` tex
\documentclass{article}
\usepackage[superscripts]{markdown}
\begin{document}
\begin{markdown}
2^10^ is 1024.
\end{markdown}
\end{document}
```````
Next, invoke LuaTeX from the terminal:
``` sh
lualatex document.tex
``````
A PDF document named `document.pdf` should be produced and contain the
following text:

> 2^10^ is 1024.

##### \Hologo{ConTeXt} Example {.unnumbered}

Using a text editor, create a text document named `document.tex` with the
following content:
``` tex
\usemodule[t][markdown]
\setupmarkdown[superscripts = yes]
\starttext
\startmarkdown
2^10^ is 1024.
\stopmarkdown
\stoptext
```````
Next, invoke LuaTeX from the terminal:
``` sh
context --luatex document.tex
``````
A PDF document named `document.pdf` should be produced and contain the
following text:

> 2^10^ is 1024.

%</manual-options>
%<*tex>
% \fi
%  \begin{macrocode}
\@@_add_lua_option:nnn
  { superscripts }
  { boolean }
  { false }
%    \end{macrocode}
% \iffalse
%</tex>
%<*lua,lua-cli>
% \fi
%  \begin{macrocode}
defaultOptions.superscripts = false
%    \end{macrocode}
% \par
% \iffalse
%</lua,lua-cli>
%<*manual-options>

#### Option `tableAttributes`

`tableAttributes` (default value: `false`)

% \fi
% \begin{markdown}
%
% \Optitem[false]{tableAttributes}{\opt{true}, \opt{false}}
%
:    true

     :  Enable the assignment of HTML attributes to
%       table captions (see the \Opt{tableCaptions} option).
%       \iffalse
        [table captions](#table-captions).
%       \fi

        ``` md
        | Right | Left | Default | Center |
        |------:|:-----|---------|:------:|
        |   12  |  12  |    12   |    12  |
        |  123  |  123 |   123   |   123  |
        |    1  |    1 |     1   |     1  |

          : Demonstration of pipe table syntax. {#example-table}
        ```

:    false

     :  Disable the assignment of HTML attributes to table captions.

% \end{markdown}
% \iffalse

##### \LaTeX{} Example {.unnumbered}

Using a text editor, create a text document named `document.tex` with the
following content:
``` tex
\documentclass{article}
\usepackage[
  pipeTables,
  tableCaptions,
  tableAttributes,
  relativeReferences,
]{markdown}
\usepackage{expl3}
\ExplSyntaxOn
\markdownSetup{
  renderers = {
    tableAttributeContextBegin = {
      \group_begin:
      \markdownSetup{
        renderers = {
          attributeIdentifier = {
            \markdownSetup{
              renderers = {
                tableAttributeContextEnd = {
                  \label{##1}
                  \group_end:
                },
              },
            }
          },
        },
      }
    },
    tableAttributeContextEnd = {
      \group_end:
    },
  },
}
\ExplSyntaxOff
\begin{document}
\begin{markdown}
See Table <#example-table>.

| Right | Left | Default | Center |
|------:|:-----|---------|:------:|
|   12  |  12  |    12   |    12  |
|  123  |  123 |   123   |   123  |
|    1  |    1 |     1   |     1  |

  : Demonstration of pipe table syntax. {#example-table}
\end{markdown}
\end{document}
```````
Next, invoke LuaTeX from the terminal:
``` sh
lualatex document.tex
``````
A PDF document named `document.pdf` should be produced and contain the
following text:

> See Table 1.
>
> | Right | Left | Default | Center |
> |------:|:-----|---------|:------:|
> |   12  |  12  |    12   |    12  |
> |  123  |  123 |   123   |   123  |
> |    1  |    1 |     1   |     1  |
>
> : Table 1. Demonstration of pipe table syntax.

%</manual-options>
%<*tex>
% \fi
%  \begin{macrocode}
\@@_add_lua_option:nnn
  { tableAttributes }
  { boolean }
  { false }
%    \end{macrocode}
% \iffalse
%</tex>
%<*lua,lua-cli>
% \fi
%  \begin{macrocode}
defaultOptions.tableAttributes = false
%    \end{macrocode}
% \par
% \iffalse
%</lua,lua-cli>
%<*manual-options>

#### Option `tableCaptions` {#table-captions}

`tableCaptions` (default value: `false`)

% \fi
% \begin{markdown}
%
% \Optitem[false]{tableCaptions}{\opt{true}, \opt{false}}
%
:    true

     :  Enable the Pandoc [table caption syntax
        extension][pandoc-table-captions] for
%       pipe tables (see the \Opt{pipeTables} option).
%       \iffalse
        [pipe tables](#pipe-tables).
%       \fi

        ``` md
        | Right | Left | Default | Center |
        |------:|:-----|---------|:------:|
        |   12  |  12  |    12   |    12  |
        |  123  |  123 |   123   |   123  |
        |    1  |    1 |     1   |     1  |

          : Demonstration of pipe table syntax.
        ``````

:    false

     :  Disable the Pandoc table caption syntax extension.

 [pandoc-table-captions]: https://pandoc.org/MANUAL.html#extension-table_captions

% \end{markdown}
% \iffalse

##### \LaTeX{} Example {.unnumbered}

Using a text editor, create a text document named `document.tex` with the
following content:
``` tex
\documentclass{article}
\usepackage[pipeTables, tableCaptions]{markdown}
\begin{document}
\begin{markdown}
| Right | Left | Default | Center |
|------:|:-----|---------|:------:|
|   12  |  12  |    12   |    12  |
|  123  |  123 |   123   |   123  |
|    1  |    1 |     1   |     1  |

: Demonstration of pipe table syntax.
\end{markdown}
\end{document}
```````
Next, invoke LuaTeX from the terminal:
``` sh
lualatex document.tex
``````
A PDF document named `document.pdf` should be produced and contain the
following text:

> | Right | Left | Default | Center |
> |------:|:-----|---------|:------:|
> |   12  |  12  |    12   |    12  |
> |  123  |  123 |   123   |   123  |
> |    1  |    1 |     1   |     1  |
>
> : Demonstration of pipe table syntax.

##### \Hologo{ConTeXt} Example {.unnumbered}

Using a text editor, create a text document named `document.tex` with the
following content:
``` tex
\usemodule[t][markdown]
\setupmarkdown
  [
    pipeTables = yes,
    tableCaptions = yes,
  ]
\starttext
\startmarkdown
| Right | Left | Default | Center |
|------:|:-----|---------|:------:|
|   12  |  12  |    12   |    12  |
|  123  |  123 |   123   |   123  |
|    1  |    1 |     1   |     1  |

: Demonstration of pipe table syntax.
\stopmarkdown
\stoptext
````````
Next, invoke LuaTeX from the terminal:
``` sh
context --luatex document.tex
`````
A PDF document named `document.pdf` should be produced and contain the
following text:

> | Right | Left | Default | Center |
> |------:|:-----|---------|:------:|
> |   12  |  12  |    12   |    12  |
> |  123  |  123 |   123   |   123  |
> |    1  |    1 |     1   |     1  |
>
> : Demonstration of pipe table syntax.

%</manual-options>
%<*tex>
% \fi
%  \begin{macrocode}
\@@_add_lua_option:nnn
  { tableCaptions }
  { boolean }
  { false }
%    \end{macrocode}
% \iffalse
%</tex>
%<*lua,lua-cli>
% \fi
%  \begin{macrocode}
defaultOptions.tableCaptions = false
%    \end{macrocode}
% \par
% \iffalse
%</lua,lua-cli>
%<*manual-options>

#### Option `taskLists`

`taskLists` (default value: `false`)

% \fi
% \begin{markdown}
%
% \Optitem[false]{taskLists}{\opt{true}, \opt{false}}
%
:    true

     :  Enable the Pandoc [task list syntax
        extension][pandoc-task-lists]:


        ``` md
        - [ ] an unticked task list item
        - [/] a half-checked task list item
        - [X] a ticked task list item
        ``````

:    false

     :  Disable the Pandoc task list syntax extension.

 [pandoc-task-lists]: https://pandoc.org/MANUAL.html#extension-task_lists

% \end{markdown}
% \iffalse

##### \LaTeX{} Example {.unnumbered}

Using a text editor, create a text document named `document.tex` with the
following content:
``` tex
\documentclass{article}
\usepackage[taskLists]{markdown}
\markdownSetup{
  renderers = {
    untickedBox = No,
    halfTickedBox = Maybe,
    tickedBox = Yes,
  },
}
\begin{document}
\begin{markdown}
- [ ] you can't.
- [/] I can?
- [X] I can!
\end{markdown}
\end{document}
```````
Next, invoke LuaTeX from the terminal:
``` sh
lualatex document.tex
``````
A PDF document named `document.pdf` should be produced and contain the
following text:

> - No you can't.
> - Maybe I can?
> - Yes I can!

##### \Hologo{ConTeXt} Example {.unnumbered}

Using a text editor, create a text document named `document.tex` with the
following content:
``` tex
\usemodule[t][markdown]
\setupmarkdown[taskLists = yes]
\def\markdownRendererUntickedBox{No}
\def\markdownRendererHalftickedBox{Maybe}
\def\markdownRendererTickedBox{Yes}
\starttext
\startmarkdown
- [ ] you can't.
- [/] I can?
- [X] I can!
\stopmarkdown
\stoptext
````````
Next, invoke LuaTeX from the terminal:
``` sh
context --luatex document.tex
`````
A PDF document named `document.pdf` should be produced and contain the
following text:

> - No you can't.
> - Maybe I can?
> - Yes I can!

%</manual-options>
%<*tex>
% \fi
%  \begin{macrocode}
\@@_add_lua_option:nnn
  { taskLists }
  { boolean }
  { false }
%    \end{macrocode}
% \iffalse
%</tex>
%<*lua,lua-cli>
% \fi
%  \begin{macrocode}
defaultOptions.taskLists = false
%    \end{macrocode}
% \par
% \iffalse
%</lua,lua-cli>
%<*manual-options>

#### Option `texComments`

`texComments` (default value: `false`)

% \fi
% \markdownBegin
%
% \Optitem[false]{texComments}{\opt{true}, \opt{false}}
%
:    true

     :   Strip \TeX{}-style comments.

         ``` tex
         \documentclass{article}
         \usepackage[texComments]{markdown}
         \begin{document}
         \begin{markdown}
         Hel%  this is a comment
            lo *world*!
         \end{markdown}
         \end{document}
         ```````

         Always enabled when \Opt{hybrid} is enabled.

:    false

     :   Do not strip \TeX{}-style comments.

% \markdownEnd
% \iffalse

##### Plain \TeX{} Example {.unnumbered}

Using a text editor, create a text document named `document.tex` with the
following content:
``` tex
\input markdown
\def\markdownOptionTexComments{true}
\markdownBegin
Hel%  this is a comment
   lo *world*!
\markdownEnd
\bye
```````
Next, invoke LuaTeX from the terminal:
``` sh
luatex document.tex
``````
A PDF document named `document.pdf` should be produced and contain the text
“Hello *world*!”

##### \LaTeX{} Example {.unnumbered}

Using a text editor, create a text document named `document.tex` with the
following content:
``` tex
\documentclass{article}
\usepackage[texComments]{markdown}
\begin{document}
\begin{markdown}
Hel%  this is a comment
   lo *world*!
\end{markdown}
\end{document}
```````
Next, invoke LuaTeX from the terminal:
``` sh
lualatex document.tex
``````
A PDF document named `document.pdf` should be produced and contain the
text “Hello *world*!”

##### \Hologo{ConTeXt} Example {.unnumbered}

Using a text editor, create a text document named `document.tex` with the
following content:
``` tex
\usemodule[t][markdown]
\setupmarkdown[texComments = yes]
\starttext
\startmarkdown
Hel%  this is a comment
   lo *world*!
\stopmarkdown
\stoptext
````````
Next, invoke LuaTeX from the terminal:
``` sh
context --luatex document.tex
`````
A PDF document named `document.pdf` should be produced and contain the
text “Hello *world*!”

%</manual-options>
%<*tex>
% \fi
%  \begin{macrocode}
\@@_add_lua_option:nnn
  { texComments }
  { boolean }
  { false }
%    \end{macrocode}
% \iffalse
%</tex>
%<*lua,lua-cli>
% \fi
%  \begin{macrocode}
defaultOptions.texComments = false
%    \end{macrocode}
% \par
% \iffalse
%</lua,lua-cli>
%<*manual-options>

#### Option `texMathDollars`

`texMathDollars` (default value: `false`)

% \fi
% \begin{markdown}
%
% \Optitem[false]{texMathDollars}{\opt{true}, \opt{false}}
%
:    true

     :  Enable the Pandoc [dollar math syntax
        extension][pandoc-tex-math-dollars]:

        ``` md
        inline math: $E=mc^2$

        display math: $$E=mc^2$$
        ```

:    false

     :  Disable the Pandoc dollar math syntax extension.

 [pandoc-tex-math-dollars]: https://pandoc.org/MANUAL.html#extension-tex_math_dollars

% \end{markdown}
% \iffalse

##### Lua Module Example {.unnumbered}

Using a text editor, create a text document named `document.tex` with the
following content:
``` tex
\input markdown
\input lmfonts
\begingroup
\catcode`\%=12
\catcode`\#=12
\directlua{
  local markdown = require("markdown")
  local newline = [[^^J^^J]]
  local convert = markdown.new({texMathDollars = true})
  local input =
    [[$E=mc^2$]] .. newline .. newline ..
    [[$$\hat{f} \left ( \xi  \right )= \int_{-\infty}^{\infty} f\left ( x  \right ) e^{-i2\pi \xi x} dx$$]]
  tex.sprint(convert(input)) }
\endgroup
\bye
```````
Then, invoke LuaTeX from the terminal:
``` sh
luatex document.tex
```````
A PDF document named `document.pdf` should be produced and contain the
following text:

> $E=mc^2$
>
> $$\hat{f} \left ( \xi  \right )= \int_{-\infty}^{\infty} f\left ( x  \right ) e^{-i2\pi \xi x} dx$$

##### Lua CLI Example {.unnumbered}

Using a text editor, create a text document named `document.tex` with the
following content:
``` tex
\input markdown
\input lmfonts
\begingroup
\catcode`\%=12
\catcode`\#=12
\input optionfalse
\par
\input optiontrue
\endgroup
\bye
```````
Using a text editor, create a text document named `content.md` with the
following content:
``` md
$E=mc^2$

$$\hat{f} \left ( \xi  \right )= \int_{-\infty}^{\infty} f\left ( x  \right ) e^{-i2\pi \xi x} dx$$
``````
Next, invoke LuaTeX from the terminal:
``` sh
texlua ⟨CLI pathname⟩ -- content.md optionfalse.tex
texlua ⟨CLI pathname⟩ texMathDollars=true -- content.md optiontrue.tex
luatex document.tex
```````
where \meta{CLI pathname} corresponds to the location of the Lua CLI script file,
such as `~/texmf/scripts/markdown/markdown-cli.lua` on UN\*X systems or
`C:\Users\`\meta{Your username}`\texmf\scripts\markdown\markdown-cli.lua` on Windows
systems. Use the command `kpsewhich -a markdown-cli.lua` to locate the Lua CLI
script file using [Kpathsea][].

A PDF document named `document.pdf` should be produced and contain the
following text:

> \$E=mc^2\$
>
> \$\$\hat{f} \left ( \xi  \right )= \int_{-\infty}^{\infty} f\left ( x  \right ) e^{-i2\pi \xi x} dx\$\$
>
> $E=mc^2$
>
> $$\hat{f} \left ( \xi  \right )= \int_{-\infty}^{\infty} f\left ( x  \right ) e^{-i2\pi \xi x} dx$$

##### Plain \TeX{} Example {.unnumbered}

Using a text editor, create a text document named `document.tex` with the
following content:
``` tex
\input markdown

\def\markdownOptionTexMathDollars{true}
\markdownBegin
$E=mc^2$

$$\hat{f} \left ( \xi  \right )= \int_{-\infty}^{\infty} f\left ( x  \right ) e^{-i2\pi \xi x} dx$$
\markdownEnd

\bye
```````
Next, invoke LuaTeX from the terminal:
``` sh
luatex document.tex
``````
A PDF document named `document.pdf` should be produced and contain the
following text:

> $E=mc^2$
>
> $$\hat{f} \left ( \xi  \right )= \int_{-\infty}^{\infty} f\left ( x  \right ) e^{-i2\pi \xi x} dx$$

##### \LaTeX{} Example {.unnumbered}

Using a text editor, create a text document named `document.tex` with the
following content:
``` tex
\documentclass{article}
\usepackage[texMathDollars]{markdown}
\begin{document}

\begin{markdown}
$E=mc^2$

$$\hat{f} \left ( \xi  \right )= \int_{-\infty}^{\infty} f\left ( x  \right ) e^{-i2\pi \xi x} dx$$
\end{markdown}

\end{document}
```````
Next, invoke LuaTeX from the terminal:
``` sh
lualatex document.tex
``````
A PDF document named `document.pdf` should be produced and contain the
following text:

> $E=mc^2$
>
> $$\hat{f} \left ( \xi  \right )= \int_{-\infty}^{\infty} f\left ( x  \right ) e^{-i2\pi \xi x} dx$$

##### \Hologo{ConTeXt} Example {.unnumbered}

Using a text editor, create a text document named `document.tex` with the
following content:
``` tex
\usemodule[t][markdown]
\setupmarkdown[texMathDollars = yes]
\starttext

\startmarkdown
$E=mc^2$

$$\hat{f} \left ( \xi  \right )= \int_{-\infty}^{\infty} f\left ( x  \right ) e^{-i2\pi \xi x} dx$$
\stopmarkdown

\stoptext
````````
Next, invoke LuaTeX from the terminal:
``` sh
context --luatex document.tex
`````
A PDF document named `document.pdf` should be produced and contain the
following text:

> $E=mc^2$
>
> $$\hat{f} \left ( \xi  \right )= \int_{-\infty}^{\infty} f\left ( x  \right ) e^{-i2\pi \xi x} dx$$

%</manual-options>
%<*tex>
% \fi
%  \begin{macrocode}
\@@_add_lua_option:nnn
  { texMathDollars }
  { boolean }
  { false }
%    \end{macrocode}
% \iffalse
%</tex>
%<*lua,lua-cli>
% \fi
%  \begin{macrocode}
defaultOptions.texMathDollars = false
%    \end{macrocode}
% \par
% \iffalse
%</lua,lua-cli>
%<*manual-options>

#### Option `texMathDoubleBackslash`

`texMathDoubleBackslash` (default value: `false`)

% \fi
% \begin{markdown}
%
% \Optitem[false]{texMathDoubleBackslash}{\opt{true}, \opt{false}}
%
:    true

     :  Enable the Pandoc [double backslash math syntax
        extension][pandoc-tex-math-double-backslash]:

        ``` md
        inline math: \\(E=mc^2\\)

        display math: \\[E=mc^2\\]
        ```

:    false

     :  Disable the Pandoc double backslash math syntax extension.

 [pandoc-tex-math-double-backslash]: https://pandoc.org/MANUAL.html#extension-tex_math_double_backslash

% \end{markdown}
% \iffalse

##### Lua Module Example {.unnumbered}

Using a text editor, create a text document named `document.tex` with the
following content:
``` tex
\input markdown
\input lmfonts
\begingroup
\catcode`\%=12
\catcode`\#=12
\directlua{
  local markdown = require("markdown")
  local newline = [[^^J^^J]]
  local convert = markdown.new({texMathDoubleBackslash = true})
  local input =
    [[\\(E=mc^2\\)]] .. newline .. newline ..
    [[\\[\hat{f} \left ( \xi  \right )= \int_{-\infty}^{\infty} f\left ( x  \right ) e^{-i2\pi \xi x} dx\\]]]
  tex.sprint(convert(input)) }
\endgroup
\bye
```````
Then, invoke LuaTeX from the terminal:
``` sh
luatex document.tex
```````
A PDF document named `document.pdf` should be produced and contain the
following text:

> \(E=mc^2\)
>
> \[\hat{f} \left ( \xi  \right )= \int_{-\infty}^{\infty} f\left ( x  \right ) e^{-i2\pi \xi x} dx\]

##### Lua CLI Example {.unnumbered}

Using a text editor, create a text document named `document.tex` with the
following content:
``` tex
\input markdown
\input lmfonts
\begingroup
\catcode`\%=12
\catcode`\#=12
\input optionfalse
\par
\input optiontrue
\endgroup
\bye
```````
Using a text editor, create a text document named `content.md` with the
following content:
``` md
\\(E=mc^2\\)

\\[\hat{f} \left ( \xi  \right )= \int_{-\infty}^{\infty} f\left ( x  \right ) e^{-i2\pi \xi x} dx\\]
``````
Next, invoke LuaTeX from the terminal:
``` sh
texlua ⟨CLI pathname⟩ -- content.md optionfalse.tex
texlua ⟨CLI pathname⟩ texMathDoubleBackslash=true -- content.md optiontrue.tex
luatex document.tex
```````
where \meta{CLI pathname} corresponds to the location of the Lua CLI script file,
such as `~/texmf/scripts/markdown/markdown-cli.lua` on UN\*X systems or
`C:\Users\`\meta{Your username}`\texmf\scripts\markdown\markdown-cli.lua` on Windows
systems. Use the command `kpsewhich -a markdown-cli.lua` to locate the Lua CLI
script file using [Kpathsea][].

A PDF document named `document.pdf` should be produced and contain the
following text:

> \\(E=mc^2\\)
>
> \\[\hat{f} \left ( \xi  \right )= \int_{-\infty}^{\infty} f\left ( x  \right ) e^{-i2\pi \xi x} dx\\]
>
> \(E=mc^2\)
>
> \[\hat{f} \left ( \xi  \right )= \int_{-\infty}^{\infty} f\left ( x  \right ) e^{-i2\pi \xi x} dx\]

##### Plain \TeX{} Example {.unnumbered}

Using a text editor, create a text document named `document.tex` with the
following content:
``` tex
\input markdown

\def\markdownOptionTexMathDoubleBackslash{true}
\markdownBegin
\\(E=mc^2\\)

\\[\hat{f} \left ( \xi  \right )= \int_{-\infty}^{\infty} f\left ( x  \right ) e^{-i2\pi \xi x} dx\\]
\markdownEnd

\bye
```````
Next, invoke LuaTeX from the terminal:
``` sh
luatex document.tex
``````
A PDF document named `document.pdf` should be produced and contain the
following text:

> \(E=mc^2\)
>
> \[\hat{f} \left ( \xi  \right )= \int_{-\infty}^{\infty} f\left ( x  \right ) e^{-i2\pi \xi x} dx\]

##### \LaTeX{} Example {.unnumbered}

Using a text editor, create a text document named `document.tex` with the
following content:
``` tex
\documentclass{article}
\usepackage[texMathDoubleBackslash]{markdown}
\begin{document}

\begin{markdown}
\\(E=mc^2\\)

\\[\hat{f} \left ( \xi  \right )= \int_{-\infty}^{\infty} f\left ( x  \right ) e^{-i2\pi \xi x} dx\\]
\end{markdown}

\end{document}
```````
Next, invoke LuaTeX from the terminal:
``` sh
lualatex document.tex
``````
A PDF document named `document.pdf` should be produced and contain the
following text:

> \(E=mc^2\)
>
> \[\hat{f} \left ( \xi  \right )= \int_{-\infty}^{\infty} f\left ( x  \right ) e^{-i2\pi \xi x} dx\]

##### \Hologo{ConTeXt} Example {.unnumbered}

Using a text editor, create a text document named `document.tex` with the
following content:
``` tex
\usemodule[t][markdown]
\setupmarkdown[texMathDoubleBackslash = yes]
\starttext

\startmarkdown
\\(E=mc^2\\)

\\[\hat{f} \left ( \xi  \right )= \int_{-\infty}^{\infty} f\left ( x  \right ) e^{-i2\pi \xi x} dx\\]
\stopmarkdown

\stoptext
````````
Next, invoke LuaTeX from the terminal:
``` sh
context --luatex document.tex
`````
A PDF document named `document.pdf` should be produced and contain the
following text:

> \(E=mc^2\)
>
> \[\hat{f} \left ( \xi  \right )= \int_{-\infty}^{\infty} f\left ( x  \right ) e^{-i2\pi \xi x} dx\]

%</manual-options>
%<*tex>
% \fi
%  \begin{macrocode}
\@@_add_lua_option:nnn
  { texMathDoubleBackslash }
  { boolean }
  { false }
%    \end{macrocode}
% \iffalse
%</tex>
%<*lua,lua-cli>
% \fi
%  \begin{macrocode}
defaultOptions.texMathDoubleBackslash = false
%    \end{macrocode}
% \par
% \iffalse
%</lua,lua-cli>
%<*manual-options>

#### Option `texMathSingleBackslash`

`texMathSingleBackslash` (default value: `false`)

% \fi
% \begin{markdown}
%
% \Optitem[false]{texMathSingleBackslash}{\opt{true}, \opt{false}}
%
:    true

     :  Enable the Pandoc [single backslash math syntax
        extension][pandoc-tex-math-single-backslash]:

        ``` md
        inline math: \(E=mc^2\)

        display math: \[E=mc^2\]
        ```

:    false

     :  Disable the Pandoc single backslash math syntax extension.

 [pandoc-tex-math-single-backslash]: https://pandoc.org/MANUAL.html#extension-tex_math_single_backslash

% \end{markdown}
% \iffalse

##### Lua Module Example {.unnumbered}

Using a text editor, create a text document named `document.tex` with the
following content:
``` tex
\input markdown
\input lmfonts
\begingroup
\catcode`\%=12
\catcode`\#=12
\directlua{
  local markdown = require("markdown")
  local newline = [[^^J^^J]]
  local convert = markdown.new({texMathSingleBackslash = true})
  local input =
    [[\(E=mc^2\)]] .. newline .. newline ..
    [[\[\hat{f} \left ( \xi  \right )= \int_{-\infty}^{\infty} f\left ( x  \right ) e^{-i2\pi \xi x} dx\]]]
  tex.sprint(convert(input)) }
\endgroup
\bye
```````
Then, invoke LuaTeX from the terminal:
``` sh
luatex document.tex
```````
A PDF document named `document.pdf` should be produced and contain the
following text:

> \(E=mc^2\)
>
> \[\hat{f} \left ( \xi  \right )= \int_{-\infty}^{\infty} f\left ( x  \right ) e^{-i2\pi \xi x} dx\]

##### Lua CLI Example {.unnumbered}

Using a text editor, create a text document named `document.tex` with the
following content:
``` tex
\input markdown
\input lmfonts
\begingroup
\catcode`\%=12
\catcode`\#=12
\input optionfalse
\par
\input optiontrue
\endgroup
\bye
```````
Using a text editor, create a text document named `content.md` with the
following content:
``` md
\(E=mc^2\)

\[\hat{f} \left ( \xi  \right )= \int_{-\infty}^{\infty} f\left ( x  \right ) e^{-i2\pi \xi x} dx\]
``````
Next, invoke LuaTeX from the terminal:
``` sh
texlua ⟨CLI pathname⟩ -- content.md optionfalse.tex
texlua ⟨CLI pathname⟩ texMathSingleBackslash=true -- content.md optiontrue.tex
luatex document.tex
```````
where \meta{CLI pathname} corresponds to the location of the Lua CLI script file,
such as `~/texmf/scripts/markdown/markdown-cli.lua` on UN\*X systems or
`C:\Users\`\meta{Your username}`\texmf\scripts\markdown\markdown-cli.lua` on Windows
systems. Use the command `kpsewhich -a markdown-cli.lua` to locate the Lua CLI
script file using [Kpathsea][].

A PDF document named `document.pdf` should be produced and contain the
following text:

> (E=mc^2)
>
> [\hat{f} \left ( \xi  \right )= \int_{-\infty}^{\infty} f\left ( x  \right ) e^{-i2\pi \xi x} dx]
>
> \(E=mc^2\)
>
> \[\hat{f} \left ( \xi  \right )= \int_{-\infty}^{\infty} f\left ( x  \right ) e^{-i2\pi \xi x} dx\]

##### Plain \TeX{} Example {.unnumbered}

Using a text editor, create a text document named `document.tex` with the
following content:
``` tex
\input markdown

\def\markdownOptionTexMathSingleBackslash{true}
\markdownBegin
\(E=mc^2\)

\[\hat{f} \left ( \xi  \right )= \int_{-\infty}^{\infty} f\left ( x  \right ) e^{-i2\pi \xi x} dx\]
\markdownEnd

\bye
```````
Next, invoke LuaTeX from the terminal:
``` sh
luatex document.tex
``````
A PDF document named `document.pdf` should be produced and contain the
following text:

> \(E=mc^2\)
>
> \[\hat{f} \left ( \xi  \right )= \int_{-\infty}^{\infty} f\left ( x  \right ) e^{-i2\pi \xi x} dx\]

##### \LaTeX{} Example {.unnumbered}

Using a text editor, create a text document named `document.tex` with the
following content:
``` tex
\documentclass{article}
\usepackage[texMathSingleBackslash]{markdown}
\begin{document}

\begin{markdown}
\(E=mc^2\)

\[\hat{f} \left ( \xi  \right )= \int_{-\infty}^{\infty} f\left ( x  \right ) e^{-i2\pi \xi x} dx\]
\end{markdown}

\end{document}
```````
Next, invoke LuaTeX from the terminal:
``` sh
lualatex document.tex
``````
A PDF document named `document.pdf` should be produced and contain the
following text:

> \(E=mc^2\)
>
> \[\hat{f} \left ( \xi  \right )= \int_{-\infty}^{\infty} f\left ( x  \right ) e^{-i2\pi \xi x} dx\]

##### \Hologo{ConTeXt} Example {.unnumbered}

Using a text editor, create a text document named `document.tex` with the
following content:
``` tex
\usemodule[t][markdown]
\setupmarkdown[texMathSingleBackslash = yes]
\starttext

\startmarkdown
\(E=mc^2\)

\[\hat{f} \left ( \xi  \right )= \int_{-\infty}^{\infty} f\left ( x  \right ) e^{-i2\pi \xi x} dx\]
\stopmarkdown

\stoptext
````````
Next, invoke LuaTeX from the terminal:
``` sh
context --luatex document.tex
`````
A PDF document named `document.pdf` should be produced and contain the
following text:

> \(E=mc^2\)
>
> \[\hat{f} \left ( \xi  \right )= \int_{-\infty}^{\infty} f\left ( x  \right ) e^{-i2\pi \xi x} dx\]

%</manual-options>
%<*tex>
% \fi
%  \begin{macrocode}
\@@_add_lua_option:nnn
  { texMathSingleBackslash }
  { boolean }
  { false }
%    \end{macrocode}
% \iffalse
%</tex>
%<*lua,lua-cli>
% \fi
%  \begin{macrocode}
defaultOptions.texMathSingleBackslash = false
%    \end{macrocode}
% \par
% \iffalse
%</lua,lua-cli>
%<*manual-options>

#### Option `tightLists`

`tightLists` (default value: `true`)

% \fi
% \begin{markdown}
%
% \Optitem[true]{tightLists}{\opt{true}, \opt{false}}
%
:    true

     :  Unordered and ordered lists whose items do not consist of multiple
        paragraphs will be considered *tight*. Tight lists will produce tight
        renderers that may produce different output than lists that are not
        tight:

        ``` md
        - This is
        - a tight
        - unordered list.

        - This is

          not a tight

        - unordered list.
        ```

:    false

     :  Unordered and ordered lists whose items consist of multiple paragraphs
        will be treated the same way as lists that consist of multiple paragraphs.

% \end{markdown}
% \iffalse

##### \LaTeX{} Example {.unnumbered}

Using a text editor, create a text document named `document.tex` with the
following content:
``` tex
\documentclass{article}
\usepackage{markdown}
\begin{document}

\begin{markdown}
The following list is tight:

- first item
- second item
- third item

The following list is loose:

- first item
- second item that spans

  multiple paragraphs
- third item
\end{markdown}

\begin{markdown}[tightLists=false]
The following list is now also loose:

- first item
- second item
- third item
\end{markdown}

\end{document}
```````
Next, invoke LuaTeX from the terminal:
``` sh
lualatex document.tex
``````
A PDF document named `document.pdf` should be produced and contain the
following text:

> The following list is tight:
>
> - first item
> - second item
> - third item
>
> The following list is loose:
>
> - first item
> - second item that spans
>
>   multiple paragraphs
> - third item
>
> The following list is now also loose:
>
> - first item
>
> - second item
>
> - third item

%</manual-options>
%<*tex>
% \fi
%  \begin{macrocode}
\@@_add_lua_option:nnn
  { tightLists }
  { boolean }
  { true }
%    \end{macrocode}
% \iffalse
%</tex>
%<*lua,lua-cli>
% \fi
%  \begin{macrocode}
defaultOptions.tightLists = true
%    \end{macrocode}
% \par
% \iffalse
%</lua,lua-cli>
%<*manual-options>

#### Option `underscores`

`underscores` (default value: `true`)

% \fi
% \begin{markdown}
%
% \Optitem[true]{underscores}{\opt{true}, \opt{false}}
%
:    true

     :  Both underscores and asterisks can be used to denote emphasis and
        strong emphasis:

        ``` md
        *single asterisks*
        _single underscores_
        **double asterisks**
        __double underscores__
        ``````

:    false

     :  Only asterisks can be used to denote emphasis and strong emphasis.
        This makes it easy to write math with the \Opt{hybrid} option
        without the need to constantly escape subscripts.

% \end{markdown}
% \iffalse

##### Plain \TeX{} Example {.unnumbered}

Using a text editor, create a text document named `document.tex` with the
following content:
``` tex
\input markdown
\def\markdownOptionHybrid{true}

\markdownBegin
This is _emphasized text_ and this is a math subscript: $m\_n$.
\markdownEnd

\def\markdownOptionUnderscores{false}
\markdownBegin
This is *emphasized text* and this is a math subscript: $m_n$.
\markdownEnd

\bye
```````
Next, invoke LuaTeX from the terminal:
``` sh
luatex document.tex
``````
A PDF document named `document.pdf` should be produced and contain the
following text:

> This is _emphasized text_ and this is a math subscript: *mₙ*.
>
> This is _emphasized text_ and this is a math subscript: *mₙ*.

##### \LaTeX{} Example {.unnumbered}

Using a text editor, create a text document named `document.tex` with the
following content:
``` tex
\documentclass{article}
\usepackage[hybrid]{markdown}
\begin{document}

\begin{markdown}
This is _emphasized text_ and this is a math subscript: $m\_n$.
\end{markdown}

\begin{markdown}[underscores=false]
This is *emphasized text* and this is a math subscript: $m_n$.
\end{markdown}

\end{document}
```````
Next, invoke LuaTeX from the terminal:
``` sh
lualatex document.tex
``````
A PDF document named `document.pdf` should be produced and contain the
following text:

> This is _emphasized text_ and this is a math subscript: *mₙ*.
>
> This is _emphasized text_ and this is a math subscript: *mₙ*.

##### \Hologo{ConTeXt} Example {.unnumbered}

Using a text editor, create a text document named `document.tex` with the
following content:
``` tex
\usemodule[t][markdown]
\setupmarkdown[hybrid = yes]
\starttext

\startmarkdown
This is _emphasized text_ and this is a math subscript: $m\_n$.
\stopmarkdown

\setupmarkdown[underscores = yes]
\startmarkdown
This is *emphasized text* and this is a math subscript: $m_n$.
\stopmarkdown

\stoptext
````````
Next, invoke LuaTeX from the terminal:
``` sh
context --luatex document.tex
`````
A PDF document named `document.pdf` should be produced and contain the
following text:

> This is _emphasized text_ and this is a math subscript: *mₙ*.
>
> This is _emphasized text_ and this is a math subscript: *mₙ*.

%</manual-options>
%<*tex>
% \fi
%  \begin{macrocode}
\@@_add_lua_option:nnn
  { underscores }
  { boolean }
  { true }
\ExplSyntaxOff
%    \end{macrocode}
% \iffalse
%</tex>
%<*lua,lua-cli>
% \fi
%  \begin{macrocode}
defaultOptions.underscores = true
%    \end{macrocode}
% \endgroup
% \iffalse
%</lua,lua-cli>
%<*lua-cli>
% \fi
% \par
% \begin{markdown}
%
%### Command-Line Interface {#lua-cli-interface}
%
% The high-level operation of the Markdown package involves the communication
% between several programming layers: the plain \TeX{} layer hands markdown
% documents to the Lua layer. Lua converts the documents to \TeX{}, and hands
% the converted documents back to plain \TeX{} layer for typesetting, see
% Figure <#fig:sequence-diagram-tex-interface>.
%
% This procedure has the advantage of being fully automated. However, it also
% has several important disadvantages: The converted \TeX{} documents are
% cached on the file system, taking up increasing amount of space. Unless the
% \TeX{} engine includes a Lua interpreter, the package also requires shell
% access, which opens the door for a malicious actor to access the system.
% Last, but not least, the complexity of the procedure impedes debugging.
%
% A solution to the above problems is to decouple the conversion from the
% typesetting. For this reason, a command-line Lua interface for converting a
% markdown document to \TeX{} is also provided, see
% Figure <#fig:sequence-diagram-lua-cli>.
%
% \end{markdown}
% \begin{figure}
% \centering
% \begin{sequencediagram}
%   \newthread{user}{User}{}
%   \newinst[4]{tex}{\TeX{}}{}
%   \newinst[4]{lua}{Lua}{}
%   \begin{call}{user}{\footnotesize\cs{jobname.tex}}{tex}{\footnotesize\cs{jobname.pdf}}
%   \begin{call}{tex}{\footnotesize\cs{jobname.markdown.in}}{lua}{\footnotesize\cs{jobname.markdown.out}}
%   \end{call}
%   \begin{call}{tex}{\footnotesize\cs{input}\cs{jobname.markdown.out}}{tex}{}
%   \end{call}
%   \end{call}
% \end{sequencediagram}
% \caption[A sequence diagram of typesetting a document using the \TeX{}
%   interface]{A sequence diagram of the Markdown package typesetting a
%   markdown document using the \TeX{} interface}
% \label{fig:sequence-diagram-tex-interface}
% \end{figure}
% \begin{figure}
% \centering
% \begin{sequencediagram}
%   \newthread{user}{User}{}
%   \newinst[4]{tex}{\TeX{}}{}
%   \newinst[4]{lua}{Lua}{}
%   \begin{call}{user}{\footnotesize$\langle$\textit{document}$\rangle$\texttt{.md}}{lua}{\footnotesize$\langle$\textit{document}$\rangle$\texttt{.tex}}
%   \end{call}
%   \begin{call}{user}{\footnotesize\cs{jobname.tex}}{tex}{\footnotesize\cs{jobname.pdf}}
%   \begin{call}{tex}{\footnotesize\cs{input} $\langle$\textit{document}$\rangle$}{tex}{}
%   \end{call}
%   \end{call}
% \end{sequencediagram}
% \caption[A sequence diagram of typesetting a document using the Lua CLI]%
%   {A sequence diagram of the Markdown package typesetting a markdown document
%   using the Lua command-line interface}
% \label{fig:sequence-diagram-lua-cli}
% \end{figure}
%  \begin{macrocode}

local HELP_STRING = [[
Usage: texlua ]] .. arg[0] .. [[ [OPTIONS] -- [INPUT_FILE] [OUTPUT_FILE]
where OPTIONS are documented in the Lua interface section of the
technical Markdown package documentation.

When OUTPUT_FILE is unspecified, the result of the conversion will be
written to the standard output. When INPUT_FILE is also unspecified, the
result of the conversion will be read from the standard input.

Report bugs to: witiko@mail.muni.cz
Markdown package home page: <https://github.com/witiko/markdown>]]

local VERSION_STRING = [[
markdown-cli.lua (Markdown) ]] .. metadata.version .. [[

Copyright (C) ]] .. table.concat(metadata.copyright,
                                 "\nCopyright (C) ") .. [[

License: ]] .. metadata.license

local function warn(s)
  io.stderr:write("Warning: " .. s .. "\n") end

local function error(s)
  io.stderr:write("Error: " .. s .. "\n")
  os.exit(1)
end
%    \end{macrocode}
% \begin{markdown}
%
% To make it easier to copy-and-paste options from Pandoc [@macfarlane22] such
% as `fancy_lists`, `header_attributes`, and `pipe_tables`, we accept
% snake\\\_case in addition to camelCase variants of options. As a bonus,
% studies [@sharif10] also show that snake\\\_case is faster to read than
% camelCase.
%
% \end{markdown}
%  \begin{macrocode}
local function camel_case(option_name)
  local cased_option_name = option_name:gsub("_(%l)", function(match)
    return match:sub(2, 2):upper()
  end)
  return cased_option_name
end

local function snake_case(option_name)
  local cased_option_name = option_name:gsub("%l%u", function(match)
    return match:sub(1, 1) .. "_" .. match:sub(2, 2):lower()
  end)
  return cased_option_name
end

local cases = {camel_case, snake_case}
local various_case_options = {}
for option_name, _ in pairs(defaultOptions) do
  for _, case in ipairs(cases) do
    various_case_options[case(option_name)] = option_name
  end
end

local process_options = true
local options = {}
local input_filename
local output_filename
for i = 1, #arg do
  if process_options then
%    \end{macrocode}
% \begin{markdown}
% After the optional `-`{}`-` argument has been specified, the remaining
% arguments are assumed to be input and output filenames. This argument is
% optional, but encouraged, because it helps resolve ambiguities when
% deciding whether an option or a filename has been specified.
% \end{markdown}
%  \begin{macrocode}
    if arg[i] == "--" then
      process_options = false
      goto continue
%    \end{macrocode}
% \begin{markdown}
% Unless the `-`{}`-` argument has been specified before, an argument
% containing the equals sign (`=`) is assumed to be an option specification in
% a \meta{key}`=`\meta{value} format. The available options are listed in
% Section <#sec:lua-options>.
% \end{markdown}
%  \begin{macrocode}
    elseif arg[i]:match("=") then
      local key, value = arg[i]:match("(.-)=(.*)")
      if defaultOptions[key] == nil and
         various_case_options[key] ~= nil then
        key = various_case_options[key]
      end
%    \end{macrocode}
% \begin{markdown}
% The \luamref{defaultOptions} table is consulted to identify whether \meta{value}
% should be parsed as a string, number, table, or boolean.
% \end{markdown}
%  \begin{macrocode}
      local default_type = type(defaultOptions[key])
      if default_type == "boolean" then
        options[key] = (value == "true")
      elseif default_type == "number" then
        options[key] = tonumber(value)
      elseif default_type == "table" then
        options[key] = {}
        for item in value:gmatch("[^ ,]+") do
          table.insert(options[key], item)
        end
      else
        if default_type ~= "string" then
          if default_type == "nil" then
            warn('Option "' .. key .. '" not recognized.')
          else
            warn('Option "' .. key .. '" type not recognized, please file ' ..
                 'a report to the package maintainer.')
          end
          warn('Parsing the ' .. 'value "' .. value ..'" of option "' ..
               key .. '" as a string.')
        end
        options[key] = value
      end
      goto continue
%    \end{macrocode}
% \begin{markdown}
% Unless the `-`{}`-` argument has been specified before, an argument
% `-`{}`-help`, or `-h` causes a brief documentation for how to invoke the
% program to be printed to the standard output.
% \end{markdown}
%  \begin{macrocode}
    elseif arg[i] == "--help" or arg[i] == "-h" then
      print(HELP_STRING)
      os.exit()
%    \end{macrocode}
% \begin{markdown}
% Unless the `-`{}`-` argument has been specified before, an argument
% `-`{}`-version`, or `-v` causes the program to print information about its
% name, version, origin and legal status, all on standard output.
% \end{markdown}
%  \begin{macrocode}
    elseif arg[i] == "--version" or arg[i] == "-v" then
      print(VERSION_STRING)
      os.exit()
    end
  end
%    \end{macrocode}
% \begin{markdown}
% The first argument that matches none of the above patterns is assumed to be
% the input filename. The input filename should correspond to the Markdown
% document that is going to be converted to a \TeX{} document.
% \end{markdown}
%  \begin{macrocode}
  if input_filename == nil then
    input_filename = arg[i]
%    \end{macrocode}
% \begin{markdown}
% The first argument that matches none of the above patterns is assumed to be
% the output filename. The output filename should correspond to the \TeX{}
% document that will result from the conversion.
% \end{markdown}
%  \begin{macrocode}
  elseif output_filename == nil then
    output_filename = arg[i]
  else
    error('Unexpected argument: "' .. arg[i] .. '".')
  end
  ::continue::
end
%    \end{macrocode}
% \par
% \begin{markdown}
%
% The command-line Lua interface is implemented by the `markdown-cli.lua`
% file that can be invoked from the command line as follows:
% ``` sh
% texlua /path/to/markdown-cli.lua cacheDir=. -- hello.md hello.tex
% ``````
% \noindent to convert the Markdown document `hello.md` to a \TeX{} document
% `hello.tex`.  After the Markdown package for our \TeX{} format has been
% loaded, the converted document can be typeset as follows:
% ``` tex
% \input hello
% ```````
%
% \end{markdown}
% \iffalse
%</lua-cli>
%<*tex>
% \fi
% \par
% \begin{markdown}
%
% Plain \TeX{} Interface {#texinterface}
%------------------------
%
% \end{markdown}
% \iffalse
%</tex>
%<*manual-interfaces>

### Plain \TeX{}

The plain \TeX{} interface provides \TeX{} commands that typeset markdown
documents by using the Lua interface behind the scenes. Unlike the Lua
interface, the plain TeX interface does not provide low-level tools for
converting markdown to \TeX{}. Instead, its goal is to provide high-level
typesetting capabilities.

The plain \TeX{} interface accepts the same options as the `markdown` Lua
module, in addition to its own options, but now the options are specified as
\TeX{} commands.

Using a text editor, create a text document named `document.tex` with the
following content:
``` tex
\input markdown
\input lmfonts

\markdownBegin
$\sqrt{-1}$ *equals* $i$.
\markdownEnd

\def\markdownOptionTexMathDollars{true}
\markdownBegin
$\sqrt{-1}$ *equals* $i$.
\markdownEnd

\bye
```````
Next, invoke LuaTeX from the terminal:
``` sh
luatex document.tex
``````
A PDF document named `document.pdf` should be produced and contain the
following text:

> \$\\sqrt{-1}\$ *equals* \$i\$.
>
> <math><msqrt><mo>−</mo><mn>1</mn></msqrt></math>
> *equals*
> <math><mi>i</mi></math>.

Invoking pdfTeX should have the same effect:
``` sh
pdftex --shell-escape document.tex
``````

%</manual-interfaces>
%<*tex>
% \fi
% \begin{markdown}
%
% The plain \TeX{} interface provides macros for the typesetting of markdown
% input from within plain \TeX{}, for setting the Lua interface options (see
% Section <#sec:lua-options>) used during the conversion from markdown to
% plain \TeX{} and for changing the way markdown the tokens are rendered.
%
% \end{markdown}
%  \begin{macrocode}
\def\markdownLastModified{(((LASTMODIFIED)))}%
\def\markdownVersion{(((VERSION)))}%
%    \end{macrocode}
% \par
% \begin{markdown}
%
% The plain \TeX{} interface is implemented by the `markdown.tex` file that can
% be loaded as follows:
% ``` tex
% \input markdown
% ```````
% \noindent It is expected that the special plain \TeX{} characters have the
% expected category codes, when `\input`ting the file.
%
%### Typesetting Markdown {#tex-typesetting}
%
% The interface exposes the \mdef{markdownBegin}, \mdef{markdownEnd},
% \mdef{markinline}, \mdef{markdownInput}, and \mdef{markdownEscape} macros.
%
% The \mref{markdownBegin} macro marks the beginning of a markdown document
% fragment and the \mref{markdownEnd} macro marks its end.
%
% \end{markdown}
%  \begin{macrocode}
\let\markdownBegin\relax
\let\markdownEnd\relax
%    \end{macrocode}
% \par
% \begin{markdown}
%
% You may prepend your own code to the \mref{markdownBegin} macro and redefine the
% \mref{markdownEnd} macro to produce special effects before and after the
% markdown block.
%
% There are several limitations to the macros you need to be aware of.
% The first limitation concerns the \mref{markdownEnd} macro, which must be
% visible directly from the input line buffer (it may not be produced as a
% result of input expansion). Otherwise, it will not be recognized as the end
% of the markdown string. As a corrolary, the \mref{markdownEnd} string
% may not appear anywhere inside the markdown input.
%
% Another limitation concerns spaces at the right end of an input line. In
% markdown, these are used to produce a forced line break. However, any such
% spaces are removed before the lines enter the input buffer of
% \TeX{}~[@knuth86a, p. 46]. As a corrolary, the \mref{markdownBegin} macro also
% ignores them.
%
% The \mref{markdownBegin} and \mref{markdownEnd} macros will also consume the rest
% of the lines at which they appear.  In the following example plain \TeX{}
% code, the characters `c`, `e`, and `f` will not appear in the output.
%
% ``` tex
% \input markdown
% a
% b \markdownBegin c
% d
% e \markdownEnd   f
% g
% \bye
% ```````
%
% Note that you may also not nest the \mref{markdownBegin} and \mref{markdownEnd}
% macros.
%
% The following example plain \TeX{} code showcases the usage of the
% \mref{markdownBegin} and \mref{markdownEnd} macros:
%
% ``` tex
% \input markdown
% \markdownBegin
% _Hello_ **world** ...
% \markdownEnd
% \bye
% ```````
%
% You can use the \mref{markinline} macro to input inline markdown content.
%
% \end{markdown}
%  \begin{macrocode}
\let\markinline\relax
%    \end{macrocode}
% \par
% \begin{markdown}
%
% The following example plain \TeX{} code showcases the usage of the
% \mref{markinline} macro:
%
% ``` tex
% \input markdown
% \markinline{_Hello_ **world**}
% \bye
% ```````
%
% The above code has the same effect as the below code:
%
% ``` tex
% \input markdown
% \markdownSetup{contentLevel=inline}
% \markdownBegin
% _Hello_ **world** ...
% \markdownEnd
% \bye
% ```````
%
% The \mref{markinline} macro is subject to the same limitations as the
% \mref{markdownBegin} and \mref{markdownEnd} macros.
%
% You can use the \mref{markdownInput} macro to include markdown documents,
% similarly to how you might use the \mref{input} \TeX{} primitive to include
% \TeX{} documents. The \mref{markdownInput} macro accepts a single parameter
% with the filename of a markdown document and expands to the result of the
% conversion of the input markdown document to plain \TeX{}.
%
% \end{markdown}
%  \begin{macrocode}
\let\markdownInput\relax
%    \end{macrocode}
% \par
% \begin{markdown}
%
% This macro is not subject to the limitations of the
% \mref{markdownBegin} and \mref{markdownEnd} macros.
%
% The following example plain \TeX{} code showcases the usage of the
% \mref{markdownInput} macro:
%
% ``` tex
% \input markdown
% \markdownInput{hello.md}
% \bye
% ```````
%
% The \mref{markdownEscape} macro accepts a single parameter with the filename
% of a \TeX{} document and executes the \TeX{} document in the middle of a
% markdown document fragment. Unlike the `\input` built-in of \TeX,
% \mref{markdownEscape} guarantees that the standard catcode regime of your
% \TeX{} format will be used.
%
% \end{markdown}
%  \begin{macrocode}
\let\markdownEscape\relax
%    \end{macrocode}
% \par
% \begin{markdown}
%
%### Options {#tex-options}
%
% The plain \TeX{} options are represented by \TeX{} commands. Some of them map
% directly to the options recognized by the Lua interface (see
% Section <#sec:lua-options>), while some of them are specific to the plain
% \TeX{} interface.
%
% To determine whether plain \TeX{} is the top layer or if there are other
% layers above plain \TeX{}, we take a look on whether the
% \mdef{c_@@_top_layer_tl} token list has already been defined. If not,
% we will assume that plain \TeX{} is the top layer.
%
% \end{markdown}
%  \begin{macrocode}
\ExplSyntaxOn
\tl_const:Nn \c_@@_option_layer_plain_tex_tl { plain_tex }
\cs_generate_variant:Nn
  \tl_const:Nn
  { NV }
\tl_if_exist:NF
  \c_@@_top_layer_tl
  {
    \tl_const:NV
      \c_@@_top_layer_tl
      \c_@@_option_layer_plain_tex_tl
  }
%    \end{macrocode}
% \begin{markdown}
%
% To enable the enumeration of plain \TeX{} options, we will maintain the
% \mdef{g_\@\@_plain_tex_options_seq} sequence.
%
% \end{markdown}
%  \begin{macrocode}
\seq_new:N \g_@@_plain_tex_options_seq
%    \end{macrocode}
% \begin{markdown}
%
% To enable the reflection of default plain \TeX{} options and their types, we
% will maintain the \mdef{g_\@\@_default_plain_tex_options_prop} and
% \mdef{g_\@\@_plain_tex_option_types_prop} property lists, respectively.
%
% \end{markdown}
%  \begin{macrocode}
\prop_new:N \g_@@_plain_tex_option_types_prop
\prop_new:N \g_@@_default_plain_tex_options_prop
\seq_gput_right:NV \g_@@_option_layers_seq \c_@@_option_layer_plain_tex_tl
\cs_new:Nn
  \@@_add_plain_tex_option:nnn
  {
    \@@_add_option:Vnnn
      \c_@@_option_layer_plain_tex_tl
      { #1 }
      { #2 }
      { #3 }
  }
%    \end{macrocode}
% \iffalse
%</tex>
%<*manual-options>

### Plain \TeX{}

Plain \TeX{} options control the communication between the \TeX{} interface and
the `markdown` Lua module. They are supported by all higher-level interfaces of
the Markdown package, i.e. the plain \TeX{}, \LaTeX{} and \Hologo{ConTeXt}
interfaces.

#### Setting Lua options from plain \TeX{}

As a rule of thumb, you can set all Lua options directly from plain \TeX{}. For
example, to set the \Opt{taskLists} Lua option to `true`, you would include the
following code in your plain \TeX{} document:

``` tex
\def\markdownOptionTaskLists{true}
```

%</manual-options>
%<*tex>
% \fi
% \par
% \begin{markdown}
%
% The plain \TeX{} options may be also be specified via the \mdef{markdownSetup}
% macro. Here, the plain \TeX{} options are represented by a comma-delimited
% list of \meta{key}`=`\meta{value} pairs. For boolean options, the
% `=`\meta{value} part is optional, and \meta{key} will be interpreted as
% \meta{key}`=true` if the `=`\meta{value} part has been omitted.
% The \mref{markdownSetup} macro receives the options to set up as its only
% argument.
%
% \end{markdown}
%  \begin{macrocode}
\cs_new:Nn
  \@@_setup:n
  {
    \keys_set:nn
      { markdown/options }
      { #1 }
  }
\cs_gset_eq:NN
  \markdownSetup
  \@@_setup:n
%    \end{macrocode}
% \iffalse
%</tex>
%<*manual-options>

Alternatively, you can also set plain \TeX{} options using the `\markdownSetup`
\TeX{} macro. For example, to set the \Opt{taskLists} Lua option to `true`,
you would include the following code in your plain \TeX{} document:

``` tex
\markdownSetup{taskLists = true}
```

%</manual-options>
%<*tex>
% \fi
% \par
% \begin{markdown}
%
% The
% \mdef{markdownIfOption}`{`\meta{name}`}{`\meta{iftrue}`}{`\meta{iffalse}`}`
% macro is provided for testing, whether the value of
% `\markdownOption`\meta{name} is `true`. If the value is `true`, then
% \meta{iftrue} is expanded, otherwise \meta{iffalse} is expanded.
%
% \end{markdown}
%  \begin{macrocode}
\prg_new_conditional:Nnn
  \@@_if_option:n
  { TF, T, F }
  {
    \@@_get_option_type:nN
      { #1 }
      \l_tmpa_tl
    \str_if_eq:NNF
      \l_tmpa_tl
      \c_@@_option_type_boolean_tl
      {
        \msg_error:nnxx
          { markdown }
          { expected-boolean-option }
          { #1 }
          { \l_tmpa_tl }
      }
    \@@_get_option_value:nN
      { #1 }
      \l_tmpa_tl
    \str_if_eq:NNTF
      \l_tmpa_tl
      \c_@@_option_value_true_tl
      { \prg_return_true: }
      { \prg_return_false: }
  }
\msg_new:nnn
  { markdown }
  { expected-boolean-option }
  {
    Option~#1~has~type~#2,~
    but~a~boolean~was~expected.
  }
\let\markdownIfOption=\@@_if_option:nTF
%    \end{macrocode}
% \begin{markdown}
%
%#### Finalizing and Freezing the Cache
% The \mdef{markdownOptionFinalizeCache} option corresponds to the Lua
% interface \Opt{finalizeCache} option, which creates an output file
% \Opt{frozenCacheFileName} (frozen cache) that contains a mapping
% between an enumeration of the markdown documents in the plain \TeX{} document
% and their auxiliary files cached in the \Opt{cacheDir} directory.
%
% \iffalse
%</tex>
%<*manual-options>

#### Finalizing and Freezing the Cache

% \fi
%
The \mdef{markdownOptionFrozenCache} option uses the mapping previously
% created by the \Opt{finalizeCache} option,
% \iffalse
created by the Lua interface \Opt{finalizeCache} option,
% \fi
and uses it to typeset the plain \TeX{} document without invoking Lua. As a
result, the plain \TeX{} document becomes more portable, but further changes
in the order and the content of markdown documents will not be reflected. It
defaults to `false`.

% \end{markdown}
% \iffalse
%</manual-options>
%<*tex>
% \fi
%  \begin{macrocode}
\@@_add_plain_tex_option:nnn
  { frozenCache }
  { boolean }
  { false }
%    \end{macrocode}
% \iffalse
%</tex>
%<*manual-options>
% \fi
% \begin{markdown}
%
% The standard usage of the above two options is as follows:
% \iffalse
The standard usage of the \Opt{finalizeCache} and \Opt{frozenCache}
options is as follows:
% \fi

1. Remove the \Opt{cacheDir} cache directory with stale auxiliary cache
   files.
% 2. Enable the \Opt{finalizeCache} option.
% \iffalse
2. Enable the \Opt{finalizeCache} option.
% \fi
3. Typeset the plain \TeX{} document to populate and finalize the cache.
4. Enable the \Opt{frozenCache} option.
5. Publish the source code of the plain \TeX{} document and the
   \Opt{cacheDir} directory.

% \iffalse
For more information, see the examples for the \Opt{finalizeCache} option.
% \fi

#### File and Directory Names

% \end{markdown}
% \iffalse
%</manual-options>
%<*tex>
% \fi
% \begin{markdown}
%
% The \mdef{markdownOptionInputTempFileName} macro sets the filename of the
% temporary input file that is created during the buffering of markdown text
% from a \TeX{} source. It defaults to `\jobname.markdown.in`.
%
% The expansion of this macro must not contain quotation marks (`"`) or
% backslash symbols (`\`). Mind that \TeX{} engines tend to
% put quotation marks around `\jobname`, when it contains spaces.
%
% \end{markdown}
%  \begin{macrocode}
\@@_add_plain_tex_option:nnn
  { inputTempFileName }
  { path }
  { \jobname.markdown.in }
%    \end{macrocode}
% \par
% \begin{markdown}
%
% The \mdef{markdownOptionOutputDir} macro sets the path to the directory that
% will contain the auxiliary cache files produced by the Lua implementation and
% also the auxiliary files produced by the plain \TeX{} implementation. The
% option defaults to `.` or, since \TeX{} Live 2024, to the value of the
% `-output-directory` option of your \TeX{} engine.
%
% The path must be set to the same value as the `-output-directory` option of
% your \TeX{} engine for the package to function correctly. We need this macro
% to make the Lua implementation aware where it should store the helper files.
% The same limitations apply here as in the case of the
% \Opt{inputTempFileName} macro.
%
% The \mref{markdownOptionOutputDir} macro has been deprecated and will be
% removed in the next major version of the Markdown package. <!-- After the
% macro has been removed, we should remove support for TeX Live 2023 and
% earlier, where the automatic detection does not work. -->
%
% \end{markdown}
%  \begin{macrocode}
\@@_add_plain_tex_option:nnn
  { outputDir }
  { path }
  { . }
%    \end{macrocode}
% \iffalse
%</tex>
%<*manual-options>

The plain \TeX{} interface provides the following commands that you can use to
specify the location of temporary files produced during the conversion from
Markdown to \TeX{}:

- `\markdownOptionInputTempFileName`,
- `\markdownOptionOutputDir`,
- `\markdownOptionCacheDir`, and
- `\markdownOptionFrozenCacheFileName`.

The `\markdownOptionCacheDir` and `\markdownOptionFrozenCacheFileName` commands
correspond to the `cacheDir` and `frozenCacheFileName` Lua options.

Using a text editor, create a folder named `output-directory` and a text
document named `document.tex` with the following content:
``` tex
\input lmfonts
\input markdown
\def\markdownOptionInputTempFileName{temporary-input.md}
\def\markdownOptionOutputDir{output-directory}
\def\markdownOptionCacheDir{output-directory/cache-directory}
\def\markdownOptionEagerCache{true}
\def\markdownOptionFinalizeCache{true}
\def\markdownOptionFrozenCacheFileName{output-directory/cache-directory/frozen-cache.tex}
\markdownBegin
Hello *world*!
\markdownEnd
\bye
```````
Next, invoke LuaTeX from the terminal:
``` sh
luatex --output-directory output-directory document.tex
``````
A text document named `temporary-input.md` should be produced in the folder
named `output-directory` and contain the following text:

> Hello \*world\*!

A folder named `output-directory/cache-directory` should also be produced and
contain fragments of the converted markdown document. LuaTeX does not need
other temporary files to perform the conversion from markdown to \TeX{}. To
produce the remaining temporary files, invoke pdfTeX from the terminal:
``` sh
pdftex --output-directory output-directory --shell-escape document.tex
``````
Text document named `temporary-output.md` should be produced in the folder
named `output-directory`.  The document will contain the input markdown
document converted to \TeX{}.

% \fi
% \begin{markdown}

#### No default token renderer prototypes {#plain}

The Markdown package provides default definitions for token renderer prototypes
using the `witiko/markdown/defaults`
% theme (see Section~<sec:#themes>).
% \iffalse
theme.
% \fi
Although these default definitions provide a useful starting point for authors,
they use extra resources, especially with higher-level \TeX{} formats such as
\LaTeX{} and \Hologo{ConTeXt}. Furthermore, the default definitions may change
at any time, which may pose a problem for maintainers of Markdown themes and
templates who may require a stable output.

The \mdef{markdownOptionPlain} macro specifies whether higher-level \TeX{}
formats should only use the plain \TeX{} default definitions or whether they
should also use the format-specific default definitions. Whereas plain \TeX{}
default definitions only provide definitions for simple elements such as
emphasis, strong emphasis, and paragraph separators, format-specific default
definitions add support for more complex elements such as lists, tables, and
citations. On the flip side, plain \TeX{} default definitions load no extra
resources and are rather stable, whereas format-specific default definitions
load extra resources and are subject to a more rapid change.

Here is how you would enable the macro in a \LaTeX{} document:

``` tex
\usepackage[plain]{markdown}
```````

Here is how you would enable the macro in a \Hologo{ConTeXt} document:

``` tex
\def\markdownOptionPlain{true}
\usemodule[t][markdown]
```````

The macro must be set before or during the loading of the package. Setting the
macro after loading the package has no effect.

% \end{markdown}
% \iffalse
%</manual-options>
%<*tex>
% \fi
%  \begin{macrocode}
\@@_add_plain_tex_option:nnn
  { plain }
  { boolean }
  { false }
%    \end{macrocode}
% \iffalse
%</tex>
%<*manual-options>
% \fi
% \begin{markdown}

The \mdef{markdownOptionNoDefaults} macro specifies whether we should prevent
the loading of default definitions or not. This is useful in contexts, where
we want to have total control over how all elements are rendered.

Here is how you would enable the macro in a \LaTeX{} document:

``` tex
\usepackage[noDefaults]{markdown}
```````

Here is how you would enable the macro in a \Hologo{ConTeXt} document:

``` tex
\def\markdownOptionNoDefaults{true}
\usemodule[t][markdown]
```````

The macro must be set before or during the loading of the package. Setting the
macro after loading the package has no effect.

% \end{markdown}
% \iffalse
%</manual-options>
%<*tex>
% \fi
%  \begin{macrocode}
\@@_add_plain_tex_option:nnn
  { noDefaults }
  { boolean }
  { false }
%    \end{macrocode}
% \begin{markdown}
%
%#### Miscellaneous Options
%
% The \mdef{markdownOptionStripPercentSigns} macro controls whether a percent
% sign (`\%`) at the beginning of a line will be discarded when buffering
% Markdown input (see sections <#sec:buffering-block> and
% <#sec:buffering-inline>) or not. Notably, this enables the use of markdown
% when writing \TeX{} package documentation using the \pkg{Doc}
% \LaTeX{}~package~[@mittelbach17] or similar. The recognized values of the
% macro are `true` (discard) and `false` (retain). It defaults to `false`.
%
% \end{markdown}
%  \begin{macrocode}
\seq_gput_right:Nn
  \g_@@_plain_tex_options_seq
  { stripPercentSigns }
\prop_gput:Nnn
  \g_@@_plain_tex_option_types_prop
  { stripPercentSigns }
  { boolean }
\prop_gput:Nnx
  \g_@@_default_plain_tex_options_prop
  { stripPercentSigns }
  { false }
%    \end{macrocode}
% \iffalse
%</tex>
%<*manual-options>

#### Package Documentation

The \mdef{markdownOptionStripPercentSigns} macro controls whether a percent
sign (`\%`) at the beginning of a line will be discarded when reading Markdown
input from a \TeX{} document. This enables the use of markdown when writing
\TeX{} package documentation using the [Doc \LaTeX{} package][doc] by Frank
Mittelbach. The recognized values of the macro are `true` (discard) and `false`
(retain). It defaults to `false`.

 [doc]: https://ctan.org/pkg/doc (doc – Format LaTeX documentation)

Using a text editor, create a text document named `document.dtx` with the
following content:

 ``` tex
 % \iffalse
 \documentclass{ltxdoc}
 \usepackage[stripPercentSigns]{markdown}
 \begin{document}
 \DocInput{document.dtx}
 \end{document}
 % \fi
 %
 % \begin{markdown}
 % Hello *world*!
 % \end{markdown}
 ```````

Next, invoke LuaTeX from the terminal:
``` sh
lualatex document.dtx
``````
A PDF document named `document.pdf` should be produced and contain the text
“Hello *world*!”

%</manual-options>
%<*tex>
% \fi
% \begin{markdown}
%
%#### Generating Plain \TeX{} Option Macros and Key-Values
%
% We define the command \mdef{@@_define_option_commands_and_keyvals:} that
% defines plain \TeX{} macros and the key-value interface
% of the \mref{markdownSetup} macro for the above plain \TeX{} options.
%
% The command also defines macros and key-values that map
% directly to the options recognized by the Lua interface, such as
% \mdef{markdownOptionHybrid} for the \Opt{hybrid} Lua option (see Section
% <#sec:lua-options>), which are not processed by the plain \TeX{}
% implementation, only passed along to Lua.
%
% Furthermore, the command also defines options and key-values
% for subsequently loaded layers that correspond to higher-level \TeX{} formats
% such as \LaTeX{} and \Hologo{ConTeXt}.
%
% For the macros that correspond to the non-boolean options recognized by the
% Lua interface, the same limitations apply here in the case of the
% \Opt{inputTempFileName} macro.
%
% \end{markdown}
%  \begin{macrocode}
\cs_new:Nn
  \@@_define_option_commands_and_keyvals:
  {
    \seq_map_inline:Nn
      \g_@@_option_layers_seq
      {
        \seq_map_inline:cn
          { g_@@_ ##1 _options_seq }
          {
              \@@_define_option_command:n
                { ####1 }
%    \end{macrocode}
% \begin{markdown}
%
% To make it easier to copy-and-paste options from Pandoc [@macfarlane22] such
% as `fancy_lists`, `header_attributes`, and `pipe_tables`, we accept
% snake\\\_case in addition to camelCase variants of options. As a bonus,
% studies [@sharif10] also show that snake\\\_case is faster to read than
% camelCase.
%
% \end{markdown}
%  \begin{macrocode}
              \@@_with_various_cases:nn
                { ####1 }
                {
                  \@@_define_option_keyval:nnn
                    { ##1 }
                    { ####1 }
                    { ########1 }
                }
          }
      }
  }
\cs_new:Nn
  \@@_define_option_command:n
  {
%    \end{macrocode}
% \begin{markdown}
%
% Use the \pkg{lt3luabridge} library to determine the default value of the
% \mref{markdownOptionOutputDir} macro by using the environmental variable
% `TEXMF_OUTPUT_DIRECTORY` that is available since TeX~Live 2024.
%
% \end{markdown}
%  \begin{macrocode}
    \str_if_eq:nnTF
      { #1 }
      { outputDir }
      { \@@_define_option_command_output_dir: }
      {
%    \end{macrocode}
% \begin{markdown}
%
% Do not override options defined before loading the package.
%
% \end{markdown}
%  \begin{macrocode}
        \@@_option_tl_to_csname:nN
          { #1 }
          \l_tmpa_tl
        \cs_if_exist:cF
          { \l_tmpa_tl }
          {
            \@@_get_default_option_value:nN
              { #1 }
              \l_tmpa_tl
            \@@_set_option_value:nV
              { #1 }
              \l_tmpa_tl
          }
      }
  }
\ExplSyntaxOff
\input lt3luabridge.tex
\ExplSyntaxOn
\cs_new:Nn
  \@@_define_option_command_output_dir:
  {
    \cs_if_free:NT
      \markdownOptionOutputDir
      {
        \bool_if:nTF
          {
            \cs_if_exist_p:N
              \luabridge_tl_set:Nn &&
            (
              \int_compare_p:nNn
                { \g_luabridge_method_int }
                =
                { \c_luabridge_method_directlua_int } ||
              \sys_if_shell_unrestricted_p:
            )
          }
          {
            \luabridge_tl_set:Nn
              \l_tmpa_tl
              { print(os.getenv("TEXMF_OUTPUT_DIRECTORY") or ".") }
            \tl_gset:NV
              \markdownOptionOutputDir
              \l_tmpa_tl
          }
          {
            \tl_gset:Nn
              \markdownOptionOutputDir
              { . }
          }
      }
  }
\cs_new:Nn
  \@@_set_option_value:nn
  {
    \@@_define_option:n
      { #1 }
    \@@_get_option_type:nN
      { #1 }
      \l_tmpa_tl
    \str_if_eq:NNTF
      \c_@@_option_type_counter_tl
      \l_tmpa_tl
      {
        \@@_option_tl_to_csname:nN
          { #1 }
          \l_tmpa_tl
        \int_gset:cn
          { \l_tmpa_tl }
          { #2 }
      }
      {
        \@@_option_tl_to_csname:nN
          { #1 }
          \l_tmpa_tl
        \cs_set:cpn
          { \l_tmpa_tl }
          { #2 }
      }
  }
\cs_generate_variant:Nn
  \@@_set_option_value:nn
  { nV }
\cs_new:Nn
  \@@_define_option:n
  {
    \@@_option_tl_to_csname:nN
      { #1 }
      \l_tmpa_tl
    \cs_if_free:cT
      { \l_tmpa_tl }
      {
        \@@_get_option_type:nN
          { #1 }
          \l_tmpb_tl
        \str_if_eq:NNT
          \c_@@_option_type_counter_tl
          \l_tmpb_tl
          {
            \@@_option_tl_to_csname:nN
              { #1 }
              \l_tmpa_tl
            \int_new:c
              { \l_tmpa_tl }
          }
      }
  }
\cs_new:Nn
  \@@_define_option_keyval:nnn
  {
    \prop_get:cnN
      { g_@@_ #1 _option_types_prop }
      { #2 }
      \l_tmpa_tl
    \str_if_eq:VVTF
      \l_tmpa_tl
      \c_@@_option_type_boolean_tl
      {
        \keys_define:nn
          { markdown/options }
          {
%    \end{macrocode}
% \begin{markdown}
%
% For boolean options, we also accept `yes` as an alias for
% `true` and `no` as an alias for `false`.
%
% \end{markdown}
%  \begin{macrocode}
            #3 .code:n = {
              \tl_set:Nx
                \l_tmpa_tl
                {
                  \str_case:nnF
                    { ##1 }
                    {
                      { yes } { true }
                      { no } { false }
                    }
                    { ##1 }
                }
              \@@_set_option_value:nV
                { #2 }
                \l_tmpa_tl
            },
            #3 .default:n = { true },
          }
      }
      {
        \keys_define:nn
          { markdown/options }
          {
            #3 .code:n = {
              \@@_set_option_value:nn
                { #2 }
                { ##1 }
            },
          }
      }
%    \end{macrocode}
% \begin{markdown}
%
% For options of type `clist`, we assume that \meta{key} is a regular English
% noun in plural (such as `extensions`) and we also define the
% \meta{singular key}`=`\meta{value} interface, where \meta{singular key} is
% \meta{key} after stripping the trailing -s (such as `extension`). Rather
% than setting the option to \meta{value}, this interface appends \meta{value}
% to the current value as the rightmost item in the list.
%
% \end{markdown}
%  \begin{macrocode}
    \str_if_eq:VVT
      \l_tmpa_tl
      \c_@@_option_type_clist_tl
      {
        \tl_set:Nn
          \l_tmpa_tl
          { #3 }
        \tl_reverse:N
          \l_tmpa_tl
        \str_if_eq:enF
          {
            \tl_head:V
              \l_tmpa_tl
          }
          { s }
          {
            \msg_error:nnn
              { markdown }
              { malformed-name-for-clist-option }
              { #3 }
          }
        \tl_set:Nx
          \l_tmpa_tl
          {
            \tl_tail:V
              \l_tmpa_tl
          }
        \tl_reverse:N
          \l_tmpa_tl
        \tl_put_right:Nn
          \l_tmpa_tl
          {
            .code:n = {
              \@@_get_option_value:nN
                { #2 }
                \l_tmpa_tl
              \clist_set:NV
                \l_tmpa_clist
                { \l_tmpa_tl, { ##1 } }
              \@@_set_option_value:nV
                { #2 }
                \l_tmpa_clist
            }
          }
        \keys_define:nV
          { markdown/options }
          \l_tmpa_tl
      }
  }
\cs_generate_variant:Nn
  \clist_set:Nn
  { NV }
\cs_generate_variant:Nn
  \keys_define:nn
  { nV }
\cs_generate_variant:Nn
  \@@_set_option_value:nn
  { nV }
\prg_generate_conditional_variant:Nnn
  \str_if_eq:nn
  { en }
  { F }
\msg_new:nnn
  { markdown }
  { malformed-name-for-clist-option }
  {
    Clist~option~name~#1~does~not~end~with~-s.
  }
%    \end{macrocode}
% \begin{markdown}
%
% If plain \TeX{} is the top layer, we use the
% \mref{@@_define_option_commands_and_keyvals:} macro to define plain \TeX{}
% option macros and key-values immediately. Otherwise, we
% postpone the definition until the upper layers have been loaded.
%
% \end{markdown}
%  \begin{macrocode}
\str_if_eq:VVT
  \c_@@_top_layer_tl
  \c_@@_option_layer_plain_tex_tl
  {
    \@@_define_option_commands_and_keyvals:
  }
\ExplSyntaxOff
%    \end{macrocode}
% \iffalse
%</tex>
%<*manual-options>
% \fi
% \begin{markdown}

### Themes {#themes}

User-defined themes for the Markdown package provide a domain-specific
interpretation of Markdown tokens. Themes allow the authors to achieve
a specific look and other high-level goals without low-level programming.

% The key-values `theme`=\meta{theme name} and `import`=\meta{theme name} load
% a \TeX{} document (further referred to as *a theme*) named
% `markdowntheme`\meta{munged theme name}`.tex`, where the *munged theme name*
% is the *theme name* after the substitution of all forward slashes (`/`) for
% an underscore (`_`). The theme name is *qualified* and contains no
% underscores. A theme name is qualified if and only if it contains at least
% one forward slash. Themes are inspired by the Beamer \LaTeX{} package, which
% provides similar functionality with its `\usetheme` macro [@tantau21, Section
% 15.1].
%
% Theme names must be qualified to minimize naming conflicts between different
% themes with a similar purpose. The preferred format of a theme name is
% \meta{theme author}`/`\meta{theme purpose}`/`\meta{private naming
% scheme}, where the *private naming scheme* may contain additional forward
% slashes. For example, a theme by a user `witiko` for the MU theme of the
% Beamer document class may have the name `witiko/beamer/MU`.
%
% Theme names are munged to allow structure inside theme names without
% dictating where the themes should be located inside the \TeX{} directory
% structure. For example, loading a theme named `witiko/beamer/MU` would
% load a \TeX{} document package named `markdownthemewitiko_beamer_MU.tex`.
%
% \end{markdown}
% \iffalse
%</manual-options>
%<*tex>
% \fi
%  \begin{macrocode}
\ExplSyntaxOn
\keys_define:nn
  { markdown/options }
  {
    theme .code:n = {
      \@@_set_theme:n
        { #1 }
    },
    import .code:n = {
      \tl_set:Nn
        \l_tmpa_tl
        { #1 }
%    \end{macrocode}
% \begin{markdown}
%
% To ensure that keys containing forward slashes get passed correctly, we
% replace all forward slashes in the input with backslash tokens with category
% code letter and then undo the replacement. This means that if any unbraced
% backslash tokens with category code letter exist in the input, they will be
% replaced with forward slashes. However, this should be extremely rare.
%
% \end{markdown}
%  \begin{macrocode}
      \tl_replace_all:NnV
        \l_tmpa_tl
        { / }
        \c_backslash_str
      \keys_set:nV
        { markdown/options/import }
        \l_tmpa_tl
    },
  }
%    \end{macrocode}
% \begin{markdown}
%
% To keep track of the current theme when themes are nested, we will
% maintain the \mdef{g_\@\@_themes_seq} stack of theme names.
% For convenience, the name of the current theme is also available in the
% \mdef{g_@@_current_theme_tl} macro.
%
% \end{markdown}
%  \begin{macrocode}
\seq_new:N
  \g_@@_themes_seq
\tl_new:N
  \g_@@_current_theme_tl
\tl_gset:Nn
  \g_@@_current_theme_tl
  { }
\seq_gput_right:NV
  \g_@@_themes_seq
  \g_@@_current_theme_tl
\cs_new:Nn
  \@@_set_theme:n
  {
%    \end{macrocode}
% \begin{markdown}
%
% First, we validate the theme name.
%
% \end{markdown}
%  \begin{macrocode}
    \str_if_in:nnF
      { #1 }
      { / }
      {
        \msg_error:nnn
          { markdown }
          { unqualified-theme-name }
          { #1 }
      }
    \str_if_in:nnT
      { #1 }
      { _ }
      {
        \msg_error:nnn
          { markdown }
          { underscores-in-theme-name }
          { #1 }
      }
%    \end{macrocode}
% \begin{markdown}
%
% Next, we munge the theme name.
%
% \end{markdown}
%  \begin{macrocode}
    \str_set:Nn
      \l_tmpa_str
      { #1 }
    \str_replace_all:Nnn
      \l_tmpa_str
      { / }
      { _ }
%    \end{macrocode}
% \begin{markdown}
%
% Finally, we load the theme.
%
% \end{markdown}
%  \begin{macrocode}
    \tl_gset:Nn
      \g_@@_current_theme_tl
      { #1 / }
    \seq_gput_right:NV
      \g_@@_themes_seq
      \g_@@_current_theme_tl
    \@@_load_theme:nV
      { #1 }
      \l_tmpa_str
    \seq_gpop_right:NN
      \g_@@_themes_seq
      \l_tmpa_tl
    \seq_get_right:NN
      \g_@@_themes_seq
      \l_tmpa_tl
    \tl_gset:NV
      \g_@@_current_theme_tl
      \l_tmpa_tl
  }
\msg_new:nnnn
  { markdown }
  { unqualified-theme-name }
  { Won't~load~theme~with~unqualified~name~#1 }
  { Theme~names~must~contain~at~least~one~forward~slash }
\msg_new:nnnn
  { markdown }
  { underscores-in-theme-name }
  { Won't~load~theme~with~an~underscore~in~its~name~#1 }
  { Theme~names~must~not~contain~underscores~in~their~names }
\cs_generate_variant:Nn
  \tl_replace_all:Nnn
  { NnV }
\ExplSyntaxOff
%    \end{macrocode}
% \iffalse
%</tex>
%<*manual-options>
% \fi
% \par
% \begin{markdown}

Built-in plain \TeX{} themes provided with the Markdown package include:

\pkg{witiko/tilde}

:    A theme that makes tilde (`~`) always typeset the non-breaking space even
     when the \Opt{hybrid} Lua option is disabled.
%    ``` tex
%    \input markdown
%    \markdownSetup{import=witiko/tilde}
%    \markdownBegin
%    Bartel~Leendert van~der~Waerden
%    \markdownEnd
%    \bye
%    ```````
%    Typesetting the above document produces the following text:
%    “Bartel~Leendert van~der~Waerden”.
%
% \end{markdown}
% \iffalse

##### Plain \TeX{} Example {.unnumbered}

Using a text editor, create a text document named `document.tex` with the
following content:
``` tex
\input markdown
\markdownSetup{import=witiko/tilde}
\markdownBegin
Bartel~Leendert van~der~Waerden
\markdownEnd
\bye
```````
Next, invoke LuaTeX from the terminal:
``` sh
luatex document.tex
``````
A PDF document named `document.pdf` should be produced and contain the
following text, where the middot (`·`) denotes a non-breaking space:

> Bartel·Leendert van·der·Waerden

% \fi
% \par
% \begin{markdown}

\pkg{witiko/markdown/defaults}

:    A plain \TeX{} theme with the default definitions of token renderer
     prototypes for plain \TeX{}. This theme is loaded automatically together
     with the package and explicitly loading it has no effect.

% Please, see Section <#sec:themes-implementation> for implementation
% details of the built-in plain \TeX{} themes.

### Snippets {#snippets}

% \end{markdown}
% \iffalse

User-defined themes provide global control over high-level goals.
However, it is often desirable to change only some local aspects of a document.
Snippets provide syntactic sugar for defining and invoking various
options locally.

%</manual-options>
%<*tex>
% \fi
% \par
% \begin{markdown}
%
% We may set up options as *snippets* using the
% \mdef{markdownSetupSnippet} macro and invoke them later. The
% \mref{markdownSetupSnippet} macro receives two arguments: the name
% of the snippet and the options to store.
%
% \end{markdown}
%  \begin{macrocode}
\ExplSyntaxOn
\prop_new:N
  \g_@@_snippets_prop
\cs_new:Nn
  \@@_setup_snippet:nn
  {
    \tl_if_empty:nT
      { #1 }
      {
        \msg_error:nnn
          { markdown }
          { empty-snippet-name }
          { #1 }
      }
    \tl_set:NV
      \l_tmpa_tl
      \g_@@_current_theme_tl
    \tl_put_right:Nn
      \l_tmpa_tl
      { #1 }
    \@@_if_snippet_exists:nT
      { #1 }
      {
        \msg_warning:nnV
          { markdown }
          { redefined-snippet }
          \l_tmpa_tl
      }
    \prop_gput:NVn
      \g_@@_snippets_prop
      \l_tmpa_tl
      { #2 }
  }
\cs_gset_eq:NN
  \markdownSetupSnippet
  \@@_setup_snippet:nn
\msg_new:nnnn
  { markdown }
  { empty-snippet-name }
  { Empty~snippet~name~#1 }
  { Pick~a~non-empty~name~for~your~snippet }
\msg_new:nnn
  { markdown }
  { redefined-snippet }
  { Redefined~snippet~#1 }
%    \end{macrocode}
% \begin{markdown}
%
% To decide whether a snippet exists, we can use the
% \mdef{markdownIfSnippetExists} macro.
%
% \end{markdown}
%  \begin{macrocode}
\prg_new_conditional:Nnn
  \@@_if_snippet_exists:n
  { TF, T, F }
  {
    \tl_set:NV
      \l_tmpa_tl
      \g_@@_current_theme_tl
    \tl_put_right:Nn
      \l_tmpa_tl
      { #1 }
    \prop_get:NVNTF
      \g_@@_snippets_prop
      \l_tmpa_tl
      \l_tmpb_tl
      { \prg_return_true: }
      { \prg_return_false: }
  }
\cs_gset_eq:NN
  \markdownIfSnippetExists
  \@@_if_snippet_exists:nTF
%    \end{macrocode}
% \begin{markdown}
%
% The option with key `snippet` invokes a snippet named \meta{value}.
%
% \end{markdown}
%  \begin{macrocode}
\keys_define:nn
  { markdown/options }
  {
    snippet .code:n = {
      \tl_set:NV
        \l_tmpa_tl
        \g_@@_current_theme_tl
      \tl_put_right:Nn
        \l_tmpa_tl
        { #1 }
      \@@_if_snippet_exists:nTF
        { #1 }
        {
          \prop_get:NVN
            \g_@@_snippets_prop
            \l_tmpa_tl
            \l_tmpb_tl
          \@@_setup:V
            \l_tmpb_tl
        }
        {
          \msg_error:nnV
            { markdown }
            { undefined-snippet }
            \l_tmpa_tl
        }
    }
  }
\msg_new:nnn
  { markdown }
  { undefined-snippet }
  { Can't~invoke~undefined~snippet~#1 }
\cs_generate_variant:Nn
  \@@_setup:n
  { V }
\ExplSyntaxOff
%    \end{macrocode}
% \iffalse
%</tex>
%<*manual-options>
% \fi
% \par
% \markdownBegin

Here is how we can use snippets to store options and invoke them later
in \LaTeX{}:

``` tex
\markdownSetupSnippet{romanNumerals}{
  renderers = {
      olItemWithNumber = {%
          \item[\romannumeral#1\relax.]%
      },
  },
}
\begin{markdown}

The following ordered list will be preceded by arabic numerals:

1. wahid
2. aithnayn

\end{markdown}
\begin{markdown}[snippet=romanNumerals]

The following ordered list will be preceded by roman numerals:

3. tres
4. quattuor

\end{markdown}
```````

If the `romanNumerals` snippet were defined in the `jdoe/lists` theme,
we could import the `jdoe/lists` theme and use the qualified name
`jdoe/lists/romanNumerals` to invoke the snippet:

``` tex
\markdownSetup{import=jdoe/lists}
\begin{markdown}[snippet=jdoe/lists/romanNumerals]

The following ordered list will be preceded by roman numerals:

3. tres
4. quattuor

\end{markdown}
```````

Alternatively, we can use the extended variant of the `import` \LaTeX{}
option that allows us to import the `romanNumerals` snippet to the
current namespace for easier access:

``` tex
\markdownSetup{
  import = {
    jdoe/lists = romanNumerals,
  },
}
\begin{markdown}[snippet=romanNumerals]

The following ordered list will be preceded by roman numerals:

3. tres
4. quattuor

\end{markdown}
```````

Furthermore, we can also specify the name of the snippet in the current
namespace, which can be different from the name of the snippet in the
`jdoe/lists` theme. For example, we can make the snippet
`jdoe/lists/romanNumerals` available under the name `roman`.

``` tex
\markdownSetup{
  import = {
    jdoe/lists = romanNumerals as roman,
  },
}
\begin{markdown}[snippet=roman]

The following ordered list will be preceded by roman numerals:

3. tres
4. quattuor

\end{markdown}
```````

Several themes and/or snippets can be loaded at once using the extended
variant of the `import` \LaTeX{} option:

``` tex
\markdownSetup{
  import = {
    jdoe/longpackagename/lists = {
      arabic as arabic1,
      roman,
      alphabetic,
    },
    jdoe/anotherlongpackagename/lists = {
      arabic as arabic2,
    },
    jdoe/yetanotherlongpackagename,
  },
}
```````

% \markdownEnd
% \iffalse
%</manual-options>
%<*tex>
% \fi
%  \begin{macrocode}
\ExplSyntaxOn
\tl_new:N
  \l_@@_import_current_theme_tl
\keys_define:nn
  { markdown/options/import }
  {
%    \end{macrocode}
% \begin{markdown}
%
% If a theme name is given without a list of snippets to import,
% we assume that an empty list was given.
%
% \end{markdown}
%  \begin{macrocode}
    unknown .default:n = {},
    unknown .code:n = {
%    \end{macrocode}
% \begin{markdown}
%
% To ensure that keys containing forward slashes get passed correctly, we
% replace all forward slashes in the input with backslash tokens with category
% code letter and then undo the replacement. This means that if any unbraced
% backslash tokens with category code letter exist in the input, they will be
% replaced with forward slashes. However, this should be extremely rare.
%
% \end{markdown}
%  \begin{macrocode}
      \tl_set_eq:NN
        \l_@@_import_current_theme_tl
        \l_keys_key_str
      \tl_replace_all:NVn
        \l_@@_import_current_theme_tl
        \c_backslash_str
        { / }
%    \end{macrocode}
% \begin{markdown}
%
% Here, we import the snippets.
%
% \end{markdown}
%  \begin{macrocode}
      \clist_map_inline:nn
        { #1 }
        {
          \regex_extract_once:nnNTF
            { ^(.*?)\s+as\s+(.*?)$ }
            { ##1 }
            \l_tmpa_seq
            {
              \seq_pop:NN
                \l_tmpa_seq
                \l_tmpa_tl
              \seq_pop:NN
                \l_tmpa_seq
                \l_tmpa_tl
              \seq_pop:NN
                \l_tmpa_seq
                \l_tmpb_tl
            }
            {
              \tl_set:Nn
                \l_tmpa_tl
                { ##1 }
              \tl_set:Nn
                \l_tmpb_tl
                { ##1 }
            }
          \tl_put_left:Nn
            \l_tmpa_tl
            { / }
          \tl_put_left:NV
            \l_tmpa_tl
            \l_@@_import_current_theme_tl
          \@@_setup_snippet:Vx
            \l_tmpb_tl
            { snippet = { \l_tmpa_tl } }
        }
%    \end{macrocode}
% \begin{markdown}
%
% Here, we load the theme.
%
% \end{markdown}
%  \begin{macrocode}
      \@@_set_theme:V
        \l_@@_import_current_theme_tl
    },
  }
\cs_generate_variant:Nn
  \tl_replace_all:Nnn
  { NVn }
\cs_generate_variant:Nn
  \@@_set_theme:n
  { V }
\cs_generate_variant:Nn
  \@@_setup_snippet:nn
  { Vx }
%    \end{macrocode}
% \iffalse
%</tex>
%<*manual-tokens>

## Markdown Tokens

A key feature of the Markdown package is the support for manipulating markdown
tokens, such as headings, emphasized text, links, and lists, in \TeX{}. Instead
of reducing \TeX{} to a PDF document producer, the Markdown package allows the
user to specify how every markdown token should be processed and rendered.

% \fi
% \begin{markdown}

### Token Renderers {#texrenderersuser}

% \end{markdown}
% \iffalse

Token renderers are user-defined \TeX{} macros, which render markdown tokens.
In this section, I will describe the individual token renderers.

% \fi
% \begin{markdown}
%
% The following \TeX{} macros may occur inside the output of the
% converter functions exposed by the Lua interface (see Section
% <#sec:lua-conversion>) and represent the parsed markdown tokens. These
% macros are intended to be redefined by the user who is typesetting a
% document. By default, they point to the corresponding prototypes (see Section
% <#sec:texrendererprototypes>).
%
% \end{markdown}
%
% \iffalse
%</manual-tokens>
%<*tex>
% \fi
% \begin{markdown}
%
% To enable the enumeration of token renderers, we will maintain the
% \mdef{g_\@\@_renderers_seq} sequence.
%
% \end{markdown}
%  \begin{macrocode}
\ExplSyntaxOn
\seq_new:N \g_@@_renderers_seq
%    \end{macrocode}
% \begin{markdown}
%
% To enable the reflection of token renderers and their parameters, we will
% maintain the \mdef{g_\@\@_renderer_arities_prop} property list.
%
% \end{markdown}
%  \begin{macrocode}
\prop_new:N \g_@@_renderer_arities_prop
\ExplSyntaxOff
%    \end{macrocode}
% \par
%
% \iffalse
%</tex>
%<*manual-tokens>
% \fi
%
% \begin{markdown}

#### Attribute Renderers
The following macros are only produced, when at least one of the following
options for markdown attributes on different elements is enabled:

- \Opt{autoIdentifiers}
- \Opt{fencedCodeAttributes}
- \Opt{gfmAutoIdentifiers}
- \Opt{headerAttributes}
- \Opt{inlineCodeAttributes}
- \Opt{linkAttributes}

\mdef{markdownRendererAttributeIdentifier} represents the \meta{identifier} of
a markdown element (`id="`\meta{identifier}`"` in HTML and `#`\meta{identifier}
in markdown attributes). The macro receives a single attribute that corresponds
to the \meta{identifier}.

\mdef{markdownRendererAttributeClassName} represents the \meta{class name} of a
markdown element (`class="`\meta{class name} ...`"` in HTML and
`.`\meta{class name} in markdown attributes).  The macro receives a single
attribute that corresponds to the \meta{class name}.

\mdef{markdownRendererAttributeKeyValue} represents a HTML attribute in the form
\meta{key}`=`\meta{value} that is neither an identifier nor a class name.
The macro receives two attributes that correspond to the \meta{key} and the
\meta{value}, respectively.

% \end{markdown}
%
% \iffalse

##### \LaTeX{} Example {.unnumbered}

Using a text editor, create a text document named `document.tex` with the
following content:
``` tex
\documentclass{article}
\usepackage[headerAttributes, underscores=false]{markdown}
\markdownSetup{
  renderers = {
    attributeIdentifier = {%
      \par
      \emph{(Identifier: #1)}
      \par
    },
    attributeClassName = {%
      \par
      \emph{(Class name: #1)}
      \par
    },
    attributeKeyValue = {%
      \par
      \emph{(Key: #1, Value: #2)}
      \par
    },
  },
}
\begin{document}
\begin{markdown}

# First top-level heading {jane=doe}

## A subheading {#identifier}

# Second top-level heading {.class_name}

\end{markdown}
\end{document}
```````
Next, invoke LuaTeX from the terminal:
``` sh
lualatex document.tex
``````
A PDF document named `document.pdf` should be produced and contain the
following text:

> # First top-level heading
>
> *(Key: Jane, Value: Doe)*
>
> ## A subheading
>
> *(Identifier: identifier)*
>
> # Second top-level heading
>
> *(Class name: class\_name)*

%</manual-tokens>
%<*tex>
% \fi
%
%  \begin{macrocode}
\def\markdownRendererAttributeIdentifier{%
  \markdownRendererAttributeIdentifierPrototype}%
\ExplSyntaxOn
\seq_gput_right:Nn
  \g_@@_renderers_seq
  { attributeIdentifier }
\prop_gput:Nnn
  \g_@@_renderer_arities_prop
  { attributeIdentifier }
  { 1 }
\ExplSyntaxOff
\def\markdownRendererAttributeClassName{%
  \markdownRendererAttributeClassNamePrototype}%
\ExplSyntaxOn
\seq_gput_right:Nn
  \g_@@_renderers_seq
  { attributeClassName }
\prop_gput:Nnn
  \g_@@_renderer_arities_prop
  { attributeClassName }
  { 1 }
\ExplSyntaxOff
\def\markdownRendererAttributeKeyValue{%
  \markdownRendererAttributeKeyValuePrototype}%
\ExplSyntaxOn
\seq_gput_right:Nn
  \g_@@_renderers_seq
  { attributeKeyValue }
\prop_gput:Nnn
  \g_@@_renderer_arities_prop
  { attributeKeyValue }
  { 2 }
\ExplSyntaxOff
%    \end{macrocode}
% \par
%
% \iffalse
%</tex>
%<*manual-tokens>
% \fi
%
% \begin{markdown}

#### Block Quote Renderers
The \mdef{markdownRendererBlockQuoteBegin} macro represents the beginning of
a block quote. The macro receives no arguments.

% \end{markdown}
%
% \iffalse
%</manual-tokens>
%<*tex>
% \fi
%
%  \begin{macrocode}
\def\markdownRendererBlockQuoteBegin{%
  \markdownRendererBlockQuoteBeginPrototype}%
\ExplSyntaxOn
\seq_gput_right:Nn
  \g_@@_renderers_seq
  { blockQuoteBegin }
\prop_gput:Nnn
  \g_@@_renderer_arities_prop
  { blockQuoteBegin }
  { 0 }
\ExplSyntaxOff
%    \end{macrocode}
% \par
%
% \iffalse
%</tex>
%<*manual-tokens>
% \fi
%
% \begin{markdown}

The \mdef{markdownRendererBlockQuoteEnd} macro represents the end of a block
quote. The macro receives no arguments.

% \end{markdown}
%
% \iffalse

##### Plain \TeX{} Example {.unnumbered}

Using a text editor, create a text document named `document.tex` with the
following content:
``` tex
\input markdown
\def\markdownRendererBlockQuoteBegin{%
  \begingroup
  \vskip\parindent
  \leftskip=2\parindent
  \parindent=0pt
}
\def\markdownRendererBlockQuoteEnd{%
  \par
  \vskip\parindent
  \endgroup
}
\markdownBegin
A quote from William Shakespeare's King Lear:

> This is the excellent foppery of the world that when we are
> sick in fortune---often the surfeit of our own behavior---we
> make guilty of our disasters the sun, the moon, and the
> stars [...]
\markdownEnd
\bye
```````
Next, invoke LuaTeX from the terminal:
``` sh
luatex document.tex
``````
A PDF document named `document.pdf` should be produced and contain the
following text:

> A quote from William Shakespeare's King Lear:
>
> > This is the excellent foppery of the world that when we are
> > sick in fortune—often the surfeit of our own behavior—we
> > make guilty of our disasters the sun, the moon, and the
> > stars [...]

##### \LaTeX{} Example {.unnumbered}

Using a text editor, create a text document named `document.tex` with the
following content:
``` tex
\documentclass{article}
\usepackage{markdown}
\markdownSetup{
  renderers = {
    blockQuoteBegin = {\begin{quote}},
    blockQuoteEnd = {\end{quote}},
  },
}
\begin{document}
\begin{markdown}
A quote from William Shakespeare's King Lear:

> This is the excellent foppery of the world that when we are
> sick in fortune---often the surfeit of our own behavior---we
> make guilty of our disasters the sun, the moon, and the
> stars [...]
\end{markdown}
\end{document}
```````
Next, invoke LuaTeX from the terminal:
``` sh
lualatex document.tex
``````
A PDF document named `document.pdf` should be produced and contain the
following text:

> A quote from William Shakespeare's King Lear:
>
> > This is the excellent foppery of the world that when we are
> > sick in fortune—often the surfeit of our own behavior—we
> > make guilty of our disasters the sun, the moon, and the
> > stars [...]

##### \Hologo{ConTeXt} Example {.unnumbered}

Using a text editor, create a text document named `document.tex` with the
following content:
``` tex
\usemodule[t][markdown]
\def\markdownRendererBlockQuoteBegin{\startquotation}
\def\markdownRendererBlockQuoteEnd{\stopquotation}
\starttext
\startmarkdown
A quote from William Shakespeare's King Lear:

> This is the excellent foppery of the world that when we are
> sick in fortune---often the surfeit of our own behavior---we
> make guilty of our disasters the sun, the moon, and the
> stars [...]
\stopmarkdown
\stoptext
```````
Next, invoke LuaTeX from the terminal:
``` sh
context --luatex document.tex
``````
A PDF document named `document.pdf` should be produced and contain the
following text:

> A quote from William Shakespeare's King Lear:
>
> > This is the excellent foppery of the world that when we are
> > sick in fortune—often the surfeit of our own behavior—we
> > make guilty of our disasters the sun, the moon, and the
> > stars [...]

%</manual-tokens>
%<*tex>
% \fi
%
%  \begin{macrocode}
\def\markdownRendererBlockQuoteEnd{%
  \markdownRendererBlockQuoteEndPrototype}%
\ExplSyntaxOn
\seq_gput_right:Nn
  \g_@@_renderers_seq
  { blockQuoteEnd }
\prop_gput:Nnn
  \g_@@_renderer_arities_prop
  { blockQuoteEnd }
  { 0 }
\ExplSyntaxOff
%    \end{macrocode}
% \par
%
% \iffalse
%</tex>
%<*manual-tokens>
% \fi
%
% \begin{markdown}

#### Bracketed Spans Attribute Context Renderers
The following macros are only produced, when the \Opt{bracketedSpans} option is
enabled.

The \mdef{markdownRendererBracketedSpanAttributeContextBegin} and
\mdef{markdownRendererBracketedSpanAttributeContextEnd} macros represent the
beginning and the end of a context in which the attributes of
an inline bracketed span apply. The macros receive no arguments.

% \end{markdown}
%
% \iffalse

##### \LaTeX{} Example {.unnumbered}

Using a text editor, create a text document named `document.tex` with the
following content:
``` tex
\documentclass{article}
\usepackage[bracketedSpans]{markdown}
\markdownSetup{
  renderers = {
    bracketedSpanAttributeContextBegin = {(},
    bracketedSpanAttributeContextEnd = {)},
  },
}
\begin{document}
\begin{markdown}

[foo [bar]{#identifier}]{key=value} [baz]{.class_name}

\end{markdown}
\end{document}
```````
Next, invoke LuaTeX from the terminal:
``` sh
lualatex document.tex
``````
A PDF document named `document.pdf` should be produced and contain the
following text:

> (foo (bar)) (baz)

%</manual-tokens>
%<*tex>
% \fi
%
%  \begin{macrocode}
\def\markdownRendererBracketedSpanAttributeContextBegin{%
  \markdownRendererBracketedSpanAttributeContextBeginPrototype}%
\ExplSyntaxOn
\seq_gput_right:Nn
  \g_@@_renderers_seq
  { bracketedSpanAttributeContextBegin }
\prop_gput:Nnn
  \g_@@_renderer_arities_prop
  { bracketedSpanAttributeContextBegin }
  { 0 }
\ExplSyntaxOff
\def\markdownRendererBracketedSpanAttributeContextEnd{%
  \markdownRendererBracketedSpanAttributeContextEndPrototype}%
\ExplSyntaxOn
\seq_gput_right:Nn
  \g_@@_renderers_seq
  { bracketedSpanAttributeContextEnd }
\prop_gput:Nnn
  \g_@@_renderer_arities_prop
  { bracketedSpanAttributeContextEnd }
  { 0 }
\ExplSyntaxOff
%    \end{macrocode}
% \par
%
% \iffalse
%</tex>
%<*manual-tokens>
% \fi
%
% \begin{markdown}

#### Bullet List Renderers
The \mdef{markdownRendererUlBegin} macro represents the beginning of a
bulleted list that contains an item with several paragraphs of text (the
list is not tight). The macro receives no arguments.

% \end{markdown}
%
% \iffalse
%</manual-tokens>
%<*tex>
% \fi
%
%  \begin{macrocode}
\def\markdownRendererUlBegin{%
  \markdownRendererUlBeginPrototype}%
\ExplSyntaxOn
\seq_gput_right:Nn
  \g_@@_renderers_seq
  { ulBegin }
\prop_gput:Nnn
  \g_@@_renderer_arities_prop
  { ulBegin }
  { 0 }
\ExplSyntaxOff
%    \end{macrocode}
% \par
%
% \iffalse
%</tex>
%<*manual-tokens>
% \fi
%
% \begin{markdown}

The \mdef{markdownRendererUlBeginTight} macro represents the beginning of a
bulleted list that contains no item with several paragraphs of text (the list
is tight). This macro will only be produced, when the \Opt{tightLists} option
is disabled. The macro receives no arguments.

% \end{markdown}
%
% \iffalse
%</manual-tokens>
%<*tex>
% \fi
%
%  \begin{macrocode}
\def\markdownRendererUlBeginTight{%
  \markdownRendererUlBeginTightPrototype}%
\ExplSyntaxOn
\seq_gput_right:Nn
  \g_@@_renderers_seq
  { ulBeginTight }
\prop_gput:Nnn
  \g_@@_renderer_arities_prop
  { ulBeginTight }
  { 0 }
\ExplSyntaxOff
%    \end{macrocode}
% \par
%
% \iffalse
%</tex>
%<*manual-tokens>
% \fi
%
% \begin{markdown}

The \mdef{markdownRendererUlItem} macro represents an item in a bulleted
list. The macro receives no arguments.

% \end{markdown}
%
% \iffalse
%</manual-tokens>
%<*tex>
% \fi
%
%  \begin{macrocode}
\def\markdownRendererUlItem{%
  \markdownRendererUlItemPrototype}%
\ExplSyntaxOn
\seq_gput_right:Nn
  \g_@@_renderers_seq
  { ulItem }
\prop_gput:Nnn
  \g_@@_renderer_arities_prop
  { ulItem }
  { 0 }
\ExplSyntaxOff
%    \end{macrocode}
% \par
%
% \iffalse
%</tex>
%<*manual-tokens>
% \fi
%
% \begin{markdown}

The \mdef{markdownRendererUlItemEnd} macro represents the end of an item in a
bulleted list. The macro receives no arguments.

% \end{markdown}
%
% \iffalse
%</manual-tokens>
%<*tex>
% \fi
%
%  \begin{macrocode}
\def\markdownRendererUlItemEnd{%
  \markdownRendererUlItemEndPrototype}%
\ExplSyntaxOn
\seq_gput_right:Nn
  \g_@@_renderers_seq
  { ulItemEnd }
\prop_gput:Nnn
  \g_@@_renderer_arities_prop
  { ulItemEnd }
  { 0 }
\ExplSyntaxOff
%    \end{macrocode}
% \par
%
% \iffalse
%</tex>
%<*manual-tokens>
% \fi
%
% \begin{markdown}

The \mdef{markdownRendererUlEnd} macro represents the end of a bulleted list
that contains an item with several paragraphs of text (the list is not
tight). The macro receives no arguments.

% \end{markdown}
%
% \iffalse
%</manual-tokens>
%<*tex>
% \fi
%
%  \begin{macrocode}
\def\markdownRendererUlEnd{%
  \markdownRendererUlEndPrototype}%
\ExplSyntaxOn
\seq_gput_right:Nn
  \g_@@_renderers_seq
  { ulEnd }
\prop_gput:Nnn
  \g_@@_renderer_arities_prop
  { ulEnd }
  { 0 }
\ExplSyntaxOff
%    \end{macrocode}
% \par
%
% \iffalse
%</tex>
%<*manual-tokens>
% \fi
%
% \begin{markdown}

The \mdef{markdownRendererUlEndTight} macro represents the end of a bulleted
list that contains no item with several paragraphs of text (the list is
tight). This macro will only be produced, when the \Opt{tightLists} option is
disabled. The macro receives no arguments.

% \end{markdown}
%
% \iffalse

##### Plain \TeX{} Example {.unnumbered}

Using a text editor, create a text document named `document.tex` with the
following content:
``` tex
\input markdown
\def\markdownOptionTightLists{true}

\def\markdownRendererInterblockSeparator{}
\def\markdownRendererUlBeginTight{ (}
\def\markdownRendererUlItem{%
  \def\markdownRendererUlItem{%
    ,
    \def\markdownRendererUlItem{, and }%
  }%
}
\def\markdownRendererUlItemEnd{}
\def\markdownRendererUlEndTight{).}

\markdownBegin
This is a tight list

- the first item
- the second item
- the third item
\markdownEnd

\def\markdownRendererInterblockSeparator{%
  :\par
  \def\markdownRendererInterblockSeparator{\par}%
}
\def\markdownRendererUlBegin{}
\def\markdownRendererUlItem{--\kern 0.5em}
\def\markdownRendererUlItemEnd{.\par}
\def\markdownRendererUlEnd{}

\markdownBegin
This is a loose list

- This is the first item

- This is the second item

- This is the third item
\markdownEnd

\bye
```````
Next, invoke LuaTeX from the terminal:
``` sh
luatex document.tex
``````
A PDF document named `document.pdf` should be produced and contain the
following text:

> This is a tight list (the first item, the second item, and the third item).
>
> This is a loose list:
>
> - This is the first item.
>
> - This is the second item.
>
> - This is the third item.

##### \LaTeX{} Example {.unnumbered}

Using a text editor, create a text document named `document.tex` with the
following content:
``` tex
\documentclass{article}
\usepackage[tightLists]{markdown}
\begin{document}

\begin{markdown}[
  renderers = {
    interblockSeparator = {},
    ulBeginTight = { (},
    ulItem = {%
      \def\markdownRendererUlItem{%
        ,
        \def\markdownRendererUlItem{, and }%
      }%
    },
    ulItemEnd = {},
    ulEndTight = {).},
  },
]
This is a tight list

- the first item
- the second item
- the third item
\end{markdown}

\begin{markdown}[
  renderers = {
    interblockSeparator = {%
      :\par
      \def\markdownRendererInterblockSeparator{\par}%
    },
    ulBeginTight = {\begin{itemize}},
    ulItem = {\item},
    ulItemEnd = {.},
    ulEnd = {\end{itemize}},
  },
]
This is a loose list

- This is the first item

- This is the second item

- This is the third item
\end{markdown}

\end{document}
```````
Next, invoke LuaTeX from the terminal:
``` sh
lualatex document.tex
``````
A PDF document named `document.pdf` should be produced and contain the
following text:

> This is a tight list (the first item, the second item, and the third item).
>
> This is a loose list:
>
> - This is the first item.
>
> - This is the second item.
>
> - This is the third item.

##### \Hologo{ConTeXt} Example {.unnumbered}

Using a text editor, create a text document named `document.tex` with the
following content:
``` tex
\usemodule[t][markdown]
\setupmarkdown[tightLists = yes]
\starttext

\def\markdownRendererInterblockSeparator{}
\def\markdownRendererUlBeginTight{ (}
\def\markdownRendererUlItem{%
  \def\markdownRendererUlItem{%
    ,
    \def\markdownRendererUlItem{, and }%
  }%
}
\def\markdownRendererUlItemEnd{}
\def\markdownRendererUlEndTight{).}

\startmarkdown
This is a tight list

- the first item
- the second item
- the third item
\stopmarkdown

\def\markdownRendererInterblockSeparator{%
  :\par
  \def\markdownRendererInterblockSeparator{\par}%
}
\def\markdownRendererUlBegin{\startitemize}
\def\markdownRendererUlItem{\item}
\def\markdownRendererUlItemEnd{.}
\def\markdownRendererUlEnd{\stopitemize}

\startmarkdown
This is a loose list

- This is the first item

- This is the second item

- This is the third item
\stopmarkdown

\stoptext
```````
Next, invoke LuaTeX from the terminal:
``` sh
context --luatex document.tex
``````
A PDF document named `document.pdf` should be produced and contain the
following text:

> This is a tight list (the first item, the second item, and the third item).
>
> This is a loose list:
>
> - This is the first item.
>
> - This is the second item.
>
> - This is the third item.

%</manual-tokens>
%<*tex>
% \fi
%
%  \begin{macrocode}
\def\markdownRendererUlEndTight{%
  \markdownRendererUlEndTightPrototype}%
\ExplSyntaxOn
\seq_gput_right:Nn
  \g_@@_renderers_seq
  { ulEndTight }
\prop_gput:Nnn
  \g_@@_renderer_arities_prop
  { ulEndTight }
  { 0 }
\ExplSyntaxOff
%    \end{macrocode}
% \par
%
% \iffalse
%</tex>
%<*manual-tokens>
% \fi
%
% \begin{markdown}

#### Citation Renderers
The \mdef{markdownRendererCite} macro represents a string of one or more
parenthetical citations. This macro will only be produced, when the
\Opt{citations} option is enabled. The macro receives the parameter
`{`\meta{number of citations}`}` followed by \meta{suppress author}
`{`\meta{prenote}`}{`\meta{postnote}`}{`\meta{name}`}` repeated
\meta{number of citations} times. The \meta{suppress author} parameter is
either the token `-`, when the author's name is to be suppressed, or `+`
otherwise.

% \end{markdown}
%
% \iffalse

##### \LaTeX{} Example {.unnumbered}

Using a text editor, create a text document named `document.tex` with the
following content:
``` tex
\documentclass{article}
\usepackage[citations]{markdown}
\newcount\citationsCounter
\newcount\citationsTotal
\makeatletter
\def\citations#1#2#3#4{%
  a parenthesized citation \emph{#4}
  \advance\citationsCounter by 1\relax
  \ifx\relax#2\relax
    \ifx\relax#3\relax\else
      with a postfix \emph{#3}%
    \fi
  \else
    with a prefix \emph{#2}%
    \ifx\relax#3\relax\else
      \ and a postfix \emph{#3}%
    \fi
  \fi
  \ifnum\citationsCounter>\citationsTotal\relax
    .%
    \expandafter\@gobble
  \else
    , and
  \fi\citations}
\makeatother
\markdownSetup{
  renderers = {
    cite = {%
      \citationsCounter=1%
      \citationsTotal=#1%
      This is
      \expandafter\citations
    },
  },
}
\begin{document}
\begin{markdown}
[see @abrahams90, pp. 12; @eijkhout91, pp. 34]
\end{markdown}
\end{document}
```````
Next, invoke LuaTeX from the terminal:
``` sh
lualatex document.tex
``````
A PDF document named `document.pdf` should be produced and contain the
following text:

> This is a parenthesized citation *abrahams90* with a prefix see
> and a postfix *pp. 12*, and a citation *eijkhout91* with a
> postfix *pp. 34*.

%</manual-tokens>
%<*tex>
% \fi
%
%  \begin{macrocode}
\def\markdownRendererCite{%
  \markdownRendererCitePrototype}%
\ExplSyntaxOn
\seq_gput_right:Nn
  \g_@@_renderers_seq
  { cite }
\prop_gput:Nnn
  \g_@@_renderer_arities_prop
  { cite }
  { 1 }
\ExplSyntaxOff
%    \end{macrocode}
% \par
%
% \iffalse
%</tex>
%<*manual-tokens>
% \fi
%
% \begin{markdown}

The \mdef{markdownRendererTextCite} macro represents a string of one or more
text citations. This macro will only be produced, when the
\Opt{citations} option is enabled. The macro receives parameters in the same
format as the \mref{markdownRendererCite} macro.

% \end{markdown}
%
% \iffalse

##### \LaTeX{} Example {.unnumbered}

Using a text editor, create a text document named `document.tex` with the
following content:
``` tex
\documentclass{article}
\usepackage[citations]{markdown}
\newcount\citationsCounter
\newcount\citationsTotal
\makeatletter
\def\citations#1#2#3#4{%
  a text citation \emph{#4}
  \advance\citationsCounter by 1\relax
  \ifx\relax#2\relax
    \ifx\relax#3\relax\else
      with a postfix \emph{#3}%
    \fi
  \else
    with a prefix \emph{#2}%
    \ifx\relax#3\relax\else
      \ and a postfix \emph{#3}%
    \fi
  \fi
  \ifnum\citationsCounter>\citationsTotal\relax
    .%
    \expandafter\@gobble
  \else
    , and
  \fi\citations}
\makeatother
\markdownSetup{
  renderers = {
    textCite = {%
      \citationsCounter=1%
      \citationsTotal=#1%
      This is
      \expandafter\citations
    },
  },
}
\begin{document}
\begin{markdown}
@abrahams90 [pp. 12; also @eijkhout91]
\end{markdown}
\end{document}
```````
Next, invoke LuaTeX from the terminal:
``` sh
lualatex document.tex
``````
A PDF document named `document.pdf` should be produced and contain the
following text:

> This is a text citation *abrahams90* with a postfix *pp. 12*,
> and a citation *eijkhout91* with a prefix *also*.

%</manual-tokens>
%<*tex>
% \fi
%
%  \begin{macrocode}
\def\markdownRendererTextCite{%
  \markdownRendererTextCitePrototype}%
\ExplSyntaxOn
\seq_gput_right:Nn
  \g_@@_renderers_seq
  { textCite }
\prop_gput:Nnn
  \g_@@_renderer_arities_prop
  { textCite }
  { 1 }
\ExplSyntaxOff
%    \end{macrocode}
% \par
%
% \iffalse
%</tex>
%<*manual-tokens>
% \fi
%
% \begin{markdown}

#### Code Block Renderers
The \mdef{markdownRendererInputVerbatim} macro represents a code
block. The macro receives a single argument that corresponds to the
filename of a file containing the code block contents.

% \end{markdown}
%
% \iffalse
%</manual-tokens>
%<*tex>
% \fi
%
%  \begin{macrocode}
\def\markdownRendererInputVerbatim{%
  \markdownRendererInputVerbatimPrototype}%
\ExplSyntaxOn
\seq_gput_right:Nn
  \g_@@_renderers_seq
  { inputVerbatim }
\prop_gput:Nnn
  \g_@@_renderer_arities_prop
  { inputVerbatim }
  { 1 }
\ExplSyntaxOff
%    \end{macrocode}
% \par
%
% \iffalse
%</tex>
%<*manual-tokens>
% \fi
%
% \begin{markdown}

The \mdef{markdownRendererInputFencedCode} macro represents a fenced code
block. This macro will only be produced, when the \Opt{fencedCode} option is
enabled. The macro receives three arguments that correspond to the filename of
a file containing the code block contents, the fully escaped code fence infostring
that can be directly typeset, and the raw code fence infostring that can be used
outside typesetting.

% \end{markdown}
%
% \iffalse

##### \LaTeX{} Example {.unnumbered}

Using a text editor, create a text document named `document.tex` with the
following content:
``` tex
\documentclass{article}
\usepackage{verbatim}
\usepackage[hyphens]{url}
\usepackage[fencedCode]{markdown}
\markdownSetup{
  renderers = {
    interblockSeparator = {
      \def\markdownRendererInterblockSeparator{%
        \par
        \def\markdownRendererInterblockSeparator{%
          \def\markdownRendererInterblockSeparator{%
            \par
          }%
        }%
      }%
    },
    inputVerbatim = {
      is contained in file \url{#1}:%
      \verbatiminput{#1}%
    },
    inputFencedCode = {
      in #2 \markdownRendererInputVerbatim{#1}%
    },
  },
}
\begin{document}
\begin{markdown}
The following code

    def foo(bar):
      if len(bar) <= 1:
        return bar[0]
      elif len(bar) == 2:
        return sorted(bar)
      else:
        baz = len(bar) // 2
        return foo(bar[baz:], bar[:baz])

The following code

~~~ Python
>>> foo([4, 2, 1, 3])
[1, 2, 3, 4]
~~~~~~~~~~
\end{markdown}
\end{document}
```````
Next, invoke LuaTeX from the terminal:
``` sh
lualatex document.tex
``````
A PDF document named `document.pdf` should be produced and contain the
following text except for the filename, which may differ:

> The following code is contained in file
> `./_markdown_document/882453149edcf288976647f6fe147ada.verbatim`:
> ``` py
> def foo(bar):
>   if len(bar) <= 1:
>     return bar[:1]
>   elif len(bar) == 2:
>     return sorted(bar)
>   else:
>     baz = bar[len(bar) // 2]
>     return (
>       foo([qux for qux in bar if qux < baz]) + [baz] +
>       foo([qux for qux in bar if qux > baz])
>     )
> ``````
> The following code in Python contained in file
> `./_markdown_document/cf2a96e2120cef5b1fae5fea36fcc27b.verbatim`:
> ``` py
> >>> foo([4, 2, 1, 3])
> [1, 2, 3, 4]
> ``````

%</manual-tokens>
%<*tex>
% \fi
%
%  \begin{macrocode}
\def\markdownRendererInputFencedCode{%
  \markdownRendererInputFencedCodePrototype}%
\ExplSyntaxOn
\seq_gput_right:Nn
  \g_@@_renderers_seq
  { inputFencedCode }
\prop_gput:Nnn
  \g_@@_renderer_arities_prop
  { inputFencedCode }
  { 3 }
\ExplSyntaxOff
%    \end{macrocode}
% \par
%
% \iffalse
%</tex>
%<*manual-tokens>
% \fi
% \begin{markdown}

#### Code Span Renderer
The \mdef{markdownRendererCodeSpan} macro represents inline code span in the
input text. It receives a single argument that corresponds to the inline
code span.

% \end{markdown}
%
% \iffalse

##### Plain \TeX{} Example {.unnumbered}

Using a text editor, create a text document named `document.tex` with the
following content:
``` tex
\input markdown
\input lmfonts

\def\markdownRendererCodeSpan#1{#1}

\markdownBegin
`$\sqrt{-1}$ *equals* $i$`

$\sqrt{-1}$ *equals* $i$
\markdownEnd

\def\markdownOptionHybrid{true}
\markdownBegin
$\sqrt{-1}$ *equals* $i$
\markdownEnd

\bye
```````
Next, invoke LuaTeX from the terminal:
``` sh
luatex document.tex
``````
A PDF document named `document.pdf` should be produced and contain the
following text:

> \$\\sqrt{-1}\$ \*equals\* \$i\$.
>
> \$\\sqrt{-1}\$ *equals* \$i\$.
>
> <math><msqrt><mo>−</mo><mn>1</mn></msqrt></math>
> *equals*
> <math><mi>i</mi></math>.

##### \LaTeX{} Example {.unnumbered}

Using a text editor, create a text document named `document.tex` with the
following content:
``` tex
\documentclass{article}
\usepackage[smartEllipses]{markdown}
\markdownSetup{
  renderers = {
    codeSpan = {#1},
  },
}
\begin{document}

\begin{markdown}
`$\sqrt{-1}$ *equals* $i$`

$\sqrt{-1}$ *equals* $i$
\end{markdown}

\begin{markdown}[hybrid]
$\sqrt{-1}$ *equals* $i$
\end{markdown}

\end{document}
```````
Next, invoke LuaTeX from the terminal:
``` sh
lualatex document.tex
``````
A PDF document named `document.pdf` should be produced and contain the
following text:

> \$\\sqrt{-1}\$ \*equals\* \$i\$.
>
> \$\\sqrt{-1}\$ *equals* \$i\$.
>
> <math><msqrt><mo>−</mo><mn>1</mn></msqrt></math>
> *equals*
> <math><mi>i</mi></math>.

##### \Hologo{ConTeXt} Example {.unnumbered}

Using a text editor, create a text document named `document.tex` with the
following content:
``` tex
\usemodule[t][markdown]
\def\markdownRendererCodeSpan#1{#1}
\starttext

\startmarkdown
`$\sqrt{-1}$ *equals* $i$`

$\sqrt{-1}$ *equals* $i$
\stopmarkdown

\setupmarkdown[hybrid = yes]
\startmarkdown
$\sqrt{-1}$ *equals* $i$
\stopmarkdown

\bye
```````
Next, invoke LuaTeX from the terminal:
``` sh
context --luatex document.tex
``````
A PDF document named `document.pdf` should be produced and contain the
following text:

> \$\\sqrt{-1}\$ \*equals\* \$i\$.
>
> \$\\sqrt{-1}\$ *equals* \$i\$.
>
> <math><msqrt><mo>−</mo><mn>1</mn></msqrt></math>
> *equals*
> <math><mi>i</mi></math>.

%</manual-tokens>
%<*tex>
% \fi
%
%  \begin{macrocode}
\def\markdownRendererCodeSpan{%
  \markdownRendererCodeSpanPrototype}%
\ExplSyntaxOn
\seq_gput_right:Nn
  \g_@@_renderers_seq
  { codeSpan }
\prop_gput:Nnn
  \g_@@_renderer_arities_prop
  { codeSpan }
  { 1 }
\ExplSyntaxOff
%    \end{macrocode}
% \par
%
% \iffalse
%</tex>
%<*manual-tokens>
% \fi
%
% \begin{markdown}

#### Code Span Attribute Context Renderers
The following macros are only produced, when the \Opt{inlineCodeAttributes}
option is enabled.

The \mdef{markdownRendererCodeSpanAttributeContextBegin} and
\mdef{markdownRendererCodeSpanAttributeContextEnd} macros represent the beginning
and the end of a context in which the attributes of an inline code span apply.
The macros receive no arguments.

% \end{markdown}
%
% \iffalse

##### \LaTeX{} Example {.unnumbered}

Using a text editor, create a text document named `document.tex` with the
following content:
``` tex
\documentclass{article}
\usepackage[inlineCodeAttributes]{markdown}
\markdownSetup{
  renderers = {
    codeSpanAttributeContextBegin = {(},
    codeSpan = {#1},
    codeSpanAttributeContextEnd = {)},
  },
}
\begin{document}
\begin{markdown}

foo `bar`{key=value} baz

\end{markdown}
\end{document}
```````
Next, invoke LuaTeX from the terminal:
``` sh
lualatex document.tex
``````
A PDF document named `document.pdf` should be produced and contain the
following text:

> foo (bar) baz

%</manual-tokens>
%<*tex>
% \fi
%
%  \begin{macrocode}
\def\markdownRendererCodeSpanAttributeContextBegin{%
  \markdownRendererCodeSpanAttributeContextBeginPrototype}%
\ExplSyntaxOn
\seq_gput_right:Nn
  \g_@@_renderers_seq
  { codeSpanAttributeContextBegin }
\prop_gput:Nnn
  \g_@@_renderer_arities_prop
  { codeSpanAttributeContextBegin }
  { 0 }
\ExplSyntaxOff
\def\markdownRendererCodeSpanAttributeContextEnd{%
  \markdownRendererCodeSpanAttributeContextEndPrototype}%
\ExplSyntaxOn
\seq_gput_right:Nn
  \g_@@_renderers_seq
  { codeSpanAttributeContextEnd }
\prop_gput:Nnn
  \g_@@_renderer_arities_prop
  { codeSpanAttributeContextEnd }
  { 0 }
\ExplSyntaxOff
%    \end{macrocode}
% \par
%
% \iffalse
%</tex>
%<*manual-tokens>
% \fi
%
% \begin{markdown}

#### Content Block Renderers {#texcontentblockrenderers}

The \mdef{markdownRendererContentBlock} macro represents an iA\\,Writer content
block. It receives four arguments: the local file or online image filename
extension cast to the lower case, the fully escaped \acro{uri} that can be
directly typeset, the raw \acro{uri} that can be used outside typesetting,
and the title of the content block.

% \end{markdown}
%
% \iffalse
%</manual-tokens>
%<*tex>
% \fi
%
%  \begin{macrocode}
\def\markdownRendererContentBlock{%
  \markdownRendererContentBlockPrototype}%
\ExplSyntaxOn
\seq_gput_right:Nn
  \g_@@_renderers_seq
  { contentBlock }
\prop_gput:Nnn
  \g_@@_renderer_arities_prop
  { contentBlock }
  { 4 }
\ExplSyntaxOff
%    \end{macrocode}
% \par
%
% \iffalse
%</tex>
%<*manual-tokens>
% \fi
%
% \begin{markdown}

The \mdef{markdownRendererContentBlockOnlineImage} macro represents an
iA\\,Writer online image content block. The macro receives the same arguments
as \mref{markdownRendererContentBlock}.

% \end{markdown}
%
% \iffalse
%</manual-tokens>
%<*tex>
% \fi
%
%  \begin{macrocode}
\def\markdownRendererContentBlockOnlineImage{%
  \markdownRendererContentBlockOnlineImagePrototype}%
\ExplSyntaxOn
\seq_gput_right:Nn
  \g_@@_renderers_seq
  { contentBlockOnlineImage }
\prop_gput:Nnn
  \g_@@_renderer_arities_prop
  { contentBlockOnlineImage }
  { 4 }
\ExplSyntaxOff
%    \end{macrocode}
% \par
%
% \iffalse
%</tex>
%<*manual-tokens>
% \fi
%
% \begin{markdown}

The \mdef{markdownRendererContentBlockCode} macro represents an iA\\,Writer
content block that was recognized as a file in a known programming language
by its filename extension $s$. If any `markdown-languages.json` file found
by \pkg{kpathsea}^[Filenames other than `markdown-languages.json` may
be specified using the `contentBlocksLanguageMap` Lua option.] contains a
record $(k, v)$, then a non-online-image content block with the filename
extension $s, s$`:lower()`${}=k$ is considered to be in a known programming
language $v$.
The macro receives five arguments: the local file name extension $s$ cast to
the lower case, the language $v$, the fully escaped \acro{uri} that can be
directly typeset, the raw \acro{uri} that can be used outside typesetting,
and the title of the content block.

Note that you will need to place place a `markdown-languages.json` file
inside your working directory or inside your local \TeX{} directory
structure.  In this file, you will define a mapping between filename
extensions and the language names recognized by your favorite syntax
highlighter; there may exist other creative uses beside syntax highlighting.
% The `Languages.json` file provided by @sotkov17 is a good starting point.
% \end{markdown}
%
% \iffalse
[The `Languages.json` file provided by Anton Sotkov][sotkov17] is a good
starting point.

 [sotkov17]: https://github.com/iainc/Markdown-Content-Blocks
             (File transclusion syntax for Markdown)

##### Plain \TeX{} Example {.unnumbered}

Using a text editor, create a text document named `document.tex` with the
following content:
``` tex
\input markdown
\def\markdownOptionContentBlocks{true}
\def\markdownRendererContentBlock#1#2#3#4{%
  This is {\tt #2}, #4.
}
\def\markdownRendererContentBlockOnlineImage#1#2#3#4{%
  This is the image {\tt #2}, #4.
}
\def\markdownRendererContentBlockCode#1#2#3#4#5{%
  This is the #2 (\uppercase{#1}) document {\tt #3}, #5.
}
\markdownBegin
/document.tex (the document that we are currently typesetting)
/markdown-languages.json (the mapping between filename extensions
                          and programming language names)
https://tug.org/tugboat/noword.jpg (the logotype of TUGboat)
\markdownEnd
\bye
```````
Create also a text document named `markdown-languages.json` with the following
content:
``` js
{
  "json": "JavaScript Object Notation",
}
``````
Next, invoke LuaTeX from the terminal:
``` sh
luatex document.tex
``````
A PDF document named `document.pdf` should be produced and contain the
following text:

> This is `document.tex`, the document that we are currently typesetting.
>
> This is the JavaScript Object Notation (JSON) document
> `markdown-languages.json`, the mapping between filename extensions and
> programming language names.
>
> This is the image `https://tug.org/tugboat/noword.jpg`, the logotype of
> TUGboat.

##### \LaTeX{} Example {.unnumbered}

Using a text editor, create a text document named `document.tex` with the
following content:
``` tex
\documentclass{article}
\begin{filecontents}[overwrite,nosearch,noheader]{markdown-languages.json}
{
  "json": "JavaScript Object Notation",
}
\end{filecontents}
\usepackage[contentBlocks]{markdown}
\markdownSetup{
  renderers = {
    contentBlock = {This is \texttt{#2}, #4.},
    contentBlockOnlineImage = {This is the image \texttt{#2}, #4.},
    contentBlockCode = {%
      This is the #2 (\MakeUppercase{#1}) document \texttt{#3}, #5.
    },
  },
}
\begin{document}
\begin{markdown}
/document.tex (the document that we are currently typesetting)
/markdown-languages.json (the mapping between filename extensions
                          and programming language names)
https://tug.org/tugboat/noword.jpg (the logotype of TUGboat)
\end{markdown}
\end{document}
```````
Next, invoke LuaTeX from the terminal:
``` sh
lualatex document.tex
``````
A PDF document named `document.pdf` should be produced and contain the
following text:

> This is `document.tex`, the document that we are currently typesetting.
>
> This is the JavaScript Object Notation (JSON) document
> `markdown-languages.json`, the mapping between filename extensions and
> programming language names.
>
> This is the image `https://tug.org/tugboat/noword.jpg`, the logotype of
> TUGboat.

##### \Hologo{ConTeXt} Example {.unnumbered}

Using a text editor, create a text document named `document.tex` with the
following content:
``` tex
\usemodule[t][markdown]
\setupmarkdown[contentBlocks = yes]
\def\markdownRendererContentBlock#1#2#3#4{%
  This is {\tt #2}, #4.
}
\def\markdownRendererContentBlockOnlineImage#1#2#3#4{%
  This is the image {\tt #2}, #4.
}
\def\markdownRendererContentBlockCode#1#2#3#4#5{%
  This is the #2 (\uppercase{#1}) document {\tt #3}, #5.
}
\starttext
\startmarkdown
/document.tex (the document that we are currently typesetting)
/markdown-languages.json (the mapping between filename extensions
                          and programming language names)
https://tug.org/tugboat/noword.jpg (the logotype of TUGboat)
\stopmarkdown
\stoptext
```````
Create also a text document named `markdown-languages.json` with the following
content:
``` js
{
  "json": "JavaScript Object Notation",
}
``````
Next, invoke LuaTeX from the terminal:
``` sh
context --luatex document.tex
``````
A PDF document named `document.pdf` should be produced and contain the
following text:

> This is `document.tex`, the document that we are currently typesetting.
>
> This is the JavaScript Object Notation (JSON) document
> `markdown-languages.json`, the mapping between filename extensions and
> programming language names.
>
> This is the image `https://tug.org/tugboat/noword.jpg`, the logotype of
> TUGboat.

%</manual-tokens>
%<*tex>
% \fi
%
%  \begin{macrocode}
\def\markdownRendererContentBlockCode{%
  \markdownRendererContentBlockCodePrototype}%
\ExplSyntaxOn
\seq_gput_right:Nn
  \g_@@_renderers_seq
  { contentBlockCode }
\prop_gput:Nnn
  \g_@@_renderer_arities_prop
  { contentBlockCode }
  { 5 }
\ExplSyntaxOff
%    \end{macrocode}
% \par
%
% \iffalse
%</tex>
%<*manual-tokens>
% \fi
%
% \begin{markdown}

#### Definition List Renderers
The following macros are only produced, when the \Opt{definitionLists} option
is enabled.

The \mdef{markdownRendererDlBegin} macro represents the beginning of a
definition list that contains an item with several paragraphs of text (the
list is not tight). The macro receives no arguments.

% \end{markdown}
%
% \iffalse
%</manual-tokens>
%<*tex>
% \fi
%
%  \begin{macrocode}
\def\markdownRendererDlBegin{%
  \markdownRendererDlBeginPrototype}%
\ExplSyntaxOn
\seq_gput_right:Nn
  \g_@@_renderers_seq
  { dlBegin }
\prop_gput:Nnn
  \g_@@_renderer_arities_prop
  { dlBegin }
  { 0 }
\ExplSyntaxOff
%    \end{macrocode}
% \par
%
% \iffalse
%</tex>
%<*manual-tokens>
% \fi
%
% \begin{markdown}

The \mdef{markdownRendererDlBeginTight} macro represents the beginning of a
definition list that contains no item with several paragraphs of text (the
list is tight). This macro will only be produced, when the
\Opt{tightLists} option is disabled. The macro receives no arguments.

% \end{markdown}
%
% \iffalse
%</manual-tokens>
%<*tex>
% \fi
%
%  \begin{macrocode}
\def\markdownRendererDlBeginTight{%
  \markdownRendererDlBeginTightPrototype}%
\ExplSyntaxOn
\seq_gput_right:Nn
  \g_@@_renderers_seq
  { dlBeginTight }
\prop_gput:Nnn
  \g_@@_renderer_arities_prop
  { dlBeginTight }
  { 0 }
\ExplSyntaxOff
%    \end{macrocode}
% \par
%
% \iffalse
%</tex>
%<*manual-tokens>
% \fi
%
% \begin{markdown}

The \mdef{markdownRendererDlItem} macro represents a term in a definition
list. The macro receives a single argument that corresponds to the term
being defined.

% \end{markdown}
%
% \iffalse
%</manual-tokens>
%<*tex>
% \fi
%
%  \begin{macrocode}
\def\markdownRendererDlItem{%
  \markdownRendererDlItemPrototype}%
\ExplSyntaxOn
\seq_gput_right:Nn
  \g_@@_renderers_seq
  { dlItem }
\prop_gput:Nnn
  \g_@@_renderer_arities_prop
  { dlItem }
  { 1 }
\ExplSyntaxOff
%    \end{macrocode}
% \par
%
% \iffalse
%</tex>
%<*manual-tokens>
% \fi
%
% \begin{markdown}

The \mdef{markdownRendererDlItemEnd} macro represents the end of a list of
definitions for a single term.

% \end{markdown}
%
% \iffalse
%</manual-tokens>
%<*tex>
% \fi
%
%  \begin{macrocode}
\def\markdownRendererDlItemEnd{%
  \markdownRendererDlItemEndPrototype}%
\ExplSyntaxOn
\seq_gput_right:Nn
  \g_@@_renderers_seq
  { dlItemEnd }
\prop_gput:Nnn
  \g_@@_renderer_arities_prop
  { dlItemEnd }
  { 0 }
\ExplSyntaxOff
%    \end{macrocode}
% \par
%
% \iffalse
%</tex>
%<*manual-tokens>
% \fi
%
% \begin{markdown}

The \mdef{markdownRendererDlDefinitionBegin} macro represents the beginning
of a definition in a definition list. There can be several definitions for
a single term.

% \end{markdown}
%
% \iffalse
%</manual-tokens>
%<*tex>
% \fi
%
%  \begin{macrocode}
\def\markdownRendererDlDefinitionBegin{%
  \markdownRendererDlDefinitionBeginPrototype}%
\ExplSyntaxOn
\seq_gput_right:Nn
  \g_@@_renderers_seq
  { dlDefinitionBegin }
\prop_gput:Nnn
  \g_@@_renderer_arities_prop
  { dlDefinitionBegin }
  { 0 }
\ExplSyntaxOff
%    \end{macrocode}
% \par
%
% \iffalse
%</tex>
%<*manual-tokens>
% \fi
%
% \begin{markdown}

The \mdef{markdownRendererDlDefinitionEnd} macro represents the end of a
definition in a definition list. There can be several definitions for a
single term.

% \end{markdown}
%
% \iffalse
%</manual-tokens>
%<*tex>
% \fi
%
%  \begin{macrocode}
\def\markdownRendererDlDefinitionEnd{%
  \markdownRendererDlDefinitionEndPrototype}%
\ExplSyntaxOn
\seq_gput_right:Nn
  \g_@@_renderers_seq
  { dlDefinitionEnd }
\prop_gput:Nnn
  \g_@@_renderer_arities_prop
  { dlDefinitionEnd }
  { 0 }
\ExplSyntaxOff
%    \end{macrocode}
% \par
%
% \iffalse
%</tex>
%<*manual-tokens>
% \fi
%
% \begin{markdown}

The \mdef{markdownRendererDlEnd} macro represents the end of a definition
list that contains an item with several paragraphs of text (the list is not
tight). The macro receives no arguments.

% \end{markdown}
%
% \iffalse
%</manual-tokens>
%<*tex>
% \fi
%
%  \begin{macrocode}
\def\markdownRendererDlEnd{%
  \markdownRendererDlEndPrototype}%
\ExplSyntaxOn
\seq_gput_right:Nn
  \g_@@_renderers_seq
  { dlEnd }
\prop_gput:Nnn
  \g_@@_renderer_arities_prop
  { dlEnd }
  { 0 }
\ExplSyntaxOff
%    \end{macrocode}
% \par
%
% \iffalse
%</tex>
%<*manual-tokens>
% \fi
%
% \begin{markdown}

The \mdef{markdownRendererDlEndTight} macro represents the end of a
definition list that contains no item with several paragraphs of text (the
list is tight). This macro will only be produced, when the \Opt{tightLists}
option is disabled. The macro receives no arguments.

% \end{markdown}
%
% \iffalse

##### Plain \TeX{} Example {.unnumbered}

Using a text editor, create a text document named `document.tex` with the
following content:
``` tex
\input markdown
\def\markdownOptionDefinitionLists{true}
\def\markdownOptionTightLists{true}

\def\markdownRendererInterblockSeparator{%
  :%
  \def\markdownRendererInterblockSeparator{\par}%
}
\def\markdownRendererDlBeginTight{%
  \begingroup
  \parindent=0pt
}
\def\markdownRendererDlItem#1{%
  \par{\bf#1}%
  \def\markdownRendererDlDefinitionEnd{%
    ,
    \def\markdownRendererDlDefinitionEnd{%
      , and
      \def\markdownRendererDlDefinitionEnd{.}%
    }%
  }%
}
\def\markdownRendererDlItemEnd{}
\def\markdownRendererDlDefinitionBegin{\par--\kern 0.5em}
\def\markdownRendererDlEndTight{\endgroup}

\markdownBegin
This is a tight definition list

Coffee
:   black hot drink
:   prepared from roasted coffee beans
:   one of the most traded agricultural commodities in the world

Milk
:   white cold drink
:   nutrient-rich
:   produced on an industrial scale
\markdownEnd

\def\markdownRendererInterblockSeparator{%
  \def\markdownRendererInterblockSeparator{\par}%
}
\def\markdownRendererDlBegin{}
\def\markdownRendererDlItem#1{%
  . #1 is a
  \def\markdownRendererDlDefinitionBegin{%
    \def\markdownRendererDlDefinitionBegin{%
      ,
      \def\markdownRendererDlDefinitionBegin{, and }%
    }%
  }%
}
\def\markdownRendererDlItemEnd{}
\def\markdownRendererDlDefinitionEnd{}
\def\markdownRendererDlEnd{.}

\markdownBegin
This is a loose definition list

Coffee

:   black hot drink

:   prepared from roasted coffee beans

:   one of the most traded agricultural commodities in the world

Milk

:   white cold drink

:   nutrient-rich

:   produced on an industrial scale
\markdownEnd

\bye
```````
Next, invoke LuaTeX from the terminal:
``` sh
luatex document.tex
``````
A PDF document named `document.pdf` should be produced and contain the
following text:

> This is a tight definition list:
>
> **Coffee**
>
> - black hot drink,
> - prepared from roasted coffee beans, and
> - one of the most traded agricultural commodities in the world.
>
> **Milk**
>
> - white cold drink,
> - nutrient-rich, and
> - produced on an industrial scale.
>
> This is a loose definition list. Coffee is a black hot drink, prepared from
> roasted coffee beans, and one of the most traded agricultural commodities in
> the world. Milk is a white cold drink, nutrient-rich, and produced on an
> industrial scale.

##### \LaTeX{} Example {.unnumbered}

Using a text editor, create a text document named `document.tex` with the
following content:
``` tex
\documentclass{article}
\usepackage[definitionLists, tightLists]{markdown}
\begin{document}

\begin{markdown}[
  renderers = {
    interblockSeparator = {%
      :%
      \def\markdownRendererInterblockSeparator{\par}%
    },
    dlBeginTight = {\begin{description}},
    dlItem = {%
      \item[#1]
      \begin{itemize}
      \def\markdownRendererDlDefinitionEnd{%
        ,
        \def\markdownRendererDlDefinitionEnd{%
          , and
          \def\markdownRendererDlDefinitionEnd{.}%
        }%
      }%
    },
    dlItemEnd = {\end{itemize}},
    dlDefinitionBegin = \item,
    dlEndTight = {\end{description}},
  },
]
This is a tight definition list

Coffee
:   black hot drink
:   prepared from roasted coffee beans
:   one of the most traded agricultural commodities in the world

Milk
:   white cold drink
:   nutrient-rich
:   produced on an industrial scale
\end{markdown}

\begin{markdown}[
  renderers = {
    interblockSeparator = {%
      \def\markdownRendererInterblockSeparator{\par}%
    },
    dlBegin = {},
    dlItem = {%
      . #1 is a
      \def\markdownRendererDlDefinitionBegin{%
        \def\markdownRendererDlDefinitionBegin{%
          ,
          \def\markdownRendererDlDefinitionBegin{, and }%
        }%
      }%
    },
    dlItemEnd = {},
    dlDefinitionEnd = {},
    dlEnd = {.},
  },
]
This is a loose definition list

Coffee

:   black hot drink

:   prepared from roasted coffee beans

:   one of the most traded agricultural commodities in the world

Milk

:   white cold drink

:   nutrient-rich

:   produced on an industrial scale
\end{markdown}

\end{document}
```````
Next, invoke LuaTeX from the terminal:
``` sh
lualatex document.tex
``````
A PDF document named `document.pdf` should be produced and contain the
following text:

> This is a tight definition list:
>
> **Coffee**
>
> - black hot drink,
> - prepared from roasted coffee beans, and
> - one of the most traded agricultural commodities in the world.
>
> **Milk**
>
> - white cold drink,
> - nutrient-rich, and
> - produced on an industrial scale.
>
> This is a loose definition list. Coffee is a black hot drink, prepared from
> roasted coffee beans, and one of the most traded agricultural commodities in
> the world. Milk is a white cold drink, nutrient-rich, and produced on an
> industrial scale.

##### \Hologo{ConTeXt} Example {.unnumbered}

Using a text editor, create a text document named `document.tex` with the
following content:
``` tex
\usemodule[t][markdown]
\setupmarkdown
  [
    definitionLists = yes,
    tightLists = yes,
  ]
\starttext

\def\markdownRendererInterblockSeparator{%
  :%
  \def\markdownRendererInterblockSeparator{\par}%
}
\def\markdownRendererDlBeginTight{}
\def\markdownRendererDlItem#1{%
  \par{\bf#1}%
  \startitemize
  \def\markdownRendererDlDefinitionEnd{%
    ,
    \def\markdownRendererDlDefinitionEnd{%
      , and
      \def\markdownRendererDlDefinitionEnd{.}%
    }%
  }%
}
\def\markdownRendererDlItemEnd{\stopitemize}
\def\markdownRendererDlDefinitionBegin{\item}
\def\markdownRendererDlEndTight{}

\startmarkdown
This is a tight definition list

Coffee
:   black hot drink
:   prepared from roasted coffee beans
:   one of the most traded agricultural commodities in the world

Milk
:   white cold drink
:   nutrient-rich
:   produced on an industrial scale
\stopmarkdown

\def\markdownRendererInterblockSeparator{%
  \def\markdownRendererInterblockSeparator{\par}%
}
\def\markdownRendererDlBegin{}
\def\markdownRendererDlItem#1{%
  . #1 is a
  \def\markdownRendererDlDefinitionBegin{%
    \def\markdownRendererDlDefinitionBegin{%
      ,
      \def\markdownRendererDlDefinitionBegin{, and }%
    }%
  }%
}
\def\markdownRendererDlItemEnd{}
\def\markdownRendererDlDefinitionEnd{}
\def\markdownRendererDlEnd{.}

\startmarkdown
This is a loose definition list

Coffee

:   black hot drink

:   prepared from roasted coffee beans

:   one of the most traded agricultural commodities in the world

Milk

:   white cold drink

:   nutrient-rich

:   produced on an industrial scale
\stopmarkdown

\stoptext
```````
Next, invoke LuaTeX from the terminal:
``` sh
context --luatex document.tex
``````
A PDF document named `document.pdf` should be produced and contain the
following text:

> This is a tight definition list:
>
> **Coffee**
>
> - black hot drink,
> - prepared from roasted coffee beans, and
> - one of the most traded agricultural commodities in the world.
>
> **Milk**
>
> - white cold drink,
> - nutrient-rich, and
> - produced on an industrial scale.
>
> This is a loose definition list. Coffee is a black hot drink, prepared from
> roasted coffee beans, and one of the most traded agricultural commodities in
> the world. Milk is a white cold drink, nutrient-rich, and produced on an
> industrial scale.

%</manual-tokens>
%<*tex>
% \fi
%
%  \begin{macrocode}
\def\markdownRendererDlEndTight{%
  \markdownRendererDlEndTightPrototype}%
\ExplSyntaxOn
\seq_gput_right:Nn
  \g_@@_renderers_seq
  { dlEndTight }
\prop_gput:Nnn
  \g_@@_renderer_arities_prop
  { dlEndTight }
  { 0 }
\ExplSyntaxOff
%    \end{macrocode}
% \par
%
% \iffalse
%</tex>
%<*manual-tokens>
% \fi
%
% \begin{markdown}

#### Ellipsis Renderer
The \mdef{markdownRendererEllipsis} macro replaces any occurrence of ASCII
ellipses in the input text. This macro will only be produced, when the
\Opt{smartEllipses} option is enabled.  The macro receives no arguments.

% \end{markdown}
%
% \iffalse

##### Plain \TeX{} Example {.unnumbered}

Using a text editor, create a text document named `document.tex` with the
following content:
``` tex
\input markdown
\def\markdownOptionSmartEllipses{true}
\def\markdownRendererEllipsis{{\it SHAZAM}!}
\markdownBegin
The secret word is ...
\markdownEnd
\bye
```````
Next, invoke LuaTeX from the terminal:
``` sh
luatex document.tex
``````
A PDF document named `document.pdf` should be produced and contain the
following text:

> The secret word is *SHAZAM*!

##### \LaTeX{} Example {.unnumbered}

Using a text editor, create a text document named `document.tex` with the
following content:
``` tex
\documentclass{article}
\usepackage[smartEllipses]{markdown}
\markdownSetup{
  renderers = {
    ellipsis = \emph{SHAZAM}!,
  },
}
\begin{document}
\begin{markdown}
The secret word is ...
\end{markdown}
\end{document}
```````
Next, invoke LuaTeX from the terminal:
``` sh
lualatex document.tex
``````
A PDF document named `document.pdf` should be produced and contain the
following text:

> The secret word is *SHAZAM*!

##### \Hologo{ConTeXt} Example {.unnumbered}

Using a text editor, create a text document named `document.tex` with the
following content:
``` tex
\usemodule[t][markdown]
\setupmarkdown[smartEllipses = yes]
\def\markdownRendererEllipsis{\emph{SHAZAM}!}
\starttext
\startmarkdown
The secret word is ...
\stopmarkdown
\stoptext
```````
Next, invoke LuaTeX from the terminal:
``` sh
context --luatex document.tex
``````
A PDF document named `document.pdf` should be produced and contain the
following text:

> Hello *world*!
>
> *(The end of a block)*
>
> _Foo_ bar!

%</manual-tokens>
%<*tex>
% \fi
%
%  \begin{macrocode}
\def\markdownRendererEllipsis{%
  \markdownRendererEllipsisPrototype}%
\ExplSyntaxOn
\seq_gput_right:Nn
  \g_@@_renderers_seq
  { ellipsis }
\prop_gput:Nnn
  \g_@@_renderer_arities_prop
  { ellipsis }
  { 0 }
\ExplSyntaxOff
%    \end{macrocode}
% \par
%
% \iffalse
%</tex>
%<*manual-tokens>
% \fi
%
% \begin{markdown}

#### Emphasis Renderers
The \mdef{markdownRendererEmphasis} macro represents an emphasized span of
text. The macro receives a single argument that corresponds to the emphasized
span of text.

% \end{markdown}
%
% \iffalse
%</manual-tokens>
%<*tex>
% \fi
%
%  \begin{macrocode}
\def\markdownRendererEmphasis{%
  \markdownRendererEmphasisPrototype}%
\ExplSyntaxOn
\seq_gput_right:Nn
  \g_@@_renderers_seq
  { emphasis }
\prop_gput:Nnn
  \g_@@_renderer_arities_prop
  { emphasis }
  { 1 }
\ExplSyntaxOff
%    \end{macrocode}
% \par
%
% \iffalse
%</tex>
%<*manual-tokens>
% \fi
%
% \begin{markdown}

The \mdef{markdownRendererStrongEmphasis} macro represents a strongly
emphasized span of text. The macro receives a single argument that
corresponds to the emphasized span of text.

% \end{markdown}
%
% \iffalse

##### Plain \TeX{} Example {.unnumbered}

Using a text editor, create a text document named `document.tex` with the
following content:
``` tex
\input markdown
\def\markdownRendererEmphasis#1{{\it#1}}
\def\markdownRendererStrongEmphasis#1{{\bf#1}}
\markdownBegin
This is *emphasis*.

This is **strong emphasis**.
\markdownEnd
\bye
```````
Next, invoke LuaTeX from the terminal:
``` sh
luatex document.tex
``````
A PDF document named `document.pdf` should be produced and contain the
following text:

> This is *emphasis*.
>
> This is **strong emphasis**.

##### \LaTeX{} Example {.unnumbered}

Using a text editor, create a text document named `document.tex` with the
following content:
``` tex
\documentclass{article}
\usepackage{markdown}
\markdownSetup{
  renderers = {
    emphasis = {\emph{#1}},
    strongEmphasis = {\textbf{#1}},
  },
}
\begin{document}
\begin{markdown}
This is *emphasis*.

This is **strong emphasis**.
\end{markdown}
\end{document}
```````
Next, invoke LuaTeX from the terminal:
``` sh
lualatex document.tex
``````
A PDF document named `document.pdf` should be produced and contain the
following text:

> This is *emphasis*.
>
> This is **strong emphasis**.

##### \Hologo{ConTeXt} Example {.unnumbered}

Using a text editor, create a text document named `document.tex` with the
following content:
``` tex
\usemodule[t][markdown]
\def\markdownRendererEmphasis#1{\emph{#1}}
\def\markdownRendererStrongEmphasis#1{\bold{#1}}
\starttext
\startmarkdown
This is *emphasis*.

This is **strong emphasis**.
\stopmarkdown
\stoptext
```````
Next, invoke LuaTeX from the terminal:
``` sh
context --luatex document.tex
``````
A PDF document named `document.pdf` should be produced and contain the
following text:

> This is *emphasis*.
>
> This is **strong emphasis**.

%</manual-tokens>
%<*tex>
% \fi
%
%  \begin{macrocode}
\def\markdownRendererStrongEmphasis{%
  \markdownRendererStrongEmphasisPrototype}%
\ExplSyntaxOn
\seq_gput_right:Nn
  \g_@@_renderers_seq
  { strongEmphasis }
\prop_gput:Nnn
  \g_@@_renderer_arities_prop
  { strongEmphasis }
  { 1 }
\ExplSyntaxOff
%    \end{macrocode}
% \par
%
% \iffalse
%</tex>
%<*manual-tokens>
% \fi
%
% \begin{markdown}

#### Fenced Code Attribute Context Renderers
The following macros are only produced, when the \Opt{fencedCode} option is
enabled.

The \mdef{markdownRendererFencedCodeAttributeContextBegin} and
\mdef{markdownRendererFencedCodeAttributeContextEnd} macros represent the
beginning and the end of a context in which the attributes of a fenced code
apply. The macros receive no arguments.

% \end{markdown}
%
% \iffalse

##### \LaTeX{} Example {.unnumbered}

Using a text editor, create a text document named `document.tex` with the
following content:
```` tex
\documentclass{article}
\usepackage[fencedCode,
            fencedCodeAttributes]{markdown}
\usepackage{minted}
\markdownSetup{
  renderers = {
    fencedCodeAttributeContextBegin = {%
      \begingroup
      \markdownSetup{
        renderers = {
          attributeKeyValue = {%
            \setminted{{#1} = {#2}}%
          },
        },
      }%
    },
    fencedCodeAttributeContextEnd = {%
      \endgroup
    },
  },
}
\begin{document}
\begin{markdown}
~~~ js {linenos=true}
if (a > 3) {
    moveShip(5 * gravity, DOWN);
}
~~~~~~
\end{markdown}
\end{document}
```````
Next, invoke LuaTeX from the terminal:
``` sh
lualatex --shell-escape document.tex
``````
A PDF document named `document.pdf` should be produced and contain the
following text:

> ``` js {.linenos}
> 1. if (a > 3) {
> 2.     moveShip(5 * gravity, DOWN);
> 3. }
> ``````

%</manual-tokens>
%<*tex>
% \fi
%
%  \begin{macrocode}
\def\markdownRendererFencedCodeAttributeContextBegin{%
  \markdownRendererFencedCodeAttributeContextBeginPrototype}%
\ExplSyntaxOn
\seq_gput_right:Nn
  \g_@@_renderers_seq
  { fencedCodeAttributeContextBegin }
\prop_gput:Nnn
  \g_@@_renderer_arities_prop
  { fencedCodeAttributeContextBegin }
  { 0 }
\ExplSyntaxOff
\def\markdownRendererFencedCodeAttributeContextEnd{%
  \markdownRendererFencedCodeAttributeContextEndPrototype}%
\ExplSyntaxOn
\seq_gput_right:Nn
  \g_@@_renderers_seq
  { fencedCodeAttributeContextEnd }
\prop_gput:Nnn
  \g_@@_renderer_arities_prop
  { fencedCodeAttributeContextEnd }
  { 0 }
\ExplSyntaxOff
%    \end{macrocode}
% \par
%
% \iffalse
%</tex>
%<*manual-tokens>
% \fi
%
% \begin{markdown}

#### Fenced Div Attribute Context Renderers
The following macros are only produced, when the \Opt{fencedDiv} option is
enabled.

The \mdef{markdownRendererFencedDivAttributeContextBegin} and
\mdef{markdownRendererFencedDivAttributeContextEnd} macros represent the beginning
and the end of a context in which the attributes of a div apply. The macros
receive no arguments.

% \end{markdown}
%
% \iffalse

##### \LaTeX{} Example {.unnumbered}

Using a text editor, create a text document named `document.tex` with the
following content:
``` tex
\documentclass{article}
\usepackage[fencedDivs]{markdown}
\markdownSetup{
  renderers = {
    fencedDivAttributeContextBegin = {%
      \par
      \emph{(The beginning of a fenced div attribute context)}
      \par
    },
    fencedDivAttributeContextEnd = {%
      \par
      \emph{(The end of a fenced div attribute context)}
      \par
    },
  },
}
\begin{document}
\begin{markdown}

::: {key=value}

foo

:::: {#identifier}

bar

::::

:::

::: {.class_name}

baz

:::

\end{markdown}
\end{document}
```````
Next, invoke LuaTeX from the terminal:
``` sh
lualatex document.tex
``````
A PDF document named `document.pdf` should be produced and contain the
following text:

> *(The beginning of a fenced div attribute context)*
>
> foo
>
> *(The beginning of a fenced div attribute context)*
>
> bar
>
> *(The end of a fenced div attribute context)*
>
> *(The end of a fenced div attribute context)*
>
> *(The beginning of a fenced div attribute context)*
>
> baz
>
> *(The end of a fenced div attribute context)*

%</manual-tokens>
%<*tex>
% \fi
%
%  \begin{macrocode}
\def\markdownRendererFencedDivAttributeContextBegin{%
  \markdownRendererFencedDivAttributeContextBeginPrototype}%
\ExplSyntaxOn
\seq_gput_right:Nn
  \g_@@_renderers_seq
  { fencedDivAttributeContextBegin }
\prop_gput:Nnn
  \g_@@_renderer_arities_prop
  { fencedDivAttributeContextBegin }
  { 0 }
\ExplSyntaxOff
\def\markdownRendererFencedDivAttributeContextEnd{%
  \markdownRendererFencedDivAttributeContextEndPrototype}%
\ExplSyntaxOn
\seq_gput_right:Nn
  \g_@@_renderers_seq
  { fencedDivAttributeContextEnd }
\prop_gput:Nnn
  \g_@@_renderer_arities_prop
  { fencedDivAttributeContextEnd }
  { 0 }
\ExplSyntaxOff
%    \end{macrocode}
% \par
%
% \iffalse
%</tex>
%<*manual-tokens>
% \fi
%
% \begin{markdown}

#### Header Attribute Context Renderers
The following macros are only produced, when the \Opt{autoIdentifiers},
\Opt{gfmAutoIdentifiers}, or \Opt{headerAttributes} options are enabled.

The \mdef{markdownRendererHeaderAttributeContextBegin} and
\mdef{markdownRendererHeaderAttributeContextEnd} macros represent the beginning
and the end of a context in which the attributes of a heading apply. The macros
receive no arguments.

% \end{markdown}
%
% \iffalse

##### \LaTeX{} Example {.unnumbered}

Using a text editor, create a text document named `document.tex` with the
following content:
``` tex
\documentclass{article}
\usepackage[headerAttributes]{markdown}
\markdownSetup{
  renderers = {
    headerAttributeContextBegin = {%
      \par
      \emph{(The beginning of a header attribute context)}
      \par
    },
    headerAttributeContextEnd = {%
      \par
      \emph{(The end of a header attribute context)}
      \par
    },
  },
}
\begin{document}
\begin{markdown}

# First top-level heading

## A subheading {#identifier}

# Second top-level heading {.class_name}

\end{markdown}
\end{document}
```````
Next, invoke LuaTeX from the terminal:
``` sh
lualatex document.tex
``````
A PDF document named `document.pdf` should be produced and contain the
following text:

> # First top-level heading
>
> *(The beginning of a header attribute context)*
>
> ## A subheading
>
> *(The end of a header attribute context)*
>
> *(The beginning of a header attribute context)*
>
> # Second top-level heading
>
> *(The end of a header attribute context)*

%</manual-tokens>
%<*tex>
% \fi
%  \begin{macrocode}
\def\markdownRendererHeaderAttributeContextBegin{%
  \markdownRendererHeaderAttributeContextBeginPrototype}%
\ExplSyntaxOn
\seq_gput_right:Nn
  \g_@@_renderers_seq
  { headerAttributeContextBegin }
\prop_gput:Nnn
  \g_@@_renderer_arities_prop
  { headerAttributeContextBegin }
  { 0 }
\ExplSyntaxOff
\def\markdownRendererHeaderAttributeContextEnd{%
  \markdownRendererHeaderAttributeContextEndPrototype}%
\ExplSyntaxOn
\seq_gput_right:Nn
  \g_@@_renderers_seq
  { headerAttributeContextEnd }
\prop_gput:Nnn
  \g_@@_renderer_arities_prop
  { headerAttributeContextEnd }
  { 0 }
\ExplSyntaxOff
%    \end{macrocode}
% \par
%
% \iffalse
%</tex>
%<*manual-tokens>
% \fi
%
% \begin{markdown}

#### Heading Renderers
The \mdef{markdownRendererHeadingOne} macro represents a first level heading.
The macro receives a single argument that corresponds to the heading text.

% \end{markdown}
%
% \iffalse
%</manual-tokens>
%<*tex>
% \fi
%
%  \begin{macrocode}
\def\markdownRendererHeadingOne{%
  \markdownRendererHeadingOnePrototype}%
\ExplSyntaxOn
\seq_gput_right:Nn
  \g_@@_renderers_seq
  { headingOne }
\prop_gput:Nnn
  \g_@@_renderer_arities_prop
  { headingOne }
  { 1 }
\ExplSyntaxOff
%    \end{macrocode}
% \par
%
% \iffalse
%</tex>
%<*manual-tokens>
% \fi
%
% \begin{markdown}

The \mdef{markdownRendererHeadingTwo} macro represents a second level
heading. The macro receives a single argument that corresponds to the heading
text.

% \end{markdown}
%
% \iffalse
%</manual-tokens>
%<*tex>
% \fi
%
%  \begin{macrocode}
\def\markdownRendererHeadingTwo{%
  \markdownRendererHeadingTwoPrototype}%
\ExplSyntaxOn
\seq_gput_right:Nn
  \g_@@_renderers_seq
  { headingTwo }
\prop_gput:Nnn
  \g_@@_renderer_arities_prop
  { headingTwo }
  { 1 }
\ExplSyntaxOff
%    \end{macrocode}
% \par
%
% \iffalse
%</tex>
%<*manual-tokens>
% \fi
%
% \begin{markdown}

The \mdef{markdownRendererHeadingThree} macro represents a third level
heading. The macro receives a single argument that corresponds to the heading
text.

% \end{markdown}
%
% \iffalse
%</manual-tokens>
%<*tex>
% \fi
%
%  \begin{macrocode}
\def\markdownRendererHeadingThree{%
  \markdownRendererHeadingThreePrototype}%
\ExplSyntaxOn
\seq_gput_right:Nn
  \g_@@_renderers_seq
  { headingThree }
\prop_gput:Nnn
  \g_@@_renderer_arities_prop
  { headingThree }
  { 1 }
\ExplSyntaxOff
%    \end{macrocode}
% \par
%
% \iffalse
%</tex>
%<*manual-tokens>
% \fi
%
% \begin{markdown}

The \mdef{markdownRendererHeadingFour} macro represents a fourth level
heading. The macro receives a single argument that corresponds to the heading
text.

% \end{markdown}
%
% \iffalse
%</manual-tokens>
%<*tex>
% \fi
%
%  \begin{macrocode}
\def\markdownRendererHeadingFour{%
  \markdownRendererHeadingFourPrototype}%
\ExplSyntaxOn
\seq_gput_right:Nn
  \g_@@_renderers_seq
  { headingFour }
\prop_gput:Nnn
  \g_@@_renderer_arities_prop
  { headingFour }
  { 1 }
\ExplSyntaxOff
%    \end{macrocode}
% \par
%
% \iffalse
%</tex>
%<*manual-tokens>
% \fi
%
% \begin{markdown}

The \mdef{markdownRendererHeadingFive} macro represents a fifth level
heading. The macro receives a single argument that corresponds to the heading
text.

% \end{markdown}
%
% \iffalse
%</manual-tokens>
%<*tex>
% \fi
%
%  \begin{macrocode}
\def\markdownRendererHeadingFive{%
  \markdownRendererHeadingFivePrototype}%
\ExplSyntaxOn
\seq_gput_right:Nn
  \g_@@_renderers_seq
  { headingFive }
\prop_gput:Nnn
  \g_@@_renderer_arities_prop
  { headingFive }
  { 1 }
\ExplSyntaxOff
%    \end{macrocode}
% \par
%
% \iffalse
%</tex>
%<*manual-tokens>
% \fi
%
% \begin{markdown}

The \mdef{markdownRendererHeadingSix} macro represents a sixth level
heading. The macro receives a single argument that corresponds to the heading
text.

% \end{markdown}
%
% \iffalse

##### Plain \TeX{} Example {.unnumbered}

Using a text editor, create a text document named `document.tex` with the
following content:
``` tex
\input markdown
\def\markdownRendererInterblockSeparator{}
\def\markdownRendererHeadingOne{1}
\def\markdownRendererHeadingTwo{2}
\def\markdownRendererHeadingThree{3}
\def\markdownRendererHeadingFour{4}
\def\markdownRendererHeadingFive{5}
\def\markdownRendererHeadingSix{6}
\markdownBegin
######
#####
#####
###
######
\markdownEnd
\bye
```````
Next, invoke LuaTeX from the terminal:
``` sh
luatex document.tex
``````
A PDF document named `document.pdf` should be produced and contain the
following text:

> 65536

##### \LaTeX{} Example {.unnumbered}

Using a text editor, create a text document named `document.tex` with the
following content:
``` tex
\documentclass{article}
\usepackage{markdown}
\markdownSetup{
  renderers = {
    interblockSeparator = {},
    headingOne = 1,
    headingTwo = 2,
    headingThree = 3,
    headingFour = 4,
    headingFive = 5,
    headingSix = 6,
  },
}
\begin{document}
\begin{markdown}
######
#####
#####
###
######
\end{markdown}
\end{document}
```````
Next, invoke LuaTeX from the terminal:
``` sh
lualatex document.tex
``````
A PDF document named `document.pdf` should be produced and contain the
following text:

> 65536

##### \Hologo{ConTeXt} Example {.unnumbered}

Using a text editor, create a text document named `document.tex` with the
following content:
``` tex
\usemodule[t][markdown]
\def\markdownRendererInterblockSeparator{}
\def\markdownRendererHeadingOne{1}
\def\markdownRendererHeadingTwo{2}
\def\markdownRendererHeadingThree{3}
\def\markdownRendererHeadingFour{4}
\def\markdownRendererHeadingFive{5}
\def\markdownRendererHeadingSix{6}
\starttext
\startmarkdown
######
#####
#####
###
######
\stopmarkdown
\stoptext
```````
Next, invoke LuaTeX from the terminal:
``` sh
context --luatex document.tex
``````
A PDF document named `document.pdf` should be produced and contain the
following text:

> 65536

%</manual-tokens>
%<*tex>
% \fi
%
%  \begin{macrocode}
\def\markdownRendererHeadingSix{%
  \markdownRendererHeadingSixPrototype}%
\ExplSyntaxOn
\seq_gput_right:Nn
  \g_@@_renderers_seq
  { headingSix }
\prop_gput:Nnn
  \g_@@_renderer_arities_prop
  { headingSix }
  { 1 }
\ExplSyntaxOff
%    \end{macrocode}
% \par
%
% \iffalse
%</tex>
%<*manual-tokens>
% \fi
%
% \begin{markdown}

#### Inline HTML Comment Renderer

The \mdef{markdownRendererInlineHtmlComment} macro represents the contents of an
inline \acro{HTML} comment. This macro will only be produced, when the
\Opt{html} option is enabled. The macro receives a single argument that
corresponds to the contents of the \acro{HTML} comment.

% \end{markdown}
%
% \iffalse

##### \LaTeX{} Example {.unnumbered}

Using a text editor, create a text document named `document.tex` with the
following content:
``` tex
\documentclass{article}
\usepackage[html]{markdown}
\usepackage{marginnote}
\markdownSetup{
  renderers = {
    inlineHtmlComment = {\marginnote{#1}},
  },
}
\begin{document}
\begin{markdown}
A useful use of inline HTML comments are side notes.
<!-- Side notes are displayed in the horizontal margins next to the relevant
passages, which makes them *easier for the reader to find* than notes. -->
\end{markdown}
\end{document}
```````
Next, invoke LuaTeX from the terminal:
``` sh
lualatex document.tex
lualatex document.tex
``````
A PDF document named `document.pdf` should be produced and contain the
following body text:

> A useful use of HTML comments are side notes.

The horizontal margins should contain the following text:

> Side notes are displayed in the horizontal margins next to the relevant
> passages, which makes them *easier for the reader to find* than notes.

%</manual-tokens>
%<*tex>
% \fi
%
%  \begin{macrocode}
\def\markdownRendererInlineHtmlComment{%
  \markdownRendererInlineHtmlCommentPrototype}%
\ExplSyntaxOn
\seq_gput_right:Nn
  \g_@@_renderers_seq
  { inlineHtmlComment }
\prop_gput:Nnn
  \g_@@_renderer_arities_prop
  { inlineHtmlComment }
  { 1 }
\ExplSyntaxOff
%    \end{macrocode}
% \par
%
% \iffalse
%</tex>
%<*manual-tokens>
% \fi
%
% \begin{markdown}

#### HTML Tag and Element Renderers

The \mdef{markdownRendererInlineHtmlTag} macro represents an opening, closing,
or empty inline \acro{HTML} tag. This macro will only be produced, when the
\Opt{html} option is enabled. The macro receives a single argument that
corresponds to the contents of the \acro{HTML} tag.

The \mdef{markdownRendererInputBlockHtmlElement} macro represents a block
\acro{HTML} element. This macro will only be produced, when the \Opt{html}
option is enabled. The macro receives a single argument that filename of a file
containing the contents of the \acro{HTML} element.

% \end{markdown}
%
% \iffalse

##### \LaTeX{} Example {.unnumbered}

Using a text editor, create a text document named `document.tex` with the
following content:
``` tex
\documentclass{article}
\usepackage[html]{markdown}
\usepackage{marginnote}
\usepackage{verbatim}
\markdownSetup{
  renderers = {
    inlineHtmlTag = {\textbf{#1}},
    inputBlockHtmlElement = {\verbatiminput{#1}},
  },
}
\begin{document}
\begin{markdown}
<b>_Hello,_ world!</b><br/>

<div>_Hello,_ world!</div>
\end{markdown}
\end{document}
```````
Next, invoke LuaTeX from the terminal:
``` sh
lualatex document.tex
lualatex document.tex
``````
A PDF document named `document.pdf` should be produced and contain the
following body text:

> **<b>**_Hello,_ world!**</b><br/>**
>
>     <div>_Hello,_ world!</div>

%</manual-tokens>
%<*tex>
% \fi
%
%  \begin{macrocode}
\def\markdownRendererInlineHtmlTag{%
  \markdownRendererInlineHtmlTagPrototype}%
\ExplSyntaxOn
\seq_gput_right:Nn
  \g_@@_renderers_seq
  { inlineHtmlTag }
\prop_gput:Nnn
  \g_@@_renderer_arities_prop
  { inlineHtmlTag }
  { 1 }
\ExplSyntaxOff
\def\markdownRendererInputBlockHtmlElement{%
  \markdownRendererInputBlockHtmlElementPrototype}%
\ExplSyntaxOn
\seq_gput_right:Nn
  \g_@@_renderers_seq
  { inputBlockHtmlElement }
\prop_gput:Nnn
  \g_@@_renderer_arities_prop
  { inputBlockHtmlElement }
  { 1 }
\ExplSyntaxOff
%    \end{macrocode}
% \par
%
% \iffalse
%</tex>
%<*manual-tokens>
% \fi
%
% \begin{markdown}

#### Image Renderer
The \mdef{markdownRendererImage} macro represents an image. It receives
four arguments: the label, the fully escaped \acro{uri} that can be directly
typeset, the raw \acro{uri} that can be used outside typesetting, and the
title of the link.

% \end{markdown}
%
% \iffalse

##### \LaTeX{} Example {.unnumbered}

Using a text editor, create a text document named `document.tex` with the
following content:
``` tex
\documentclass{article}
\usepackage{markdown}
\begingroup
\catcode`\@=11
\catcode`\%=12
\catcode`\^^A=14
\global\def\markdownRendererImage#1#2#3#4{^^A
  \immediate\write18{^^A
```
``` sh
    if printf '%s' "#3" | grep -q ^http; then
      OUTPUT="$(printf '%s' "#3" | md5sum | cut -d' ' -f1).^^A
              $(printf '%s' "#3" | sed 's/.*[.]//')";
      if ! [ -e "$OUTPUT" ]; then
        wget -O "$OUTPUT" '#3' || rm "$OUTPUT";
        convert "$OUTPUT" png:"$OUTPUT";
      fi;
      printf '%s%%' "$OUTPUT" > \jobname.fetched;
    else
      printf '%s%%' "#3"      > \jobname.fetched;
    fi^^A
```
``` tex
  }^^A
  {^^A
    \everyeof={\noexpand}^^A
    \edef\filename{\@@@@input"\jobname.fetched" }^^A
    \includegraphics[width=\textwidth]{\filename}^^A
  }^^A
}
\endgroup
\begin{document}
\begin{markdown}
![TUGboat](https://tug.org/tugboat/noword.jpg)
\end{markdown}
\end{document}
``````
Next, invoke LuaTeX from the terminal:
``` sh
lualatex --shell-escape document.tex
``````
A PDF document named `document.pdf` should be produced and contain the
following content. This assumes that you use a Unix-like operating system with
Bourne or Bourne again shell as the default shell of the current user. It also
assumes that the `md5sum`, `wget`, and `convert` binaries are installed and
that the \TeX{} engine has shell access.

> ![TUGboat](https://tug.org/tugboat/noword.jpg "The Communications of the TeX Users Group")

%</manual-tokens>
%<*tex>
% \fi
%
%  \begin{macrocode}
\def\markdownRendererImage{%
  \markdownRendererImagePrototype}%
\ExplSyntaxOn
\seq_gput_right:Nn
  \g_@@_renderers_seq
  { image }
\prop_gput:Nnn
  \g_@@_renderer_arities_prop
  { image }
  { 4 }
\ExplSyntaxOff
%    \end{macrocode}
% \par
%
% \iffalse
%</tex>
%<*manual-tokens>
% \fi
%
% \begin{markdown}

#### Image Attribute Context Renderers
The following macros are only produced, when the \Opt{linkAttributes} option
is enabled.

The \mdef{markdownRendererImageAttributeContextBegin} and
\mdef{markdownRendererImageAttributeContextEnd} macros represent the beginning
and the end of a context in which the attributes of an image apply. The macros
receive no arguments.

% \end{markdown}
%
% \iffalse

##### \LaTeX{} Example {.unnumbered}

Using a text editor, create a text document named `document.tex` with the
following content:
``` tex
\documentclass{article}
\usepackage[linkAttributes]{markdown}
\markdownSetup{
  renderers = {
    imageAttributeContextBegin = {(},
    image = {#1},
    imageAttributeContextEnd = {)},
  },
}
\begin{document}
\begin{markdown}

foo ![bar](#bar){key=value} baz

\end{markdown}
\end{document}
```````
Next, invoke LuaTeX from the terminal:
``` sh
lualatex document.tex
``````
A PDF document named `document.pdf` should be produced and contain the
following text:

> foo (bar) baz

%</manual-tokens>
%<*tex>
% \fi
%
%  \begin{macrocode}
\def\markdownRendererImageAttributeContextBegin{%
  \markdownRendererImageAttributeContextBeginPrototype}%
\ExplSyntaxOn
\seq_gput_right:Nn
  \g_@@_renderers_seq
  { imageAttributeContextBegin }
\prop_gput:Nnn
  \g_@@_renderer_arities_prop
  { imageAttributeContextBegin }
  { 0 }
\ExplSyntaxOff
\def\markdownRendererImageAttributeContextEnd{%
  \markdownRendererImageAttributeContextEndPrototype}%
\ExplSyntaxOn
\seq_gput_right:Nn
  \g_@@_renderers_seq
  { imageAttributeContextEnd }
\prop_gput:Nnn
  \g_@@_renderer_arities_prop
  { imageAttributeContextEnd }
  { 0 }
\ExplSyntaxOff
%    \end{macrocode}
% \par
%
% \iffalse
%</tex>
%<*manual-tokens>
% \fi
%
% \begin{markdown}

#### Interblock Separator Renderers
The \mdef{markdownRendererInterblockSeparator} macro represents an interblock
separator between two markdown block elements. The macro receives no
arguments.

% \end{markdown}
%
% \iffalse

##### Plain \TeX{} Example {.unnumbered}

Using a text editor, create a text document named `document.tex` with the
following content:
``` tex
\input markdown
\def\markdownRendererInterblockSeparator{%
  \par
  {\it(The end of a block)}%
  \par
}
\markdownBegin
- Hello *world*!

_Foo_ bar!
\markdownEnd
\bye
```````
Next, invoke LuaTeX from the terminal:
``` sh
luatex document.tex
``````
A PDF document named `document.pdf` should be produced and contain the
following text:

> - Hello *world*!
>
> *(The end of a block)*
>
> _Foo_ bar!

##### \LaTeX{} Example {.unnumbered}

Using a text editor, create a text document named `document.tex` with the
following content:
``` tex
\documentclass{article}
\usepackage{markdown}
\markdownSetup{
  renderers = {
    interblockSeparator = {%
      \par
      \emph{(The end of a block)}%
      \par
    },
  },
}
\begin{document}
\begin{markdown}
- Hello *world*!

_Foo_ bar!
\end{markdown}
\end{document}
```````
Next, invoke LuaTeX from the terminal:
``` sh
lualatex document.tex
``````
A PDF document named `document.pdf` should be produced and contain the
following text:

> - Hello *world*!
>
> *(The end of a block)*
>
> _Foo_ bar!

##### \Hologo{ConTeXt} Example {.unnumbered}

Using a text editor, create a text document named `document.tex` with the
following content:
``` tex
\usemodule[t][markdown]
\def\markdownRendererInterblockSeparator{%
  \par
  \emph{(The end of a block)}%
  \par
}
\starttext
\startmarkdown
- Hello *world*!

_Foo_ bar!
\stopmarkdown
\stoptext
```````
Next, invoke LuaTeX from the terminal:
``` sh
context --luatex document.tex
``````
A PDF document named `document.pdf` should be produced and contain the
following text:

> - Hello *world*!
>
> *(The end of a block)*
>
> _Foo_ bar!

%</manual-tokens>
%<*tex>
% \fi
%
%  \begin{macrocode}
\def\markdownRendererInterblockSeparator{%
  \markdownRendererInterblockSeparatorPrototype}%
\ExplSyntaxOn
\seq_gput_right:Nn
  \g_@@_renderers_seq
  { interblockSeparator }
\prop_gput:Nnn
  \g_@@_renderer_arities_prop
  { interblockSeparator }
  { 0 }
\ExplSyntaxOff
%    \end{macrocode}
% \par
%
% \iffalse
%</tex>
%<*manual-tokens>
% \fi
%
% \begin{markdown}

Users can use more than one blank line to delimit two block to indicate the
end of a series of blocks that make up a logical paragraph. This produces a
paragraph separator instead of an interblock separator. Between some blocks,
such as markdown paragraphs, a paragraph separator is always produced.

The \mdef{markdownRendererParagraphSeparator} macro represents a paragraph
separator. The macro receives no arguments.

% \end{markdown}
%
% \iffalse

##### \LaTeX{} Example {.unnumbered}

Using a text editor, create a text document named `document.tex` with the
following content:
``` tex
\documentclass{article}
\usepackage{markdown}
\markdownSetup{
  renderers = {
    paragraphSeparator = {%
      \par
      \emph{(The end of a paragraph)}%
      \par
    },
  },
}
\begin{document}
\begin{markdown}
Hello *world*!

_Foo_ bar!
\end{markdown}
\end{document}
```````
Next, invoke LuaTeX from the terminal:
``` sh
lualatex document.tex
``````
A PDF document named `document.pdf` should be produced and contain the
following text:

> Hello *world*!
>
> *(The end of a paragraph)*
>
> _Foo_ bar!

%</manual-tokens>
%<*tex>
% \fi
%
%  \begin{macrocode}
\def\markdownRendererParagraphSeparator{%
  \markdownRendererParagraphSeparatorPrototype}%
\ExplSyntaxOn
\seq_gput_right:Nn
  \g_@@_renderers_seq
  { paragraphSeparator }
\prop_gput:Nnn
  \g_@@_renderer_arities_prop
  { paragraphSeparator }
  { 0 }
\ExplSyntaxOff
%    \end{macrocode}
% \par
%
% \iffalse
%</tex>
%<*manual-tokens>
% \fi
%
% \begin{markdown}

#### Line Block Renderers
The following macros are only produced, when the \Opt{lineBlocks} option
is enabled.

The \mdef{markdownRendererLineBlockBegin} and \mdef{markdownRendererLineBlockEnd} macros
represent the beginning and the end of a line block. The macros receive no arguments.

% \end{markdown}
%
% \iffalse

##### Plain \TeX{} Example {.unnumbered}

Using a text editor, create a text document named `document.tex` with the
following content:
``` tex
\input markdown
\def\markdownOptionLineBlocks{true}
\markdownBegin
| I would spread the cloths under your feet:
| But I, being poor, have only my dreams;
| I have spread my dreams under your feet;
| Tread softly because you tread on my dreams.
\markdownEnd
\bye
```````
Next, invoke LuaTeX from the terminal:
``` sh
luatex document.tex
``````
A PDF document named `document.pdf` should be produced and contain the
following text:

> | I would spread the cloths under your feet:
> | But I, being poor, have only my dreams;
> | I have spread my dreams under your feet;
> | Tread softly because you tread on my dreams.

##### \LaTeX{} Example {.unnumbered}

Using a text editor, create a text document named `document.tex` with the
following content:
``` tex
\documentclass{article}
\usepackage[lineBlocks]{markdown}
\begin{document}
\begin{markdown}
| I would spread the cloths under your feet:
| But I, being poor, have only my dreams;
| I have spread my dreams under your feet;
| Tread softly because you tread on my dreams.
\end{markdown}
\end{document}
```````
Next, invoke LuaTeX from the terminal:
``` sh
lualatex document.tex
``````
A PDF document named `document.pdf` should be produced and contain the
following text:

> | I would spread the cloths under your feet:
> | But I, being poor, have only my dreams;
> | I have spread my dreams under your feet;
> | Tread softly because you tread on my dreams.

##### \Hologo{ConTeXt} Example {.unnumbered}

Using a text editor, create a text document named `document.tex` with the
following content:
``` tex
\usemodule[t][markdown]
\setupmarkdown[lineBlocks = yes]
\starttext
\startmarkdown
| I would spread the cloths under your feet:
| But I, being poor, have only my dreams;
| I have spread my dreams under your feet;
| Tread softly because you tread on my dreams.
\stopmarkdown
\stoptext
```````
Next, invoke LuaTeX from the terminal:
``` sh
context --luatex document.tex
``````
A PDF document named `document.pdf` should be produced and contain the
following text:

> | I would spread the cloths under your feet:
> | But I, being poor, have only my dreams;
> | I have spread my dreams under your feet;
> | Tread softly because you tread on my dreams.

%</manual-tokens>
%<*tex>
% \fi
%
%  \begin{macrocode}
\def\markdownRendererLineBlockBegin{%
  \markdownRendererLineBlockBeginPrototype}%
\ExplSyntaxOn
\seq_gput_right:Nn
  \g_@@_renderers_seq
  { lineBlockBegin }
\prop_gput:Nnn
  \g_@@_renderer_arities_prop
  { lineBlockBegin }
  { 0 }
\ExplSyntaxOff
\def\markdownRendererLineBlockEnd{%
  \markdownRendererLineBlockEndPrototype}%
\ExplSyntaxOn
\seq_gput_right:Nn
  \g_@@_renderers_seq
  { lineBlockEnd }
\prop_gput:Nnn
  \g_@@_renderer_arities_prop
  { lineBlockEnd }
  { 0 }
\ExplSyntaxOff
%    \end{macrocode}
% \par
%
% \iffalse
%</tex>
%<*manual-tokens>
% \fi
%
% \begin{markdown}

#### Line Break Renderers

The \mdef{markdownRendererSoftLineBreak} macro represents a soft line break.
The macro receives no arguments.

% \end{markdown}
%
% \iffalse

##### Plain \TeX{} Example {.unnumbered}

Using a text editor, create a text document named `document.tex` with the
following content:
``` tex
\input markdown
\def\markdownRendererSoftLineBreak{%
  \par
  {\it(A soft line break)}%
  \par
}
\markdownInput{example.md}
\bye
```````
Using a text editor, create a text document named `example.md` with the
following content:

``` md
Hello world!
_Foo_ bar!
``````
Next, invoke LuaTeX from the terminal:
``` sh
luatex document.tex
``````
A PDF document named `document.pdf` should be produced and contain the
following text:

> Hello *world*!
>
> *(A soft line break)*
>
> _Foo_ bar!

##### \LaTeX{} Example {.unnumbered}

Using a text editor, create a text document named `document.tex` with the
following content:
``` tex
\documentclass{article}
\usepackage{markdown}
\markdownSetup{
  renderers = {
    softLineBreak = {%
      \par
      \emph{(A soft line break)}%
      \par
    },
  },
}
\begin{document}
\markdownInput{example.md}
\end{document}
```````
Using a text editor, create a text document named `example.md` with the
following content:

``` md
Hello world!
_Foo_ bar!
``````
Next, invoke LuaTeX from the terminal:
``` sh
lualatex document.tex
``````
A PDF document named `document.pdf` should be produced and contain the
following text:

> Hello *world*!
>
> *(A soft line break)*
>
> _Foo_ bar!

%</manual-tokens>
%<*tex>
% \fi
%  \begin{macrocode}
\def\markdownRendererSoftLineBreak{%
  \markdownRendererSoftLineBreakPrototype}%
\ExplSyntaxOn
\seq_gput_right:Nn
  \g_@@_renderers_seq
  { softLineBreak }
\prop_gput:Nnn
  \g_@@_renderer_arities_prop
  { softLineBreak }
  { 0 }
\ExplSyntaxOff
%    \end{macrocode}
% \par
%
% \iffalse
%</tex>
%<*manual-tokens>
% \fi
%
% \begin{markdown}

The \mdef{markdownRendererHardLineBreak} macro represents a hard line break.
The macro receives no arguments.

% \end{markdown}
%
% \iffalse

##### Plain \TeX{} Example {.unnumbered}

Using a text editor, create a text document named `document.tex` with the
following content:
``` tex
\input markdown
\def\markdownRendererHardLineBreak{%
  \par
  {\it(A hard line break)}%
  \par
}
\markdownInput{example.md}
\bye
```````
Using a text editor, create a text document named `example.md` with the
following content.  Note the two spaces at the end of the first line, which
specify a hard line break.  Due to the limitations of the \TeX{} input
processor, hard line breaks would be ignored if we typed them directly into the
`document.tex` document.

<pre><code>Hello world!  <br/>_Foo_ bar!</code></pre>

Next, invoke LuaTeX from the terminal:
``` sh
luatex document.tex
``````
A PDF document named `document.pdf` should be produced and contain the
following text:

> Hello *world*!
>
> *(A hard line break)*
>
> _Foo_ bar!

##### \LaTeX{} Example {.unnumbered}

Using a text editor, create a text document named `document.tex` with the
following content:
``` tex
\documentclass{article}
\usepackage{markdown}
\markdownSetup{
  renderers = {
    hardLineBreak = {%
      \par
      \emph{(A hard line break)}%
      \par
    },
  },
}
\begin{document}
\markdownInput{example.md}
\end{document}
```````
Using a text editor, create a text document named `example.md` with the
following content.  Note the two spaces at the end of the first line, which
specify a hard line break.  Due to the limitations of the \TeX{} input
processor, hard line breaks would be ignored if we typed them directly into the
`document.tex` document.

<pre><code>Hello world!  <br/>_Foo_ bar!</code></pre>

Next, invoke LuaTeX from the terminal:
``` sh
lualatex document.tex
``````
A PDF document named `document.pdf` should be produced and contain the
following text:

> Hello *world*!
>
> *(A hard line break)*
>
> _Foo_ bar!

%</manual-tokens>
%<*tex>
% \fi
%  \begin{macrocode}
\def\markdownRendererHardLineBreak{%
  \markdownRendererHardLineBreakPrototype}%
\ExplSyntaxOn
\seq_gput_right:Nn
  \g_@@_renderers_seq
  { hardLineBreak }
\prop_gput:Nnn
  \g_@@_renderer_arities_prop
  { hardLineBreak }
  { 0 }
\ExplSyntaxOff
%    \end{macrocode}
% \par
%
% \iffalse
%</tex>
%<*manual-tokens>
% \fi
%
% \begin{markdown}

#### Link Renderer
The \mdef{markdownRendererLink} macro represents a hyperlink. It receives
four arguments: the label, the fully escaped \acro{uri} that can be directly
typeset, the raw \acro{uri} that can be used outside typesetting, and the
title of the link.

% \end{markdown}
%
% \iffalse

##### Plain \TeX{} Example {.unnumbered}

Using a text editor, create a text document named `document.tex` with the
following content:
``` tex
\input markdown
\def\markdownRendererLink#1#2#3#4{%
  #1 {\tt#2} titled {\it#4}%
}
\markdownBegin
Please visit [the link][ctan].

 [ctan]: https://ctan.org/
         (the Comprehensive TeX Archive Network)
\markdownEnd
\bye
```````
Next, invoke LuaTeX from the terminal:
``` sh
luatex document.tex
``````
A PDF document named `document.pdf` should be produced and contain the
following text:

> Please visit the link <https://ctan.org/> titled *the Comprehensive TeX
> Archive Network*.

##### \LaTeX{} Example {.unnumbered}

Using a text editor, create a text document named `document.tex` with the
following content:
``` tex
\documentclass{article}
\usepackage{markdown}
\markdownSetup{
  renderers = {
    link = {%
      #1 \texttt{#2} titled \emph{#4}%
    },
  },
}
\begin{document}
\begin{markdown}
Please visit [the link][ctan].

 [ctan]: https://ctan.org/
         (the Comprehensive TeX Archive Network)
\end{markdown}
\end{document}
```````
Next, invoke LuaTeX from the terminal:
``` sh
lualatex document.tex
``````
A PDF document named `document.pdf` should be produced and contain the
following text:

> Please visit the link <https://ctan.org/> titled *the Comprehensive TeX
> Archive Network*.

##### \Hologo{ConTeXt} Example {.unnumbered}

Using a text editor, create a text document named `document.tex` with the
following content:
``` tex
\usemodule[t][markdown]
\def\markdownRendererLink#1#2#3#4{%
  #1 {\tt#2} titled \emph{#4}%
}
\starttext
\startmarkdown
Please visit [the link][ctan].

 [ctan]: https://ctan.org/
         (the Comprehensive TeX Archive Network)
\stopmarkdown
\stoptext
```````
Next, invoke LuaTeX from the terminal:
``` sh
context --luatex document.tex
``````
A PDF document named `document.pdf` should be produced and contain the
following text:

> Please visit the link <https://ctan.org/> titled *the Comprehensive TeX
> Archive Network*.

%</manual-tokens>
%<*tex>
% \fi
%
%  \begin{macrocode}
\def\markdownRendererLink{%
  \markdownRendererLinkPrototype}%
\ExplSyntaxOn
\seq_gput_right:Nn
  \g_@@_renderers_seq
  { link }
\prop_gput:Nnn
  \g_@@_renderer_arities_prop
  { link }
  { 4 }
\ExplSyntaxOff
%    \end{macrocode}
% \par
%
% \iffalse
%</tex>
%<*manual-tokens>
% \fi
%
% \begin{markdown}

#### Link Attribute Context Renderers
The following macros are only produced, when the \Opt{linkAttributes} option
is enabled.

The \mdef{markdownRendererLinkAttributeContextBegin} and
\mdef{markdownRendererLinkAttributeContextEnd} macros represent the beginning
and the end of a context in which the attributes of a hyperlink apply.
The macros receive no arguments.

% \end{markdown}
%
% \iffalse

##### \LaTeX{} Example {.unnumbered}

Using a text editor, create a text document named `document.tex` with the
following content:
``` tex
\documentclass{article}
\usepackage[linkAttributes]{markdown}
\markdownSetup{
  renderers = {
    linkAttributeContextBegin = {(},
    link = {#1},
    linkAttributeContextEnd = {)},
  },
}
\begin{document}
\begin{markdown}

foo [bar](#bar){key=value} baz

\end{markdown}
\end{document}
```````
Next, invoke LuaTeX from the terminal:
``` sh
lualatex document.tex
``````
A PDF document named `document.pdf` should be produced and contain the
following text:

> foo (bar) baz

%</manual-tokens>
%<*tex>
% \fi
%
%  \begin{macrocode}
\def\markdownRendererLinkAttributeContextBegin{%
  \markdownRendererLinkAttributeContextBeginPrototype}%
\ExplSyntaxOn
\seq_gput_right:Nn
  \g_@@_renderers_seq
  { linkAttributeContextBegin }
\prop_gput:Nnn
  \g_@@_renderer_arities_prop
  { linkAttributeContextBegin }
  { 0 }
\ExplSyntaxOff
\def\markdownRendererLinkAttributeContextEnd{%
  \markdownRendererLinkAttributeContextEndPrototype}%
\ExplSyntaxOn
\seq_gput_right:Nn
  \g_@@_renderers_seq
  { linkAttributeContextEnd }
\prop_gput:Nnn
  \g_@@_renderer_arities_prop
  { linkAttributeContextEnd }
  { 0 }
\ExplSyntaxOff
%    \end{macrocode}
% \par
%
% \iffalse
%</tex>
%<*manual-tokens>
% \fi
%
% \begin{markdown}

#### Marked Text Renderer
The following macro is only produced, when the \Opt{mark} option is enabled.

The \mdef{markdownRendererMark} macro represents a span of marked or
highlighted text. The macro receives a single argument that corresponds to
the marked text.

% \end{markdown}
%
% \iffalse

##### \LaTeX{} Example {.unnumbered}

Using a text editor, create a text document named `document.tex` with the
following content:
``` tex
\documentclass{article}
\usepackage[mark]{markdown}
\usepackage{soul}
\markdownSetup{
  renderers = {
    mark = {\hl{#1}},
  },
}
\begin{document}
\begin{markdown}

This ==is highlighted text.==

\end{markdown}
\end{document}
```````
Next, invoke LuaTeX from the terminal:
``` sh
lualatex document.tex
``````
A PDF document named `document.pdf` should be produced and contain the
following text:

> This <mark>is highlighted text.</mark>

%</manual-tokens>
%<*tex>
% \fi
%
%  \begin{macrocode}
\def\markdownRendererMark{%
  \markdownRendererMarkPrototype}%
\ExplSyntaxOn
\seq_gput_right:Nn
  \g_@@_renderers_seq
  { mark }
\prop_gput:Nnn
  \g_@@_renderer_arities_prop
  { mark }
  { 1 }
\ExplSyntaxOff
%    \end{macrocode}
% \par
%
% \iffalse
%</tex>
%<*manual-tokens>
% \fi
%
% \begin{markdown}

#### Markdown Document Renderers
The \mdef{markdownRendererDocumentBegin} and \mdef{markdownRendererDocumentEnd}
macros represent the beginning and the end of a *markdown* document. The macros
receive no arguments.

A \TeX{} document may contain any number of markdown documents. Additionally,
markdown documents may appear not only in a sequence, but several markdown
documents may also be *nested*. Redefinitions of the macros should take this
into account.

% \end{markdown}
%
% \iffalse

##### \LaTeX{} Example {.unnumbered}

Using a text editor, create a text document named `nested.md` with the
following content:
``` md
This is a *nested* markdown document.
```

Using a text editor, create a text document named `document.tex` with the
following content:
``` tex
\documentclass{article}
\usepackage[contentBlocks]{markdown}
\markdownSetup{
  renderers = {
    contentBlock = {%
      \markdownInput{#3}%
    },
    documentBegin = {%
      \par
      \emph{(The beginning of a document)}
      \par
      \begingroup
      \markdownSetup{snippet=first-nesting-level}%
    },
    documentEnd = {%
      \endgroup
      \par
      \emph{(The end of a document)}
      \par
    },
  },
}
\markdownSetupSnippet{first-nesting-level}{
  renderers = {
    documentBegin = {
      \par
      \emph{(The beginning of a nested document)}
      \par
      \begingroup
      \markdownSetup{snippet=second-nesting-level-and-below}
    },
  },
}
\markdownSetupSnippet{second-nesting-level-and-below}{
  renderers = {
    documentBegin = {
      \par
      \emph{(The beginning of a nested document)}
      \par
      \begingroup
    },
    documentEnd = {
      \endgroup
      \par
      \emph{(The end of a nested document)}
      \par
    },
  },
}
\begin{document}
\begin{markdown}
Hello *world*!

/nested.md

_Foo_ bar!
\end{markdown}
\begin{markdown}

Bar baz!

\end{markdown}
\end{document}
```````
Next, invoke LuaTeX from the terminal:
``` sh
lualatex document.tex
``````
A PDF document named `document.pdf` should be produced and contain the
following text:

> *(The beginning of a document)*
>
> Hello *world*!
>
> *(The beginning of a nested document)*
>
> This is a *nested* markdown document.
>
> *(The end of a nested document)*
>
> _Foo_ bar!
>
> *(The end of a document)*
>
> *(The beginning of a document)*
>
> Bar baz!
>
> *(The end of a document)*

%</manual-tokens>
%<*tex>
% \fi
%
%  \begin{macrocode}
\def\markdownRendererDocumentBegin{%
  \markdownRendererDocumentBeginPrototype}%
\ExplSyntaxOn
\seq_gput_right:Nn
  \g_@@_renderers_seq
  { documentBegin }
\prop_gput:Nnn
  \g_@@_renderer_arities_prop
  { documentBegin }
  { 0 }
\ExplSyntaxOff
\def\markdownRendererDocumentEnd{%
  \markdownRendererDocumentEndPrototype}%
\ExplSyntaxOn
\seq_gput_right:Nn
  \g_@@_renderers_seq
  { documentEnd }
\prop_gput:Nnn
  \g_@@_renderer_arities_prop
  { documentEnd }
  { 0 }
\ExplSyntaxOff
%    \end{macrocode}
% \par
%
% \iffalse
%</tex>
%<*manual-tokens>
% \fi
%
% \begin{markdown}

#### Non-Breaking Space Renderer
The \mdef{markdownRendererNbsp} macro represents a non-breaking space.

% \end{markdown}
%
% \iffalse

##### \LaTeX{} Example {.unnumbered}

Using a text editor, create a text document named `document.bib` with the
following content:
``` bib
@book{knuth:tex,
  author    = "Knuth, Donald Ervin",
  title     = "The \TeX book, volume A of Computers and typesetting",
  publisher = "Addison-Wesley",
  year      = "1984"
}
```````
Using a text editor, create a text document named `document.tex` with the
following content:
``` tex
\documentclass{article}
\usepackage[
  citations,
  citationNbsps,
]{markdown}
\markdownSetup{
  renderers = {
    nbsp = {$\cdot$},
  },
}
\begin{document}

\begin{markdown}
The TeXbook [@knuth:tex, p. 123 and 130] is good.
\end{markdown}

\bibliographystyle{plain}
\bibliography{document.bib}

\end{document}
```````
Next, invoke LuaTeX and BibTeX from the terminal:
``` sh
lualatex document.tex
bibtex document.aux
lualatex document.tex
lualatex document.tex
``````
A PDF document named `document.pdf` should be produced and contain the
following text:

> The TeXbook [1, p.·123·and·130] is good.
>
> ### References
> [1] Donald Ervin Knuth. _The TeXbook, volume A of Computers and typesetting._
> Addison-Wesley, 1984.

%</manual-tokens>
%<*tex>
% \fi
%
%  \begin{macrocode}
\def\markdownRendererNbsp{%
  \markdownRendererNbspPrototype}%
\ExplSyntaxOn
\seq_gput_right:Nn
  \g_@@_renderers_seq
  { nbsp }
\prop_gput:Nnn
  \g_@@_renderer_arities_prop
  { nbsp }
  { 0 }
\ExplSyntaxOff
%    \end{macrocode}
% \par
%
% \iffalse
%</tex>
%<*manual-tokens>
% \fi
%
% \begin{markdown}

#### Note Renderer
The \mdef{markdownRendererNote} macro represents a note. This macro
will only be produced, when the \Opt{notes} option is enabled. The
macro receives a single argument that corresponds to the note text.

% \end{markdown}
%
% \iffalse

##### Plain \TeX{} Example {.unnumbered}

Using a text editor, create a text document named `document.tex` with the
following content:
``` tex
\input markdown
\def\markdownOptionNotes{true}
\def\markdownRendererNote#1{ (and \lowercase{#1})}
\markdownBegin
This is some text[^1] and this is some other text[^2].

 [^1]: this is a note

 [^2]: this is some other note
\markdownEnd
\bye
```````
Next, invoke LuaTeX from the terminal:
``` sh
luatex document.tex
``````
A PDF document named `document.pdf` should be produced and contain the
following text:

> This is some text (and this is a note) and this is some other
> text (and this is some other note).

##### \LaTeX{} Example {.unnumbered}

Using a text editor, create a text document named `document.tex` with the
following content:
``` tex
\documentclass{article}
\usepackage[notes]{markdown}
\markdownSetup{
  renderers = {
    note = { (and \MakeLowercase{#1})},
  },
}
\begin{document}
\begin{markdown}
This is some text[^1] and this is some other text[^2].

 [^1]: this is a note

 [^2]: this is some other note
\end{markdown}
\end{document}
```````
Next, invoke LuaTeX from the terminal:
``` sh
lualatex document.tex
``````
A PDF document named `document.pdf` should be produced and contain the
following text:

> This is some text (and this is a note) and this is some other
> text (and this is some other note).

##### \Hologo{ConTeXt} Example {.unnumbered}

Using a text editor, create a text document named `document.tex` with the
following content:
``` tex
\usemodule[t][markdown]
\setupmarkdown[notes = yes]
\def\markdownRendererNote#1{ (and \lowercase{#1})}
\starttext
\startmarkdown
This is some text[^1] and this is some other text[^2].

 [^1]: this is a note

 [^2]: this is some other note
\stopmarkdown
\stoptext
```````
Next, invoke LuaTeX from the terminal:
``` sh
context --luatex document.tex
``````
A PDF document named `document.pdf` should be produced and contain the
following text:

> This is some text (and this is a note) and this is some other
> text (and this is some other note).

%</manual-tokens>
%<*tex>
% \fi
%  \begin{macrocode}
\def\markdownRendererNote{%
  \markdownRendererNotePrototype}%
\ExplSyntaxOn
\seq_gput_right:Nn
  \g_@@_renderers_seq
  { note }
\prop_gput:Nnn
  \g_@@_renderer_arities_prop
  { note }
  { 1 }
\ExplSyntaxOff
%    \end{macrocode}
% \par
%
% \iffalse
%</tex>
%<*manual-tokens>
% \fi
%
% \begin{markdown}

#### Ordered List Renderers
The \mdef{markdownRendererOlBegin} macro represents the beginning of an
ordered list that contains an item with several paragraphs of text (the
list is not tight). This macro will only be produced, when the \Opt{fancyLists}
option is disabled. The macro receives no arguments.

% \end{markdown}
%
% \iffalse
%</manual-tokens>
%<*tex>
% \fi
%
%  \begin{macrocode}
\def\markdownRendererOlBegin{%
  \markdownRendererOlBeginPrototype}%
\ExplSyntaxOn
\seq_gput_right:Nn
  \g_@@_renderers_seq
  { olBegin }
\prop_gput:Nnn
  \g_@@_renderer_arities_prop
  { olBegin }
  { 0 }
\ExplSyntaxOff
%    \end{macrocode}
% \par
%
% \iffalse
%</tex>
%<*manual-tokens>
% \fi
%
% \begin{markdown}

The \mdef{markdownRendererOlBeginTight} macro represents the beginning of an
ordered list that contains no item with several paragraphs of text (the
list is tight). This macro will only be produced, when the \Opt{tightLists}
option is enabled and the \Opt{fancyLists} option is disabled. The macro
receives no arguments.

% \end{markdown}
%
% \iffalse
%</manual-tokens>
%<*tex>
% \fi
%
%  \begin{macrocode}
\def\markdownRendererOlBeginTight{%
  \markdownRendererOlBeginTightPrototype}%
\ExplSyntaxOn
\seq_gput_right:Nn
  \g_@@_renderers_seq
  { olBeginTight }
\prop_gput:Nnn
  \g_@@_renderer_arities_prop
  { olBeginTight }
  { 0 }
\ExplSyntaxOff
%    \end{macrocode}
% \par
%
% \iffalse
%</tex>
%<*manual-tokens>
% \fi
%
% \begin{markdown}

The \mdef{markdownRendererFancyOlBegin} macro represents the beginning of a
fancy ordered list that contains an item with several paragraphs of text (the
list is not tight). This macro will only be produced, when the \Opt{fancyLists}
option is enabled. The macro receives two arguments: the style of the list item
labels (`Decimal`, `LowerRoman`, `UpperRoman`, `LowerAlpha`, and `UpperAlpha`),
and the style of delimiters between list item labels and texts (`Default`,
`OneParen`, and `Period`).

% \end{markdown}
%
% \iffalse
%</manual-tokens>
%<*tex>
% \fi
%
%  \begin{macrocode}
\def\markdownRendererFancyOlBegin{%
  \markdownRendererFancyOlBeginPrototype}%
\ExplSyntaxOn
\seq_gput_right:Nn
  \g_@@_renderers_seq
  { fancyOlBegin }
\prop_gput:Nnn
  \g_@@_renderer_arities_prop
  { fancyOlBegin }
  { 2 }
\ExplSyntaxOff
%    \end{macrocode}
% \par
%
% \iffalse
%</tex>
%<*manual-tokens>
% \fi
%
% \begin{markdown}

The \mdef{markdownRendererFancyOlBeginTight} macro represents the beginning of
a fancy ordered list that contains no item with several paragraphs of text (the
list is tight). This macro will only be produced, when the \Opt{fancyLists}
and \Opt{tightLists} options are enabled. The macro receives two arguments: the
style of the list item labels, and the style of delimiters between list
item labels and texts. See the \mref{markdownRendererFancyOlBegin} macro for
the valid style values.

% \end{markdown}
%
% \iffalse
%</manual-tokens>
%<*tex>
% \fi
%
%  \begin{macrocode}
\def\markdownRendererFancyOlBeginTight{%
  \markdownRendererFancyOlBeginTightPrototype}%
\ExplSyntaxOn
\seq_gput_right:Nn
  \g_@@_renderers_seq
  { fancyOlBeginTight }
\prop_gput:Nnn
  \g_@@_renderer_arities_prop
  { fancyOlBeginTight }
  { 2 }
\ExplSyntaxOff
%    \end{macrocode}
% \par
%
% \iffalse
%</tex>
%<*manual-tokens>
% \fi
%
% \begin{markdown}

The \mdef{markdownRendererOlItem} macro represents an item in an ordered list.
This macro will only be produced, when the \Opt{startNumber} option is
disabled and the \Opt{fancyLists} option is disabled. The macro receives no
arguments.

% \end{markdown}
%
% \iffalse
%</manual-tokens>
%<*tex>
% \fi
%
%  \begin{macrocode}
\def\markdownRendererOlItem{%
  \markdownRendererOlItemPrototype}%
\ExplSyntaxOn
\seq_gput_right:Nn
  \g_@@_renderers_seq
  { olItem }
\prop_gput:Nnn
  \g_@@_renderer_arities_prop
  { olItem }
  { 0 }
\ExplSyntaxOff
%    \end{macrocode}
% \par
%
% \iffalse
%</tex>
%<*manual-tokens>
% \fi
%
% \begin{markdown}

The \mdef{markdownRendererOlItemEnd} macro represents the end of an item in
an ordered list. This macro will only be produced, when the \Opt{fancyLists}
option is disabled. The macro receives no arguments.

% \end{markdown}
%
% \iffalse
%</manual-tokens>
%<*tex>
% \fi
%
%  \begin{macrocode}
\def\markdownRendererOlItemEnd{%
  \markdownRendererOlItemEndPrototype}%
\ExplSyntaxOn
\seq_gput_right:Nn
  \g_@@_renderers_seq
  { olItemEnd }
\prop_gput:Nnn
  \g_@@_renderer_arities_prop
  { olItemEnd }
  { 0 }
\ExplSyntaxOff
%    \end{macrocode}
% \par
%
% \iffalse
%</tex>
%<*manual-tokens>
% \fi
%
% \begin{markdown}

The \mdef{markdownRendererOlItemWithNumber} macro represents an item in an
ordered list.  This macro will only be produced, when the \Opt{startNumber}
option is enabled and the \Opt{fancyLists} option is disabled.  The macro
receives a single numeric argument that corresponds to the item number.

% \end{markdown}
%
% \iffalse
%</manual-tokens>
%<*tex>
% \fi
%
%  \begin{macrocode}
\def\markdownRendererOlItemWithNumber{%
  \markdownRendererOlItemWithNumberPrototype}%
\ExplSyntaxOn
\seq_gput_right:Nn
  \g_@@_renderers_seq
  { olItemWithNumber }
\prop_gput:Nnn
  \g_@@_renderer_arities_prop
  { olItemWithNumber }
  { 1 }
\ExplSyntaxOff
%    \end{macrocode}
% \par
%
% \iffalse
%</tex>
%<*manual-tokens>
% \fi
%
% \begin{markdown}

The \mdef{markdownRendererFancyOlItem} macro represents an item in a fancy
ordered list. This macro will only be produced, when the \Opt{startNumber}
option is disabled and the \Opt{fancyLists} option is enabled. The macro receives
no arguments.

% \end{markdown}
%
% \iffalse
%</manual-tokens>
%<*tex>
% \fi
%
%  \begin{macrocode}
\def\markdownRendererFancyOlItem{%
  \markdownRendererFancyOlItemPrototype}%
\ExplSyntaxOn
\seq_gput_right:Nn
  \g_@@_renderers_seq
  { fancyOlItem }
\prop_gput:Nnn
  \g_@@_renderer_arities_prop
  { fancyOlItem }
  { 0 }
\ExplSyntaxOff
%    \end{macrocode}
% \par
%
% \iffalse
%</tex>
%<*manual-tokens>
% \fi
%
% \begin{markdown}

The \mdef{markdownRendererFancyOlItemEnd} macro represents the end of an item in
a fancy ordered list. This macro will only be produced, when the \Opt{fancyLists}
option is enabled. The macro receives no arguments.

% \end{markdown}
%
% \iffalse
%</manual-tokens>
%<*tex>
% \fi
%
%  \begin{macrocode}
\def\markdownRendererFancyOlItemEnd{%
  \markdownRendererFancyOlItemEndPrototype}%
\ExplSyntaxOn
\seq_gput_right:Nn
  \g_@@_renderers_seq
  { fancyOlItemEnd }
\prop_gput:Nnn
  \g_@@_renderer_arities_prop
  { fancyOlItemEnd }
  { 0 }
\ExplSyntaxOff
%    \end{macrocode}
% \par
%
% \iffalse
%</tex>
%<*manual-tokens>
% \fi
%
% \begin{markdown}

The \mdef{markdownRendererFancyOlItemWithNumber} macro represents an item in a
fancy ordered list. This macro will only be produced, when the \Opt{startNumber}
and \Opt{fancyLists} options are enabled. The macro receives a single numeric
argument that corresponds to the item number.

% \end{markdown}
%
% \iffalse
%</manual-tokens>
%<*tex>
% \fi
%
%  \begin{macrocode}
\def\markdownRendererFancyOlItemWithNumber{%
  \markdownRendererFancyOlItemWithNumberPrototype}%
\ExplSyntaxOn
\seq_gput_right:Nn
  \g_@@_renderers_seq
  { fancyOlItemWithNumber }
\prop_gput:Nnn
  \g_@@_renderer_arities_prop
  { fancyOlItemWithNumber }
  { 1 }
\ExplSyntaxOff
%    \end{macrocode}
% \par
%
% \iffalse
%</tex>
%<*manual-tokens>
% \fi
%
% \begin{markdown}

The \mdef{markdownRendererOlEnd} macro represents the end of an ordered list
that contains an item with several paragraphs of text (the list is not
tight). This macro will only be produced, when the \Opt{fancyLists} option is
disabled. The macro receives no arguments.

% \end{markdown}
%
% \iffalse
%</manual-tokens>
%<*tex>
% \fi
%
%  \begin{macrocode}
\def\markdownRendererOlEnd{%
  \markdownRendererOlEndPrototype}%
\ExplSyntaxOn
\seq_gput_right:Nn
  \g_@@_renderers_seq
  { olEnd }
\prop_gput:Nnn
  \g_@@_renderer_arities_prop
  { olEnd }
  { 0 }
\ExplSyntaxOff
%    \end{macrocode}
% \par
%
% \iffalse
%</tex>
%<*manual-tokens>
% \fi
%
% \begin{markdown}

The \mdef{markdownRendererOlEndTight} macro represents the end of an ordered
list that contains no item with several paragraphs of text (the list is
tight). This macro will only be produced, when the \Opt{tightLists} option is
enabled and the \Opt{fancyLists} option is disabled. The macro receives no
arguments.

% \end{markdown}
%
% \iffalse
%</manual-tokens>
%<*tex>
% \fi
%
%  \begin{macrocode}
\def\markdownRendererOlEndTight{%
  \markdownRendererOlEndTightPrototype}%
\ExplSyntaxOn
\seq_gput_right:Nn
  \g_@@_renderers_seq
  { olEndTight }
\prop_gput:Nnn
  \g_@@_renderer_arities_prop
  { olEndTight }
  { 0 }
\ExplSyntaxOff
%    \end{macrocode}
% \par
%
% \iffalse
%</tex>
%<*manual-tokens>
% \fi
%
% \begin{markdown}

The \mdef{markdownRendererFancyOlEnd} macro represents the end of a fancy
ordered list that contains an item with several paragraphs of text (the list
is not tight). This macro will only be produced, when the \Opt{fancyLists}
option is enabled. The macro receives no arguments.

% \end{markdown}
%
% \iffalse
%</manual-tokens>
%<*tex>
% \fi
%
%  \begin{macrocode}
\def\markdownRendererFancyOlEnd{%
  \markdownRendererFancyOlEndPrototype}%
\ExplSyntaxOn
\seq_gput_right:Nn
  \g_@@_renderers_seq
  { fancyOlEnd }
\prop_gput:Nnn
  \g_@@_renderer_arities_prop
  { fancyOlEnd }
  { 0 }
\ExplSyntaxOff
%    \end{macrocode}
% \par
%
% \iffalse
%</tex>
%<*manual-tokens>
% \fi
%
% \begin{markdown}

The \mdef{markdownRendererFancyOlEndTight} macro represents the end of a
fancy ordered list that contains no item with several paragraphs of text (the
list is tight). This macro will only be produced, when the \Opt{fancyLists}
and \Opt{tightLists} options are enabled. The macro receives no arguments.

% \end{markdown}
%
% \iffalse

##### Plain \TeX{} Example {.unnumbered}

Using a text editor, create a text document named `document.tex` with the
following content:
``` tex
\input markdown
\def\markdownOptionTightLists{true}
\def\markdownOptionStartNumber{true}

\def\markdownRendererInterblockSeparator{}
\def\markdownRendererOlBeginTight{ (}
\def\markdownRendererOlItemWithNumber#1{%
  \ifnum #1=1\relax
    the first
  \else
    \ifnum #1=2\relax
      , the second
    \else
      , and the third
    \fi
  \fi
}
\def\markdownRendererOlItemEnd{}
\def\markdownRendererOlEndTight{).}

\markdownBegin
This is a tight list

1. item
2. item
3. item
\markdownEnd

\def\markdownRendererInterblockSeparator{%
  :\par
  \def\markdownRendererInterblockSeparator{\par}%
}
\def\markdownRendererOlBegin{}
\def\markdownRendererOlItemWithNumber#1{%
  #1.\kern 0.5em%
  This is the
  \ifnum #1=1\relax
    first
  \else
    \ifnum #1=2\relax
      second
    \else
      third
    \fi
  \fi
}
\def\markdownRendererOlItemEnd{.\par}
\def\markdownRendererOlEnd{}

\markdownBegin
This is a loose list

1. item

2. item

3. item
\markdownEnd

\bye
```````
Next, invoke LuaTeX from the terminal:
``` sh
luatex document.tex
``````
A PDF document named `document.pdf` should be produced and contain the
following text:

> This is a tight list (the first item, the second item, and the third item).
>
> This is a loose list:
>
> 1. This is the first item.
>
> 2. This is the second item.
>
> 3. This is the third item.

##### \LaTeX{} Example {.unnumbered}

Using a text editor, create a text document named `document.tex` with the
following content:
``` tex
\documentclass{article}
\usepackage[tightLists, startNumber]{markdown}
\begin{document}

\begin{markdown}[
  renderers = {
    interblockSeparator = {},
    olBeginTight = { (},
    olItemWithNumber = {%
      \ifnum #1=1\relax
        the first
      \else
        \ifnum #1=2\relax
          , the second
        \else
          , and the third
        \fi
      \fi
    },
    olItemEnd = {},
    olEndTight = {).},
  },
]
This is a tight list

1. item
2. item
3. item
\end{markdown}

\begin{markdown}[
  renderers = {
    interblockSeparator = {%
      :\par
      \def\markdownRendererInterblockSeparator{\par}%
    },
    olBeginTight = {\begin{enumerate}},
    olItemWithNumber = {%
      \item This is the
      \ifnum #1=1\relax
        first
      \else
        \ifnum #1=2\relax
          second
        \else
          third
        \fi
      \fi
    },
    olItemEnd = {.},
    olEnd = {\end{enumerate}},
  },
]
This is a loose list

1. item

2. item

3. item
\end{markdown}

\end{document}
```````
Next, invoke LuaTeX from the terminal:
``` sh
lualatex document.tex
``````
A PDF document named `document.pdf` should be produced and contain the
following text:

> This is a tight list (the first item, the second item, and the third item).
>
> This is a loose list:
>
> 1. This is the first item.
>
> 2. This is the second item.
>
> 3. This is the third item.

##### \Hologo{ConTeXt} Example {.unnumbered}

Using a text editor, create a text document named `document.tex` with the
following content:
``` tex
\usemodule[t][markdown]
\setupmarkdown
  [
    tightLists = yes,
    startNumber = yes,
  ]
\starttext

\def\markdownRendererInterblockSeparator{}
\def\markdownRendererOlBeginTight{ (}
\def\markdownRendererOlItemWithNumber#1{%
  \ifnum #1=1\relax
    the first
  \else
    \ifnum #1=2\relax
      , the second
    \else
      , and the third
    \fi
  \fi
}
\def\markdownRendererOlItemEnd{}
\def\markdownRendererOlEndTight{).}

\startmarkdown
This is a tight list

1. item
2. item
3. item
\stopmarkdown

\def\markdownRendererInterblockSeparator{%
  :\par
  \def\markdownRendererInterblockSeparator{\par}%
}
\def\markdownRendererOlBegin{\startitemize}
\def\markdownRendererOlItemWithNumber#1{%
  \sym{#1.}
  This is the
  \ifnum #1=1\relax
    first
  \else
    \ifnum #1=2\relax
      second
    \else
      third
    \fi
  \fi
}
\def\markdownRendererOlItemEnd{.\par}
\def\markdownRendererOlEnd{\stopitemize}

\startmarkdown
This is a loose list

1. item

2. item

3. item
\stopmarkdown

\stoptext
```````
Next, invoke LuaTeX from the terminal:
``` sh
context --luatex document.tex
``````
A PDF document named `document.pdf` should be produced and contain the
following text:

> This is a tight list (the first item, the second item, and the third item).
>
> This is a loose list:
>
> 1. This is the first item.
>
> 2. This is the second item.
>
> 3. This is the third item.

%</manual-tokens>
%<*tex>
% \fi
%
%  \begin{macrocode}
\def\markdownRendererFancyOlEndTight{%
  \markdownRendererFancyOlEndTightPrototype}%
\ExplSyntaxOn
\seq_gput_right:Nn
  \g_@@_renderers_seq
  { fancyOlEndTight }
\prop_gput:Nnn
  \g_@@_renderer_arities_prop
  { fancyOlEndTight }
  { 0 }
\ExplSyntaxOff
%    \end{macrocode}
% \par
%
% \iffalse
%</tex>
%<*manual-tokens>
% \fi
%
% \begin{markdown}

#### Raw Content Renderers
The \mdef{markdownRendererInputRawInline} macro represents an inline raw span.
The macro receives two arguments: the filename of a file containing the inline
raw span contents and the raw attribute that designates the format of the
inline raw span. This macro will only be produced, when the \Opt{rawAttribute}
option is enabled.

% \end{markdown}
%
% \iffalse
%</manual-tokens>
%<*tex>
% \fi
%
%  \begin{macrocode}
\def\markdownRendererInputRawInline{%
  \markdownRendererInputRawInlinePrototype}%
\ExplSyntaxOn
\seq_gput_right:Nn
  \g_@@_renderers_seq
  { inputRawInline }
\prop_gput:Nnn
  \g_@@_renderer_arities_prop
  { inputRawInline }
  { 2 }
\ExplSyntaxOff
%    \end{macrocode}
% \par
%
% \iffalse
%</tex>
%<*manual-tokens>
% \fi
%
% \begin{markdown}

The \mdef{markdownRendererInputRawBlock} macro represents a raw block. The
macro receives two arguments: the filename of a file containing the raw block
and the raw attribute that designates the format of the raw block. This macro
will only be produced, when the \Opt{rawAttribute} and \Opt{fencedCode} options
are enabled.

% \end{markdown}
%
% \iffalse

##### \LaTeX{} Example {.unnumbered}

Using a text editor, create a text document named `document.tex` with the
following content:
```` tex
\documentclass{article}
\usepackage[rawAttribute, fencedCode]{markdown}
\usepackage{expl3}
\ExplSyntaxOn
\cs_new:Nn
  \display_raw_content:nn
  {
    % If the raw attribute is TeX, execute the content as a TeX document.
    \str_if_eq:nnTF
      { #2 }
      { tex }
      { \markdownEscape { #1 } }
      % Otherwise, ignore the content.
      { }
  }
\markdownSetup{
  renderers = {
    rawInline = { \display_raw_content:nn { #1 } { #2 } },
    rawBlock  = { \display_raw_content:nn { #1 } { #2 } }
  },
}
\ExplSyntaxOff
\begin{document}
\begin{markdown}
`$H_2 O$`{=tex} is a liquid.

``` {=html}
<p>Here is some HTML content that will be ignored.</p>
```
\end{markdown}
\end{document}
````
Next, invoke LuaTeX from the terminal:
``` sh
lualatex document.tex
``````
A PDF document named `document.pdf` should be produced and contain the
following text:

> H~2~O is a liquid.

%</manual-tokens>
%<*tex>
% \fi
%
%  \begin{macrocode}
\def\markdownRendererInputRawBlock{%
  \markdownRendererInputRawBlockPrototype}%
\ExplSyntaxOn
\seq_gput_right:Nn
  \g_@@_renderers_seq
  { inputRawBlock }
\prop_gput:Nnn
  \g_@@_renderer_arities_prop
  { inputRawBlock }
  { 2 }
\ExplSyntaxOff
%    \end{macrocode}
% \par
%
% \iffalse
%</tex>
%<*manual-tokens>
% \fi
%
% \begin{markdown}

#### Section Renderers
The \mdef{markdownRendererSectionBegin} and \mdef{markdownRendererSectionEnd}
macros represent the beginning and the end of a section based on headings.

% \end{markdown}
%
% \iffalse
%</manual-tokens>
%<*tex>
% \fi
%
%  \begin{macrocode}
\def\markdownRendererSectionBegin{%
  \markdownRendererSectionBeginPrototype}%
\ExplSyntaxOn
\seq_gput_right:Nn
  \g_@@_renderers_seq
  { sectionBegin }
\prop_gput:Nnn
  \g_@@_renderer_arities_prop
  { sectionBegin }
  { 0 }
\ExplSyntaxOff
\def\markdownRendererSectionEnd{%
  \markdownRendererSectionEndPrototype}%
\ExplSyntaxOn
\seq_gput_right:Nn
  \g_@@_renderers_seq
  { sectionEnd }
\prop_gput:Nnn
  \g_@@_renderer_arities_prop
  { sectionEnd }
  { 0 }
\ExplSyntaxOff
%    \end{macrocode}
% \par
%
% \iffalse
%</tex>
%<*manual-tokens>
% \fi
%
% \begin{markdown}

#### Replacement Character Renderers
The \mdef{markdownRendererReplacementCharacter} macro represents the U+0000
and U+FFFD Unicode characters. The macro receives no arguments.

% \end{markdown}
%
% \iffalse
%</manual-tokens>
%<*tex>
% \fi
%
%  \begin{macrocode}
\def\markdownRendererReplacementCharacter{%
  \markdownRendererReplacementCharacterPrototype}%
\ExplSyntaxOn
\seq_gput_right:Nn
  \g_@@_renderers_seq
  { replacementCharacter }
\prop_gput:Nnn
  \g_@@_renderer_arities_prop
  { replacementCharacter }
  { 0 }
\ExplSyntaxOff
%    \end{macrocode}
% \par
%
% \iffalse
%</tex>
%<*manual-tokens>
% \fi
%
% \begin{markdown}

#### Special Character Renderers
The following macros replace any special plain \TeX{} characters, including
% \iffalse
the active pipe character (`|`) of \Hologo{ConTeXt}, in the input text:

- \mdef{markdownRendererAmpersand} replaces the ampersand (`&`).
- \mdef{markdownRendererBackslash} replaces the backslash (`\`).
- \mdef{markdownRendererCircumflex} replaces the circumflex (`^`).
- \mdef{markdownRendererDollarSign} replaces the dollar sign (`$`).
- \mdef{markdownRendererHash} replaces the hash sign (`#`).
- \mdef{markdownRendererLeftBrace} replaces the left brace (`{`).
- \mdef{markdownRendererPercentSign} replaces the percent sign (`%`).
- \mdef{markdownRendererPipe} replaces the pipe character (`|`).
- \mdef{markdownRendererRightBrace} replaces the right brace (`}`).
- \mdef{markdownRendererTilde} replaces the tilde (`~`).
- \mdef{markdownRendererUnderscore} replaces the underscore (`_`).

% \fi
% the active pipe character (`|`) of \Hologo{ConTeXt}, in the input text.
% These macros will only be produced, when the \Opt{hybrid} option is
% `false`.

% \end{markdown}
%
% \iffalse

##### Plain \TeX{} Example {.unnumbered}

Using a text editor, create a text document named `document.tex` with the
following content. We will make the tilde behave as if it were written in
\TeX{}, where it represents a non-breaking space.
``` tex
\input markdown
\def\markdownRendererTilde{~}
\markdownBegin
Bartel~Leendert van~der~Waerden
\markdownEnd
\bye
```````
Next, invoke LuaTeX from the terminal:
``` sh
luatex document.tex
``````
A PDF document named `document.pdf` should be produced and contain the
following text, where the middot (`·`) denotes a non-breaking space:

> Bartel·Leendert van·der·Waerden

##### \LaTeX{} Example {.unnumbered}

Using a text editor, create a text document named `document.tex` with the
following content. We will make the tilde behave as if it were written in
\TeX{}, where it represents a non-breaking space.
``` tex
\documentclass{article}
\usepackage{markdown}
\markdownSetup{
  renderers = {
    tilde = ~,
  },
}
\begin{document}
\begin{markdown}
Bartel~Leendert van~der~Waerden
\end{markdown}
\end{document}
```````
Next, invoke LuaTeX from the terminal:
``` sh
lualatex document.tex
``````
A PDF document named `document.pdf` should be produced and contain the
following text, where the middot (`·`) denotes a non-breaking space:

> Bartel·Leendert van·der·Waerden

##### \Hologo{ConTeXt} Example {.unnumbered}

Using a text editor, create a text document named `document.tex` with the
following content. We will make the tilde behave as if it were written in
\TeX{}, where it represents a non-breaking space.
``` tex
\usemodule[t][markdown]
\def\markdownRendererTilde{~}
\starttext
\startmarkdown
Bartel~Leendert van~der~Waerden
\stopmarkdown
\stoptext
```````
Next, invoke LuaTeX from the terminal:
``` sh
context --luatex document.tex
``````
A PDF document named `document.pdf` should be produced and contain the
following text, where the middot (`·`) denotes a non-breaking space:

> Bartel·Leendert van·der·Waerden

%</manual-tokens>
%<*tex>
% \fi
%
%  \begin{macrocode}
\def\markdownRendererLeftBrace{%
  \markdownRendererLeftBracePrototype}%
\ExplSyntaxOn
\seq_gput_right:Nn
  \g_@@_renderers_seq
  { leftBrace }
\prop_gput:Nnn
  \g_@@_renderer_arities_prop
  { leftBrace }
  { 0 }
\ExplSyntaxOff
\def\markdownRendererRightBrace{%
  \markdownRendererRightBracePrototype}%
\ExplSyntaxOn
\seq_gput_right:Nn
  \g_@@_renderers_seq
  { rightBrace }
\prop_gput:Nnn
  \g_@@_renderer_arities_prop
  { rightBrace }
  { 0 }
\ExplSyntaxOff
\def\markdownRendererDollarSign{%
  \markdownRendererDollarSignPrototype}%
\ExplSyntaxOn
\seq_gput_right:Nn
  \g_@@_renderers_seq
  { dollarSign }
\prop_gput:Nnn
  \g_@@_renderer_arities_prop
  { dollarSign }
  { 0 }
\ExplSyntaxOff
\def\markdownRendererPercentSign{%
  \markdownRendererPercentSignPrototype}%
\ExplSyntaxOn
\seq_gput_right:Nn
  \g_@@_renderers_seq
  { percentSign }
\prop_gput:Nnn
  \g_@@_renderer_arities_prop
  { percentSign }
  { 0 }
\ExplSyntaxOff
\def\markdownRendererAmpersand{%
  \markdownRendererAmpersandPrototype}%
\ExplSyntaxOn
\seq_gput_right:Nn
  \g_@@_renderers_seq
  { ampersand }
\prop_gput:Nnn
  \g_@@_renderer_arities_prop
  { ampersand }
  { 0 }
\ExplSyntaxOff
\def\markdownRendererUnderscore{%
  \markdownRendererUnderscorePrototype}%
\ExplSyntaxOn
\seq_gput_right:Nn
  \g_@@_renderers_seq
  { underscore }
\prop_gput:Nnn
  \g_@@_renderer_arities_prop
  { underscore }
  { 0 }
\ExplSyntaxOff
\def\markdownRendererHash{%
  \markdownRendererHashPrototype}%
\ExplSyntaxOn
\seq_gput_right:Nn
  \g_@@_renderers_seq
  { hash }
\prop_gput:Nnn
  \g_@@_renderer_arities_prop
  { hash }
  { 0 }
\ExplSyntaxOff
\def\markdownRendererCircumflex{%
  \markdownRendererCircumflexPrototype}%
\ExplSyntaxOn
\seq_gput_right:Nn
  \g_@@_renderers_seq
  { circumflex }
\prop_gput:Nnn
  \g_@@_renderer_arities_prop
  { circumflex }
  { 0 }
\ExplSyntaxOff
\def\markdownRendererBackslash{%
  \markdownRendererBackslashPrototype}%
\ExplSyntaxOn
\seq_gput_right:Nn
  \g_@@_renderers_seq
  { backslash }
\prop_gput:Nnn
  \g_@@_renderer_arities_prop
  { backslash }
  { 0 }
\ExplSyntaxOff
\def\markdownRendererTilde{%
  \markdownRendererTildePrototype}%
\ExplSyntaxOn
\seq_gput_right:Nn
  \g_@@_renderers_seq
  { tilde }
\prop_gput:Nnn
  \g_@@_renderer_arities_prop
  { tilde }
  { 0 }
\ExplSyntaxOff
\def\markdownRendererPipe{%
  \markdownRendererPipePrototype}%
\ExplSyntaxOn
\seq_gput_right:Nn
  \g_@@_renderers_seq
  { pipe }
\prop_gput:Nnn
  \g_@@_renderer_arities_prop
  { pipe }
  { 0 }
\ExplSyntaxOff
%    \end{macrocode}
% \par
%
% \iffalse
%</tex>
%<*manual-tokens>
% \fi
%
% \begin{markdown}

#### Strike-Through Renderer
The \mdef{markdownRendererStrikeThrough} macro represents a strike-through span of
text. The macro receives a single argument that corresponds to the striked-out
span of text. This macro will only be produced, when the \Opt{strikeThrough}
option is enabled.

% \end{markdown}
%
% \iffalse

##### Plain \TeX{} Example {.unnumbered}

Using a text editor, create a text document named `document.tex` with the
following content:
``` tex
\input markdown
\def\markdownOptionStrikeThrough{true}
\input soulutf8.sty
\def\markdownRendererStrikeThrough#1{\st{#1}}
\markdownBegin
This is ~~a lunar roving vehicle~~ strike-through text.
\markdownEnd
\bye
```````
Next, invoke LuaTeX from the terminal:
``` sh
luatex document.tex
``````
A PDF document named `document.pdf` should be produced and contain the
following text:

> This is ~~a lunar roving vehicle~~ strike-through text.

##### \LaTeX{} Example {.unnumbered}

Using a text editor, create a text document named `document.tex` with the
following content:
``` tex
\documentclass{article}
\usepackage[strikeThrough]{markdown}
\usepackage{soulutf8}
\markdownSetup{
  renderers = {
    strikeThrough = {\st{#1}},
  },
}
\begin{document}
\begin{markdown}
This is ~~a lunar roving vehicle~~ strike-through text.
\end{markdown}
\end{document}
```````
Next, invoke LuaTeX from the terminal:
``` sh
lualatex document.tex
``````
A PDF document named `document.pdf` should be produced and contain the
following text:

> This is ~~a lunar roving vehicle~~ strike-through text.

##### \Hologo{ConTeXt} Example {.unnumbered}

Using a text editor, create a text document named `document.tex` with the
following content:
``` tex
\usemodule[t][markdown]
\setupmarkdown[strikeThrough = yes]
\def\markdownRendererStrikeThrough#1{\overstrikes{#1}}
\starttext
\startmarkdown
This is ~~a lunar roving vehicle~~ strike-through text.
\stopmarkdown
\stoptext
```````
Next, invoke LuaTeX from the terminal:
``` sh
context --luatex document.tex
``````
A PDF document named `document.pdf` should be produced and contain the
following text:

> This is ~~a lunar roving vehicle~~ strike-through text.

%</manual-tokens>
%<*tex>
% \fi
%
%  \begin{macrocode}
\def\markdownRendererStrikeThrough{%
  \markdownRendererStrikeThroughPrototype}%
\ExplSyntaxOn
\seq_gput_right:Nn
  \g_@@_renderers_seq
  { strikeThrough }
\prop_gput:Nnn
  \g_@@_renderer_arities_prop
  { strikeThrough }
  { 1 }
\ExplSyntaxOff
%    \end{macrocode}
% \par
%
% \iffalse
%</tex>
%<*manual-tokens>
% \fi
%
% \begin{markdown}

#### Subscript Renderer
The \mdef{markdownRendererSubscript} macro represents a subscript span of
text. The macro receives a single argument that corresponds to the subscript
span of text. This macro will only be produced, when the \Opt{subscripts}
option is enabled.

% \end{markdown}
%
% \iffalse

##### Plain \TeX{} Example {.unnumbered}

Using a text editor, create a text document named `document.tex` with the
following content:
``` tex
\input markdown
\def\markdownOptionSubscripts{true}
\def\markdownRendererSubscript#1{ (#1 moles) and }
\markdownBegin
H~2~O is a liquid.
\markdownEnd
\bye
```````
Next, invoke LuaTeX from the terminal:
``` sh
luatex document.tex
``````
A PDF document named `document.pdf` should be produced and contain the
following text:

> H (2 moles) and O is a liquid.

##### \LaTeX{} Example {.unnumbered}

Using a text editor, create a text document named `document.tex` with the
following content:
``` tex
\documentclass{article}
\usepackage[subscripts]{markdown}
\markdownSetup{
  renderers = {
    subscript = { (#1 moles) and },
  },
}
\begin{document}
\begin{markdown}
H~2~O is a liquid.
\end{markdown}
\end{document}
```````
Next, invoke LuaTeX from the terminal:
``` sh
lualatex document.tex
``````
A PDF document named `document.pdf` should be produced and contain the
following text:

> H (2 moles) and O is a liquid.

##### \Hologo{ConTeXt} Example {.unnumbered}

Using a text editor, create a text document named `document.tex` with the
following content:
``` tex
\usemodule[t][markdown]
\setupmarkdown[subscripts = yes]
\def\markdownRendererSubscript#1{ (#1 moles) and }
\starttext
\startmarkdown
H~2~O is a liquid.
\stopmarkdown
\stoptext
```````
Next, invoke LuaTeX from the terminal:
``` sh
context --luatex document.tex
``````
A PDF document named `document.pdf` should be produced and contain the
following text:

> H (2 moles) and O is a liquid.

%</manual-tokens>
%<*tex>
% \fi
%
%  \begin{macrocode}
\def\markdownRendererSubscript{%
  \markdownRendererSubscriptPrototype}%
\ExplSyntaxOn
\seq_gput_right:Nn
  \g_@@_renderers_seq
  { subscript }
\prop_gput:Nnn
  \g_@@_renderer_arities_prop
  { subscript }
  { 1 }
%    \end{macrocode}
% \par
%
% \iffalse
%</tex>
%<*manual-tokens>
% \fi
%
% \begin{markdown}

#### Superscript Renderer
The \mdef{markdownRendererSuperscript} macro represents a superscript span of
text. The macro receives a single argument that corresponds to the superscript
span of text. This macro will only be produced, when the \Opt{superscripts}
option is enabled.

% \end{markdown}
%
% \iffalse

##### Plain \TeX{} Example {.unnumbered}

Using a text editor, create a text document named `document.tex` with the
following content:
``` tex
\input markdown
\def\markdownOptionSuperscripts{true}
\def\markdownRendererSuperscript#1{ taken to the power of #1}
\markdownBegin
2^10^ is 1024.
\markdownEnd
\bye
```````
Next, invoke LuaTeX from the terminal:
``` sh
luatex document.tex
``````
A PDF document named `document.pdf` should be produced and contain the
following text:

> 2 taken to the power of 10 is 1024.

##### \LaTeX{} Example {.unnumbered}

Using a text editor, create a text document named `document.tex` with the
following content:
``` tex
\documentclass{article}
\usepackage[superscripts]{markdown}
\markdownSetup{
  renderers = {
    superscript = { taken to the power of #1},
  },
}
\begin{document}
\begin{markdown}
2^10^ is 1024.
\end{markdown}
\end{document}
```````
Next, invoke LuaTeX from the terminal:
``` sh
lualatex document.tex
``````
A PDF document named `document.pdf` should be produced and contain the
following text:

> 2 taken to the power of 10 is 1024.

##### \Hologo{ConTeXt} Example {.unnumbered}

Using a text editor, create a text document named `document.tex` with the
following content:
``` tex
\usemodule[t][markdown]
\setupmarkdown[superscripts = yes]
\def\markdownRendererSuperscript#1{ taken to the power of #1}
\starttext
\startmarkdown
2^10^ is 1024.
\stopmarkdown
\stoptext
```````
Next, invoke LuaTeX from the terminal:
``` sh
context --luatex document.tex
``````
A PDF document named `document.pdf` should be produced and contain the
following text:

> 2 taken to the power of 10 is 1024.

%</manual-tokens>
%<*tex>
% \fi
%
%  \begin{macrocode}
\def\markdownRendererSuperscript{%
  \markdownRendererSuperscriptPrototype}%
\ExplSyntaxOn
\seq_gput_right:Nn
  \g_@@_renderers_seq
  { superscript }
\prop_gput:Nnn
  \g_@@_renderer_arities_prop
  { superscript }
  { 1 }
\ExplSyntaxOff
%    \end{macrocode}
% \par
%
% \iffalse
%</tex>
%<*manual-tokens>
% \fi
%
% \begin{markdown}

#### Table Attribute Context Renderers
The following macros are only produced, when the \Opt{tableCaptions} and
\Opt{tableAttributes} options are enabled.

The \mdef{markdownRendererTableAttributeContextBegin} and
\mdef{markdownRendererTableAttributeContextEnd} macros represent the
beginning and the end of a context in which the attributes of a table
apply. The macros receive no arguments.

% \end{markdown}
%
% \iffalse

##### \LaTeX{} Example {.unnumbered}

Using a text editor, create a text document named `document.tex` with the
following content:
``` tex
\documentclass{article}
\usepackage[
  pipeTables,
  tableCaptions,
  tableAttributes,
  relativeReferences,
]{markdown}
\usepackage{expl3}
\ExplSyntaxOn
\markdownSetup{
  renderers = {
    tableAttributeContextBegin = {
      \group_begin:
      \markdownSetup{
        renderers = {
          attributeIdentifier = {
            \markdownSetup{
              renderers = {
                tableAttributeContextEnd = {
                  \label{##1}
                  \group_end:
                },
              },
            }
          },
        },
      }
    },
    tableAttributeContextEnd = {
      \group_end:
    },
  },
}
\ExplSyntaxOff
\begin{document}
\begin{markdown}
See Table <#example-table>.

| Right | Left | Default | Center |
|------:|:-----|---------|:------:|
|   12  |  12  |    12   |    12  |
|  123  |  123 |   123   |   123  |
|    1  |    1 |     1   |     1  |

  : Demonstration of pipe table syntax. {#example-table}
\end{markdown}
\end{document}
```````
Next, invoke LuaTeX from the terminal:
``` sh
lualatex document.tex
``````
A PDF document named `document.pdf` should be produced and contain the
following text:

> See Table 1.
>
> | Right | Left | Default | Center |
> |------:|:-----|---------|:------:|
> |   12  |  12  |    12   |    12  |
> |  123  |  123 |   123   |   123  |
> |    1  |    1 |     1   |     1  |
>
> : Table 1. Demonstration of pipe table syntax.

%</manual-tokens>
%<*tex>
% \fi
%
%  \begin{macrocode}
\def\markdownRendererTableAttributeContextBegin{%
  \markdownRendererTableAttributeContextBeginPrototype}%
\ExplSyntaxOn
\seq_gput_right:Nn
  \g_@@_renderers_seq
  { tableAttributeContextBegin }
\prop_gput:Nnn
  \g_@@_renderer_arities_prop
  { tableAttributeContextBegin }
  { 0 }
\ExplSyntaxOff
\def\markdownRendererTableAttributeContextEnd{%
  \markdownRendererTableAttributeContextEndPrototype}%
\ExplSyntaxOn
\seq_gput_right:Nn
  \g_@@_renderers_seq
  { tableAttributeContextEnd }
\prop_gput:Nnn
  \g_@@_renderer_arities_prop
  { tableAttributeContextEnd }
  { 0 }
\ExplSyntaxOff
%    \end{macrocode}
% \par
%
% \iffalse
%</tex>
%<*manual-tokens>
% \fi
%
% \begin{markdown}

#### Table Renderer
The \mdef{markdownRendererTable} macro represents a table. This macro will only
be produced, when the \Opt{pipeTables} option is enabled. The macro receives the
parameters `{`\meta{caption}`}{`\meta{number of rows}`}{`\meta{number of columns}`}`
followed by `{`\meta{alignments}`}` and then by `{`\meta{row}`}` repeated
\meta{number of rows} times, where \meta{row} is `{`\meta{column}`}` repeated
\meta{number of columns} times, \meta{alignments} is \meta{alignment} repeated
\meta{number of columns} times, and \meta{alignment} is one of the following:

- `d` -- The corresponding column has an unspecified (default) alignment.
- `l` -- The corresponding column is left-aligned.
- `c` -- The corresponding column is centered.
- `r` -- The corresponding column is right-aligned.

% \end{markdown}
%
% \iffalse

##### \LaTeX{} Example {.unnumbered}

Using a text editor, create a text document named `document.tex` with the
following content:
``` tex
\documentclass{article}
\usepackage[pipeTables, tableCaptions]{markdown}
\newcount\rowCounter
\newcount\columnCounter
\makeatletter
\def\processRow#1{%
  \columnCounter=1%
  \ifnum\rowCounter=0\relax
    As for the alignment,
  \else
    In row \the\rowCounter,
  \fi
  \processColumn#1
  \advance\rowCounter by 1\relax
  \ifnum\rowCounter>\rowTotal\relax
    \expandafter\@gobble
  \fi\processRow}%
\def\processColumn#1{%
  column number \the\columnCounter{}
  \ifnum\rowCounter=0\relax
    \if#1d{}has default alignment\fi
    \if#1l{}is left-aligned\fi
    \if#1c{}is centered\fi
    \if#1r{}is right-aligned\fi
  \else
    says \emph{#1}%
  \fi
  \advance\columnCounter by 1\relax
  \ifnum\columnCounter<\columnTotal\relax, \fi
  \ifnum\columnCounter=\columnTotal\relax, and \fi
  \ifnum\columnCounter>\columnTotal\relax
    .\expandafter\@gobble
  \fi\processColumn}%
\makeatother
\markdownSetup{
  renderers = {
    table = {%
      This is a table with caption \emph{#1} that is #3 columns wide
      and #2 rows long.
      \rowCounter=0%
      \def\rowTotal{#2}%
      \def\columnTotal{#3}%
      \processRow
    },
  },
}
\begin{document}
\begin{markdown}
| Right | Left | Default | Center |
|------:|:-----|---------|:------:|
|   12  |  12  |    12   |    12  |
|  123  |  123 |   123   |   123  |
|    1  |    1 |     1   |     1  |

  : Demonstration of pipe table syntax
\end{markdown}
\end{document}
```````
Next, invoke LuaTeX from the terminal:
``` sh
lualatex document.tex
``````
A PDF document named `document.pdf` should be produced and contain the
following text:

> This is a table with caption *Demonstration of pipe table syntax* that is 4
> columns wide and 4 rows long. As for the alignment, column number 1 is
> right-aligned, column number 2 is left-aligned, column number 3 has default
> alignment, and column number 4 is centered. In row 1, column number 1 says
> *Right*, column number 2 says *Left*, column number 3 says *Default*, and
> column number 4 says *Center*. In row 2, column number 1 says *12*, column
> number 2 says *12*, column number 3 says *12*, and column number 4 says *12*.
> In row 3, column number 1 says *123*, column number 2 says *123*, column
> number 3 says *123*, and column number 4 says *123*. In row 4, column number
> 1 says *1*, column number 2 says *1*, column number 3 says *1*, and column
> number 4 says *1*.


%</manual-tokens>
%<*tex>
% \fi
%
%  \begin{macrocode}
\def\markdownRendererTable{%
  \markdownRendererTablePrototype}%
\ExplSyntaxOn
\seq_gput_right:Nn
  \g_@@_renderers_seq
  { table }
\prop_gput:Nnn
  \g_@@_renderer_arities_prop
  { table }
  { 3 }
\ExplSyntaxOff
%    \end{macrocode}
% \par
%
% \iffalse
%</tex>
%<*manual-tokens>
% \fi
%
% \begin{markdown}

#### \TeX{} Math Renderers
The \mdef{markdownRendererInlineMath} and \mdef{markdownRendererDisplayMath} macros
represent inline and display \TeX{} math.
Both macros receive a single argument that corresponds to the \TeX{} math content.
These macros will only be produced, when the \Opt{texMathDollars},
\Opt{texMathSingleBackslash}, or \Opt{texMathDoubleBackslash} option are enabled.

% \end{markdown}
%
% \iffalse

##### Plain \TeX{} Example {.unnumbered}

Using a text editor, create a text document named `document.tex` with the
following content:
``` tex
\input markdown
\def\markdownOptionTexMathDollars{true}
\def\markdownRendererInlineMath#1{$#1\dots$}
\def\markdownRendererDisplayMath#1{$$#1\nonumber$$}
\markdownBegin
$E=mc^2$

$$\hat{f} \left ( \xi  \right )= \int_{-\infty}^{\infty} f\left ( x  \right ) e^{-i2\pi \xi x} dx$$
\markdownEnd
\bye
```````
Next, invoke LuaTeX from the terminal:
``` sh
luatex document.tex
``````
A PDF document named `document.pdf` should be produced and contain the
following text:

> $E=mc^2\dots$
>
> $$\hat{f} \left ( \xi  \right )= \int_{-\infty}^{\infty} f\left ( x  \right ) e^{-i2\pi \xi x} dx\nonumber$$

##### \LaTeX{} Example {.unnumbered}

Using a text editor, create a text document named `document.tex` with the
following content:
``` tex
\documentclass{article}
\usepackage[texMathDollars]{markdown}
\usepackage{amsmath}
\def\markdownRendererInlineMath#1{\begin{math}#1\dots\end{math}}
\def\markdownRendererDisplayMath#1{\begin{equation}#1\end{equation}}
\begin{document}
\begin{markdown}
$E=mc^2$

$$\hat{f} \left ( \xi  \right )= \int_{-\infty}^{\infty} f\left ( x  \right ) e^{-i2\pi \xi x} dx$$
\end{markdown}
\end{document}
```````
Next, invoke LuaTeX from the terminal:
``` sh
lualatex document.tex
``````
A PDF document named `document.pdf` should be produced and contain the
following text:

> $E=mc^2\dots$
>
> $$\hat{f} \left ( \xi  \right )= \int_{-\infty}^{\infty} f\left ( x  \right ) e^{-i2\pi \xi x} dx\quad(1)$$

##### \Hologo{ConTeXt} Example {.unnumbered}

Using a text editor, create a text document named `document.tex` with the
following content:
``` tex
\usemodule[t][markdown]
\setupmarkdown[texMathDollars = yes]
\def\markdownRendererInlineMath#1{$#1\dots$}%
\def\markdownRendererDisplayMath#1{\placeformula\startformula#1\stopformula}%
\starttext
\startmarkdown
$E=mc^2$

$$\hat{f} \left ( \xi  \right )= \int_{-\infty}^{\infty} f\left ( x  \right ) e^{-i2\pi \xi x} dx$$
\stopmarkdown
\stoptext
```````
Next, invoke LuaTeX from the terminal:
``` sh
context --luatex document.tex
``````
A PDF document named `document.pdf` should be produced and contain the
following text:

> $E=mc^2\dots$
>
> $$\hat{f} \left ( \xi  \right )= \int_{-\infty}^{\infty} f\left ( x  \right ) e^{-i2\pi \xi x} dx\quad(1)$$

%</manual-tokens>
%<*tex>
% \fi
%
%  \begin{macrocode}
\def\markdownRendererInlineMath{%
  \markdownRendererInlineMathPrototype}%
\ExplSyntaxOn
\seq_gput_right:Nn
  \g_@@_renderers_seq
  { inlineMath }
\prop_gput:Nnn
  \g_@@_renderer_arities_prop
  { inlineMath }
  { 1 }
\ExplSyntaxOff
\def\markdownRendererDisplayMath{%
  \markdownRendererDisplayMathPrototype}%
\ExplSyntaxOn
\seq_gput_right:Nn
  \g_@@_renderers_seq
  { displayMath }
\prop_gput:Nnn
  \g_@@_renderer_arities_prop
  { displayMath }
  { 1 }
\ExplSyntaxOff
%    \end{macrocode}
% \par
%
% \iffalse
%</tex>
%<*manual-tokens>
% \fi
%
% \begin{markdown}

#### Thematic Break Renderer
The \mdef{markdownRendererThematicBreak} macro represents a thematic break.
The macro receives no arguments.

% \end{markdown}
%
% \iffalse

##### Plain \TeX{} Example {.unnumbered}

Using a text editor, create a text document named `document.tex` with the
following content:
``` tex
\input markdown
\def\markdownRendererThematicBreak{\vfil\break}
\markdownBegin
This is the first page.

***

This is the second page.
\markdownEnd
\bye
```````
Next, invoke LuaTeX from the terminal:
``` sh
luatex document.tex
``````
A PDF document named `document.pdf` should be produced and contain the
following text:

> This is the first page.
>
> ***
>
> This is the second page.

##### \LaTeX{} Example {.unnumbered}

Using a text editor, create a text document named `document.tex` with the
following content:
``` tex
\documentclass{article}
\usepackage{markdown}
\markdownSetup{
  renderers = {
    thematicBreak = \newpage,
  },
}
\begin{document}
\begin{markdown}
This is the first page.

***

This is the second page.
\end{markdown}
\end{document}
```````
Next, invoke LuaTeX from the terminal:
``` sh
lualatex document.tex
``````
A PDF document named `document.pdf` should be produced and contain the
following text:

> This is the first page.
>
> ***
>
> This is the second page.

##### \Hologo{ConTeXt} Example {.unnumbered}

Using a text editor, create a text document named `document.tex` with the
following content:
``` tex
\usemodule[t][markdown]
\def\markdownRendererThematicBreak{\page[yes]}
\starttext
\startmarkdown
This is the first page.

***

This is the second page.
\stopmarkdown
\stoptext
```````
Next, invoke LuaTeX from the terminal:
``` sh
context --luatex document.tex
``````
A PDF document named `document.pdf` should be produced and contain the
following text:

> This is the first page.
>
> ***
>
> This is the second page.

%</manual-tokens>
%<*tex>
% \fi
%  \begin{macrocode}
\def\markdownRendererThematicBreak{%
  \markdownRendererThematicBreakPrototype}%
\ExplSyntaxOn
\seq_gput_right:Nn
  \g_@@_renderers_seq
  { thematicBreak }
\prop_gput:Nnn
  \g_@@_renderer_arities_prop
  { thematicBreak }
  { 0 }
\ExplSyntaxOff
%    \end{macrocode}
% \par
%
% \iffalse
%</tex>
%<*manual-tokens>
% \fi
%
% \begin{markdown}

#### Tickbox Renderers
The macros named \mdef{markdownRendererTickedBox},
\mdef{markdownRendererHalfTickedBox}, and \mdef{markdownRendererUntickedBox}
represent ticked and unticked boxes, respectively. These macros will either be
produced, when the \Opt{taskLists} option is enabled, or when the Ballot Box
with X (☒, U+2612), Hourglass (⌛, U+231B) or Ballot Box (☐, U+2610) Unicode
characters are encountered in the markdown input, respectively.

% \end{markdown}
%
% \iffalse

##### \LaTeX{} Example {.unnumbered}

Using a text editor, create a text document named `document.tex` with the
following content:
``` tex
\documentclass{article}
\usepackage[taskLists]{markdown}
\markdownSetup{
  renderers = {
    untickedBox = No,
    tickedBox = Yes,
  },
}
\begin{document}
\begin{markdown}
- [ ] you can't.
- [x] I can!
\end{markdown}
\end{document}
```````
Next, invoke LuaTeX from the terminal:
``` sh
lualatex document.tex
``````
A PDF document named `document.pdf` should be produced and contain the
following text:

> - No you can't.
> - Yes I can!

##### \Hologo{ConTeXt} Example {.unnumbered}

Using a text editor, create a text document named `document.tex` with the
following content:
``` tex
\usemodule[t][markdown]
\setupmarkdown[taskLists = yes]
\def\markdownRendererUntickedBox{No}
\def\markdownRendererTickedBox{Yes}
\starttext
\startmarkdown
- [ ] you can't.
- [x] I can!
\stopmarkdown
\stoptext
````````
Next, invoke LuaTeX from the terminal:
``` sh
context --luatex document.tex
`````
A PDF document named `document.pdf` should be produced and contain the
following text:

> - No you can't.
> - Yes I can!

%</manual-tokens>
%<*tex>
% \fi
%
%  \begin{macrocode}
\def\markdownRendererTickedBox{%
  \markdownRendererTickedBoxPrototype}%
\ExplSyntaxOn
\seq_gput_right:Nn
  \g_@@_renderers_seq
  { tickedBox }
\prop_gput:Nnn
  \g_@@_renderer_arities_prop
  { tickedBox }
  { 0 }
\ExplSyntaxOff
\def\markdownRendererHalfTickedBox{%
  \markdownRendererHalfTickedBoxPrototype}%
\ExplSyntaxOn
\seq_gput_right:Nn
  \g_@@_renderers_seq
  { halfTickedBox }
\prop_gput:Nnn
  \g_@@_renderer_arities_prop
  { halfTickedBox }
  { 0 }
\ExplSyntaxOff
\def\markdownRendererUntickedBox{%
  \markdownRendererUntickedBoxPrototype}%
\ExplSyntaxOn
\seq_gput_right:Nn
  \g_@@_renderers_seq
  { untickedBox }
\prop_gput:Nnn
  \g_@@_renderer_arities_prop
  { untickedBox }
  { 0 }
\ExplSyntaxOff
%    \end{macrocode}
% \par
%
% \iffalse
%</tex>
%<*manual-tokens>
% \fi
%
% \begin{markdown}

#### Warning and Error Renderers

The \mdef{markdownRendererWarning} and \mdef{markdownRendererError} macros
represent warnings and errors produced by the markdown parser. Both macros
receive a single parameter with the text of the warning or error.

% \end{markdown}
%
% \iffalse
%</manual-tokens>
%<*tex>
% \fi
%
%  \begin{macrocode}
\def\markdownRendererWarning{%
  \markdownRendererWarningPrototype}%
\def\markdownRendererError{%
  \markdownRendererErrorPrototype}%
\ExplSyntaxOn
\seq_gput_right:Nn
  \g_@@_renderers_seq
  { warning }
\prop_gput:Nnn
  \g_@@_renderer_arities_prop
  { warning }
  { 1 }
\seq_gput_right:Nn
  \g_@@_renderers_seq
  { error }
\prop_gput:Nnn
  \g_@@_renderer_arities_prop
  { error }
  { 1 }
\ExplSyntaxOff
%    \end{macrocode}
% \par
%
% \iffalse
%</tex>
%<*manual-tokens>
% \fi
%
% \begin{markdown}

#### YAML Metadata Renderers {#yamlmetadatarenderers}

The \mdef{markdownRendererJekyllDataBegin} macro represents the beginning of a
\acro{yaml} document. This macro will only be produced when the \Opt{jekyllData}
option is enabled. The macro receives no arguments.

% \end{markdown}
%
% \iffalse
%</manual-tokens>
%<*tex>
% \fi
%
%  \begin{macrocode}
\def\markdownRendererJekyllDataBegin{%
  \markdownRendererJekyllDataBeginPrototype}%
\ExplSyntaxOn
\seq_gput_right:Nn
  \g_@@_renderers_seq
  { jekyllDataBegin }
\prop_gput:Nnn
  \g_@@_renderer_arities_prop
  { jekyllDataBegin }
  { 0 }
\ExplSyntaxOff
%    \end{macrocode}
% \par
%
% \iffalse
%</tex>
%<*manual-tokens>
% \fi
%
% \begin{markdown}

The \mdef{markdownRendererJekyllDataEnd} macro represents the end of a
\acro{yaml} document. This macro will only be produced when the
\Opt{jekyllData} option is enabled. The macro receives no arguments.

% \end{markdown}
%
% \iffalse
%</manual-tokens>
%<*tex>
% \fi
%
%  \begin{macrocode}
\def\markdownRendererJekyllDataEnd{%
  \markdownRendererJekyllDataEndPrototype}%
\ExplSyntaxOn
\seq_gput_right:Nn
  \g_@@_renderers_seq
  { jekyllDataEnd }
\prop_gput:Nnn
  \g_@@_renderer_arities_prop
  { jekyllDataEnd }
  { 0 }
\ExplSyntaxOff
%    \end{macrocode}
% \par
%
% \iffalse
%</tex>
%<*manual-tokens>
% \fi
%
% \begin{markdown}

The \mdef{markdownRendererJekyllDataMappingBegin} macro represents the
beginning of a mapping in a \acro{yaml} document. This macro will only be
produced when the \Opt{jekyllData} option is enabled. The macro receives two
arguments: the scalar key in the parent structure, cast to a string following
\acro{yaml} serialization rules, and the number of items in the mapping.

% \end{markdown}
%
% \iffalse
%</manual-tokens>
%<*tex>
% \fi
%
%  \begin{macrocode}
\def\markdownRendererJekyllDataMappingBegin{%
  \markdownRendererJekyllDataMappingBeginPrototype}%
\ExplSyntaxOn
\seq_gput_right:Nn
  \g_@@_renderers_seq
  { jekyllDataMappingBegin }
\prop_gput:Nnn
  \g_@@_renderer_arities_prop
  { jekyllDataMappingBegin }
  { 2 }
\ExplSyntaxOff
%    \end{macrocode}
% \par
%
% \iffalse
%</tex>
%<*manual-tokens>
% \fi
%
% \begin{markdown}

The \mdef{markdownRendererJekyllDataMappingEnd} macro represents the
end of a mapping in a \acro{yaml} document. This macro will only be produced
when the \Opt{jekyllData} option is enabled. The macro receives no arguments.

% \end{markdown}
%
% \iffalse
%</manual-tokens>
%<*tex>
% \fi
%
%  \begin{macrocode}
\def\markdownRendererJekyllDataMappingEnd{%
  \markdownRendererJekyllDataMappingEndPrototype}%
\ExplSyntaxOn
\seq_gput_right:Nn
  \g_@@_renderers_seq
  { jekyllDataMappingEnd }
\prop_gput:Nnn
  \g_@@_renderer_arities_prop
  { jekyllDataMappingEnd }
  { 0 }
\ExplSyntaxOff
%    \end{macrocode}
% \par
%
% \iffalse
%</tex>
%<*manual-tokens>
% \fi
%
% \begin{markdown}

The \mdef{markdownRendererJekyllDataSequenceBegin} macro represents the
beginning of a sequence in a \acro{yaml} document. This macro will only be
produced when the \Opt{jekyllData} option is enabled. The macro receives two
arguments: the scalar key in the parent structure, cast to a string following
\acro{yaml} serialization rules, and the number of items in the sequence.

% \end{markdown}
%
% \iffalse
%</manual-tokens>
%<*tex>
% \fi
%
%  \begin{macrocode}
\def\markdownRendererJekyllDataSequenceBegin{%
  \markdownRendererJekyllDataSequenceBeginPrototype}%
\ExplSyntaxOn
\seq_gput_right:Nn
  \g_@@_renderers_seq
  { jekyllDataSequenceBegin }
\prop_gput:Nnn
  \g_@@_renderer_arities_prop
  { jekyllDataSequenceBegin }
  { 2 }
\ExplSyntaxOff
%    \end{macrocode}
% \par
%
% \iffalse
%</tex>
%<*manual-tokens>
% \fi
%
% \begin{markdown}

The \mdef{markdownRendererJekyllDataSequenceEnd} macro represents the
end of a sequence in a \acro{yaml} document. This macro will only be produced
when the \Opt{jekyllData} option is enabled. The macro receives no arguments.

% \end{markdown}
%
% \iffalse
%</manual-tokens>
%<*tex>
% \fi
%
%  \begin{macrocode}
\def\markdownRendererJekyllDataSequenceEnd{%
  \markdownRendererJekyllDataSequenceEndPrototype}%
\ExplSyntaxOn
\seq_gput_right:Nn
  \g_@@_renderers_seq
  { jekyllDataSequenceEnd }
\prop_gput:Nnn
  \g_@@_renderer_arities_prop
  { jekyllDataSequenceEnd }
  { 0 }
\ExplSyntaxOff
%    \end{macrocode}
% \par
%
% \iffalse
%</tex>
%<*manual-tokens>
% \fi
%
% \begin{markdown}

The \mdef{markdownRendererJekyllDataBoolean} macro represents a boolean scalar
value in a \acro{yaml} document. This macro will only be produced when the
\Opt{jekyllData} option is enabled. The macro receives two arguments: the
scalar key in the parent structure, and the scalar value, both cast to a string
following \acro{yaml} serialization rules.

% \end{markdown}
%
% \iffalse
%</manual-tokens>
%<*tex>
% \fi
%
%  \begin{macrocode}
\def\markdownRendererJekyllDataBoolean{%
  \markdownRendererJekyllDataBooleanPrototype}%
\ExplSyntaxOn
\seq_gput_right:Nn
  \g_@@_renderers_seq
  { jekyllDataBoolean }
\prop_gput:Nnn
  \g_@@_renderer_arities_prop
  { jekyllDataBoolean }
  { 2 }
\ExplSyntaxOff
%    \end{macrocode}
% \par
%
% \iffalse
%</tex>
%<*manual-tokens>
% \fi
%
% \begin{markdown}

The \mdef{markdownRendererJekyllDataNumber} macro represents a numeric scalar
value in a \acro{yaml} document. This macro will only be produced when the
\Opt{jekyllData} option is enabled. The macro receives two arguments: the
scalar key in the parent structure, and the scalar value, both cast to a string
following \acro{yaml} serialization rules.

% \end{markdown}
%
% \iffalse
%</manual-tokens>
%<*tex>
% \fi
%
%  \begin{macrocode}
\def\markdownRendererJekyllDataNumber{%
  \markdownRendererJekyllDataNumberPrototype}%
\ExplSyntaxOn
\seq_gput_right:Nn
  \g_@@_renderers_seq
  { jekyllDataNumber }
\prop_gput:Nnn
  \g_@@_renderer_arities_prop
  { jekyllDataNumber }
  { 2 }
\ExplSyntaxOff
%    \end{macrocode}
% \par
%
% \iffalse
%</tex>
%<*manual-tokens>
% \fi
%
% \begin{markdown}

The \mdef{markdownRendererJekyllDataTypographicString} and
\mdef{markdownRendererJekyllDataProgrammaticString} macros represent string
scalar values in a \acro{yaml} document. This macro will only be produced when
the \Opt{jekyllData} option is enabled. The macro receives two arguments: the
scalar key in the parent structure, cast to a string following \acro{yaml}
serialization rules, and the scalar value.

For each string scalar value, both macros are produced. Whereas
\mref{markdownRendererJekyllDataTypographicString} receives the scalar value
after all markdown markup and special \TeX{} characters in the string have been
replaced by \TeX{} macros, \mref{markdownRendererJekyllDataProgrammaticString}
receives the raw scalar value. Therefore, whereas the
\mref{markdownRendererJekyllDataTypographicString} macro is more appropriate
for texts that are supposed to be typeset with \TeX{}, such as document titles,
author names, or exam questions, the
\mref{markdownRendererJekyllDataProgrammaticString} macro is more appropriate
for identifiers and other programmatic text that won't be typeset by \TeX{}.

% \end{markdown}
%
% \iffalse
%</manual-tokens>
%<*tex>
% \fi
%
%  \begin{macrocode}
\def\markdownRendererJekyllDataTypographicString{%
  \markdownRendererJekyllDataTypographicStringPrototype}%
\def\markdownRendererJekyllDataProgrammaticString{%
  \markdownRendererJekyllDataProgrammaticStringPrototype}%
\ExplSyntaxOn
\seq_gput_right:Nn
  \g_@@_renderers_seq
  { jekyllDataTypographicString }
\prop_gput:Nnn
  \g_@@_renderer_arities_prop
  { jekyllDataTypographicString }
  { 2 }
\seq_gput_right:Nn
  \g_@@_renderers_seq
  { jekyllDataProgrammaticString }
\prop_gput:Nnn
  \g_@@_renderer_arities_prop
  { jekyllDataProgrammaticString }
  { 2 }
\ExplSyntaxOff
%    \end{macrocode}
% \par
%
% \iffalse
%</tex>
%<*manual-tokens>
% \fi
%
% \begin{markdown}

Before Markdown 3.7.0, the \mref{markdownRendererJekyllDataTypographicString}
macro was named \mref{markdownRendererJekyllDataString} and the
\mref{markdownRendererJekyllDataProgrammaticString} macro was not produced.
The \mref{markdownRendererJekyllDataString} has been deprecated and will be
removed in Markdown 4.0.0.

% \end{markdown}
%
% \iffalse
%</manual-tokens>
%<*tex>
% \fi
%
%  \begin{macrocode}
\ExplSyntaxOn
\cs_gset:Npn
  \markdownRendererJekyllDataTypographicString
  {
    \cs_if_exist:NTF
      \markdownRendererJekyllDataString
      {
        \markdownWarning
          {
            The~jekyllDataString~renderer~has~been~deprecated,~
            to~be~removed~in~Markdown~4.0.0
          }
        \markdownRendererJekyllDataString
      }
      {
        \cs_if_exist:NTF
          \markdownRendererJekyllDataStringPrototype
          {
            \markdownWarning
              {
                The~jekyllDataString~renderer~prototype~has~been~deprecated,~
                to~be~removed~in~Markdown~4.0.0
              }
            \markdownRendererJekyllDataStringPrototype
          }
          {
            \markdownRendererJekyllDataTypographicStringPrototype
          }
      }
  }
\seq_gput_right:Nn
  \g_@@_renderers_seq
  { jekyllDataString }
\prop_gput:Nnn
  \g_@@_renderer_arities_prop
  { jekyllDataString }
  { 2 }
\ExplSyntaxOff
%    \end{macrocode}
% \par
%
% \iffalse
%</tex>
%<*manual-tokens>
% \fi
%
% \begin{markdown}

The \mdef{markdownRendererJekyllDataEmpty} macro represents an empty scalar
value in a \acro{yaml} document. This macro will only be produced when the
\Opt{jekyllData} option is enabled. The macro receives one argument: the
scalar key in the parent structure, cast to a string following \acro{yaml}
serialization rules.

% See also Section <#sec:expl3yamlmetadata> for the description of the
% high-level expl3 interface that you can also use to react to \acro{yaml}
% metadata.
%
% \end{markdown}
%
% \iffalse
%</manual-tokens>
%<*tex>
% \fi
%
%  \begin{macrocode}
\def\markdownRendererJekyllDataEmpty{%
  \markdownRendererJekyllDataEmptyPrototype}%
\ExplSyntaxOn
\seq_gput_right:Nn
  \g_@@_renderers_seq
  { jekyllDataEmpty }
\prop_gput:Nnn
  \g_@@_renderer_arities_prop
  { jekyllDataEmpty }
  { 1 }
\ExplSyntaxOff
%    \end{macrocode}
% \par
%
% \iffalse
%</tex>
%<*manual-tokens>

##### Plain \TeX{} Example {.unnumbered}

Using a text editor, create a text document named `document.tex` with the
following content:
``` tex
\input markdown
\def\markdownOptionJekyllData{true}
\def\markdownRendererJekyllDataTypographicString#1#2{\gdef\name{#2}}
\def\markdownRendererJekyllDataNumber#1#2{\gdef\age{#2}}
\def\markdownRendererJekyllDataEnd{%
  \name{} is \age{} years old.}
\markdownBegin
---
name: Jane Doe
age:  99
---
\markdownEnd
\bye
```````
Next, invoke LuaTeX from the terminal:
``` sh
luatex document.tex
``````
A PDF document named `document.pdf` should be produced and contain the
following text:

> Jane Doe is 99 years old.

##### \LaTeX{} Example {.unnumbered}

Using a text editor, create a text document named `document.tex` with the
following content:
``` tex
\documentclass{article}
\usepackage[jekyllData]{markdown}
\markdownSetup{
  renderers = {
    jekyllDataTypographicString = {\gdef\name{#2}},
    jekyllDataNumber = {\gdef\age{#2}},
    jekyllDataEnd = {\name{} is \age{} years old.},
  }
}
\begin{document}

\begin{markdown}
---
name: Jane Doe
age:  99
---
\end{markdown}

\end{document}
```````
Next, invoke LuaTeX from the terminal:
``` sh
lualatex document.tex
``````
A PDF document named `document.pdf` should be produced and contain the
following text:

> Jane Doe is 99 years old.

##### \Hologo{ConTeXt} Example {.unnumbered}

Using a text editor, create a text document named `document.tex` with the
following content:
``` tex
\usemodule[t][markdown]
\setupmarkdown[jekyllData = yes]
\def\markdownRendererJekyllDataTypographicString#1#2{\gdef\name{#2}}
\def\markdownRendererJekyllDataNumber#1#2{\gdef\age{#2}}
\def\markdownRendererJekyllDataEnd{%
  \name{} is \age{} years old.}
\starttext

\startmarkdown
---
name: Jane Doe
age:  99
---
\stopmarkdown

\stoptext
```````
Next, invoke LuaTeX from the terminal:
``` sh
context --luatex document.tex
``````
A PDF document named `document.pdf` should be produced and contain the
following text:

> Jane Doe is 99 years old.

%</manual-tokens>
%<*tex>
% \fi
% \begin{markdown}
%
%#### Generating Plain \TeX{} Token Renderer Macros and Key-Values {#plain-tex-renderers}
%
% We define the command \mdef{@@_define_renderers:} that defines plain \TeX{}
% macros for token renderers. Furthermore, the `\markdownSetup` macro also accepts
% the `renderers` key, whose value must be a list of key-values, where the keys
% correspond to the markdown token renderer macros and the values are new
% definitions of these token renderers.
%
% \end{markdown}
%  \begin{macrocode}
\ExplSyntaxOn
\cs_new:Nn \@@_define_renderers:
  {
    \seq_map_inline:Nn
      \g_@@_renderers_seq
      {
%    \end{macrocode}
% \par
% \begin{markdown}
%
% The \mref{markdownRendererJekyllDataString} has been deprecated and will be
% removed in Markdown 4.0.0.
%
% \end{markdown}
%  \begin{macrocode}
        \str_if_eq:nnF
          { ##1 }
          { jekyllDataString }
          {
            \@@_define_renderer:n
              { ##1 }
          }
      }
  }
\cs_new:Nn \@@_define_renderer:n
  {
    \@@_renderer_tl_to_csname:nN
      { #1 }
      \l_tmpa_tl
    \prop_get:NnN
      \g_@@_renderer_arities_prop
      { #1 }
      \l_tmpb_tl
    \@@_define_renderer:ncV
      { #1 }
      { \l_tmpa_tl }
      \l_tmpb_tl
  }
\cs_new:Nn \@@_renderer_tl_to_csname:nN
  {
    \tl_set:Nn
      \l_tmpa_tl
      { \str_uppercase:n { #1 } }
    \tl_set:Nx
      #2
      {
        markdownRenderer
        \tl_head:f { \l_tmpa_tl }
        \tl_tail:n { #1 }
      }
  }
\tl_new:N
  \l_@@_renderer_definition_tl
\bool_new:N
  \g_@@_appending_renderer_bool
\cs_new:Nn \@@_define_renderer:nNn
  {
    \keys_define:nn
      { markdown/options/renderers }
      {
        #1 .code:n = {
          \tl_set:Nn
            \l_@@_renderer_definition_tl
            { ##1 }
          \regex_replace_all:nnN
            { \cP\#0 }
            { #1 }
            \l_@@_renderer_definition_tl
          \bool_if:NT
            \g_@@_appending_renderer_bool
            {
              \@@_tl_set_from_cs:NNn
                \l_tmpa_tl
                #2
                { #3 }
              \tl_put_left:NV
                \l_@@_renderer_definition_tl
                \l_tmpa_tl
            }
          \cs_generate_from_arg_count:NNnV
            #2
            \cs_set:Npn
            { #3 }
            \l_@@_renderer_definition_tl
        },
      }
  }
%    \end{macrocode}
% \par
% \begin{markdown}
%
% We define the function \mdef{@@_tl_set_from_cs:NNn}
% [@starynovotny24]. The function takes a token list, a control sequence with
% undelimited parameters, and the number of parameters the control sequence
% accepts, and locally assigns the replacement text of the control sequence
% to the token list.
%
% \end{markdown}
%  \begin{macrocode}
\cs_new_protected:Nn
  \@@_tl_set_from_cs:NNn
  {
    \tl_set:Nn
      \l_tmpa_tl
      { #2 }
    \int_step_inline:nn
      { #3 }
      {
        \exp_args:NNc
          \tl_put_right:Nn
          \l_tmpa_tl
          { @@_tl_set_from_cs_parameter_ ##1 }
      }
    \exp_args:NNV
      \tl_set:No
      \l_tmpb_tl
      \l_tmpa_tl
    \regex_replace_all:nnN
      { \cP. }
      { \0\0 }
      \l_tmpb_tl
    \int_step_inline:nn
      { #3 }
      {
        \regex_replace_all:nnN
          { \c { @@_tl_set_from_cs_parameter_ ##1 } }
          { \cP\# ##1 }
          \l_tmpb_tl
      }
    \tl_set:NV
      #1
      \l_tmpb_tl
  }
\cs_generate_variant:Nn
  \@@_define_renderer:nNn
  { ncV }
\cs_generate_variant:Nn
  \cs_generate_from_arg_count:NNnn
  { NNnV }
\cs_generate_variant:Nn
  \tl_put_left:Nn
  { Nv }
\keys_define:nn
  { markdown/options }
  {
    renderers .code:n = {
      \keys_set:nn
        { markdown/options/renderers }
        { #1 }
    },
  }
%    \end{macrocode}
% \par
% \begin{markdown}
%
% The following example code showcases a possible configuration of the
% \mref{markdownRendererLink} and \mref{markdownRendererEmphasis} token
% renderer macros.
% ``` tex
% \markdownSetup{
%   renderers = {
%     link = {#4},                   \% Render links as the link title.
%     emphasis = {{\it #1}},   \% Render emphasized text using italics.
%   }
% }
% ```````
%
% \end{markdown}
%  \begin{macrocode}
\tl_new:N
  \l_@@_renderer_glob_definition_tl
\seq_new:N
  \l_@@_renderer_glob_results_seq
\regex_const:Nn
  \c_@@_appending_key_regex
  { \s*+$ }
\keys_define:nn
  { markdown/options/renderers }
  {
    unknown .code:n = {
%    \end{macrocode}
% \par
% \begin{markdown}
%
% Besides defining renderers at once, we can also define them incrementally
% using the appending operator (`+=`). This can be especially useful in
% defining rules for processing different \acro{HTML} class names and
% identifiers:
% ``` tex
% \markdownSetup{
%   renderers = {
%     \% Start with empty renderers.
%     headerAttributeContextBegin = {},
%     attributeClassName = {},
%     attributeIdentifier = {},
%     \% Define the processing of a single specific HTML class name.
%     headerAttributeContextBegin += {
%       \markdownSetup{
%         renderers = {
%           attributeClassName += {...},
%         },
%       }
%     },
%     \% Define the processing of a single specific HTML identifier.
%     headerAttributeContextBegin += {
%       \markdownSetup{
%         renderers = {
%           attributeIdentifier += {...},
%         },
%       }
%     },
%   },
% }
% ```````
%
% \end{markdown}
%  \begin{macrocode}
      \regex_match:NVTF
        \c_@@_appending_key_regex
        \l_keys_key_str
        {
          \bool_gset_true:N
            \g_@@_appending_renderer_bool
          \tl_set:NV
            \l_tmpa_tl
            \l_keys_key_str
          \regex_replace_once:NnN
            \c_@@_appending_key_regex
            { }
            \l_tmpa_tl
          \tl_set:Nx
            \l_tmpb_tl
            { { \l_tmpa_tl } = }
          \tl_put_right:Nn
            \l_tmpb_tl
            { { #1 } }
          \keys_set:nV
            { markdown/options/renderers }
            \l_tmpb_tl
          \bool_gset_false:N
            \g_@@_appending_renderer_bool
        }
%    \end{macrocode}
% \par
% \begin{markdown}
%
% In addition to exact token renderer names, we also support wildcards (`*`)
% and enumerations (`|`) that match multiple token renderer names:
% ``` tex
% \markdownSetup{
%   renderers = {
%     heading* = {{\bf #1}},    \% Render headings using the bold face.
%     jekyllData(String|Number) = {\%  \% Render YAML string and numbers
%       {\it #2}\%                                     \% using italics.
%     },
%   }
% }
% ```````
%
% Wildcards and enumerations can be combined:
% ``` tex
% \markdownSetup{
%   renderers = {
%     *lItem(|End) = {"},           \% Quote ordered/bullet list items.
%   }
% }
% ```````
%
% To determine the current token renderer, you can use the
% pseudo-parameter `#0`:
% ``` tex
% \markdownSetup{
%   renderers = {
%     heading* = {#0: #1},      \% Render headings as the renderer name
%   }                                  \% followed by the heading text.
% }
% ```````
%
% \end{markdown}
%  \begin{macrocode}
        {
          \@@_glob_seq:VnN
            \l_keys_key_str
            { g_@@_renderers_seq }
            \l_@@_renderer_glob_results_seq
          \seq_if_empty:NTF
            \l_@@_renderer_glob_results_seq
            {
              \msg_error:nnV
                { markdown }
                { undefined-renderer }
                \l_keys_key_str
            }
            {
              \tl_set:Nn
                \l_@@_renderer_glob_definition_tl
                { \exp_not:n { #1 } }
              \seq_map_inline:Nn
                \l_@@_renderer_glob_results_seq
                {
                  \tl_set:Nn
                    \l_tmpa_tl
                    { { ##1 } = }
                  \tl_put_right:Nx
                    \l_tmpa_tl
                    { { \l_@@_renderer_glob_definition_tl } }
                  \keys_set:nV
                    { markdown/options/renderers }
                    \l_tmpa_tl
                }
            }
        }
    },
  }
\msg_new:nnn
  { markdown }
  { undefined-renderer }
  {
    Renderer~#1~is~undefined.
  }
\cs_generate_variant:Nn
  \@@_glob_seq:nnN
  { VnN }
\cs_generate_variant:Nn
  \cs_generate_from_arg_count:NNnn
  { cNVV }
\cs_generate_variant:Nn
  \msg_error:nnn
  { nnV }
\prg_generate_conditional_variant:Nnn
  \regex_match:Nn
  { NV }
  { TF }
\prop_new:N
  \g_@@_glob_cache_prop
\tl_new:N
  \l_@@_current_glob_tl
\cs_new:Nn
  \@@_glob_seq:nnN
  {
    \tl_set:Nn
      \l_@@_current_glob_tl
      { ^ #1 $ }
    \prop_get:NeNTF
      \g_@@_glob_cache_prop
      { #2 / \l_@@_current_glob_tl }
      \l_tmpa_clist
      {
        \seq_set_from_clist:NN
          #3
          \l_tmpa_clist
      }
      {
        \seq_clear:N
          #3
        \regex_replace_all:nnN
          { \* }
          { .* }
          \l_@@_current_glob_tl
        \regex_set:NV
          \l_tmpa_regex
          \l_@@_current_glob_tl
        \seq_map_inline:cn
          { #2 }
          {
            \regex_match:NnT
              \l_tmpa_regex
              { ##1 }
              {
                \seq_put_right:Nn
                  #3
                  { ##1 }
              }
          }
        \clist_set_from_seq:NN
          \l_tmpa_clist
          #3
        \prop_gput:NeV
          \g_@@_glob_cache_prop
          { #2 / \l_@@_current_glob_tl }
          \l_tmpa_clist
      }
  }
% TODO: Remove in TeX Live 2023.
\prg_generate_conditional_variant:Nnn
  \prop_get:NnN
  { NeN }
  { TF }
\cs_generate_variant:Nn
  \regex_set:Nn
  { NV }
\cs_generate_variant:Nn
  \prop_gput:Nnn
  { NeV }
%    \end{macrocode}
% \begin{markdown}
%
% If plain \TeX{} is the top layer, we use the \mref{@@_define_renderers:}
% macro to define plain \TeX{} token renderer macros and key-values
% immediately. Otherwise, we postpone the definition until the upper layers
% have been loaded.
%
% \end{markdown}
%  \begin{macrocode}
\str_if_eq:VVT
  \c_@@_top_layer_tl
  \c_@@_option_layer_plain_tex_tl
  {
    \@@_define_renderers:
  }
\ExplSyntaxOff
%    \end{macrocode}
% \iffalse
%</tex>
%<*manual-tokens>
% \fi
% \begin{markdown}

### Token Renderer Prototypes {#texrendererprototypes}

% \end{markdown}
% \iffalse

By default, token renderers point to package-defined \TeX{} macros, further
referred to as *prototypes*, which provide useful default definitions.

##### Plain \TeX{} Example {.unnumbered}

Using a text editor, create a text document named `document.tex` with the
following content:
``` tex
\input markdown
\def\markdownRendererTildePrototype{%
  Packages can specify token renderer prototypes.%
}
\markdownBegin
~
\markdownEnd

\def\markdownRendererTilde{%
  User-defined token renderers take precedence.%
}
\markdownBegin
~
\markdownEnd
\bye
```````
Next, invoke LuaTeX from the terminal:
``` sh
luatex document.tex
``````
A PDF document named `document.pdf` should be produced and contain the
following text:

> Packages can specify token renderer prototypes.
>
> User-defined token renderers take precedence.

##### \LaTeX{} Example {.unnumbered}

Using a text editor, create a text document named `document.tex` with the
following content:
``` tex
\documentclass{article}
\usepackage{markdown}
\markdownSetup{
  rendererPrototypes = {
    tilde = {Packages can specify token renderer prototypes.},
  },
}
\begin{document}
\begin{markdown}
~
\end{markdown}

\begin{markdown}[
  renderers = {
    tilde = {User-defined token renderers take precedence.},
  },
]
~
\end{markdown}
\end{document}
```````
Next, invoke LuaTeX from the terminal:
``` sh
lualatex document.tex
``````
A PDF document named `document.pdf` should be produced and contain the
following text:

> Packages can specify token renderer prototypes.
>
> User-defined token renderers take precedence.

##### \Hologo{ConTeXt} Example {.unnumbered}

Using a text editor, create a text document named `document.tex` with the
following content:
``` tex
\usemodule[t][markdown]
\def\markdownRendererTildePrototype{%
  Packages can specify token renderer prototypes.%
}
\starttext
\startmarkdown
~
\stopmarkdown

\def\markdownRendererTilde{%
  User-defined token renderers take precedence.%
}
\startmarkdown
~
\stopmarkdown
\stoptext
```````
Next, invoke LuaTeX from the terminal:
``` sh
context --luatex document.tex
``````
A PDF document named `document.pdf` should be produced and contain the
following text:

> Packages can specify token renderer prototypes.
>
> User-defined token renderers take precedence.

% \fi
%
% \begin{markdown}

#### YAML Metadata Renderer Prototypes {#expl3yamlmetadata}

By default, the renderer prototypes for YAML metadata provide a high-level
interface that can be programmed using the `markdown/jekyllData` key--values
from the l3keys module of the \LaTeX{}3 kernel.

% \end{markdown}
% \iffalse

##### Plain \TeX{} Example {.unnumbered}

Using a text editor, create a text document named `document.tex` with the
following content:
``` tex
\input markdown
\def\markdownOptionJekyllData{true}
\ExplSyntaxOn
\keys_define:nn
  { markdown/jekyllData }
  {
    name .code:n = { \gdef\name{#1} },
    age  .code:n = { \gdef\age{#1}  },
  }
\ExplSyntaxOff
\def\markdownRendererJekyllDataEnd{%
  \name{} is \age{} years old.}
\markdownBegin
---
name: Jane Doe
age:  99
---
\markdownEnd
\bye
```````
Next, invoke LuaTeX from the terminal:
``` sh
luatex document.tex
``````
A PDF document named `document.pdf` should be produced and contain the
following text:

> Jane Doe is 99 years old.

##### \LaTeX{} Example {.unnumbered}

Using a text editor, create a text document named `document.tex` with the
following content:
``` tex
\documentclass{article}
\usepackage[jekyllData]{markdown}
\markdownSetup{
  jekyllDataRenderers = {
    name = {\gdef\name{#1}},
    code = {\gdef\age{#1}},
  },
  renderers = {
    jekyllDataEnd = {\name{} is \age{} years old.},
  }
}
\begin{document}

\begin{markdown}
---
name: Jane Doe
age:  99
---
\end{markdown}

\end{document}
```````
Next, invoke LuaTeX from the terminal:
``` sh
lualatex document.tex
``````
A PDF document named `document.pdf` should be produced and contain the
following text:

> Jane Doe is 99 years old.

##### \Hologo{ConTeXt} Example {.unnumbered}

Using a text editor, create a text document named `document.tex` with the
following content:
``` tex
\usemodule[t][markdown]
\setupmarkdown[jekyllData = yes]
\ExplSyntaxOn
\keys_define:nn
  { markdown/jekyllData }
  {
    name .code:n = { \gdef\name{#1} },
    age  .code:n = { \gdef\age{#1}  },
  }
\ExplSyntaxOff
\def\markdownRendererJekyllDataEnd{%
  \name{} is \age{} years old.}
\starttext

\startmarkdown
---
name: Jane Doe
age:  99
---
\stopmarkdown

\stoptext
```````
Next, invoke LuaTeX from the terminal:
``` sh
context --luatex document.tex
``````
A PDF document named `document.pdf` should be produced and contain the
following text:

> Jane Doe is 99 years old.

%</manual-tokens>
%<*tex>
% \fi
%  \begin{macrocode}
\ExplSyntaxOn
\keys_define:nn
  { markdown/jekyllData }
  { }
\ExplSyntaxOff
%    \end{macrocode}
% \begin{markdown}
%
% The `jekyllDataRenderers` key can be used as a syntactic sugar for setting
% the `markdown/jekyllData` key--values without using the expl3 language.
%
% \end{markdown}
%  \begin{macrocode}
\ExplSyntaxOn
\@@_with_various_cases:nn
  { jekyllDataRenderers }
  {
    \keys_define:nn
      { markdown/options }
      {
        #1 .code:n = {
          \tl_set:Nn
            \l_tmpa_tl
            { ##1 }
%    \end{macrocode}
% \begin{markdown}
%
% To ensure that keys containing forward slashes get passed correctly, we
% replace all forward slashes in the input with backslash tokens with category
% code letter and then undo the replacement. This means that if any unbraced
% backslash tokens with category code letter exist in the input, they will be
% replaced with forward slashes. However, this should be extremely rare.
%
% \end{markdown}
%  \begin{macrocode}
          \tl_replace_all:NnV
            \l_tmpa_tl
            { / }
            \c_backslash_str
          \keys_set:nV
            { markdown/options/jekyll-data-renderers }
            \l_tmpa_tl
        },
      }
  }
\keys_define:nn
  { markdown/options/jekyll-data-renderers }
  {
    unknown .code:n = {
      \tl_set_eq:NN
        \l_tmpa_tl
        \l_keys_key_str
      \tl_replace_all:NVn
        \l_tmpa_tl
        \c_backslash_str
        { / }
      \tl_put_right:Nn
        \l_tmpa_tl
        {
          .code:n = { #1 }
        }
      \keys_define:nV
        { markdown/jekyllData }
        \l_tmpa_tl
    }
  }
\cs_generate_variant:Nn
  \keys_define:nn
  { nV }
\ExplSyntaxOff
%    \end{macrocode}
% \begin{markdown}
%
%#### Generating Plain \TeX{} Token Renderer Prototype Macros and Key-Values {#plain-tex-renderer-prototypes}
%
% We define the command \mdef{@@_define_renderer_prototypes:} that defines plain \TeX{}
% macros for token renderer prototypes. Furthermore, the `\markdownSetup` macro also accepts
% the `rendererPrototype` key, whose value must be a list of key-values, where the keys
% correspond to the markdown token renderer prototype macros and the values are new
% definitions of these token renderer prototypes.
%
% \end{markdown}
%  \begin{macrocode}
\ExplSyntaxOn
\cs_new:Nn \@@_define_renderer_prototypes:
  {
    \seq_map_inline:Nn
      \g_@@_renderers_seq
      {
%    \end{macrocode}
% \par
% \begin{markdown}
%
% The \mref{markdownRendererJekyllDataString} has been deprecated and will be
% removed in Markdown 4.0.0.
%
% \end{markdown}
%  \begin{macrocode}
        \str_if_eq:nnF
          { ##1 }
          { jekyllDataString }
          {
            \@@_define_renderer_prototype:n
              { ##1 }
          }
      }
  }
\cs_new:Nn \@@_define_renderer_prototype:n
  {
    \@@_renderer_prototype_tl_to_csname:nN
      { #1 }
      \l_tmpa_tl
    \prop_get:NnN
      \g_@@_renderer_arities_prop
      { #1 }
      \l_tmpb_tl
    \@@_define_renderer_prototype:ncV
      { #1 }
      { \l_tmpa_tl }
      \l_tmpb_tl
  }
\cs_new:Nn \@@_renderer_prototype_tl_to_csname:nN
  {
    \tl_set:Nn
      \l_tmpa_tl
      { \str_uppercase:n { #1 } }
    \tl_set:Nx
      #2
      {
        markdownRenderer
        \tl_head:f { \l_tmpa_tl }
        \tl_tail:n { #1 }
        Prototype
      }
  }
\tl_new:N
  \l_@@_renderer_prototype_definition_tl
\bool_new:N
  \g_@@_appending_renderer_prototype_bool
\cs_new:Nn \@@_define_renderer_prototype:nNn
  {
    \keys_define:nn
      { markdown/options/renderer-prototypes }
      {
        #1 .code:n = {
          \tl_set:Nn
            \l_@@_renderer_prototype_definition_tl
            { ##1 }
          \regex_replace_all:nnN
            { \cP\#0 }
            { #1 }
            \l_@@_renderer_prototype_definition_tl
          \bool_if:NT
            \g_@@_appending_renderer_prototype_bool
            {
              \@@_tl_set_from_cs:NNn
                \l_tmpa_tl
                #2
                { #3 }
              \tl_put_left:NV
                \l_@@_renderer_prototype_definition_tl
                \l_tmpa_tl
            }
          \cs_generate_from_arg_count:NNnV
            #2
            \cs_set:Npn
            { #3 }
            \l_@@_renderer_prototype_definition_tl
        },
      }
%    \end{macrocode}
% \begin{markdown}
%
% Unless the token renderer prototype macro has already been defined,
% we provide an empty definition.
%
% \end{markdown}
%  \begin{macrocode}
    \cs_if_free:NT
      #2
      {
        \cs_generate_from_arg_count:NNnn
          #2
          \cs_set:Npn
          { #3 }
          { }
      }
  }
\cs_generate_variant:Nn
  \@@_define_renderer_prototype:nNn
  { ncV }
%    \end{macrocode}
% \par
% \begin{markdown}
%
% The following example code showcases a possible configuration of the
% `\markdownRendererImagePrototype` and `\markdownRendererCodeSpanPrototype`
% token renderer prototype macros.
% ``` tex
% \markdownSetup{
%   rendererPrototypes = {
%     image = {\pdfximage{#2}},    \% Embed PDF images in the document.
%     codeSpan = {{\tt #1}},     \% Render inline code using monospace.
%   }
% }
% ```````
%
% \end{markdown}
%  \begin{macrocode}
\keys_define:nn
  { markdown/options/renderer-prototypes }
  {
    unknown .code:n = {
%    \end{macrocode}
% \par
% \begin{markdown}
%
% Besides defining renderer prototypes at once, we can also define them
% incrementally using the appending operator (`+=`). This can be especially
% useful in defining rules for processing different \acro{HTML} class names
% and identifiers:
% ``` tex
% \markdownSetup{
%   rendererPrototypes = {
%     \% Start with empty renderer prototypes.
%     headerAttributeContextBegin = {},
%     attributeClassName = {},
%     attributeIdentifier = {},
%     \% Define the processing of a single specific HTML class name.
%     headerAttributeContextBegin += {
%       \markdownSetup{
%         rendererPrototypes = {
%           attributeClassName += {...},
%         },
%       }
%     },
%     \% Define the processing of a single specific HTML identifier.
%     headerAttributeContextBegin += {
%       \markdownSetup{
%         rendererPrototypes = {
%           attributeIdentifier += {...},
%         },
%       }
%     },
%   },
% }
% ```````
%
% \end{markdown}
%  \begin{macrocode}
      \regex_match:NVTF
        \c_@@_appending_key_regex
        \l_keys_key_str
        {
          \bool_gset_true:N
            \g_@@_appending_renderer_prototype_bool
          \tl_set:NV
            \l_tmpa_tl
            \l_keys_key_str
          \regex_replace_once:NnN
            \c_@@_appending_key_regex
            { }
            \l_tmpa_tl
          \tl_set:Nx
            \l_tmpb_tl
            { { \l_tmpa_tl } = }
          \tl_put_right:Nn
            \l_tmpb_tl
            { { #1 } }
          \keys_set:nV
            { markdown/options/renderer-prototypes }
            \l_tmpb_tl
          \bool_gset_false:N
            \g_@@_appending_renderer_prototype_bool
        }
%    \end{macrocode}
% \par
% \begin{markdown}
%
% In addition to exact token renderer prototype names, we also support
% wildcards (`*`) and enumerations (`|`) that match multiple token renderer
% prototype names:
% ``` tex
% \markdownSetup{
%   rendererPrototypes = {
%     heading* = {{\bf #1}},    \% Render headings using the bold face.
%     jekyllData(String|Number) = {   \% Render YAML string and numbers
%       {\it #2}\%                                     \% using italics.
%     },
%   }
% }
% ```````
%
% Wildcards and enumerations can be combined:
% ``` tex
% \markdownSetup{
%   rendererPrototypes = {
%     *lItem(|End) = {"},           \% Quote ordered/bullet list items.
%   }
% }
% ```````
%
% To determine the current token renderer prototype, you can use the
% pseudo-parameter `#0`:
% ``` tex
% \markdownSetup{
%   rendererPrototypes = {
%     heading* = {#0: #1}, \% Render headings as the renderer prototype
%   }                             \% name followed by the heading text.
% }
% ```````
%
% \end{markdown}
%  \begin{macrocode}
        {
          \@@_glob_seq:VnN
            \l_keys_key_str
            { g_@@_renderers_seq }
            \l_@@_renderer_glob_results_seq
          \seq_if_empty:NTF
            \l_@@_renderer_glob_results_seq
            {
              \msg_error:nnV
                { markdown }
                { undefined-renderer-prototype }
                \l_keys_key_str
            }
            {
              \tl_set:Nn
                \l_@@_renderer_glob_definition_tl
                { \exp_not:n { #1 } }
              \seq_map_inline:Nn
                \l_@@_renderer_glob_results_seq
                {
                  \tl_set:Nn
                    \l_tmpa_tl
                    { { ##1 } = }
                  \tl_put_right:Nx
                    \l_tmpa_tl
                    { { \l_@@_renderer_glob_definition_tl } }
                  \keys_set:nV
                    { markdown/options/renderer-prototypes }
                    \l_tmpa_tl
                }
            }
        }
    },
  }
\msg_new:nnn
  { markdown }
  { undefined-renderer-prototype }
  {
    Renderer~prototype~#1~is~undefined.
  }
\@@_with_various_cases:nn
  { rendererPrototypes }
  {
    \keys_define:nn
      { markdown/options }
      {
        #1 .code:n = {
          \keys_set:nn
            { markdown/options/renderer-prototypes }
            { ##1 }
        },
      }
  }
%    \end{macrocode}
% \begin{markdown}
%
% If plain \TeX{} is the top layer, we use the \mref{@@_define_renderer_prototypes:}
% macro to define plain \TeX{} token renderer prototype macros and key-values
% immediately. Otherwise, we postpone the definition until the upper layers
% have been loaded.
%
% \end{markdown}
%  \begin{macrocode}
\str_if_eq:VVT
  \c_@@_top_layer_tl
  \c_@@_option_layer_plain_tex_tl
  {
    \@@_define_renderer_prototypes:
  }
\ExplSyntaxOff
%    \end{macrocode}
% \begin{markdown}
%
%### Logging Facilities
% The \mdef{markdownInfo}, \mdef{markdownWarning}, and \mdef{markdownError}
% macros perform logging for the Markdown package. Their first argument
% specifies the text of the info, warning, or error message.
% The \mref{markdownError} macro receives a second argument that provides a help
% text. You may redefine these macros to redirect and process the info,
% warning, and error messages.
%
% The \mref{markdownInfo}, \mref{markdownWarning}, and \mref{markdownError}
% macros have been deprecated and will be removed in the next major version of
% the Markdown package.
%
%### Miscellanea
% The \mdef{markdownMakeOther} macro is used by the package, when a \TeX{}
% engine that does not support direct Lua access is starting to buffer a text.
% The plain \TeX{} implementation changes the category code of plain \TeX{}
% special characters to other, but there may be other active characters that
% may break the output. This macro should temporarily change the category of
% these to *other*.
%
% \end{markdown}
%  \begin{macrocode}
\let\markdownMakeOther\relax
%    \end{macrocode}
% \par
% \begin{markdown}
%
% The \mdef{markdownReadAndConvert} macro implements the \mref{markdownBegin}
% macro. The first argument specifies the token sequence that will terminate
% the markdown input (\mref{markdownEnd} in the instance of the \mref{markdownBegin}
% macro) when the plain \TeX{} special characters have had their category
% changed to *other*. The second argument specifies the token sequence
% that will actually be inserted into the document, when the ending token
% sequence has been found.
%
% \end{markdown}
%  \begin{macrocode}
\let\markdownReadAndConvert\relax
\begingroup
%    \end{macrocode}
% \begin{markdown}
% Locally swap the category code of the backslash symbol (`\`) with the pipe
% symbol (`|`). This is required in order that all the special symbols in the
% first argument of the `markdownReadAndConvert` macro have the category code
% *other*.
%
% \end{markdown}
%  \begin{macrocode}
  \catcode`\|=0\catcode`\\=12%
  |gdef|markdownBegin{%
    |markdownReadAndConvert{\markdownEnd}%
                           {|markdownEnd}}%
|endgroup
%    \end{macrocode}
% \begin{markdown}
% The macro is exposed in the interface, so that users can create their own
% markdown environments. Due to the way the arguments are passed to Lua, the
% first argument may not contain the string `]]` (regardless of the category
% code of the bracket symbol).
%
% The `code` key, which can be used to immediately expand and execute code.
% \end{markdown}
%  \begin{macrocode}
\ExplSyntaxOn
\keys_define:nn
  { markdown/options }
  {
    code .code:n = { #1 },
  }
\ExplSyntaxOff
%    \end{macrocode}
% \begin{markdown}
% This can be especially useful in snippets.
%
% \end{markdown}
% \iffalse
%</tex>
% \fi
% \par
% \begin{markdown}
%
% \LaTeX{} Interface {#latexinterface}
%--------------------
%
% \end{markdown}
% \iffalse
%<*manual-interfaces>

### \LaTeX{}

The \LaTeX{} interface provides the same level of functionality as the plain
\TeX{} interface by using the plain \TeX{} interface behind the scenes. Unlike
the plain \TeX{} interface, the \LaTeX{} interface uses familiar \LaTeX{}
idioms, such as package options and environments.

The \LaTeX{} interface accepts the same options as the plain \TeX{} interface,
but now the options are specified as \meta{key}${}={}$\meta{value} pairs and
they are passed either as package options, in the \mref{markdownSetup} command, or
as parameters for the \envmref{markdown*} \LaTeX{} environment.

Using a text editor, create a text document named `document.tex` with the
following content:
``` tex
\documentclass{article}
\usepackage{markdown}
\begin{document}

\begin{markdown}
$\sqrt{-1}$ *equals* $i$
\end{markdown}

\begin{markdown}[texMathDollars]
$\sqrt{-1}$ *equals* $i$
\end{markdown}

\end{document}
```````
Next, invoke LuaTeX from the terminal:
``` sh
lualatex document.tex
``````
A PDF document named `document.pdf` should be produced and contain the
following text:

> \$\\sqrt{-1}\$ *equals* \$i\$.
>
> <math><msqrt><mo>−</mo><mn>1</mn></msqrt></math>
> *equals*
> <math><mi>i</mi></math>.

Invoking pdfTeX should have the same effect:
``` sh
pdflatex --shell-escape document.tex
``````

%</manual-interfaces>
%<*latex>
% \fi
% \begin{markdown}
%
% The \LaTeX{} interface provides \LaTeX{} environments for the typesetting of
% markdown input from within \LaTeX{}, facilities for setting Lua, plain \TeX,
% and \LaTeX{} options used during the conversion from markdown to plain
% \TeX{}, and facilities for changing the way markdown tokens are rendered. The
% rest of the interface is inherited from the plain \TeX{} interface (see
% Section <#sec:texinterface>).
%
% To determine whether \LaTeX{} is the top layer or if there are other
% layers above \LaTeX{}, we take a look on whether the
% \mref{c_@@_top_layer_tl} token list has already been defined. If not,
% we will assume that \LaTeX{} is the top layer.
%
% \end{markdown}
%  \begin{macrocode}
\ExplSyntaxOn
\tl_const:Nn \c_@@_option_layer_latex_tl { latex }
\cs_generate_variant:Nn
  \tl_const:Nn
  { NV }
\tl_if_exist:NF
  \c_@@_top_layer_tl
  {
    \tl_const:NV
      \c_@@_top_layer_tl
      \c_@@_option_layer_latex_tl
  }
\ExplSyntaxOff
\input markdown/markdown
%    \end{macrocode}
% \begin{markdown}
%
% The \LaTeX{} interface is implemented by the `markdown.sty` file, which
% can be loaded from the \LaTeX{} document preamble as follows:
% \end{markdown}
% \begin{Verbatim}[commandchars=\\\{\},gobble=2]
% \textbackslash{}usepackage[\textrm{\meta{options}}]\{markdown\}
% \end{Verbatim}
% \begin{markdown}
% where \meta{options} are the \LaTeX{} interface options (see Section
% <#sec:latex-options>). Note that \meta{options} inside the `\usepackage`
% macro may not set the `markdownRenderers` (see Section
% <#sec:plain-tex-renderers>) and `markdownRendererPrototypes` (see Section
% <#sec:plain-tex-renderer-prototypes>) keys. Furthermore, although the
% base variant of the `import` key that loads a single \LaTeX{} theme
% (see Section <#sec:latexthemes>) can be used, the extended variant
% that can load multiple themes and import snippets from them (see
% Section <#sec:snippets>) cannot. This limitation is due to the way
% \Hologo{LaTeX2e} parses package options.
%
% \end{markdown}
% \par
% \begin{markdown}
%
%### Typesetting Markdown
% The interface exposes the \envmdef{markdown} and \envmdef{markdown*}
% \LaTeX{} environments, and redefines the \mref{markinline} and
% \mref{markdownInput} commands.
%
% The \envmref{markdown} and \envmref{markdown*} \LaTeX{} environments are used
% to typeset markdown document fragments. Both \LaTeX{} environments accept
% \LaTeX{} interface options (see ection <#sec:latex-options>) as the only
% argument. This argument is optional for the \envmref{markdown} environment
% and mandatory for the \envmref{markdown*} environment.
%
% The \envmref{markdown*} environment has been deprecated and will be removed
% in the next major version of the Markdown package.
%
% \end{markdown}
%  \begin{macrocode}
\newenvironment{markdown}\relax\relax
\newenvironment{markdown*}[1]\relax\relax
%    \end{macrocode}
% \markdownBegin
%
% You may prepend your own code to the \mdef{markdown} macro and append your own
% code to the \mref{markdownEnd} macro to produce special effects before and after
% the \envmref{markdown} \LaTeX{} environment (and likewise for the starred
% version).
%
% Note that the \envmref{markdown} and \envmref{markdown*} \LaTeX{} environments are
% subject to the same limitations as the \mref{markdownBegin} and \mref{markdownEnd}
% macros exposed by the plain \TeX{} interface.
%
% The following example \LaTeX{} code showcases the usage of the
% \envmref{markdown} and \envmref{markdown*} environments:
% ``` tex
% \documentclass{article}            \documentclass{article}
% \usepackage{markdown}              \usepackage{markdown}
% \begin{document}                   \begin{document}
% \% ...                              \% ...
% \begin{markdown}[smartEllipses]    \begin{markdown*}{smartEllipses}
% _Hello_ **world** ...              _Hello_ **world** ...
% \end{markdown}                     \end{markdown*}
% \% ...                              \% ...
% \end{document}                     \end{document}
% ```````
%
% The \mref{markinline} macro accepts a single mandatory parameter containing
% inline markdown content and expands to the result of the conversion of the
% input markdown document to plain \TeX{}.  Unlike the \mref{markinline} macro
% provided by the plain \TeX{} interface, this macro also accepts \LaTeX{}
% interface options (see Section <#sec:latex-options>) as its optional
% argument. These options will only influnce this markdown content.
%
% The \mref{markdownInput} macro accepts a single mandatory parameter containing
% the filename of a markdown document and expands to the result of the
% conversion of the input markdown document to plain \TeX{}.  Unlike the
% \mref{markdownInput} macro provided by the plain \TeX{} interface, this macro
% also accepts \LaTeX{} interface options (see Section <#sec:latex-options>)
% as its optional argument. These options will only influnce this markdown
% document.
%
% The following example \LaTeX{} code showcases the usage of the
% \mref{markdownInput} macro:
% ``` tex
% \documentclass{article}
% \usepackage{markdown}
% \begin{document}
% % ...
% \markdownInput[smartEllipses]{hello.md}
% % ...
% \end{document}
% ```````
%
%### Options {#latex-options}
%
% The \LaTeX{} options are represented by a comma-delimited list of
% \meta{key}`=`\meta{value} pairs. For boolean options, the `=`\meta{value}
% part is optional, and \meta{key} will be interpreted as \meta{key}`=true`
% if the `=`\meta{value} part has been omitted.
%
% \LaTeX{} options map directly to the options recognized by the plain
% \TeX{} interface (see Section <#sec:tex-options>) and to the markdown token
% renderers and their prototypes recognized by the plain \TeX{} interface (see
% Sections <#sec:texrenderersuser> and <#sec:texrendererprototypes>).
%
% The \LaTeX{} options may be specified when loading the \LaTeX{} package, when
% using the \envmref{markdown*} \LaTeX{} environment or the \mref{markdownInput}
% macro (see Section <#sec:latexinterface>), or via the \mref{markdownSetup}
% macro.
%
%#### Finalizing and Freezing the Cache
%
% To ensure compatibility with the `minted` package [@poore17, Section
% 5.1], which supports the `finalizecache` and `frozencache` package options
% with similar semantics to the \Opt{finalizeCache} and \Opt{frozenCache} plain
% \TeX{} options, the Markdown package also recognizes these as aliases
% and accepts them as document class options. By passing `finalizecache` and
% `frozencache` as document class options, you may conveniently control the
% behavior of both packages at once:
%
% ``` tex
% \documentclass[frozencache]{article}
% \usepackage{markdown,minted}
% \begin{document}
% % ...
% \end{document}
% ```````
%
% We hope that other packages will support the `finalizecache` and
% `frozencache` package options in the future, so that they can become a
% standard interface for preparing \LaTeX{} document sources for distribution.
%
% \markdownEnd
%  \begin{macrocode}
\DeclareOption{finalizecache}{\markdownSetup{finalizeCache}}
\DeclareOption{frozencache}{\markdownSetup{frozenCache}}
%    \end{macrocode}
% \iffalse
%</latex>
%<*manual-options>

### \LaTeX{}

\LaTeX{} options allow us to disable the redefinition of the default renderer
prototypes from plain \TeX{}, load user-defined themes, and invoke user-defined
set-up snippets.

#### Setting Lua and plain \TeX{} options from \LaTeX{}

As a rule of thumb, we can set all Lua options directly from \LaTeX{}. For
example, to set the \Opt{taskLists} Lua option to `true`, we would include the
following code in our \LaTeX{} document:

``` tex
\markdownSetup{
  taskLists = true,
}
```

We can also set all plain \TeX{} options directly from \LaTeX{}. For example,
to set the `\markdownOptionInputTempFileName` plain \TeX{} option to
`helper-script.lua`, we would include the following code in our \LaTeX{}
document:

``` tex
\markdownSetup{
  inputTempFileName = temporary-input.md,
}
```

%</manual-options>
%<*latex>
% \fi
% \begin{markdown}
%
%#### Generating Plain \TeX{} Option, Token Renderer, and Token Renderer Prototype Macros and Key-Values
%
% If \LaTeX{} is the top layer, we use the
% \mref{@@_define_option_commands_and_keyvals:}, \mref{@@_define_renderers:},
% and \mref{@@_define_renderer_prototypes:} macro to define plain \TeX{}
% option, token renderer, and token renderer prototype macros and key-values
% immediately. Otherwise, we postpone the definition until the upper layers
% have been loaded.
%
% \end{markdown}
%  \begin{macrocode}
\ExplSyntaxOn
\str_if_eq:VVT
  \c_@@_top_layer_tl
  \c_@@_option_layer_latex_tl
  {
    \@@_define_option_commands_and_keyvals:
    \@@_define_renderers:
    \@@_define_renderer_prototypes:
  }
\ExplSyntaxOff
%    \end{macrocode}
% \par
% \begin{markdown}
%
% The following example \LaTeX{} code showcases a possible configuration of
% plain \TeX{} interface options \Opt{hybrid}, \Opt{smartEllipses}, and
% \Opt{cacheDir}.
% ``` tex
% \markdownSetup{
%   hybrid,
%   smartEllipses,
%   cacheDir = /tmp,
% }
% ```````
%
% \end{markdown}
% \iffalse
%</latex>
%<*manual-options>
% \fi
% \begin{markdown}

### Themes {#latexthemes}

% In Section~\ref{sec:themes}, we described the concept of themes.
In \LaTeX{}, we expand on the concept of
% themes\iffalse
[themes](#themes)
% \fi
by allowing a theme to be a full-blown \LaTeX{} package. Specifically, the
key-values `theme`=\meta{theme name} and `import`=\meta{theme name} load a
\LaTeX{} package named `markdowntheme`\meta{munged theme name}`.sty` if it
exists and a \TeX{} document named `markdowntheme`\meta{munged theme
name}`.tex` otherwise.

Having the Markdown package automatically load either the generic `.tex`
*theme file* or the \LaTeX{}-specific `.sty` theme file allows developers
to have a single *theme file*, when the theme is small or the difference
between \TeX{} formats is unimportant, and scale up to separate theme files
native to different \TeX{} formats for large multi-format themes, where
different code is needed for different \TeX{} formats. To enable code reuse,
developers can load the `.tex` theme file from the `.sty` theme file using the
\mdef{markdownLoadPlainTeXTheme} macro.

% If the \LaTeX{} option with keys `theme` or `import` is (repeatedly)
% specified in the `\usepackage` macro, the loading of the theme(s) will be
% postponed in first-in-first-out order until after the Markdown \LaTeX{}
% package has been loaded. Otherwise, the theme(s) will be loaded immediately.
% For example, there is a theme named `witiko/dot`, which typesets fenced code
% blocks with the `dot` infostring as images of directed graphs rendered by the
% Graphviz tools. The following code would first load the Markdown package,
% then the `markdownthemewitiko_beamer_MU.sty` \LaTeX{} package, and finally
% the `markdownthemewitiko_dot.sty` \LaTeX{} package:
% \end{markdown}
% \iffalse

For example, to load themes named `witiko/beamer/MU` and `witiko/dot`, you
would use the following code in the preamble of your document:

% \fi
% \par
% \begin{markdown}

``` tex
\usepackage[
  import=witiko/beamer/MU,
  import=witiko/dot,
]{markdown}
```````

% \end{markdown}
% \iffalse
%</manual-options>
%<*latex>
% \fi
%  \begin{macrocode}
\newif\ifmarkdownLaTeXLoaded
  \markdownLaTeXLoadedfalse
%    \end{macrocode}
% \iffalse
%</latex>
%<*manual-options>
% \fi
% \par
% \begin{markdown}

Due to limitations of \LaTeX{}, themes may not be loaded after the
beginning of a \LaTeX{} document.

% \end{markdown}
% \par
% \markdownBegin

Built-in \LaTeX{} themes provided with the Markdown package include:

\pkg{witiko/dot}

:    A theme that typesets fenced code blocks with the `dot …` infostring
     as images of directed graphs rendered by the Graphviz tools. The
     right tail of the infostring is used as the image title.
%    ```` tex
%    \documentclass{article}
%    \usepackage[import=witiko/dot]{markdown}
%    \setkeys{Gin}{
%      width = \columnwidth,
%      height = 0.65\paperheight,
%      keepaspectratio}
%    \begin{document}
%    \begin{markdown}
%    ``` dot Various formats of mathemathical formulae
%    digraph tree {
%      margin = 0;
%      rankdir = "LR";
%
%      latex -> pmml;
%      latex -> cmml;
%      pmml -> slt;
%      cmml -> opt;
%      cmml -> prefix;
%      cmml -> infix;
%      pmml -> mterms [style=dashed];
%      cmml -> mterms;
%
%      latex [label = "LaTeX"];
%      pmml [label = "Presentation MathML"];
%      cmml [label = "Content MathML"];
%      slt [label = "Symbol Layout Tree"];
%      opt [label = "Operator Tree"];
%      prefix [label = "Prefix"];
%      infix [label = "Infix"];
%      mterms [label = "M-Terms"];
%    }
%    ```
%    \end{markdown}
%    \end{document}
%    ````````
%    Typesetting the above document produces the output shown in
%    Figure <#fig:witiko/dot>.
%    ``` dot Various formats of mathemathical formulae \label{fig:witiko/dot}
%    digraph tree {
%      margin = 0;
%      rankdir = "LR";
%
%      latex -> pmml;
%      latex -> cmml;
%      pmml -> slt;
%      cmml -> opt;
%      cmml -> prefix;
%      cmml -> infix;
%      pmml -> mterms [style=dashed];
%      cmml -> mterms;
%
%      latex [label = "LaTeX"];
%      pmml [label = "Presentation MathML"];
%      cmml [label = "Content MathML"];
%      slt [label = "Symbol Layout Tree"];
%      opt [label = "Operator Tree"];
%      prefix [label = "Prefix"];
%      infix [label = "Infix"];
%      mterms [label = "M-Terms"];
%    }
%    ```
     The theme requires a Unix-like operating system with GNU Diffutils and
     Graphviz installed. The theme also requires shell access unless the
     \Opt{frozenCache} plain \TeX{} option is enabled.

% \markdownEnd
% \iffalse

##### \LaTeX{} Example {.unnumbered}

Using a text editor, create a text document named `document.tex` with the
following content:
```` tex
\documentclass{article}
\usepackage[import=witiko/dot]{markdown}
\setkeys{Gin}{
  width=\columnwidth,
  height=0.65\paperheight,
  keepaspectratio}
\begin{document}
\begin{markdown}
``` dot Various formats of mathemathical formulae
digraph tree {
  margin = 0;
  rankdir = "LR";

  latex -> pmml;
  latex -> cmml;
  pmml -> slt;
  cmml -> opt;
  cmml -> prefix;
  cmml -> infix;
  pmml -> mterms [style=dashed];
  cmml -> mterms;

  latex [label = "LaTeX"];
  pmml [label = "Presentation MathML"];
  cmml [label = "Content MathML"];
  slt [label = "Symbol Layout Tree"];
  opt [label = "Operator Tree"];
  prefix [label = "Prefix"];
  infix [label = "Infix"];
  mterms [label = "M-Terms"];
}
```
\end{markdown}
\end{document}
````````
Next, invoke LuaTeX from the terminal:
``` sh
lualatex document.tex
``````
A PDF document named `document.pdf` should be produced and contain
a drawing of a directed graph similar to Figure 1 from the following
conference article:

> NOVOTNÝ, Vít, Petr SOJKA, Michal ŠTEFÁNIK and Dávid LUPTÁK. Three is Better
> than One: Ensembling Math Information Retrieval Systems. *CEUR Workshop
> Proceedings*. Thessaloniki, Greece: M. Jeusfeld c/o Redaktion Sun SITE,
> Informatik V, RWTH Aachen., 2020, vol. 2020, No 2696, p. 1-30. ISSN 1613-0073.
> <http://ceur-ws.org/Vol-2696/paper_235.pdf>

%</manual-options>
%<*themes-witiko-dot>
% \fi
%  \begin{macrocode}
\ProvidesPackage{markdownthemewitiko_dot}[2021/03/09]%
%    \end{macrocode}
% \iffalse
%</themes-witiko-dot>
%<*manual-options>
% \fi
% \par
% \markdownBegin

\pkg{witiko/graphicx/http}

:    A theme that adds support for downloading images whose URL has the
     http or https protocol.
%    ``` tex
%    \documentclass{article}
%    \usepackage[import=witiko/graphicx/http]{markdown}
%    \begin{document}
%    \begin{markdown}
%    ![img](https://github.com/witiko/markdown/raw/main/markdown.png
%           "The banner of the Markdown package")
%    \end{markdown}
%    \end{document}
%    ```````
%    Typesetting the above document produces the output shown in
%    Figure <#fig:witiko/graphicx/http>.
%    ![img](https://github.com/witiko/markdown/raw/main/markdown.png
%           "The banner of the Markdown package \label{fig:witiko/graphicx/http}")
     The theme requires the \pkg{catchfile} \LaTeX{} package and a Unix-like
     operating system with GNU Coreutils `md5sum` and either GNU Wget or cURL
     installed. The theme also requires shell access unless the
     \Opt{frozenCache} plain \TeX{} option is enabled.

% \markdownEnd
% \iffalse

##### \LaTeX{} Example {.unnumbered}

Using a text editor, create a text document named `document.tex` with the
following content:
``` tex
\documentclass{article}
\usepackage[import=witiko/graphicx/http]{markdown}
\begin{document}
\begin{markdown}
![img](https://github.com/witiko/markdown/raw/main/markdown.png
       "The banner of the Markdown package")
\end{markdown}
\end{document}
```````
Next, invoke LuaTeX from the terminal:
``` sh
lualatex document.tex
``````
A PDF document named `document.pdf` should be produced and contain the
following image:

> ![img](https://github.com/witiko/markdown/raw/main/markdown.png "The banner of the Markdown package")

%</manual-options>
%<*themes-witiko-graphicx-http>
% \fi
%  \begin{macrocode}
\ProvidesPackage{markdownthemewitiko_graphicx_http}[2021/03/22]%
%    \end{macrocode}
% \iffalse
%</themes-witiko-graphicx-http>
%<*manual-options>
% \fi
% \par
% \begin{markdown}

\pkg{witiko/markdown/defaults}

:    A \LaTeX{} theme with the default definitions of token renderer prototypes
     for plain \TeX{}. This theme is loaded automatically together with the
     package and explicitly loading it has no effect.

% \end{markdown}
% \iffalse
%</manual-options>
%<*latex>
% \fi
%  \begin{macrocode}
\AtEndOfPackage{
  \markdownLaTeXLoadedtrue
%    \end{macrocode}
% \begin{markdown}
%
% At the end of the \LaTeX{} module, we load the
% `witiko/markdown/defaults` \LaTeX{} theme (see Section <#sec:themes>) with
% the default definitions for token renderer prototypes unless the option
% `noDefaults` has been enabled (see Section <#sec:plain>).
%
% \end{markdown}
%  \begin{macrocode}
  \markdownIfOption{noDefaults}{}{
    \markdownSetup{theme=witiko/markdown/defaults}
  }
}
%    \end{macrocode}
% \iffalse
%</latex>
%<*themes-witiko-markdown-defaults-latex>
% \fi
%  \begin{macrocode}
\ProvidesPackage{markdownthemewitiko_markdown_defaults}[2024/01/03]%
%    \end{macrocode}
% \iffalse
%</themes-witiko-markdown-defaults-latex>
%<*context>
% \fi
% \par
% \begin{markdown}
%
% Please, see Section <#sec:latex-themes-implementation> for implementation
% details of the built-in \LaTeX{} themes.
%
% \Hologo{ConTeXt} Interface {#contextinterface}
%----------------------------
%
% \end{markdown}
% \iffalse
%</context>
%<*manual-interfaces>

### \Hologo{ConTeXt}

The \Hologo{ConTeXt} interface provides the same level of functionality as the
plain \TeX{} interface by using the plain \TeX{} interface behind the scenes.
Unlike the plain \TeX{} interface, the \Hologo{ConTeXt} interface uses familiar
\Hologo{ConTeXt} idioms as syntactic sugar.

The \Hologo{ConTeXt} interface accepts the same options as the plain \TeX{}
interface.

Using a text editor, create a text document named `document.tex` with the
following content:
``` tex
\usemodule[t][markdown]
\starttext

\startmarkdown
$\sqrt{-1}$ *equals* $i$.
\stopmarkdown

\setupmarkdown[texmathdollars = yes]
\startmarkdown
$\sqrt{-1}$ *equals* $i$.
\stopmarkdown

\stoptext
```````
Next, invoke LuaTeX from the terminal:
``` sh
context --luatex document.tex
``````
A PDF document named `document.pdf` should be produced and contain the
following text:

> \$\\sqrt{-1}\$ *equals* \$i\$.
>
> <math><msqrt><mo>−</mo><mn>1</mn></msqrt></math>
> *equals*
> <math><mi>i</mi></math>.

%</manual-interfaces>
%<*context>
% \fi
% \begin{markdown}
%
% To determine whether \Hologo{ConTeXt} is the top layer or if there are other
% layers above \Hologo{ConTeXt}, we take a look on whether the
% \mref{c_@@_top_layer_tl} token list has already been defined. If not,
% we will assume that \Hologo{ConTeXt} is the top layer.
%
% \end{markdown}
%  \begin{macrocode}
\ExplSyntaxOn
\tl_const:Nn \c_@@_option_layer_context_tl { context }
\cs_generate_variant:Nn
  \tl_const:Nn
  { NV }
\tl_if_exist:NF
  \c_@@_top_layer_tl
  {
    \tl_const:NV
      \c_@@_top_layer_tl
      \c_@@_option_layer_context_tl
  }
\ExplSyntaxOff
%    \end{macrocode}
% \begin{markdown}
%
% The \Hologo{ConTeXt} interface provides a start-stop macro pair for the
% typesetting of markdown input from within \Hologo{ConTeXt} and facilities for
% setting Lua, plain \TeX, and \Hologo{ConTeXt} options used during the
% conversion from markdown to plain \TeX{}. The rest of the interface is
% inherited from the plain \TeX{} interface (see Section <#sec:texinterface>).
%
% \end{markdown}
%  \begin{macrocode}
\writestatus{loading}{ConTeXt User Module / markdown}%
\startmodule[markdown]
\def\dospecials{\do\ \do\\\do\{\do\}\do\$\do\&%
  \do\#\do\^\do\_\do\%\do\~}%
\input markdown/markdown
%    \end{macrocode}
% \par
% \begin{markdown}
%
% The \Hologo{ConTeXt} interface is implemented by the
% `t-markdown.tex` \Hologo{ConTeXt} module file that can be loaded as follows:
% ``` tex
% \usemodule[t][markdown]
% ```````
% \noindent It is expected that the special plain \TeX{} characters have the
% expected category codes, when `\input`ting the file.
%
%### Typesetting Markdown
% The interface exposes the \mdef{startmarkdown} and \mdef{stopmarkdown} macro
% pair for the typesetting of a markdown document fragment, and defines the
% \mdef{inputmarkdown} macro.
%
% \end{markdown}
%  \begin{macrocode}
\let\startmarkdown\relax
\let\stopmarkdown\relax
\let\inputmarkdown\relax
%    \end{macrocode}
% \par
% \begin{markdown}
%
% You may prepend your own code to the \mref{startmarkdown} macro and redefine the
% \mref{stopmarkdown} macro to produce special effects before and after the
% markdown block.
%
% Note that the \mref{startmarkdown} and \mref{stopmarkdown} macros
% are subject to the same limitations as the \mref{markdownBegin} and
% \mref{markdownEnd} macros exposed by the plain \TeX{} interface.
%
% The following example \Hologo{ConTeXt} code showcases the usage of the
% \mref{startmarkdown} and \mref{stopmarkdown} macros:
% ``` tex
% \usemodule[t][markdown]
% \starttext
% \startmarkdown
% _Hello_ **world** ...
% \stopmarkdown
% \stoptext
% ```````
%
% The \mref{inputmarkdown} macro accepts a single mandatory parameter
% containing the filename of a markdown document and expands to the result of
% the conversion of the input markdown document to plain \TeX{}. Unlike the
% \mref{markdownInput} macro provided by the plain \TeX{} interface, this macro
% also accepts \Hologo{ConTeXt} interface options (see Section
% <#sec:context-options>) as its optional argument. These options will only
% influnce this markdown document.
%
% The following example \LaTeX{} code showcases the usage of the
% \mref{markdownInput} macro:
% ``` tex
% \usemodule[t][markdown]
% \starttext
% % ...
% \inputmarkdown[smartEllipses]{hello.md}
% % ...
% \stoptext
% ```````
%
%### Options {#context-options}
%
% The \Hologo{ConTeXt} options are represented by a comma-delimited list of
% \meta{key}`=`\meta{value} pairs. For boolean options, the `=`\meta{value}
% part is optional, and \meta{key} will be interpreted as \meta{key}`=true`
% (or, equivalently, \meta{key}`=yes`) if the `=`\meta{value} part has been
% omitted.
%
% \Hologo{ConTeXt} options map directly to the options recognized by the plain
% \TeX{} interface (see Section <#sec:tex-options>).
%
% The \Hologo{ConTeXt} options may be specified when using the
% \mref{inputmarkdown} macro (see Section <#sec:contextinterface>),
% via the \mref{markdownSetup} macro, or via the \mdef{setupmarkdown}`[#1]`
% macro, which is an alias for \mref{markdownSetup}`{#1}`.
%
% \end{markdown}
%  \begin{macrocode}
\ExplSyntaxOn
\cs_new:Npn
  \setupmarkdown
  [ #1 ]
  {
    \@@_setup:n
      { #1 }
  }
\ExplSyntaxOff
%    \end{macrocode}
% \begin{markdown}
%
%#### Generating Plain \TeX{} Option Macros and Key-Values
%
% Unlike plain \TeX{}, we also accept caseless variants of options
% in line with the style of \Hologo{ConTeXt}.
%
% \end{markdown}
%  \begin{macrocode}
\ExplSyntaxOn
\cs_new:Nn \@@_caseless:N
  {
    \regex_replace_all:nnN
      { ([a-z])([A-Z]) }
      { \1 \c { str_lowercase:n } \cB\{ \2 \cE\} }
      #1
    \tl_set:Nx
      #1
      { #1 }
  }
\seq_gput_right:Nn \g_@@_cases_seq { @@_caseless:N }
%    \end{macrocode}
% \begin{markdown}
%
% If \Hologo{ConTeXt} is the top layer, we use the
% \mref{@@_define_option_commands_and_keyvals:}, \mref{@@_define_renderers:},
% and \mref{@@_define_renderer_prototypes:} macro to define plain \TeX{}
% option, token renderer, and token renderer prototype macros and key-values
% immediately. Otherwise, we postpone the definition until the upper layers
% have been loaded.
%
% \end{markdown}
%  \begin{macrocode}
\str_if_eq:VVT
  \c_@@_top_layer_tl
  \c_@@_option_layer_context_tl
  {
    \@@_define_option_commands_and_keyvals:
    \@@_define_renderers:
    \@@_define_renderer_prototypes:
  }
\ExplSyntaxOff
%    \end{macrocode}
% \iffalse
%</context>
%<*manual-options>
% \fi
% \begin{markdown}

### Themes

% In Section~\ref{sec:themes}, we described the concept of themes.
In \Hologo{ConTeXt}, we expand on the concept of
% themes\iffalse
[themes](#themes)
% \fi
by allowing a theme to be a full-blown \Hologo{ConTeXt} module. Specifically,
the key-values `theme`=\meta{theme name} and `import`=\meta{theme name} load a
\Hologo{ConTeXt} module named `t-markdowntheme`\meta{munged theme name}`.tex`
if it exists and a \TeX{} document named `markdowntheme`\meta{munged theme
name}`.tex` otherwise.

Having the Markdown package automatically load either the generic `.tex`
*theme file* or the \Hologo{ConTeXt}-specific `t-*.tex` theme file allows
developers to have a single *theme file*, when the theme is small or the
difference between \TeX{} formats is unimportant, and scale up to separate
theme files native to different \TeX{} formats for large multi-format themes,
where different code is needed for different \TeX{} formats. To enable code
reuse, developers can load the `.tex` theme file from the `t-*.tex` theme file
using the \mref{markdownLoadPlainTeXTheme} macro.

For example, to load a theme named `witiko/tilde` in your document:

``` tex
\usemodule[t][markdown]
\setupmarkdown[import=witiko/tilde]
```````

Built-in \Hologo{ConTeXt} themes provided with the Markdown package include:

\pkg{witiko/markdown/defaults}

:    A \Hologo{ConTeXt} theme with the default definitions of token renderer
     prototypes for plain \TeX{}. This theme is loaded automatically together
     with the package and explicitly loading it has no effect.

% \end{markdown}
% \iffalse
%</manual-options>
%<*themes-witiko-markdown-defaults-context>
% \fi
%  \begin{macrocode}
\startmodule[markdownthemewitiko_markdown_defaults]
\unprotect
%    \end{macrocode}
% \iffalse
%</themes-witiko-markdown-defaults-context>
%<*manual-options>
% \fi
% \begin{markdown}
%
% Please, see Section <#sec:context-themes-implementation> for implementation
% details of the built-in \Hologo{ConTeXt} themes.
%
% Implementation {#implementation}
%================
%
% This part of the documentation describes the implementation of the interfaces
% exposed by the package (see Section <#sec:interfaces>) and is aimed at the
% developers of the package, as well as the curious users.
%
% Figure <#fig:block-diagram> shows the high-level structure of the Markdown
% package: The translation from markdown to \TeX{} *token renderers* is
% performed by the Lua layer. The plain \TeX{} layer provides default
% definitions for the token renderers. The \LaTeX{} and \Hologo{ConTeXt} layers
% correct idiosyncrasies of the respective \TeX{} formats, and provide
% format-specific default definitions for the token renderers.
%
% \end{markdown}
% \iffalse
%</manual-options>
%<*lua>
% \fi
% \begin{markdown}
%
% Lua Implementation {#luaimplementation}
%--------------------
%
% The Lua implementation implements \luamdef{writer} and \luamref{reader}
% objects, which provide the conversion from markdown to plain \TeX, and
% \luamdef{extensions} objects, which provide syntax extensions for the
% \luamref{writer} and \luamref{reader} objects.
%
% The Lunamark Lua module implements writers for the conversion to various
% other formats, such as DocBook, Groff, or \acro{HTML}. These were stripped
% from the module and the remaining markdown reader and plain \TeX{} writer
% were hidden behind the converter functions exposed by the Lua interface (see
% Section <#sec:luainterface>).
%
% \end{markdown}
%  \begin{macrocode}
local upper, format, length =
  string.upper, string.format, string.len
local P, R, S, V, C, Cg, Cb, Cmt, Cc, Ct, B, Cs, Cp, any =
  lpeg.P, lpeg.R, lpeg.S, lpeg.V, lpeg.C, lpeg.Cg, lpeg.Cb,
  lpeg.Cmt, lpeg.Cc, lpeg.Ct, lpeg.B, lpeg.Cs, lpeg.Cp, lpeg.P(1)
%    \end{macrocode}
% \par
% \begin{markdown}
%
%### Utility Functions
% This section documents the utility functions used by the plain \TeX{}
% writer and the markdown reader. These functions are encapsulated in the
% `util` object. The functions were originally located in the
% `lunamark/util.lua` file in the Lunamark Lua module.
%
% \end{markdown}
%  \begin{macrocode}
local util = {}
%    \end{macrocode}
% \par
% \begin{markdown}
%
% The \luamdef{util.err} method prints an error message `msg` and exits.
% If `exit_code` is provided, it specifies the exit code.  Otherwise, the
% exit code will be 1.
%
% \end{markdown}
%  \begin{macrocode}
function util.err(msg, exit_code)
  io.stderr:write("markdown.lua: " .. msg .. "\n")
  os.exit(exit_code or 1)
end
%    \end{macrocode}
% \par
% \begin{markdown}
%
% The \luamdef{util.cache} method computes the digest of `string` and
% `salt`, adds the `suffix` and looks into the directory `dir`, whether a
% file with such a name exists. If it does not, it gets created with
% `transform(string)` as its content. The filename is then returned.
%
% \end{markdown}
%  \begin{macrocode}
function util.cache(dir, string, salt, transform, suffix)
  local digest = md5.sumhexa(string .. (salt or ""))
  local name = util.pathname(dir, digest .. suffix)
  local file = io.open(name, "r")
  if file == nil then -- If no cache entry exists, then create a new one.
    file = assert(io.open(name, "w"),
      [[Could not open file "]] .. name .. [[" for writing]])
    local result = string
    if transform ~= nil then
      result = transform(result)
    end
    assert(file:write(result))
    assert(file:close())
  end
  return name
end
%    \end{macrocode}
% \par
% \begin{markdown}
%
% The \luamdef{util.cache_verbatim} method strips whitespaces from the
% end of `string` and calls \luamref{util.cache} with `dir`, `string`,
% no salt or transformations, and the `.verbatim` suffix.
%
% \end{markdown}
%  \begin{macrocode}
function util.cache_verbatim(dir, string)
  local name = util.cache(dir, string, nil, nil, ".verbatim")
  return name
end
%    \end{macrocode}
% \par
% \begin{markdown}
%
% The \luamdef{util.table_copy} method creates a shallow copy of a table `t`
% and its metatable.
%
% \end{markdown}
%  \begin{macrocode}
function util.table_copy(t)
  local u = { }
  for k, v in pairs(t) do u[k] = v end
  return setmetatable(u, getmetatable(t))
end
%    \end{macrocode}
% \par
% \begin{markdown}
%
% The \luamdef{util.encode_json_string} method encodes a string `s` in
% \acro{JSON}.
%
% \end{markdown}
%  \begin{macrocode}
function util.encode_json_string(s)
  s = s:gsub([[\]], [[\\]])
  s = s:gsub([["]], [[\"]])
  return [["]] .. s .. [["]]
end
%    \end{macrocode}
% \par
% \begin{markdown}
%
% The \luamdef{util.expand_tabs_in_line} expands tabs in string `s`. If
% `tabstop` is specified, it is used as the tab stop width. Otherwise,
% the tab stop width of 4 characters is used. The method is a copy of the tab
% expansion algorithm from @ierusalimschy13 [Chapter 21].
%
% \end{markdown}
%  \begin{macrocode}
function util.expand_tabs_in_line(s, tabstop)
  local tab = tabstop or 4
  local corr = 0
  return (s:gsub("()\t", function(p)
            local sp = tab - (p - 1 + corr) % tab
            corr = corr - 1 + sp
            return string.rep(" ", sp)
          end))
end
%    \end{macrocode}
% \par
% \begin{markdown}
%
% The \luamdef{util.walk} method walks a rope `t`, applying a function `f`
% to each leaf element in order. A rope is an array whose elements may be
% ropes, strings, numbers, or functions.  If a leaf element is a function, call
% it and get the return value before proceeding.
%
% \end{markdown}
%  \begin{macrocode}
function util.walk(t, f)
  local typ = type(t)
  if typ == "string" then
    f(t)
  elseif typ == "table" then
    local i = 1
    local n
    n = t[i]
    while n do
      util.walk(n, f)
      i = i + 1
      n = t[i]
    end
  elseif typ == "function" then
    local ok, val = pcall(t)
    if ok then
      util.walk(val,f)
    end
  else
    f(tostring(t))
  end
end
%    \end{macrocode}
% \par
% \begin{markdown}
%
% The \luamdef{util.flatten} method flattens an array `ary` that does not
% contain cycles and returns the result.
%
% \end{markdown}
%  \begin{macrocode}
function util.flatten(ary)
  local new = {}
  for _,v in ipairs(ary) do
    if type(v) == "table" then
      for _,w in ipairs(util.flatten(v)) do
        new[#new + 1] = w
      end
    else
      new[#new + 1] = v
    end
  end
  return new
end
%    \end{macrocode}
% \par
% \begin{markdown}
%
% The \luamdef{util.rope_to_string} method converts a rope `rope` to a
% string and returns it. For the definition of a rope, see the definition of
% the \luamref{util.walk} method.
%
% \end{markdown}
%  \begin{macrocode}
function util.rope_to_string(rope)
  local buffer = {}
  util.walk(rope, function(x) buffer[#buffer + 1] = x end)
  return table.concat(buffer)
end
%    \end{macrocode}
% \par
% \begin{markdown}
%
% The \luamdef{util.rope_last} method retrieves the last item in a rope. For
% the definition of a rope, see the definition of the \luamref{util.walk} method.
%
% \end{markdown}
%  \begin{macrocode}
function util.rope_last(rope)
  if #rope == 0 then
    return nil
  else
    local l = rope[#rope]
    if type(l) == "table" then
      return util.rope_last(l)
    else
      return l
    end
  end
end
%    \end{macrocode}
% \par
% \begin{markdown}
%
% Given an array `ary` and a string `x`, the \luamdef{util.intersperse}
% method returns an array `new`, such that `ary[i] == new[2*(i-1)+1]` and
% `new[2*i] == x` for all $1\leq`i`\leq`#ary`$.
%
% \end{markdown}
%  \begin{macrocode}
function util.intersperse(ary, x)
  local new = {}
  local l = #ary
  for i,v in ipairs(ary) do
    local n = #new
    new[n + 1] = v
    if i ~= l then
      new[n + 2] = x
    end
  end
  return new
end
%    \end{macrocode}
% \par
% \begin{markdown}
%
% Given an array `ary` and a function `f`, the \luamdef{util.map} method
% returns an array `new`, such that `new[i] == f(ary[i])` for all
% $1\leq`i`\leq`#ary`$.
%
% \end{markdown}
%  \begin{macrocode}
function util.map(ary, f)
  local new = {}
  for i,v in ipairs(ary) do
    new[i] = f(v)
  end
  return new
end
%    \end{macrocode}
% \par
% \begin{markdown}
%
% Given a table `char_escapes` mapping escapable characters to escaped
% strings and optionally a table `string_escapes` mapping escapable strings
% to escaped strings, the \luamdef{util.escaper} method returns an escaper
% function that escapes all occurrences of escapable strings and characters (in
% this order).
%
% The method uses \pkg{LPeg}, which is faster than the Lua `string.gsub`
% built-in method.
%
% \end{markdown}
%  \begin{macrocode}
function util.escaper(char_escapes, string_escapes)
%    \end{macrocode}
% \begin{markdown}
% Build a string of escapable characters.
% \end{markdown}
%  \begin{macrocode}
  local char_escapes_list = ""
  for i,_ in pairs(char_escapes) do
    char_escapes_list = char_escapes_list .. i
  end
%    \end{macrocode}
% \begin{markdown}
% Create an \pkg{LPeg} capture `escapable` that produces the escaped string
% corresponding to the matched escapable character.
% \end{markdown}
%  \begin{macrocode}
  local escapable = S(char_escapes_list) / char_escapes
%    \end{macrocode}
% \begin{markdown}
% If `string_escapes` is provided, turn `escapable` into the
% $$\sum^^B{(`k`, `v`)\in`string_escapes`}`P(k) / v` + `escapable`$$
% capture that replaces any occurrence of the string `k` with the string
% `v` for each $(`k`, `v`)\in`string_escapes`$. Note that the pattern
% summation is not commutative and its operands are inspected in the
% summation order during the matching. As a corrolary, the strings always
% take precedence over the characters.
%
% \end{markdown}
%  \begin{macrocode}
  if string_escapes then
    for k,v in pairs(string_escapes) do
      escapable = P(k) / v + escapable
    end
  end
%    \end{macrocode}
% \begin{markdown}
% Create an \pkg{LPeg} capture `escape_string` that captures anything
% `escapable` does and matches any other unmatched characters.
% \end{markdown}
%  \begin{macrocode}
  local escape_string = Cs((escapable + any)^0)
%    \end{macrocode}
% \begin{markdown}
% Return a function that matches the input string `s` against the
% `escape_string` capture.
% \end{markdown}
%  \begin{macrocode}
  return function(s)
    return lpeg.match(escape_string, s)
  end
end
%    \end{macrocode}
% \par
% \begin{markdown}
%
% The \luamdef{util.pathname} method produces a pathname out of a directory
% name `dir` and a filename `file` and returns it.
%
% \end{markdown}
%  \begin{macrocode}
function util.pathname(dir, file)
  if #dir == 0 then
    return file
  else
    return dir .. "/" .. file
  end
end
%    \end{macrocode}
% \par
% \begin{markdown}
%
%### HTML Entities
% This section documents the \acro{HTML} entities recognized by the
% markdown reader.  These functions are encapsulated in the `entities`
% object. The functions were originally located in the
% `lunamark/entities.lua` file in the Lunamark Lua module.
%
% \end{markdown}
%  \begin{macrocode}
local entities = {}

local character_entities = {
  ["Tab"] = 9,
  ["NewLine"] = 10,
  ["excl"] = 33,
  ["QUOT"] = 34,
  ["quot"] = 34,
  ["num"] = 35,
  ["dollar"] = 36,
  ["percnt"] = 37,
  ["AMP"] = 38,
  ["amp"] = 38,
  ["apos"] = 39,
  ["lpar"] = 40,
  ["rpar"] = 41,
  ["ast"] = 42,
  ["midast"] = 42,
  ["plus"] = 43,
  ["comma"] = 44,
  ["period"] = 46,
  ["sol"] = 47,
  ["colon"] = 58,
  ["semi"] = 59,
  ["LT"] = 60,
  ["lt"] = 60,
  ["nvlt"] = {60, 8402},
  ["bne"] = {61, 8421},
  ["equals"] = 61,
  ["GT"] = 62,
  ["gt"] = 62,
  ["nvgt"] = {62, 8402},
  ["quest"] = 63,
  ["commat"] = 64,
  ["lbrack"] = 91,
  ["lsqb"] = 91,
  ["bsol"] = 92,
  ["rbrack"] = 93,
  ["rsqb"] = 93,
  ["Hat"] = 94,
  ["UnderBar"] = 95,
  ["lowbar"] = 95,
  ["DiacriticalGrave"] = 96,
  ["grave"] = 96,
  ["fjlig"] = {102, 106},
  ["lbrace"] = 123,
  ["lcub"] = 123,
  ["VerticalLine"] = 124,
  ["verbar"] = 124,
  ["vert"] = 124,
  ["rbrace"] = 125,
  ["rcub"] = 125,
  ["NonBreakingSpace"] = 160,
  ["nbsp"] = 160,
  ["iexcl"] = 161,
  ["cent"] = 162,
  ["pound"] = 163,
  ["curren"] = 164,
  ["yen"] = 165,
  ["brvbar"] = 166,
  ["sect"] = 167,
  ["Dot"] = 168,
  ["DoubleDot"] = 168,
  ["die"] = 168,
  ["uml"] = 168,
  ["COPY"] = 169,
  ["copy"] = 169,
  ["ordf"] = 170,
  ["laquo"] = 171,
  ["not"] = 172,
  ["shy"] = 173,
  ["REG"] = 174,
  ["circledR"] = 174,
  ["reg"] = 174,
  ["macr"] = 175,
  ["strns"] = 175,
  ["deg"] = 176,
  ["PlusMinus"] = 177,
  ["plusmn"] = 177,
  ["pm"] = 177,
  ["sup2"] = 178,
  ["sup3"] = 179,
  ["DiacriticalAcute"] = 180,
  ["acute"] = 180,
  ["micro"] = 181,
  ["para"] = 182,
  ["CenterDot"] = 183,
  ["centerdot"] = 183,
  ["middot"] = 183,
  ["Cedilla"] = 184,
  ["cedil"] = 184,
  ["sup1"] = 185,
  ["ordm"] = 186,
  ["raquo"] = 187,
  ["frac14"] = 188,
  ["frac12"] = 189,
  ["half"] = 189,
  ["frac34"] = 190,
  ["iquest"] = 191,
  ["Agrave"] = 192,
  ["Aacute"] = 193,
  ["Acirc"] = 194,
  ["Atilde"] = 195,
  ["Auml"] = 196,
  ["Aring"] = 197,
  ["angst"] = 197,
  ["AElig"] = 198,
  ["Ccedil"] = 199,
  ["Egrave"] = 200,
  ["Eacute"] = 201,
  ["Ecirc"] = 202,
  ["Euml"] = 203,
  ["Igrave"] = 204,
  ["Iacute"] = 205,
  ["Icirc"] = 206,
  ["Iuml"] = 207,
  ["ETH"] = 208,
  ["Ntilde"] = 209,
  ["Ograve"] = 210,
  ["Oacute"] = 211,
  ["Ocirc"] = 212,
  ["Otilde"] = 213,
  ["Ouml"] = 214,
  ["times"] = 215,
  ["Oslash"] = 216,
  ["Ugrave"] = 217,
  ["Uacute"] = 218,
  ["Ucirc"] = 219,
  ["Uuml"] = 220,
  ["Yacute"] = 221,
  ["THORN"] = 222,
  ["szlig"] = 223,
  ["agrave"] = 224,
  ["aacute"] = 225,
  ["acirc"] = 226,
  ["atilde"] = 227,
  ["auml"] = 228,
  ["aring"] = 229,
  ["aelig"] = 230,
  ["ccedil"] = 231,
  ["egrave"] = 232,
  ["eacute"] = 233,
  ["ecirc"] = 234,
  ["euml"] = 235,
  ["igrave"] = 236,
  ["iacute"] = 237,
  ["icirc"] = 238,
  ["iuml"] = 239,
  ["eth"] = 240,
  ["ntilde"] = 241,
  ["ograve"] = 242,
  ["oacute"] = 243,
  ["ocirc"] = 244,
  ["otilde"] = 245,
  ["ouml"] = 246,
  ["div"] = 247,
  ["divide"] = 247,
  ["oslash"] = 248,
  ["ugrave"] = 249,
  ["uacute"] = 250,
  ["ucirc"] = 251,
  ["uuml"] = 252,
  ["yacute"] = 253,
  ["thorn"] = 254,
  ["yuml"] = 255,
  ["Amacr"] = 256,
  ["amacr"] = 257,
  ["Abreve"] = 258,
  ["abreve"] = 259,
  ["Aogon"] = 260,
  ["aogon"] = 261,
  ["Cacute"] = 262,
  ["cacute"] = 263,
  ["Ccirc"] = 264,
  ["ccirc"] = 265,
  ["Cdot"] = 266,
  ["cdot"] = 267,
  ["Ccaron"] = 268,
  ["ccaron"] = 269,
  ["Dcaron"] = 270,
  ["dcaron"] = 271,
  ["Dstrok"] = 272,
  ["dstrok"] = 273,
  ["Emacr"] = 274,
  ["emacr"] = 275,
  ["Edot"] = 278,
  ["edot"] = 279,
  ["Eogon"] = 280,
  ["eogon"] = 281,
  ["Ecaron"] = 282,
  ["ecaron"] = 283,
  ["Gcirc"] = 284,
  ["gcirc"] = 285,
  ["Gbreve"] = 286,
  ["gbreve"] = 287,
  ["Gdot"] = 288,
  ["gdot"] = 289,
  ["Gcedil"] = 290,
  ["Hcirc"] = 292,
  ["hcirc"] = 293,
  ["Hstrok"] = 294,
  ["hstrok"] = 295,
  ["Itilde"] = 296,
  ["itilde"] = 297,
  ["Imacr"] = 298,
  ["imacr"] = 299,
  ["Iogon"] = 302,
  ["iogon"] = 303,
  ["Idot"] = 304,
  ["imath"] = 305,
  ["inodot"] = 305,
  ["IJlig"] = 306,
  ["ijlig"] = 307,
  ["Jcirc"] = 308,
  ["jcirc"] = 309,
  ["Kcedil"] = 310,
  ["kcedil"] = 311,
  ["kgreen"] = 312,
  ["Lacute"] = 313,
  ["lacute"] = 314,
  ["Lcedil"] = 315,
  ["lcedil"] = 316,
  ["Lcaron"] = 317,
  ["lcaron"] = 318,
  ["Lmidot"] = 319,
  ["lmidot"] = 320,
  ["Lstrok"] = 321,
  ["lstrok"] = 322,
  ["Nacute"] = 323,
  ["nacute"] = 324,
  ["Ncedil"] = 325,
  ["ncedil"] = 326,
  ["Ncaron"] = 327,
  ["ncaron"] = 328,
  ["napos"] = 329,
  ["ENG"] = 330,
  ["eng"] = 331,
  ["Omacr"] = 332,
  ["omacr"] = 333,
  ["Odblac"] = 336,
  ["odblac"] = 337,
  ["OElig"] = 338,
  ["oelig"] = 339,
  ["Racute"] = 340,
  ["racute"] = 341,
  ["Rcedil"] = 342,
  ["rcedil"] = 343,
  ["Rcaron"] = 344,
  ["rcaron"] = 345,
  ["Sacute"] = 346,
  ["sacute"] = 347,
  ["Scirc"] = 348,
  ["scirc"] = 349,
  ["Scedil"] = 350,
  ["scedil"] = 351,
  ["Scaron"] = 352,
  ["scaron"] = 353,
  ["Tcedil"] = 354,
  ["tcedil"] = 355,
  ["Tcaron"] = 356,
  ["tcaron"] = 357,
  ["Tstrok"] = 358,
  ["tstrok"] = 359,
  ["Utilde"] = 360,
  ["utilde"] = 361,
  ["Umacr"] = 362,
  ["umacr"] = 363,
  ["Ubreve"] = 364,
  ["ubreve"] = 365,
  ["Uring"] = 366,
  ["uring"] = 367,
  ["Udblac"] = 368,
  ["udblac"] = 369,
  ["Uogon"] = 370,
  ["uogon"] = 371,
  ["Wcirc"] = 372,
  ["wcirc"] = 373,
  ["Ycirc"] = 374,
  ["ycirc"] = 375,
  ["Yuml"] = 376,
  ["Zacute"] = 377,
  ["zacute"] = 378,
  ["Zdot"] = 379,
  ["zdot"] = 380,
  ["Zcaron"] = 381,
  ["zcaron"] = 382,
  ["fnof"] = 402,
  ["imped"] = 437,
  ["gacute"] = 501,
  ["jmath"] = 567,
  ["circ"] = 710,
  ["Hacek"] = 711,
  ["caron"] = 711,
  ["Breve"] = 728,
  ["breve"] = 728,
  ["DiacriticalDot"] = 729,
  ["dot"] = 729,
  ["ring"] = 730,
  ["ogon"] = 731,
  ["DiacriticalTilde"] = 732,
  ["tilde"] = 732,
  ["DiacriticalDoubleAcute"] = 733,
  ["dblac"] = 733,
  ["DownBreve"] = 785,
  ["Alpha"] = 913,
  ["Beta"] = 914,
  ["Gamma"] = 915,
  ["Delta"] = 916,
  ["Epsilon"] = 917,
  ["Zeta"] = 918,
  ["Eta"] = 919,
  ["Theta"] = 920,
  ["Iota"] = 921,
  ["Kappa"] = 922,
  ["Lambda"] = 923,
  ["Mu"] = 924,
  ["Nu"] = 925,
  ["Xi"] = 926,
  ["Omicron"] = 927,
  ["Pi"] = 928,
  ["Rho"] = 929,
  ["Sigma"] = 931,
  ["Tau"] = 932,
  ["Upsilon"] = 933,
  ["Phi"] = 934,
  ["Chi"] = 935,
  ["Psi"] = 936,
  ["Omega"] = 937,
  ["ohm"] = 937,
  ["alpha"] = 945,
  ["beta"] = 946,
  ["gamma"] = 947,
  ["delta"] = 948,
  ["epsi"] = 949,
  ["epsilon"] = 949,
  ["zeta"] = 950,
  ["eta"] = 951,
  ["theta"] = 952,
  ["iota"] = 953,
  ["kappa"] = 954,
  ["lambda"] = 955,
  ["mu"] = 956,
  ["nu"] = 957,
  ["xi"] = 958,
  ["omicron"] = 959,
  ["pi"] = 960,
  ["rho"] = 961,
  ["sigmaf"] = 962,
  ["sigmav"] = 962,
  ["varsigma"] = 962,
  ["sigma"] = 963,
  ["tau"] = 964,
  ["upsi"] = 965,
  ["upsilon"] = 965,
  ["phi"] = 966,
  ["chi"] = 967,
  ["psi"] = 968,
  ["omega"] = 969,
  ["thetasym"] = 977,
  ["thetav"] = 977,
  ["vartheta"] = 977,
  ["Upsi"] = 978,
  ["upsih"] = 978,
  ["phiv"] = 981,
  ["straightphi"] = 981,
  ["varphi"] = 981,
  ["piv"] = 982,
  ["varpi"] = 982,
  ["Gammad"] = 988,
  ["digamma"] = 989,
  ["gammad"] = 989,
  ["kappav"] = 1008,
  ["varkappa"] = 1008,
  ["rhov"] = 1009,
  ["varrho"] = 1009,
  ["epsiv"] = 1013,
  ["straightepsilon"] = 1013,
  ["varepsilon"] = 1013,
  ["backepsilon"] = 1014,
  ["bepsi"] = 1014,
  ["IOcy"] = 1025,
  ["DJcy"] = 1026,
  ["GJcy"] = 1027,
  ["Jukcy"] = 1028,
  ["DScy"] = 1029,
  ["Iukcy"] = 1030,
  ["YIcy"] = 1031,
  ["Jsercy"] = 1032,
  ["LJcy"] = 1033,
  ["NJcy"] = 1034,
  ["TSHcy"] = 1035,
  ["KJcy"] = 1036,
  ["Ubrcy"] = 1038,
  ["DZcy"] = 1039,
  ["Acy"] = 1040,
  ["Bcy"] = 1041,
  ["Vcy"] = 1042,
  ["Gcy"] = 1043,
  ["Dcy"] = 1044,
  ["IEcy"] = 1045,
  ["ZHcy"] = 1046,
  ["Zcy"] = 1047,
  ["Icy"] = 1048,
  ["Jcy"] = 1049,
  ["Kcy"] = 1050,
  ["Lcy"] = 1051,
  ["Mcy"] = 1052,
  ["Ncy"] = 1053,
  ["Ocy"] = 1054,
  ["Pcy"] = 1055,
  ["Rcy"] = 1056,
  ["Scy"] = 1057,
  ["Tcy"] = 1058,
  ["Ucy"] = 1059,
  ["Fcy"] = 1060,
  ["KHcy"] = 1061,
  ["TScy"] = 1062,
  ["CHcy"] = 1063,
  ["SHcy"] = 1064,
  ["SHCHcy"] = 1065,
  ["HARDcy"] = 1066,
  ["Ycy"] = 1067,
  ["SOFTcy"] = 1068,
  ["Ecy"] = 1069,
  ["YUcy"] = 1070,
  ["YAcy"] = 1071,
  ["acy"] = 1072,
  ["bcy"] = 1073,
  ["vcy"] = 1074,
  ["gcy"] = 1075,
  ["dcy"] = 1076,
  ["iecy"] = 1077,
  ["zhcy"] = 1078,
  ["zcy"] = 1079,
  ["icy"] = 1080,
  ["jcy"] = 1081,
  ["kcy"] = 1082,
  ["lcy"] = 1083,
  ["mcy"] = 1084,
  ["ncy"] = 1085,
  ["ocy"] = 1086,
  ["pcy"] = 1087,
  ["rcy"] = 1088,
  ["scy"] = 1089,
  ["tcy"] = 1090,
  ["ucy"] = 1091,
  ["fcy"] = 1092,
  ["khcy"] = 1093,
  ["tscy"] = 1094,
  ["chcy"] = 1095,
  ["shcy"] = 1096,
  ["shchcy"] = 1097,
  ["hardcy"] = 1098,
  ["ycy"] = 1099,
  ["softcy"] = 1100,
  ["ecy"] = 1101,
  ["yucy"] = 1102,
  ["yacy"] = 1103,
  ["iocy"] = 1105,
  ["djcy"] = 1106,
  ["gjcy"] = 1107,
  ["jukcy"] = 1108,
  ["dscy"] = 1109,
  ["iukcy"] = 1110,
  ["yicy"] = 1111,
  ["jsercy"] = 1112,
  ["ljcy"] = 1113,
  ["njcy"] = 1114,
  ["tshcy"] = 1115,
  ["kjcy"] = 1116,
  ["ubrcy"] = 1118,
  ["dzcy"] = 1119,
  ["ensp"] = 8194,
  ["emsp"] = 8195,
  ["emsp13"] = 8196,
  ["emsp14"] = 8197,
  ["numsp"] = 8199,
  ["puncsp"] = 8200,
  ["ThinSpace"] = 8201,
  ["thinsp"] = 8201,
  ["VeryThinSpace"] = 8202,
  ["hairsp"] = 8202,
  ["NegativeMediumSpace"] = 8203,
  ["NegativeThickSpace"] = 8203,
  ["NegativeThinSpace"] = 8203,
  ["NegativeVeryThinSpace"] = 8203,
  ["ZeroWidthSpace"] = 8203,
  ["zwnj"] = 8204,
  ["zwj"] = 8205,
  ["lrm"] = 8206,
  ["rlm"] = 8207,
  ["dash"] = 8208,
  ["hyphen"] = 8208,
  ["ndash"] = 8211,
  ["mdash"] = 8212,
  ["horbar"] = 8213,
  ["Verbar"] = 8214,
  ["Vert"] = 8214,
  ["OpenCurlyQuote"] = 8216,
  ["lsquo"] = 8216,
  ["CloseCurlyQuote"] = 8217,
  ["rsquo"] = 8217,
  ["rsquor"] = 8217,
  ["lsquor"] = 8218,
  ["sbquo"] = 8218,
  ["OpenCurlyDoubleQuote"] = 8220,
  ["ldquo"] = 8220,
  ["CloseCurlyDoubleQuote"] = 8221,
  ["rdquo"] = 8221,
  ["rdquor"] = 8221,
  ["bdquo"] = 8222,
  ["ldquor"] = 8222,
  ["dagger"] = 8224,
  ["Dagger"] = 8225,
  ["ddagger"] = 8225,
  ["bull"] = 8226,
  ["bullet"] = 8226,
  ["nldr"] = 8229,
  ["hellip"] = 8230,
  ["mldr"] = 8230,
  ["permil"] = 8240,
  ["pertenk"] = 8241,
  ["prime"] = 8242,
  ["Prime"] = 8243,
  ["tprime"] = 8244,
  ["backprime"] = 8245,
  ["bprime"] = 8245,
  ["lsaquo"] = 8249,
  ["rsaquo"] = 8250,
  ["OverBar"] = 8254,
  ["oline"] = 8254,
  ["caret"] = 8257,
  ["hybull"] = 8259,
  ["frasl"] = 8260,
  ["bsemi"] = 8271,
  ["qprime"] = 8279,
  ["MediumSpace"] = 8287,
  ["ThickSpace"] = {8287, 8202},
  ["NoBreak"] = 8288,
  ["ApplyFunction"] = 8289,
  ["af"] = 8289,
  ["InvisibleTimes"] = 8290,
  ["it"] = 8290,
  ["InvisibleComma"] = 8291,
  ["ic"] = 8291,
  ["euro"] = 8364,
  ["TripleDot"] = 8411,
  ["tdot"] = 8411,
  ["DotDot"] = 8412,
  ["Copf"] = 8450,
  ["complexes"] = 8450,
  ["incare"] = 8453,
  ["gscr"] = 8458,
  ["HilbertSpace"] = 8459,
  ["Hscr"] = 8459,
  ["hamilt"] = 8459,
  ["Hfr"] = 8460,
  ["Poincareplane"] = 8460,
  ["Hopf"] = 8461,
  ["quaternions"] = 8461,
  ["planckh"] = 8462,
  ["hbar"] = 8463,
  ["hslash"] = 8463,
  ["planck"] = 8463,
  ["plankv"] = 8463,
  ["Iscr"] = 8464,
  ["imagline"] = 8464,
  ["Ifr"] = 8465,
  ["Im"] = 8465,
  ["image"] = 8465,
  ["imagpart"] = 8465,
  ["Laplacetrf"] = 8466,
  ["Lscr"] = 8466,
  ["lagran"] = 8466,
  ["ell"] = 8467,
  ["Nopf"] = 8469,
  ["naturals"] = 8469,
  ["numero"] = 8470,
  ["copysr"] = 8471,
  ["weierp"] = 8472,
  ["wp"] = 8472,
  ["Popf"] = 8473,
  ["primes"] = 8473,
  ["Qopf"] = 8474,
  ["rationals"] = 8474,
  ["Rscr"] = 8475,
  ["realine"] = 8475,
  ["Re"] = 8476,
  ["Rfr"] = 8476,
  ["real"] = 8476,
  ["realpart"] = 8476,
  ["Ropf"] = 8477,
  ["reals"] = 8477,
  ["rx"] = 8478,
  ["TRADE"] = 8482,
  ["trade"] = 8482,
  ["Zopf"] = 8484,
  ["integers"] = 8484,
  ["mho"] = 8487,
  ["Zfr"] = 8488,
  ["zeetrf"] = 8488,
  ["iiota"] = 8489,
  ["Bernoullis"] = 8492,
  ["Bscr"] = 8492,
  ["bernou"] = 8492,
  ["Cayleys"] = 8493,
  ["Cfr"] = 8493,
  ["escr"] = 8495,
  ["Escr"] = 8496,
  ["expectation"] = 8496,
  ["Fouriertrf"] = 8497,
  ["Fscr"] = 8497,
  ["Mellintrf"] = 8499,
  ["Mscr"] = 8499,
  ["phmmat"] = 8499,
  ["order"] = 8500,
  ["orderof"] = 8500,
  ["oscr"] = 8500,
  ["alefsym"] = 8501,
  ["aleph"] = 8501,
  ["beth"] = 8502,
  ["gimel"] = 8503,
  ["daleth"] = 8504,
  ["CapitalDifferentialD"] = 8517,
  ["DD"] = 8517,
  ["DifferentialD"] = 8518,
  ["dd"] = 8518,
  ["ExponentialE"] = 8519,
  ["ee"] = 8519,
  ["exponentiale"] = 8519,
  ["ImaginaryI"] = 8520,
  ["ii"] = 8520,
  ["frac13"] = 8531,
  ["frac23"] = 8532,
  ["frac15"] = 8533,
  ["frac25"] = 8534,
  ["frac35"] = 8535,
  ["frac45"] = 8536,
  ["frac16"] = 8537,
  ["frac56"] = 8538,
  ["frac18"] = 8539,
  ["frac38"] = 8540,
  ["frac58"] = 8541,
  ["frac78"] = 8542,
  ["LeftArrow"] = 8592,
  ["ShortLeftArrow"] = 8592,
  ["larr"] = 8592,
  ["leftarrow"] = 8592,
  ["slarr"] = 8592,
  ["ShortUpArrow"] = 8593,
  ["UpArrow"] = 8593,
  ["uarr"] = 8593,
  ["uparrow"] = 8593,
  ["RightArrow"] = 8594,
  ["ShortRightArrow"] = 8594,
  ["rarr"] = 8594,
  ["rightarrow"] = 8594,
  ["srarr"] = 8594,
  ["DownArrow"] = 8595,
  ["ShortDownArrow"] = 8595,
  ["darr"] = 8595,
  ["downarrow"] = 8595,
  ["LeftRightArrow"] = 8596,
  ["harr"] = 8596,
  ["leftrightarrow"] = 8596,
  ["UpDownArrow"] = 8597,
  ["updownarrow"] = 8597,
  ["varr"] = 8597,
  ["UpperLeftArrow"] = 8598,
  ["nwarr"] = 8598,
  ["nwarrow"] = 8598,
  ["UpperRightArrow"] = 8599,
  ["nearr"] = 8599,
  ["nearrow"] = 8599,
  ["LowerRightArrow"] = 8600,
  ["searr"] = 8600,
  ["searrow"] = 8600,
  ["LowerLeftArrow"] = 8601,
  ["swarr"] = 8601,
  ["swarrow"] = 8601,
  ["nlarr"] = 8602,
  ["nleftarrow"] = 8602,
  ["nrarr"] = 8603,
  ["nrightarrow"] = 8603,
  ["nrarrw"] = {8605, 824},
  ["rarrw"] = 8605,
  ["rightsquigarrow"] = 8605,
  ["Larr"] = 8606,
  ["twoheadleftarrow"] = 8606,
  ["Uarr"] = 8607,
  ["Rarr"] = 8608,
  ["twoheadrightarrow"] = 8608,
  ["Darr"] = 8609,
  ["larrtl"] = 8610,
  ["leftarrowtail"] = 8610,
  ["rarrtl"] = 8611,
  ["rightarrowtail"] = 8611,
  ["LeftTeeArrow"] = 8612,
  ["mapstoleft"] = 8612,
  ["UpTeeArrow"] = 8613,
  ["mapstoup"] = 8613,
  ["RightTeeArrow"] = 8614,
  ["map"] = 8614,
  ["mapsto"] = 8614,
  ["DownTeeArrow"] = 8615,
  ["mapstodown"] = 8615,
  ["hookleftarrow"] = 8617,
  ["larrhk"] = 8617,
  ["hookrightarrow"] = 8618,
  ["rarrhk"] = 8618,
  ["larrlp"] = 8619,
  ["looparrowleft"] = 8619,
  ["looparrowright"] = 8620,
  ["rarrlp"] = 8620,
  ["harrw"] = 8621,
  ["leftrightsquigarrow"] = 8621,
  ["nharr"] = 8622,
  ["nleftrightarrow"] = 8622,
  ["Lsh"] = 8624,
  ["lsh"] = 8624,
  ["Rsh"] = 8625,
  ["rsh"] = 8625,
  ["ldsh"] = 8626,
  ["rdsh"] = 8627,
  ["crarr"] = 8629,
  ["cularr"] = 8630,
  ["curvearrowleft"] = 8630,
  ["curarr"] = 8631,
  ["curvearrowright"] = 8631,
  ["circlearrowleft"] = 8634,
  ["olarr"] = 8634,
  ["circlearrowright"] = 8635,
  ["orarr"] = 8635,
  ["LeftVector"] = 8636,
  ["leftharpoonup"] = 8636,
  ["lharu"] = 8636,
  ["DownLeftVector"] = 8637,
  ["leftharpoondown"] = 8637,
  ["lhard"] = 8637,
  ["RightUpVector"] = 8638,
  ["uharr"] = 8638,
  ["upharpoonright"] = 8638,
  ["LeftUpVector"] = 8639,
  ["uharl"] = 8639,
  ["upharpoonleft"] = 8639,
  ["RightVector"] = 8640,
  ["rharu"] = 8640,
  ["rightharpoonup"] = 8640,
  ["DownRightVector"] = 8641,
  ["rhard"] = 8641,
  ["rightharpoondown"] = 8641,
  ["RightDownVector"] = 8642,
  ["dharr"] = 8642,
  ["downharpoonright"] = 8642,
  ["LeftDownVector"] = 8643,
  ["dharl"] = 8643,
  ["downharpoonleft"] = 8643,
  ["RightArrowLeftArrow"] = 8644,
  ["rightleftarrows"] = 8644,
  ["rlarr"] = 8644,
  ["UpArrowDownArrow"] = 8645,
  ["udarr"] = 8645,
  ["LeftArrowRightArrow"] = 8646,
  ["leftrightarrows"] = 8646,
  ["lrarr"] = 8646,
  ["leftleftarrows"] = 8647,
  ["llarr"] = 8647,
  ["upuparrows"] = 8648,
  ["uuarr"] = 8648,
  ["rightrightarrows"] = 8649,
  ["rrarr"] = 8649,
  ["ddarr"] = 8650,
  ["downdownarrows"] = 8650,
  ["ReverseEquilibrium"] = 8651,
  ["leftrightharpoons"] = 8651,
  ["lrhar"] = 8651,
  ["Equilibrium"] = 8652,
  ["rightleftharpoons"] = 8652,
  ["rlhar"] = 8652,
  ["nLeftarrow"] = 8653,
  ["nlArr"] = 8653,
  ["nLeftrightarrow"] = 8654,
  ["nhArr"] = 8654,
  ["nRightarrow"] = 8655,
  ["nrArr"] = 8655,
  ["DoubleLeftArrow"] = 8656,
  ["Leftarrow"] = 8656,
  ["lArr"] = 8656,
  ["DoubleUpArrow"] = 8657,
  ["Uparrow"] = 8657,
  ["uArr"] = 8657,
  ["DoubleRightArrow"] = 8658,
  ["Implies"] = 8658,
  ["Rightarrow"] = 8658,
  ["rArr"] = 8658,
  ["DoubleDownArrow"] = 8659,
  ["Downarrow"] = 8659,
  ["dArr"] = 8659,
  ["DoubleLeftRightArrow"] = 8660,
  ["Leftrightarrow"] = 8660,
  ["hArr"] = 8660,
  ["iff"] = 8660,
  ["DoubleUpDownArrow"] = 8661,
  ["Updownarrow"] = 8661,
  ["vArr"] = 8661,
  ["nwArr"] = 8662,
  ["neArr"] = 8663,
  ["seArr"] = 8664,
  ["swArr"] = 8665,
  ["Lleftarrow"] = 8666,
  ["lAarr"] = 8666,
  ["Rrightarrow"] = 8667,
  ["rAarr"] = 8667,
  ["zigrarr"] = 8669,
  ["LeftArrowBar"] = 8676,
  ["larrb"] = 8676,
  ["RightArrowBar"] = 8677,
  ["rarrb"] = 8677,
  ["DownArrowUpArrow"] = 8693,
  ["duarr"] = 8693,
  ["loarr"] = 8701,
  ["roarr"] = 8702,
  ["hoarr"] = 8703,
  ["ForAll"] = 8704,
  ["forall"] = 8704,
  ["comp"] = 8705,
  ["complement"] = 8705,
  ["PartialD"] = 8706,
  ["npart"] = {8706, 824},
  ["part"] = 8706,
  ["Exists"] = 8707,
  ["exist"] = 8707,
  ["NotExists"] = 8708,
  ["nexist"] = 8708,
  ["nexists"] = 8708,
  ["empty"] = 8709,
  ["emptyset"] = 8709,
  ["emptyv"] = 8709,
  ["varnothing"] = 8709,
  ["Del"] = 8711,
  ["nabla"] = 8711,
  ["Element"] = 8712,
  ["in"] = 8712,
  ["isin"] = 8712,
  ["isinv"] = 8712,
  ["NotElement"] = 8713,
  ["notin"] = 8713,
  ["notinva"] = 8713,
  ["ReverseElement"] = 8715,
  ["SuchThat"] = 8715,
  ["ni"] = 8715,
  ["niv"] = 8715,
  ["NotReverseElement"] = 8716,
  ["notni"] = 8716,
  ["notniva"] = 8716,
  ["Product"] = 8719,
  ["prod"] = 8719,
  ["Coproduct"] = 8720,
  ["coprod"] = 8720,
  ["Sum"] = 8721,
  ["sum"] = 8721,
  ["minus"] = 8722,
  ["MinusPlus"] = 8723,
  ["mnplus"] = 8723,
  ["mp"] = 8723,
  ["dotplus"] = 8724,
  ["plusdo"] = 8724,
  ["Backslash"] = 8726,
  ["setminus"] = 8726,
  ["setmn"] = 8726,
  ["smallsetminus"] = 8726,
  ["ssetmn"] = 8726,
  ["lowast"] = 8727,
  ["SmallCircle"] = 8728,
  ["compfn"] = 8728,
  ["Sqrt"] = 8730,
  ["radic"] = 8730,
  ["Proportional"] = 8733,
  ["prop"] = 8733,
  ["propto"] = 8733,
  ["varpropto"] = 8733,
  ["vprop"] = 8733,
  ["infin"] = 8734,
  ["angrt"] = 8735,
  ["ang"] = 8736,
  ["angle"] = 8736,
  ["nang"] = {8736, 8402},
  ["angmsd"] = 8737,
  ["measuredangle"] = 8737,
  ["angsph"] = 8738,
  ["VerticalBar"] = 8739,
  ["mid"] = 8739,
  ["shortmid"] = 8739,
  ["smid"] = 8739,
  ["NotVerticalBar"] = 8740,
  ["nmid"] = 8740,
  ["nshortmid"] = 8740,
  ["nsmid"] = 8740,
  ["DoubleVerticalBar"] = 8741,
  ["par"] = 8741,
  ["parallel"] = 8741,
  ["shortparallel"] = 8741,
  ["spar"] = 8741,
  ["NotDoubleVerticalBar"] = 8742,
  ["npar"] = 8742,
  ["nparallel"] = 8742,
  ["nshortparallel"] = 8742,
  ["nspar"] = 8742,
  ["and"] = 8743,
  ["wedge"] = 8743,
  ["or"] = 8744,
  ["vee"] = 8744,
  ["cap"] = 8745,
  ["caps"] = {8745, 65024},
  ["cup"] = 8746,
  ["cups"] = {8746, 65024},
  ["Integral"] = 8747,
  ["int"] = 8747,
  ["Int"] = 8748,
  ["iiint"] = 8749,
  ["tint"] = 8749,
  ["ContourIntegral"] = 8750,
  ["conint"] = 8750,
  ["oint"] = 8750,
  ["Conint"] = 8751,
  ["DoubleContourIntegral"] = 8751,
  ["Cconint"] = 8752,
  ["cwint"] = 8753,
  ["ClockwiseContourIntegral"] = 8754,
  ["cwconint"] = 8754,
  ["CounterClockwiseContourIntegral"] = 8755,
  ["awconint"] = 8755,
  ["Therefore"] = 8756,
  ["there4"] = 8756,
  ["therefore"] = 8756,
  ["Because"] = 8757,
  ["becaus"] = 8757,
  ["because"] = 8757,
  ["ratio"] = 8758,
  ["Colon"] = 8759,
  ["Proportion"] = 8759,
  ["dotminus"] = 8760,
  ["minusd"] = 8760,
  ["mDDot"] = 8762,
  ["homtht"] = 8763,
  ["Tilde"] = 8764,
  ["nvsim"] = {8764, 8402},
  ["sim"] = 8764,
  ["thicksim"] = 8764,
  ["thksim"] = 8764,
  ["backsim"] = 8765,
  ["bsim"] = 8765,
  ["race"] = {8765, 817},
  ["ac"] = 8766,
  ["acE"] = {8766, 819},
  ["mstpos"] = 8766,
  ["acd"] = 8767,
  ["VerticalTilde"] = 8768,
  ["wr"] = 8768,
  ["wreath"] = 8768,
  ["NotTilde"] = 8769,
  ["nsim"] = 8769,
  ["EqualTilde"] = 8770,
  ["NotEqualTilde"] = {8770, 824},
  ["eqsim"] = 8770,
  ["esim"] = 8770,
  ["nesim"] = {8770, 824},
  ["TildeEqual"] = 8771,
  ["sime"] = 8771,
  ["simeq"] = 8771,
  ["NotTildeEqual"] = 8772,
  ["nsime"] = 8772,
  ["nsimeq"] = 8772,
  ["TildeFullEqual"] = 8773,
  ["cong"] = 8773,
  ["simne"] = 8774,
  ["NotTildeFullEqual"] = 8775,
  ["ncong"] = 8775,
  ["TildeTilde"] = 8776,
  ["ap"] = 8776,
  ["approx"] = 8776,
  ["asymp"] = 8776,
  ["thickapprox"] = 8776,
  ["thkap"] = 8776,
  ["NotTildeTilde"] = 8777,
  ["nap"] = 8777,
  ["napprox"] = 8777,
  ["ape"] = 8778,
  ["approxeq"] = 8778,
  ["apid"] = 8779,
  ["napid"] = {8779, 824},
  ["backcong"] = 8780,
  ["bcong"] = 8780,
  ["CupCap"] = 8781,
  ["asympeq"] = 8781,
  ["nvap"] = {8781, 8402},
  ["Bumpeq"] = 8782,
  ["HumpDownHump"] = 8782,
  ["NotHumpDownHump"] = {8782, 824},
  ["bump"] = 8782,
  ["nbump"] = {8782, 824},
  ["HumpEqual"] = 8783,
  ["NotHumpEqual"] = {8783, 824},
  ["bumpe"] = 8783,
  ["bumpeq"] = 8783,
  ["nbumpe"] = {8783, 824},
  ["DotEqual"] = 8784,
  ["doteq"] = 8784,
  ["esdot"] = 8784,
  ["nedot"] = {8784, 824},
  ["doteqdot"] = 8785,
  ["eDot"] = 8785,
  ["efDot"] = 8786,
  ["fallingdotseq"] = 8786,
  ["erDot"] = 8787,
  ["risingdotseq"] = 8787,
  ["Assign"] = 8788,
  ["colone"] = 8788,
  ["coloneq"] = 8788,
  ["ecolon"] = 8789,
  ["eqcolon"] = 8789,
  ["ecir"] = 8790,
  ["eqcirc"] = 8790,
  ["circeq"] = 8791,
  ["cire"] = 8791,
  ["wedgeq"] = 8793,
  ["veeeq"] = 8794,
  ["triangleq"] = 8796,
  ["trie"] = 8796,
  ["equest"] = 8799,
  ["questeq"] = 8799,
  ["NotEqual"] = 8800,
  ["ne"] = 8800,
  ["Congruent"] = 8801,
  ["bnequiv"] = {8801, 8421},
  ["equiv"] = 8801,
  ["NotCongruent"] = 8802,
  ["nequiv"] = 8802,
  ["le"] = 8804,
  ["leq"] = 8804,
  ["nvle"] = {8804, 8402},
  ["GreaterEqual"] = 8805,
  ["ge"] = 8805,
  ["geq"] = 8805,
  ["nvge"] = {8805, 8402},
  ["LessFullEqual"] = 8806,
  ["lE"] = 8806,
  ["leqq"] = 8806,
  ["nlE"] = {8806, 824},
  ["nleqq"] = {8806, 824},
  ["GreaterFullEqual"] = 8807,
  ["NotGreaterFullEqual"] = {8807, 824},
  ["gE"] = 8807,
  ["geqq"] = 8807,
  ["ngE"] = {8807, 824},
  ["ngeqq"] = {8807, 824},
  ["lnE"] = 8808,
  ["lneqq"] = 8808,
  ["lvertneqq"] = {8808, 65024},
  ["lvnE"] = {8808, 65024},
  ["gnE"] = 8809,
  ["gneqq"] = 8809,
  ["gvertneqq"] = {8809, 65024},
  ["gvnE"] = {8809, 65024},
  ["Lt"] = 8810,
  ["NestedLessLess"] = 8810,
  ["NotLessLess"] = {8810, 824},
  ["ll"] = 8810,
  ["nLt"] = {8810, 8402},
  ["nLtv"] = {8810, 824},
  ["Gt"] = 8811,
  ["NestedGreaterGreater"] = 8811,
  ["NotGreaterGreater"] = {8811, 824},
  ["gg"] = 8811,
  ["nGt"] = {8811, 8402},
  ["nGtv"] = {8811, 824},
  ["between"] = 8812,
  ["twixt"] = 8812,
  ["NotCupCap"] = 8813,
  ["NotLess"] = 8814,
  ["nless"] = 8814,
  ["nlt"] = 8814,
  ["NotGreater"] = 8815,
  ["ngt"] = 8815,
  ["ngtr"] = 8815,
  ["NotLessEqual"] = 8816,
  ["nle"] = 8816,
  ["nleq"] = 8816,
  ["NotGreaterEqual"] = 8817,
  ["nge"] = 8817,
  ["ngeq"] = 8817,
  ["LessTilde"] = 8818,
  ["lesssim"] = 8818,
  ["lsim"] = 8818,
  ["GreaterTilde"] = 8819,
  ["gsim"] = 8819,
  ["gtrsim"] = 8819,
  ["NotLessTilde"] = 8820,
  ["nlsim"] = 8820,
  ["NotGreaterTilde"] = 8821,
  ["ngsim"] = 8821,
  ["LessGreater"] = 8822,
  ["lessgtr"] = 8822,
  ["lg"] = 8822,
  ["GreaterLess"] = 8823,
  ["gl"] = 8823,
  ["gtrless"] = 8823,
  ["NotLessGreater"] = 8824,
  ["ntlg"] = 8824,
  ["NotGreaterLess"] = 8825,
  ["ntgl"] = 8825,
  ["Precedes"] = 8826,
  ["pr"] = 8826,
  ["prec"] = 8826,
  ["Succeeds"] = 8827,
  ["sc"] = 8827,
  ["succ"] = 8827,
  ["PrecedesSlantEqual"] = 8828,
  ["prcue"] = 8828,
  ["preccurlyeq"] = 8828,
  ["SucceedsSlantEqual"] = 8829,
  ["sccue"] = 8829,
  ["succcurlyeq"] = 8829,
  ["PrecedesTilde"] = 8830,
  ["precsim"] = 8830,
  ["prsim"] = 8830,
  ["NotSucceedsTilde"] = {8831, 824},
  ["SucceedsTilde"] = 8831,
  ["scsim"] = 8831,
  ["succsim"] = 8831,
  ["NotPrecedes"] = 8832,
  ["npr"] = 8832,
  ["nprec"] = 8832,
  ["NotSucceeds"] = 8833,
  ["nsc"] = 8833,
  ["nsucc"] = 8833,
  ["NotSubset"] = {8834, 8402},
  ["nsubset"] = {8834, 8402},
  ["sub"] = 8834,
  ["subset"] = 8834,
  ["vnsub"] = {8834, 8402},
  ["NotSuperset"] = {8835, 8402},
  ["Superset"] = 8835,
  ["nsupset"] = {8835, 8402},
  ["sup"] = 8835,
  ["supset"] = 8835,
  ["vnsup"] = {8835, 8402},
  ["nsub"] = 8836,
  ["nsup"] = 8837,
  ["SubsetEqual"] = 8838,
  ["sube"] = 8838,
  ["subseteq"] = 8838,
  ["SupersetEqual"] = 8839,
  ["supe"] = 8839,
  ["supseteq"] = 8839,
  ["NotSubsetEqual"] = 8840,
  ["nsube"] = 8840,
  ["nsubseteq"] = 8840,
  ["NotSupersetEqual"] = 8841,
  ["nsupe"] = 8841,
  ["nsupseteq"] = 8841,
  ["subne"] = 8842,
  ["subsetneq"] = 8842,
  ["varsubsetneq"] = {8842, 65024},
  ["vsubne"] = {8842, 65024},
  ["supne"] = 8843,
  ["supsetneq"] = 8843,
  ["varsupsetneq"] = {8843, 65024},
  ["vsupne"] = {8843, 65024},
  ["cupdot"] = 8845,
  ["UnionPlus"] = 8846,
  ["uplus"] = 8846,
  ["NotSquareSubset"] = {8847, 824},
  ["SquareSubset"] = 8847,
  ["sqsub"] = 8847,
  ["sqsubset"] = 8847,
  ["NotSquareSuperset"] = {8848, 824},
  ["SquareSuperset"] = 8848,
  ["sqsup"] = 8848,
  ["sqsupset"] = 8848,
  ["SquareSubsetEqual"] = 8849,
  ["sqsube"] = 8849,
  ["sqsubseteq"] = 8849,
  ["SquareSupersetEqual"] = 8850,
  ["sqsupe"] = 8850,
  ["sqsupseteq"] = 8850,
  ["SquareIntersection"] = 8851,
  ["sqcap"] = 8851,
  ["sqcaps"] = {8851, 65024},
  ["SquareUnion"] = 8852,
  ["sqcup"] = 8852,
  ["sqcups"] = {8852, 65024},
  ["CirclePlus"] = 8853,
  ["oplus"] = 8853,
  ["CircleMinus"] = 8854,
  ["ominus"] = 8854,
  ["CircleTimes"] = 8855,
  ["otimes"] = 8855,
  ["osol"] = 8856,
  ["CircleDot"] = 8857,
  ["odot"] = 8857,
  ["circledcirc"] = 8858,
  ["ocir"] = 8858,
  ["circledast"] = 8859,
  ["oast"] = 8859,
  ["circleddash"] = 8861,
  ["odash"] = 8861,
  ["boxplus"] = 8862,
  ["plusb"] = 8862,
  ["boxminus"] = 8863,
  ["minusb"] = 8863,
  ["boxtimes"] = 8864,
  ["timesb"] = 8864,
  ["dotsquare"] = 8865,
  ["sdotb"] = 8865,
  ["RightTee"] = 8866,
  ["vdash"] = 8866,
  ["LeftTee"] = 8867,
  ["dashv"] = 8867,
  ["DownTee"] = 8868,
  ["top"] = 8868,
  ["UpTee"] = 8869,
  ["bot"] = 8869,
  ["bottom"] = 8869,
  ["perp"] = 8869,
  ["models"] = 8871,
  ["DoubleRightTee"] = 8872,
  ["vDash"] = 8872,
  ["Vdash"] = 8873,
  ["Vvdash"] = 8874,
  ["VDash"] = 8875,
  ["nvdash"] = 8876,
  ["nvDash"] = 8877,
  ["nVdash"] = 8878,
  ["nVDash"] = 8879,
  ["prurel"] = 8880,
  ["LeftTriangle"] = 8882,
  ["vartriangleleft"] = 8882,
  ["vltri"] = 8882,
  ["RightTriangle"] = 8883,
  ["vartriangleright"] = 8883,
  ["vrtri"] = 8883,
  ["LeftTriangleEqual"] = 8884,
  ["ltrie"] = 8884,
  ["nvltrie"] = {8884, 8402},
  ["trianglelefteq"] = 8884,
  ["RightTriangleEqual"] = 8885,
  ["nvrtrie"] = {8885, 8402},
  ["rtrie"] = 8885,
  ["trianglerighteq"] = 8885,
  ["origof"] = 8886,
  ["imof"] = 8887,
  ["multimap"] = 8888,
  ["mumap"] = 8888,
  ["hercon"] = 8889,
  ["intcal"] = 8890,
  ["intercal"] = 8890,
  ["veebar"] = 8891,
  ["barvee"] = 8893,
  ["angrtvb"] = 8894,
  ["lrtri"] = 8895,
  ["Wedge"] = 8896,
  ["bigwedge"] = 8896,
  ["xwedge"] = 8896,
  ["Vee"] = 8897,
  ["bigvee"] = 8897,
  ["xvee"] = 8897,
  ["Intersection"] = 8898,
  ["bigcap"] = 8898,
  ["xcap"] = 8898,
  ["Union"] = 8899,
  ["bigcup"] = 8899,
  ["xcup"] = 8899,
  ["Diamond"] = 8900,
  ["diam"] = 8900,
  ["diamond"] = 8900,
  ["sdot"] = 8901,
  ["Star"] = 8902,
  ["sstarf"] = 8902,
  ["divideontimes"] = 8903,
  ["divonx"] = 8903,
  ["bowtie"] = 8904,
  ["ltimes"] = 8905,
  ["rtimes"] = 8906,
  ["leftthreetimes"] = 8907,
  ["lthree"] = 8907,
  ["rightthreetimes"] = 8908,
  ["rthree"] = 8908,
  ["backsimeq"] = 8909,
  ["bsime"] = 8909,
  ["curlyvee"] = 8910,
  ["cuvee"] = 8910,
  ["curlywedge"] = 8911,
  ["cuwed"] = 8911,
  ["Sub"] = 8912,
  ["Subset"] = 8912,
  ["Sup"] = 8913,
  ["Supset"] = 8913,
  ["Cap"] = 8914,
  ["Cup"] = 8915,
  ["fork"] = 8916,
  ["pitchfork"] = 8916,
  ["epar"] = 8917,
  ["lessdot"] = 8918,
  ["ltdot"] = 8918,
  ["gtdot"] = 8919,
  ["gtrdot"] = 8919,
  ["Ll"] = 8920,
  ["nLl"] = {8920, 824},
  ["Gg"] = 8921,
  ["ggg"] = 8921,
  ["nGg"] = {8921, 824},
  ["LessEqualGreater"] = 8922,
  ["leg"] = 8922,
  ["lesg"] = {8922, 65024},
  ["lesseqgtr"] = 8922,
  ["GreaterEqualLess"] = 8923,
  ["gel"] = 8923,
  ["gesl"] = {8923, 65024},
  ["gtreqless"] = 8923,
  ["cuepr"] = 8926,
  ["curlyeqprec"] = 8926,
  ["cuesc"] = 8927,
  ["curlyeqsucc"] = 8927,
  ["NotPrecedesSlantEqual"] = 8928,
  ["nprcue"] = 8928,
  ["NotSucceedsSlantEqual"] = 8929,
  ["nsccue"] = 8929,
  ["NotSquareSubsetEqual"] = 8930,
  ["nsqsube"] = 8930,
  ["NotSquareSupersetEqual"] = 8931,
  ["nsqsupe"] = 8931,
  ["lnsim"] = 8934,
  ["gnsim"] = 8935,
  ["precnsim"] = 8936,
  ["prnsim"] = 8936,
  ["scnsim"] = 8937,
  ["succnsim"] = 8937,
  ["NotLeftTriangle"] = 8938,
  ["nltri"] = 8938,
  ["ntriangleleft"] = 8938,
  ["NotRightTriangle"] = 8939,
  ["nrtri"] = 8939,
  ["ntriangleright"] = 8939,
  ["NotLeftTriangleEqual"] = 8940,
  ["nltrie"] = 8940,
  ["ntrianglelefteq"] = 8940,
  ["NotRightTriangleEqual"] = 8941,
  ["nrtrie"] = 8941,
  ["ntrianglerighteq"] = 8941,
  ["vellip"] = 8942,
  ["ctdot"] = 8943,
  ["utdot"] = 8944,
  ["dtdot"] = 8945,
  ["disin"] = 8946,
  ["isinsv"] = 8947,
  ["isins"] = 8948,
  ["isindot"] = 8949,
  ["notindot"] = {8949, 824},
  ["notinvc"] = 8950,
  ["notinvb"] = 8951,
  ["isinE"] = 8953,
  ["notinE"] = {8953, 824},
  ["nisd"] = 8954,
  ["xnis"] = 8955,
  ["nis"] = 8956,
  ["notnivc"] = 8957,
  ["notnivb"] = 8958,
  ["barwed"] = 8965,
  ["barwedge"] = 8965,
  ["Barwed"] = 8966,
  ["doublebarwedge"] = 8966,
  ["LeftCeiling"] = 8968,
  ["lceil"] = 8968,
  ["RightCeiling"] = 8969,
  ["rceil"] = 8969,
  ["LeftFloor"] = 8970,
  ["lfloor"] = 8970,
  ["RightFloor"] = 8971,
  ["rfloor"] = 8971,
  ["drcrop"] = 8972,
  ["dlcrop"] = 8973,
  ["urcrop"] = 8974,
  ["ulcrop"] = 8975,
  ["bnot"] = 8976,
  ["profline"] = 8978,
  ["profsurf"] = 8979,
  ["telrec"] = 8981,
  ["target"] = 8982,
  ["ulcorn"] = 8988,
  ["ulcorner"] = 8988,
  ["urcorn"] = 8989,
  ["urcorner"] = 8989,
  ["dlcorn"] = 8990,
  ["llcorner"] = 8990,
  ["drcorn"] = 8991,
  ["lrcorner"] = 8991,
  ["frown"] = 8994,
  ["sfrown"] = 8994,
  ["smile"] = 8995,
  ["ssmile"] = 8995,
  ["cylcty"] = 9005,
  ["profalar"] = 9006,
  ["topbot"] = 9014,
  ["ovbar"] = 9021,
  ["solbar"] = 9023,
  ["angzarr"] = 9084,
  ["lmoust"] = 9136,
  ["lmoustache"] = 9136,
  ["rmoust"] = 9137,
  ["rmoustache"] = 9137,
  ["OverBracket"] = 9140,
  ["tbrk"] = 9140,
  ["UnderBracket"] = 9141,
  ["bbrk"] = 9141,
  ["bbrktbrk"] = 9142,
  ["OverParenthesis"] = 9180,
  ["UnderParenthesis"] = 9181,
  ["OverBrace"] = 9182,
  ["UnderBrace"] = 9183,
  ["trpezium"] = 9186,
  ["elinters"] = 9191,
  ["blank"] = 9251,
  ["circledS"] = 9416,
  ["oS"] = 9416,
  ["HorizontalLine"] = 9472,
  ["boxh"] = 9472,
  ["boxv"] = 9474,
  ["boxdr"] = 9484,
  ["boxdl"] = 9488,
  ["boxur"] = 9492,
  ["boxul"] = 9496,
  ["boxvr"] = 9500,
  ["boxvl"] = 9508,
  ["boxhd"] = 9516,
  ["boxhu"] = 9524,
  ["boxvh"] = 9532,
  ["boxH"] = 9552,
  ["boxV"] = 9553,
  ["boxdR"] = 9554,
  ["boxDr"] = 9555,
  ["boxDR"] = 9556,
  ["boxdL"] = 9557,
  ["boxDl"] = 9558,
  ["boxDL"] = 9559,
  ["boxuR"] = 9560,
  ["boxUr"] = 9561,
  ["boxUR"] = 9562,
  ["boxuL"] = 9563,
  ["boxUl"] = 9564,
  ["boxUL"] = 9565,
  ["boxvR"] = 9566,
  ["boxVr"] = 9567,
  ["boxVR"] = 9568,
  ["boxvL"] = 9569,
  ["boxVl"] = 9570,
  ["boxVL"] = 9571,
  ["boxHd"] = 9572,
  ["boxhD"] = 9573,
  ["boxHD"] = 9574,
  ["boxHu"] = 9575,
  ["boxhU"] = 9576,
  ["boxHU"] = 9577,
  ["boxvH"] = 9578,
  ["boxVh"] = 9579,
  ["boxVH"] = 9580,
  ["uhblk"] = 9600,
  ["lhblk"] = 9604,
  ["block"] = 9608,
  ["blk14"] = 9617,
  ["blk12"] = 9618,
  ["blk34"] = 9619,
  ["Square"] = 9633,
  ["squ"] = 9633,
  ["square"] = 9633,
  ["FilledVerySmallSquare"] = 9642,
  ["blacksquare"] = 9642,
  ["squarf"] = 9642,
  ["squf"] = 9642,
  ["EmptyVerySmallSquare"] = 9643,
  ["rect"] = 9645,
  ["marker"] = 9646,
  ["fltns"] = 9649,
  ["bigtriangleup"] = 9651,
  ["xutri"] = 9651,
  ["blacktriangle"] = 9652,
  ["utrif"] = 9652,
  ["triangle"] = 9653,
  ["utri"] = 9653,
  ["blacktriangleright"] = 9656,
  ["rtrif"] = 9656,
  ["rtri"] = 9657,
  ["triangleright"] = 9657,
  ["bigtriangledown"] = 9661,
  ["xdtri"] = 9661,
  ["blacktriangledown"] = 9662,
  ["dtrif"] = 9662,
  ["dtri"] = 9663,
  ["triangledown"] = 9663,
  ["blacktriangleleft"] = 9666,
  ["ltrif"] = 9666,
  ["ltri"] = 9667,
  ["triangleleft"] = 9667,
  ["loz"] = 9674,
  ["lozenge"] = 9674,
  ["cir"] = 9675,
  ["tridot"] = 9708,
  ["bigcirc"] = 9711,
  ["xcirc"] = 9711,
  ["ultri"] = 9720,
  ["urtri"] = 9721,
  ["lltri"] = 9722,
  ["EmptySmallSquare"] = 9723,
  ["FilledSmallSquare"] = 9724,
  ["bigstar"] = 9733,
  ["starf"] = 9733,
  ["star"] = 9734,
  ["phone"] = 9742,
  ["female"] = 9792,
  ["male"] = 9794,
  ["spades"] = 9824,
  ["spadesuit"] = 9824,
  ["clubs"] = 9827,
  ["clubsuit"] = 9827,
  ["hearts"] = 9829,
  ["heartsuit"] = 9829,
  ["diamondsuit"] = 9830,
  ["diams"] = 9830,
  ["sung"] = 9834,
  ["flat"] = 9837,
  ["natur"] = 9838,
  ["natural"] = 9838,
  ["sharp"] = 9839,
  ["check"] = 10003,
  ["checkmark"] = 10003,
  ["cross"] = 10007,
  ["malt"] = 10016,
  ["maltese"] = 10016,
  ["sext"] = 10038,
  ["VerticalSeparator"] = 10072,
  ["lbbrk"] = 10098,
  ["rbbrk"] = 10099,
  ["bsolhsub"] = 10184,
  ["suphsol"] = 10185,
  ["LeftDoubleBracket"] = 10214,
  ["lobrk"] = 10214,
  ["RightDoubleBracket"] = 10215,
  ["robrk"] = 10215,
  ["LeftAngleBracket"] = 10216,
  ["lang"] = 10216,
  ["langle"] = 10216,
  ["RightAngleBracket"] = 10217,
  ["rang"] = 10217,
  ["rangle"] = 10217,
  ["Lang"] = 10218,
  ["Rang"] = 10219,
  ["loang"] = 10220,
  ["roang"] = 10221,
  ["LongLeftArrow"] = 10229,
  ["longleftarrow"] = 10229,
  ["xlarr"] = 10229,
  ["LongRightArrow"] = 10230,
  ["longrightarrow"] = 10230,
  ["xrarr"] = 10230,
  ["LongLeftRightArrow"] = 10231,
  ["longleftrightarrow"] = 10231,
  ["xharr"] = 10231,
  ["DoubleLongLeftArrow"] = 10232,
  ["Longleftarrow"] = 10232,
  ["xlArr"] = 10232,
  ["DoubleLongRightArrow"] = 10233,
  ["Longrightarrow"] = 10233,
  ["xrArr"] = 10233,
  ["DoubleLongLeftRightArrow"] = 10234,
  ["Longleftrightarrow"] = 10234,
  ["xhArr"] = 10234,
  ["longmapsto"] = 10236,
  ["xmap"] = 10236,
  ["dzigrarr"] = 10239,
  ["nvlArr"] = 10498,
  ["nvrArr"] = 10499,
  ["nvHarr"] = 10500,
  ["Map"] = 10501,
  ["lbarr"] = 10508,
  ["bkarow"] = 10509,
  ["rbarr"] = 10509,
  ["lBarr"] = 10510,
  ["dbkarow"] = 10511,
  ["rBarr"] = 10511,
  ["RBarr"] = 10512,
  ["drbkarow"] = 10512,
  ["DDotrahd"] = 10513,
  ["UpArrowBar"] = 10514,
  ["DownArrowBar"] = 10515,
  ["Rarrtl"] = 10518,
  ["latail"] = 10521,
  ["ratail"] = 10522,
  ["lAtail"] = 10523,
  ["rAtail"] = 10524,
  ["larrfs"] = 10525,
  ["rarrfs"] = 10526,
  ["larrbfs"] = 10527,
  ["rarrbfs"] = 10528,
  ["nwarhk"] = 10531,
  ["nearhk"] = 10532,
  ["hksearow"] = 10533,
  ["searhk"] = 10533,
  ["hkswarow"] = 10534,
  ["swarhk"] = 10534,
  ["nwnear"] = 10535,
  ["nesear"] = 10536,
  ["toea"] = 10536,
  ["seswar"] = 10537,
  ["tosa"] = 10537,
  ["swnwar"] = 10538,
  ["nrarrc"] = {10547, 824},
  ["rarrc"] = 10547,
  ["cudarrr"] = 10549,
  ["ldca"] = 10550,
  ["rdca"] = 10551,
  ["cudarrl"] = 10552,
  ["larrpl"] = 10553,
  ["curarrm"] = 10556,
  ["cularrp"] = 10557,
  ["rarrpl"] = 10565,
  ["harrcir"] = 10568,
  ["Uarrocir"] = 10569,
  ["lurdshar"] = 10570,
  ["ldrushar"] = 10571,
  ["LeftRightVector"] = 10574,
  ["RightUpDownVector"] = 10575,
  ["DownLeftRightVector"] = 10576,
  ["LeftUpDownVector"] = 10577,
  ["LeftVectorBar"] = 10578,
  ["RightVectorBar"] = 10579,
  ["RightUpVectorBar"] = 10580,
  ["RightDownVectorBar"] = 10581,
  ["DownLeftVectorBar"] = 10582,
  ["DownRightVectorBar"] = 10583,
  ["LeftUpVectorBar"] = 10584,
  ["LeftDownVectorBar"] = 10585,
  ["LeftTeeVector"] = 10586,
  ["RightTeeVector"] = 10587,
  ["RightUpTeeVector"] = 10588,
  ["RightDownTeeVector"] = 10589,
  ["DownLeftTeeVector"] = 10590,
  ["DownRightTeeVector"] = 10591,
  ["LeftUpTeeVector"] = 10592,
  ["LeftDownTeeVector"] = 10593,
  ["lHar"] = 10594,
  ["uHar"] = 10595,
  ["rHar"] = 10596,
  ["dHar"] = 10597,
  ["luruhar"] = 10598,
  ["ldrdhar"] = 10599,
  ["ruluhar"] = 10600,
  ["rdldhar"] = 10601,
  ["lharul"] = 10602,
  ["llhard"] = 10603,
  ["rharul"] = 10604,
  ["lrhard"] = 10605,
  ["UpEquilibrium"] = 10606,
  ["udhar"] = 10606,
  ["ReverseUpEquilibrium"] = 10607,
  ["duhar"] = 10607,
  ["RoundImplies"] = 10608,
  ["erarr"] = 10609,
  ["simrarr"] = 10610,
  ["larrsim"] = 10611,
  ["rarrsim"] = 10612,
  ["rarrap"] = 10613,
  ["ltlarr"] = 10614,
  ["gtrarr"] = 10616,
  ["subrarr"] = 10617,
  ["suplarr"] = 10619,
  ["lfisht"] = 10620,
  ["rfisht"] = 10621,
  ["ufisht"] = 10622,
  ["dfisht"] = 10623,
  ["lopar"] = 10629,
  ["ropar"] = 10630,
  ["lbrke"] = 10635,
  ["rbrke"] = 10636,
  ["lbrkslu"] = 10637,
  ["rbrksld"] = 10638,
  ["lbrksld"] = 10639,
  ["rbrkslu"] = 10640,
  ["langd"] = 10641,
  ["rangd"] = 10642,
  ["lparlt"] = 10643,
  ["rpargt"] = 10644,
  ["gtlPar"] = 10645,
  ["ltrPar"] = 10646,
  ["vzigzag"] = 10650,
  ["vangrt"] = 10652,
  ["angrtvbd"] = 10653,
  ["ange"] = 10660,
  ["range"] = 10661,
  ["dwangle"] = 10662,
  ["uwangle"] = 10663,
  ["angmsdaa"] = 10664,
  ["angmsdab"] = 10665,
  ["angmsdac"] = 10666,
  ["angmsdad"] = 10667,
  ["angmsdae"] = 10668,
  ["angmsdaf"] = 10669,
  ["angmsdag"] = 10670,
  ["angmsdah"] = 10671,
  ["bemptyv"] = 10672,
  ["demptyv"] = 10673,
  ["cemptyv"] = 10674,
  ["raemptyv"] = 10675,
  ["laemptyv"] = 10676,
  ["ohbar"] = 10677,
  ["omid"] = 10678,
  ["opar"] = 10679,
  ["operp"] = 10681,
  ["olcross"] = 10683,
  ["odsold"] = 10684,
  ["olcir"] = 10686,
  ["ofcir"] = 10687,
  ["olt"] = 10688,
  ["ogt"] = 10689,
  ["cirscir"] = 10690,
  ["cirE"] = 10691,
  ["solb"] = 10692,
  ["bsolb"] = 10693,
  ["boxbox"] = 10697,
  ["trisb"] = 10701,
  ["rtriltri"] = 10702,
  ["LeftTriangleBar"] = 10703,
  ["NotLeftTriangleBar"] = {10703, 824},
  ["NotRightTriangleBar"] = {10704, 824},
  ["RightTriangleBar"] = 10704,
  ["iinfin"] = 10716,
  ["infintie"] = 10717,
  ["nvinfin"] = 10718,
  ["eparsl"] = 10723,
  ["smeparsl"] = 10724,
  ["eqvparsl"] = 10725,
  ["blacklozenge"] = 10731,
  ["lozf"] = 10731,
  ["RuleDelayed"] = 10740,
  ["dsol"] = 10742,
  ["bigodot"] = 10752,
  ["xodot"] = 10752,
  ["bigoplus"] = 10753,
  ["xoplus"] = 10753,
  ["bigotimes"] = 10754,
  ["xotime"] = 10754,
  ["biguplus"] = 10756,
  ["xuplus"] = 10756,
  ["bigsqcup"] = 10758,
  ["xsqcup"] = 10758,
  ["iiiint"] = 10764,
  ["qint"] = 10764,
  ["fpartint"] = 10765,
  ["cirfnint"] = 10768,
  ["awint"] = 10769,
  ["rppolint"] = 10770,
  ["scpolint"] = 10771,
  ["npolint"] = 10772,
  ["pointint"] = 10773,
  ["quatint"] = 10774,
  ["intlarhk"] = 10775,
  ["pluscir"] = 10786,
  ["plusacir"] = 10787,
  ["simplus"] = 10788,
  ["plusdu"] = 10789,
  ["plussim"] = 10790,
  ["plustwo"] = 10791,
  ["mcomma"] = 10793,
  ["minusdu"] = 10794,
  ["loplus"] = 10797,
  ["roplus"] = 10798,
  ["Cross"] = 10799,
  ["timesd"] = 10800,
  ["timesbar"] = 10801,
  ["smashp"] = 10803,
  ["lotimes"] = 10804,
  ["rotimes"] = 10805,
  ["otimesas"] = 10806,
  ["Otimes"] = 10807,
  ["odiv"] = 10808,
  ["triplus"] = 10809,
  ["triminus"] = 10810,
  ["tritime"] = 10811,
  ["intprod"] = 10812,
  ["iprod"] = 10812,
  ["amalg"] = 10815,
  ["capdot"] = 10816,
  ["ncup"] = 10818,
  ["ncap"] = 10819,
  ["capand"] = 10820,
  ["cupor"] = 10821,
  ["cupcap"] = 10822,
  ["capcup"] = 10823,
  ["cupbrcap"] = 10824,
  ["capbrcup"] = 10825,
  ["cupcup"] = 10826,
  ["capcap"] = 10827,
  ["ccups"] = 10828,
  ["ccaps"] = 10829,
  ["ccupssm"] = 10832,
  ["And"] = 10835,
  ["Or"] = 10836,
  ["andand"] = 10837,
  ["oror"] = 10838,
  ["orslope"] = 10839,
  ["andslope"] = 10840,
  ["andv"] = 10842,
  ["orv"] = 10843,
  ["andd"] = 10844,
  ["ord"] = 10845,
  ["wedbar"] = 10847,
  ["sdote"] = 10854,
  ["simdot"] = 10858,
  ["congdot"] = 10861,
  ["ncongdot"] = {10861, 824},
  ["easter"] = 10862,
  ["apacir"] = 10863,
  ["apE"] = 10864,
  ["napE"] = {10864, 824},
  ["eplus"] = 10865,
  ["pluse"] = 10866,
  ["Esim"] = 10867,
  ["Colone"] = 10868,
  ["Equal"] = 10869,
  ["ddotseq"] = 10871,
  ["eDDot"] = 10871,
  ["equivDD"] = 10872,
  ["ltcir"] = 10873,
  ["gtcir"] = 10874,
  ["ltquest"] = 10875,
  ["gtquest"] = 10876,
  ["LessSlantEqual"] = 10877,
  ["NotLessSlantEqual"] = {10877, 824},
  ["leqslant"] = 10877,
  ["les"] = 10877,
  ["nleqslant"] = {10877, 824},
  ["nles"] = {10877, 824},
  ["GreaterSlantEqual"] = 10878,
  ["NotGreaterSlantEqual"] = {10878, 824},
  ["geqslant"] = 10878,
  ["ges"] = 10878,
  ["ngeqslant"] = {10878, 824},
  ["nges"] = {10878, 824},
  ["lesdot"] = 10879,
  ["gesdot"] = 10880,
  ["lesdoto"] = 10881,
  ["gesdoto"] = 10882,
  ["lesdotor"] = 10883,
  ["gesdotol"] = 10884,
  ["lap"] = 10885,
  ["lessapprox"] = 10885,
  ["gap"] = 10886,
  ["gtrapprox"] = 10886,
  ["lne"] = 10887,
  ["lneq"] = 10887,
  ["gne"] = 10888,
  ["gneq"] = 10888,
  ["lnap"] = 10889,
  ["lnapprox"] = 10889,
  ["gnap"] = 10890,
  ["gnapprox"] = 10890,
  ["lEg"] = 10891,
  ["lesseqqgtr"] = 10891,
  ["gEl"] = 10892,
  ["gtreqqless"] = 10892,
  ["lsime"] = 10893,
  ["gsime"] = 10894,
  ["lsimg"] = 10895,
  ["gsiml"] = 10896,
  ["lgE"] = 10897,
  ["glE"] = 10898,
  ["lesges"] = 10899,
  ["gesles"] = 10900,
  ["els"] = 10901,
  ["eqslantless"] = 10901,
  ["egs"] = 10902,
  ["eqslantgtr"] = 10902,
  ["elsdot"] = 10903,
  ["egsdot"] = 10904,
  ["el"] = 10905,
  ["eg"] = 10906,
  ["siml"] = 10909,
  ["simg"] = 10910,
  ["simlE"] = 10911,
  ["simgE"] = 10912,
  ["LessLess"] = 10913,
  ["NotNestedLessLess"] = {10913, 824},
  ["GreaterGreater"] = 10914,
  ["NotNestedGreaterGreater"] = {10914, 824},
  ["glj"] = 10916,
  ["gla"] = 10917,
  ["ltcc"] = 10918,
  ["gtcc"] = 10919,
  ["lescc"] = 10920,
  ["gescc"] = 10921,
  ["smt"] = 10922,
  ["lat"] = 10923,
  ["smte"] = 10924,
  ["smtes"] = {10924, 65024},
  ["late"] = 10925,
  ["lates"] = {10925, 65024},
  ["bumpE"] = 10926,
  ["NotPrecedesEqual"] = {10927, 824},
  ["PrecedesEqual"] = 10927,
  ["npre"] = {10927, 824},
  ["npreceq"] = {10927, 824},
  ["pre"] = 10927,
  ["preceq"] = 10927,
  ["NotSucceedsEqual"] = {10928, 824},
  ["SucceedsEqual"] = 10928,
  ["nsce"] = {10928, 824},
  ["nsucceq"] = {10928, 824},
  ["sce"] = 10928,
  ["succeq"] = 10928,
  ["prE"] = 10931,
  ["scE"] = 10932,
  ["precneqq"] = 10933,
  ["prnE"] = 10933,
  ["scnE"] = 10934,
  ["succneqq"] = 10934,
  ["prap"] = 10935,
  ["precapprox"] = 10935,
  ["scap"] = 10936,
  ["succapprox"] = 10936,
  ["precnapprox"] = 10937,
  ["prnap"] = 10937,
  ["scnap"] = 10938,
  ["succnapprox"] = 10938,
  ["Pr"] = 10939,
  ["Sc"] = 10940,
  ["subdot"] = 10941,
  ["supdot"] = 10942,
  ["subplus"] = 10943,
  ["supplus"] = 10944,
  ["submult"] = 10945,
  ["supmult"] = 10946,
  ["subedot"] = 10947,
  ["supedot"] = 10948,
  ["nsubE"] = {10949, 824},
  ["nsubseteqq"] = {10949, 824},
  ["subE"] = 10949,
  ["subseteqq"] = 10949,
  ["nsupE"] = {10950, 824},
  ["nsupseteqq"] = {10950, 824},
  ["supE"] = 10950,
  ["supseteqq"] = 10950,
  ["subsim"] = 10951,
  ["supsim"] = 10952,
  ["subnE"] = 10955,
  ["subsetneqq"] = 10955,
  ["varsubsetneqq"] = {10955, 65024},
  ["vsubnE"] = {10955, 65024},
  ["supnE"] = 10956,
  ["supsetneqq"] = 10956,
  ["varsupsetneqq"] = {10956, 65024},
  ["vsupnE"] = {10956, 65024},
  ["csub"] = 10959,
  ["csup"] = 10960,
  ["csube"] = 10961,
  ["csupe"] = 10962,
  ["subsup"] = 10963,
  ["supsub"] = 10964,
  ["subsub"] = 10965,
  ["supsup"] = 10966,
  ["suphsub"] = 10967,
  ["supdsub"] = 10968,
  ["forkv"] = 10969,
  ["topfork"] = 10970,
  ["mlcp"] = 10971,
  ["Dashv"] = 10980,
  ["DoubleLeftTee"] = 10980,
  ["Vdashl"] = 10982,
  ["Barv"] = 10983,
  ["vBar"] = 10984,
  ["vBarv"] = 10985,
  ["Vbar"] = 10987,
  ["Not"] = 10988,
  ["bNot"] = 10989,
  ["rnmid"] = 10990,
  ["cirmid"] = 10991,
  ["midcir"] = 10992,
  ["topcir"] = 10993,
  ["nhpar"] = 10994,
  ["parsim"] = 10995,
  ["nparsl"] = {11005, 8421},
  ["parsl"] = 11005,
  ["fflig"] = 64256,
  ["filig"] = 64257,
  ["fllig"] = 64258,
  ["ffilig"] = 64259,
  ["ffllig"] = 64260,
  ["Ascr"] = 119964,
  ["Cscr"] = 119966,
  ["Dscr"] = 119967,
  ["Gscr"] = 119970,
  ["Jscr"] = 119973,
  ["Kscr"] = 119974,
  ["Nscr"] = 119977,
  ["Oscr"] = 119978,
  ["Pscr"] = 119979,
  ["Qscr"] = 119980,
  ["Sscr"] = 119982,
  ["Tscr"] = 119983,
  ["Uscr"] = 119984,
  ["Vscr"] = 119985,
  ["Wscr"] = 119986,
  ["Xscr"] = 119987,
  ["Yscr"] = 119988,
  ["Zscr"] = 119989,
  ["ascr"] = 119990,
  ["bscr"] = 119991,
  ["cscr"] = 119992,
  ["dscr"] = 119993,
  ["fscr"] = 119995,
  ["hscr"] = 119997,
  ["iscr"] = 119998,
  ["jscr"] = 119999,
  ["kscr"] = 120000,
  ["lscr"] = 120001,
  ["mscr"] = 120002,
  ["nscr"] = 120003,
  ["pscr"] = 120005,
  ["qscr"] = 120006,
  ["rscr"] = 120007,
  ["sscr"] = 120008,
  ["tscr"] = 120009,
  ["uscr"] = 120010,
  ["vscr"] = 120011,
  ["wscr"] = 120012,
  ["xscr"] = 120013,
  ["yscr"] = 120014,
  ["zscr"] = 120015,
  ["Afr"] = 120068,
  ["Bfr"] = 120069,
  ["Dfr"] = 120071,
  ["Efr"] = 120072,
  ["Ffr"] = 120073,
  ["Gfr"] = 120074,
  ["Jfr"] = 120077,
  ["Kfr"] = 120078,
  ["Lfr"] = 120079,
  ["Mfr"] = 120080,
  ["Nfr"] = 120081,
  ["Ofr"] = 120082,
  ["Pfr"] = 120083,
  ["Qfr"] = 120084,
  ["Sfr"] = 120086,
  ["Tfr"] = 120087,
  ["Ufr"] = 120088,
  ["Vfr"] = 120089,
  ["Wfr"] = 120090,
  ["Xfr"] = 120091,
  ["Yfr"] = 120092,
  ["afr"] = 120094,
  ["bfr"] = 120095,
  ["cfr"] = 120096,
  ["dfr"] = 120097,
  ["efr"] = 120098,
  ["ffr"] = 120099,
  ["gfr"] = 120100,
  ["hfr"] = 120101,
  ["ifr"] = 120102,
  ["jfr"] = 120103,
  ["kfr"] = 120104,
  ["lfr"] = 120105,
  ["mfr"] = 120106,
  ["nfr"] = 120107,
  ["ofr"] = 120108,
  ["pfr"] = 120109,
  ["qfr"] = 120110,
  ["rfr"] = 120111,
  ["sfr"] = 120112,
  ["tfr"] = 120113,
  ["ufr"] = 120114,
  ["vfr"] = 120115,
  ["wfr"] = 120116,
  ["xfr"] = 120117,
  ["yfr"] = 120118,
  ["zfr"] = 120119,
  ["Aopf"] = 120120,
  ["Bopf"] = 120121,
  ["Dopf"] = 120123,
  ["Eopf"] = 120124,
  ["Fopf"] = 120125,
  ["Gopf"] = 120126,
  ["Iopf"] = 120128,
  ["Jopf"] = 120129,
  ["Kopf"] = 120130,
  ["Lopf"] = 120131,
  ["Mopf"] = 120132,
  ["Oopf"] = 120134,
  ["Sopf"] = 120138,
  ["Topf"] = 120139,
  ["Uopf"] = 120140,
  ["Vopf"] = 120141,
  ["Wopf"] = 120142,
  ["Xopf"] = 120143,
  ["Yopf"] = 120144,
  ["aopf"] = 120146,
  ["bopf"] = 120147,
  ["copf"] = 120148,
  ["dopf"] = 120149,
  ["eopf"] = 120150,
  ["fopf"] = 120151,
  ["gopf"] = 120152,
  ["hopf"] = 120153,
  ["iopf"] = 120154,
  ["jopf"] = 120155,
  ["kopf"] = 120156,
  ["lopf"] = 120157,
  ["mopf"] = 120158,
  ["nopf"] = 120159,
  ["oopf"] = 120160,
  ["popf"] = 120161,
  ["qopf"] = 120162,
  ["ropf"] = 120163,
  ["sopf"] = 120164,
  ["topf"] = 120165,
  ["uopf"] = 120166,
  ["vopf"] = 120167,
  ["wopf"] = 120168,
  ["xopf"] = 120169,
  ["yopf"] = 120170,
  ["zopf"] = 120171,
}
%    \end{macrocode}
% \par
% \begin{markdown}
%
% Given a string `s` of decimal digits, the \luamdef{entities.dec_entity}
% returns the corresponding \acro{utf}8-encoded Unicode codepoint.
%
% \end{markdown}
%  \begin{macrocode}
function entities.dec_entity(s)
  local n = tonumber(s)
  if n == nil then
    return "&#" .. s .. ";"  -- fallback for unknown entities
  end
  return unicode.utf8.char(n)
end
%    \end{macrocode}
% \par
% \begin{markdown}
%
% Given a string `s` of hexadecimal digits, the
% \luamdef{entities.hex_entity} returns the corresponding
% \acro{utf}8-encoded Unicode codepoint.
%
% \end{markdown}
%  \begin{macrocode}
function entities.hex_entity(s)
  local n = tonumber("0x"..s)
  if n == nil then
    return "&#x" .. s .. ";"  -- fallback for unknown entities
  end
  return unicode.utf8.char(n)
end
%    \end{macrocode}
% \par
% \begin{markdown}
%
% Given a captured character `x` and a string `s` of hexadecimal digits, the
% \luamdef{entities.hex_entity_with_x_char} returns the corresponding
% \acro{utf}8-encoded Unicode codepoint or fallback with the `x` character.
%
% \end{markdown}
%  \begin{macrocode}
function entities.hex_entity_with_x_char(x, s)
  local n = tonumber("0x"..s)
  if n == nil then
    return "&#" .. x .. s .. ";"  -- fallback for unknown entities
  end
  return unicode.utf8.char(n)
end
%    \end{macrocode}
% \par
% \begin{markdown}
%
% Given a character entity name `s` (like `ouml`), the
% \luamdef{entities.char_entity} returns the corresponding
% \acro{utf}8-encoded Unicode codepoint.
%
% \end{markdown}
%  \begin{macrocode}
function entities.char_entity(s)
  local code_points = character_entities[s]
  if code_points == nil then
    return "&" .. s .. ";"
  end
  if type(code_points) ~= 'table' then
    code_points = {code_points}
  end
  local char_table = {}
    for _, code_point in ipairs(code_points) do
      table.insert(char_table, unicode.utf8.char(code_point))
    end
  return table.concat(char_table)
end
%    \end{macrocode}
% \par
% \begin{markdown}
%
%### Plain \TeX{} Writer {#tex-writer}
%
% This section documents the \luamref{writer} object, which implements the
% routines for producing the \TeX{} output. The object is an amalgamate of the
% generic, \TeX{}, \LaTeX{} writer objects that were located in the
% `lunamark/writer/generic.lua`, `lunamark/writer/tex.lua`, and
% `lunamark/writer/latex.lua` files in the Lunamark Lua module.
%
% Although not specified in the Lua interface (see Section
% <#sec:luainterface>), the \luamref{writer} object is exported, so that the
% curious user could easily tinker with the methods of the objects produced by
% the \luamref{writer.new} method described below. The user should be aware,
% however, that the implementation may change in a future revision.
%
% \end{markdown}
%  \begin{macrocode}
M.writer = {}
%    \end{macrocode}
% \par
% \begin{markdown}
%
% The \luamdef{writer.new} method creates and returns a new \TeX{} writer
% object associated with the Lua interface options (see Section
% <#sec:lua-options>) `options`. When `options` are unspecified, it is
% assumed that an empty table was passed to the method.
%
% The objects produced by the \luamref{writer.new} method expose instance methods
% and variables of their own. As a convention, I will refer to these
% \meta{member}s as `writer->`\meta{member}. All member variables are
% immutable unless explicitly stated otherwise.
%
% \end{markdown}
%  \begin{macrocode}
function M.writer.new(options)
  local self = {}
%    \end{macrocode}
% \par
% \begin{markdown}
%
% Make `options` available as \luamdef{writer->options}, so that it is
% accessible from extensions.
%
% \end{markdown}
%  \begin{macrocode}
  self.options = options
%    \end{macrocode}
% \par
% \begin{markdown}
%
% Define \luamdef{writer->flatten\_inlines}, which indicates whether or not the
% writer should produce raw text rather than text in the output format for
% inline elements. The \luamref{writer->flatten\_inlines} member variable is
% mutable.
%
% \end{markdown}
%  \begin{macrocode}
  self.flatten_inlines = false
%    \end{macrocode}
% \par
% \begin{markdown}
%
% Parse the \Opt{slice} option and define \luamdef{writer->slice\_begin},
% \luamdef{writer->slice\_end}, and \luamdef{writer->is\_writing}. The
% \luamref{writer->is\_writing} member variable is mutable.
%
% \end{markdown}
%  \begin{macrocode}
  local slice_specifiers = {}
  for specifier in options.slice:gmatch("[^%s]+") do
    table.insert(slice_specifiers, specifier)
  end

  if #slice_specifiers == 2 then
    self.slice_begin, self.slice_end = table.unpack(slice_specifiers)
    local slice_begin_type = self.slice_begin:sub(1, 1)
    if slice_begin_type ~= "^" and slice_begin_type ~= "$" then
      self.slice_begin = "^" .. self.slice_begin
    end
    local slice_end_type = self.slice_end:sub(1, 1)
    if slice_end_type ~= "^" and slice_end_type ~= "$" then
      self.slice_end = "$" .. self.slice_end
    end
  elseif #slice_specifiers == 1 then
    self.slice_begin = "^" .. slice_specifiers[1]
    self.slice_end = "$" .. slice_specifiers[1]
  end

  self.slice_begin_type = self.slice_begin:sub(1, 1)
  self.slice_begin_identifier = self.slice_begin:sub(2) or ""
  self.slice_end_type = self.slice_end:sub(1, 1)
  self.slice_end_identifier = self.slice_end:sub(2) or ""

  if self.slice_begin == "^" and self.slice_end ~= "^" then
    self.is_writing = true
  else
    self.is_writing = false
  end
%    \end{macrocode}
% \par
% \begin{markdown}
%
% Define \luamdef{writer->suffix} as the suffix of the produced cache files.
%
% \end{markdown}
%  \begin{macrocode}
  self.suffix = ".tex"
%    \end{macrocode}
% \par
% \begin{markdown}
%
% Define \luamdef{writer->space} as the output format of a space character.
%
% \end{markdown}
%  \begin{macrocode}
  self.space = " "
%    \end{macrocode}
% \par
% \begin{markdown}
%
% Define \luamdef{writer->nbsp} as the output format of a non-breaking space
% character.
%
% \end{markdown}
%  \begin{macrocode}
  self.nbsp = "\\markdownRendererNbsp{}"
%    \end{macrocode}
% \par
% \begin{markdown}
%
% Define \luamdef{writer->plain} as a function that will transform an input
% plain text block `s` to the output format.
%
% \end{markdown}
%  \begin{macrocode}
  function self.plain(s)
    return s
  end
%    \end{macrocode}
% \par
% \begin{markdown}
%
% Define \luamdef{writer->paragraph} as a function that will transform an
% input paragraph `s` to the output format.
%
% \end{markdown}
%  \begin{macrocode}
  function self.paragraph(s)
    if not self.is_writing then return "" end
    return s
  end
%    \end{macrocode}
% \par
% \begin{markdown}
%
% Define \luamdef{writer->pack} as a function that will take the filename
% `name` of the output file prepared by the reader and transform it to the
% output format.
%
% \end{markdown}
%  \begin{macrocode}
  function self.pack(name)
    return [[\input{]] .. name .. [[}\relax]]
  end
%    \end{macrocode}
% \par
% \begin{markdown}
%
% Define \luamdef{writer->interblocksep} as the output format of a block
% element separator.
%
% \end{markdown}
%  \begin{macrocode}
  self.interblocksep_text = "\\markdownRendererInterblockSeparator\n{}"
  function self.interblocksep()
    if not self.is_writing then return "" end
    return self.interblocksep_text
  end
%    \end{macrocode}
% \par
% \begin{markdown}
%
% Define \luamdef{writer->paragraphsep} as the output format of a paragraph
% separator. Users can use more than one blank line to delimit two blocks to
% indicate the end of a series of blocks that make up a paragraph. This
% produces a paragraph separator instead of an interblock separator.
%
% \end{markdown}
%  \begin{macrocode}
  self.paragraphsep_text = "\\markdownRendererParagraphSeparator\n{}"
  function self.paragraphsep()
    if not self.is_writing then return "" end
    return self.paragraphsep_text
  end
%    \end{macrocode}
% \par
% \begin{markdown}
%
% Define \luamdef{writer->undosep} as a function that will remove the output
% produced by an immediately preceding block element / paragraph separator.
%
% \end{markdown}
%  \begin{macrocode}
  self.undosep_text = "\\markdownRendererUndoSeparator\n{}"
  function self.undosep()
    if not self.is_writing then return "" end
    return self.undosep_text
  end
%    \end{macrocode}
% \par
% \begin{markdown}
%
% Define \luamdef{writer->soft_line_break} as the output format of a soft
% line break.
%
% \end{markdown}
%  \begin{macrocode}
  self.soft_line_break = function()
    if self.flatten_inlines then return "\n" end
    return "\\markdownRendererSoftLineBreak\n{}"
  end
%    \end{macrocode}
% \par
% \begin{markdown}
%
% Define \luamdef{writer->hard_line_break} as the output format of a hard
% line break.
%
% \end{markdown}
%  \begin{macrocode}
  self.hard_line_break = function()
    if self.flatten_inlines then return "\n" end
    return "\\markdownRendererHardLineBreak\n{}"
  end
%    \end{macrocode}
% \par
% \begin{markdown}
%
% Define \luamdef{writer->ellipsis} as the output format of an ellipsis.
%
% \end{markdown}
%  \begin{macrocode}
  self.ellipsis = "\\markdownRendererEllipsis{}"
%    \end{macrocode}
% \par
% \begin{markdown}
%
% Define \luamdef{writer->thematic_break} as the output format of a thematic
% break.
%
% \end{markdown}
%  \begin{macrocode}
  function self.thematic_break()
    if not self.is_writing then return "" end
    return "\\markdownRendererThematicBreak{}"
  end
%    \end{macrocode}
% \par
% \begin{markdown}
%
% Define tables \luamdef{writer->escaped_uri_chars} and
% \luamdef{writer->escaped_minimal_strings} containing the mapping from
% special plain characters and character strings that always need to be
% escaped.
%
% \end{markdown}
%  \begin{macrocode}
  self.escaped_uri_chars = {
    ["{"] = "\\markdownRendererLeftBrace{}",
    ["}"] = "\\markdownRendererRightBrace{}",
    ["\\"] = "\\markdownRendererBackslash{}",
    ["\r"] = " ",
    ["\n"] = " ",
  }
  self.escaped_minimal_strings = {
    ["^^"] = "\\markdownRendererCircumflex\\markdownRendererCircumflex ",
    ["☒"] = "\\markdownRendererTickedBox{}",
    ["⌛"] = "\\markdownRendererHalfTickedBox{}",
    ["☐"] = "\\markdownRendererUntickedBox{}",
    [entities.hex_entity('FFFD')] = "\\markdownRendererReplacementCharacter{}",
  }
%    \end{macrocode}
% \par
% \begin{markdown}
%
% Define table \luamdef{writer->escaped_strings} containing the mapping from
% character strings that need to be escaped in typeset content.
%
% \end{markdown}
%  \begin{macrocode}
  self.escaped_strings = util.table_copy(self.escaped_minimal_strings)
  self.escaped_strings[entities.hex_entity('00A0')] = self.nbsp
%    \end{macrocode}
% \par
% \begin{markdown}
%
% Define a table \luamdef{writer->escaped_chars} containing the mapping from
% special plain \TeX{} characters (including the active pipe character (`|`)
% of \Hologo{ConTeXt}) that need to be escaped in typeset content.
%
% \end{markdown}
%  \begin{macrocode}
  self.escaped_chars = {
    ["{"] = "\\markdownRendererLeftBrace{}",
    ["}"] = "\\markdownRendererRightBrace{}",
    ["%"] = "\\markdownRendererPercentSign{}",
    ["\\"] = "\\markdownRendererBackslash{}",
    ["#"] = "\\markdownRendererHash{}",
    ["$"] = "\\markdownRendererDollarSign{}",
    ["&"] = "\\markdownRendererAmpersand{}",
    ["_"] = "\\markdownRendererUnderscore{}",
    ["^"] = "\\markdownRendererCircumflex{}",
    ["~"] = "\\markdownRendererTilde{}",
    ["|"] = "\\markdownRendererPipe{}",
    [entities.hex_entity('0000')] = "\\markdownRendererReplacementCharacter{}",
  }
%    \end{macrocode}
% \par
% \begin{markdown}
%
% Use the \luamref{writer->escaped_chars}, \luamref{writer->escaped_uri_chars},
% and \luamref{writer->escaped_minimal_strings} tables to create the
% \luamdef{escape_typographic_text},
% \luamdef{escape_programmatic_text}, and
% \luamdef{escape_minimal} local escaper functions.
%
% \end{markdown}
%  \begin{macrocode}
  local function create_escaper(char_escapes, string_escapes)
    local escape = util.escaper(char_escapes, string_escapes)
    return function(s)
      if self.flatten_inlines then return s end
      return escape(s)
    end
  end
  local escape_typographic_text = create_escaper(
    self.escaped_chars, self.escaped_strings)
  local escape_programmatic_text = create_escaper(
    self.escaped_uri_chars, self.escaped_minimal_strings)
  local escape_minimal = create_escaper(
    {}, self.escaped_minimal_strings)
%    \end{macrocode}
% \par
% \begin{markdown}
%
% Define the following semantic aliases for the escaper functions:
%
% - \luamdef{writer->escape} transforms a text string that should always be
%   made printable.
% - \luamdef{writer->string} transforms a text string that should be made
%   printable only when the \Opt{hybrid} Lua option is disabled. When \Opt{hybrid}
%   is enabled, the text string should be kept as-is.
% - \luamdef{writer->math} transforms a math span.
% - \luamdef{writer->identifier} transforms an input programmatic identifier.
% - \luamdef{writer->uri} transforms an input \acro{uri}.
% - \luamdef{writer->infostring} transforms a fence code infostring.
%
% \end{markdown}
%  \begin{macrocode}
  self.escape = escape_typographic_text
  self.math = escape_minimal
  if options.hybrid then
    self.identifier = escape_minimal
    self.string = escape_minimal
    self.uri = escape_minimal
    self.infostring = escape_minimal
  else
    self.identifier = escape_programmatic_text
    self.string = escape_typographic_text
    self.uri = escape_programmatic_text
    self.infostring = escape_programmatic_text
  end
%    \end{macrocode}
% \par
% \begin{markdown}
%
% Define \luamdef{writer->warning} as a function that will transform an input
% warning `t` to the output format.
%
% \end{markdown}
%  \begin{macrocode}
  function self.warning(t)
    return {"\\markdownRendererWarning{", self.identifier(t), "}"}
  end
%    \end{macrocode}
% \par
% \begin{markdown}
%
% Define \luamdef{writer->error} as a function that will transform an input
% error `t` to the output format.
%
% \end{markdown}
%  \begin{macrocode}
  function self.error(t)
    return {"\\markdownRendererError{", self.identifier(t), "}"}
  end
%    \end{macrocode}
% \par
% \begin{markdown}
%
% Define \luamdef{writer->code} as a function that will transform an input
% inline code span `s` with optional attributes `attributes` to the output
% format.
%
% \end{markdown}
%  \begin{macrocode}
  function self.code(s, attributes)
    if self.flatten_inlines then return s end
    local buf = {}
    if attributes ~= nil then
      table.insert(buf,
                   "\\markdownRendererCodeSpanAttributeContextBegin\n")
      table.insert(buf, self.attributes(attributes))
    end
    table.insert(buf,
                 {"\\markdownRendererCodeSpan{", self.escape(s), "}"})
    if attributes ~= nil then
      table.insert(buf,
                   "\\markdownRendererCodeSpanAttributeContextEnd{}")
    end
    return buf
  end
%    \end{macrocode}
% \par
% \begin{markdown}
%
% Define \luamdef{writer->link} as a function that will transform an input
% hyperlink to the output format, where `lab` corresponds to the label,
% `src` to \acro{uri}, `tit` to the title of the link, and `attributes` to
% optional attributes.
%
% \end{markdown}
%  \begin{macrocode}
  function self.link(lab, src, tit, attributes)
    if self.flatten_inlines then return lab end
    local buf = {}
    if attributes ~= nil then
      table.insert(buf,
                   "\\markdownRendererLinkAttributeContextBegin\n")
      table.insert(buf, self.attributes(attributes))
    end
    table.insert(buf, {"\\markdownRendererLink{",lab,"}",
                       "{",self.escape(src),"}",
                       "{",self.uri(src),"}",
                       "{",self.string(tit or ""),"}"})
    if attributes ~= nil then
      table.insert(buf,
                   "\\markdownRendererLinkAttributeContextEnd{}")
    end
    return buf
  end
%    \end{macrocode}
% \par
% \begin{markdown}
%
% Define \luamdef{writer->image} as a function that will transform an input
% image to the output format, where `lab` corresponds to the label, `src`
% to the \acro{url}, `tit` to the title of the image, and `attributes` to
% optional attributes.
%
% \end{markdown}
%  \begin{macrocode}
  function self.image(lab, src, tit, attributes)
    if self.flatten_inlines then return lab end
    local buf = {}
    if attributes ~= nil then
      table.insert(buf,
                   "\\markdownRendererImageAttributeContextBegin\n")
      table.insert(buf, self.attributes(attributes))
    end
    table.insert(buf, {"\\markdownRendererImage{",lab,"}",
                       "{",self.string(src),"}",
                       "{",self.uri(src),"}",
                       "{",self.string(tit or ""),"}"})
    if attributes ~= nil then
      table.insert(buf,
                   "\\markdownRendererImageAttributeContextEnd{}")
    end
    return buf
  end
%    \end{macrocode}
% \par
% \begin{markdown}
%
% Define \luamdef{writer->bulletlist} as a function that will transform an input
% bulleted list to the output format, where `items` is an array of the list
% items and `tight` specifies, whether the list is tight or not.
%
% \end{markdown}
%  \begin{macrocode}
  function self.bulletlist(items,tight)
    if not self.is_writing then return "" end
    local buffer = {}
    for _,item in ipairs(items) do
      if item ~= "" then
        buffer[#buffer + 1] = self.bulletitem(item)
      end
    end
    local contents = util.intersperse(buffer,"\n")
    if tight and options.tightLists then
      return {"\\markdownRendererUlBeginTight\n",contents,
        "\n\\markdownRendererUlEndTight "}
    else
      return {"\\markdownRendererUlBegin\n",contents,
        "\n\\markdownRendererUlEnd "}
    end
  end
%    \end{macrocode}
% \begin{markdown}
%
% Define \luamdef{writer->bulletitem} as a function that will transform an
% input bulleted list item to the output format, where `s` is the text of
% the list item.
%
% \end{markdown}
%  \begin{macrocode}
  function self.bulletitem(s)
    return {"\\markdownRendererUlItem ",s,
            "\\markdownRendererUlItemEnd "}
  end
%    \end{macrocode}
% \par
% \begin{markdown}
%
% Define \luamdef{writer->orderedlist} as a function that will transform an
% input ordered list to the output format, where `items` is an array of the
% list items and `tight` specifies, whether the list is tight or not. If the
% optional parameter `startnum` is present, it is the number of the first list
% item.
%
% \end{markdown}
%  \begin{macrocode}
  function self.orderedlist(items,tight,startnum)
    if not self.is_writing then return "" end
    local buffer = {}
    local num = startnum
    for _,item in ipairs(items) do
      if item ~= "" then
        buffer[#buffer + 1] = self.ordereditem(item,num)
      end
      if num ~= nil and item ~= "" then
        num = num + 1
      end
    end
    local contents = util.intersperse(buffer,"\n")
    if tight and options.tightLists then
      return {"\\markdownRendererOlBeginTight\n",contents,
              "\n\\markdownRendererOlEndTight "}
    else
      return {"\\markdownRendererOlBegin\n",contents,
              "\n\\markdownRendererOlEnd "}
    end
  end
%    \end{macrocode}
% \begin{markdown}
%
% Define \luamdef{writer->ordereditem} as a function that will transform an
% input ordered list item to the output format, where `s` is the text of
% the list item. If the optional parameter `num` is present, it is the number
% of the list item.
%
% \end{markdown}
%  \begin{macrocode}
  function self.ordereditem(s,num)
    if num ~= nil then
      return {"\\markdownRendererOlItemWithNumber{",num,"}",s,
              "\\markdownRendererOlItemEnd "}
    else
      return {"\\markdownRendererOlItem ",s,
              "\\markdownRendererOlItemEnd "}
    end
  end
%    \end{macrocode}
% \par
% \begin{markdown}
%
% Define \luamdef{writer->inline_html_comment} as a function that will
% transform the contents of an inline \acro{HTML} comment, to the output
% format, where `contents` are the contents of the \acro{HTML} comment.
%
% \end{markdown}
%  \begin{macrocode}
  function self.inline_html_comment(contents)
    if self.flatten_inlines then return contents end
    return {"\\markdownRendererInlineHtmlComment{",contents,"}"}
  end
%    \end{macrocode}
% \par
% \begin{markdown}
%
% Define \luamdef{writer->inline_html_tag} as a function that will
% transform the contents of an opening, closing, or empty inline \acro{HTML}
% tag to the output format, where `contents` are the contents of the
% \acro{HTML} tag.
%
% \end{markdown}
%  \begin{macrocode}
  function self.inline_html_tag(contents)
    if self.flatten_inlines then return contents end
    return {"\\markdownRendererInlineHtmlTag{",self.string(contents),"}"}
  end
%    \end{macrocode}
% \par
% \begin{markdown}
%
% Define \luamdef{writer->block_html_element} as a function that will
% transform the contents of a block \acro{HTML} element to the output format,
% where `s` are the contents of the \acro{HTML} element.
%
% \end{markdown}
%  \begin{macrocode}
  function self.block_html_element(s)
    if not self.is_writing then return "" end
    local name = util.cache(options.cacheDir, s, nil, nil, ".verbatim")
    return {"\\markdownRendererInputBlockHtmlElement{",name,"}"}
  end
%    \end{macrocode}
% \par
% \begin{markdown}
%
% Define \luamdef{writer->emphasis} as a function that will transform an
% emphasized span `s` of input text to the output format.
%
% \end{markdown}
%  \begin{macrocode}
  function self.emphasis(s)
    if self.flatten_inlines then return s end
    return {"\\markdownRendererEmphasis{",s,"}"}
  end
%    \end{macrocode}
% \par
% \begin{markdown}
%
% Define \luamdef{writer->tickbox} as a function that will transform a
% number `f` to the output format.
%
% \end{markdown}
%  \begin{macrocode}
  function self.tickbox(f)
    if f == 1.0 then
      return "☒ "
    elseif f == 0.0 then
      return "☐ "
    else
      return "⌛ "
    end
  end
%    \end{macrocode}
% \par
% \begin{markdown}
%
% Define \luamdef{writer->strong} as a function that will transform a strongly
% emphasized span `s` of input text to the output format.
%
% \end{markdown}
%  \begin{macrocode}
  function self.strong(s)
    if self.flatten_inlines then return s end
    return {"\\markdownRendererStrongEmphasis{",s,"}"}
  end
%    \end{macrocode}
% \par
% \begin{markdown}
%
% Define \luamdef{writer->blockquote} as a function that will transform an
% input block quote `s` to the output format.
%
% \end{markdown}
%  \begin{macrocode}
  function self.blockquote(s)
    if not self.is_writing then return "" end
    return {"\\markdownRendererBlockQuoteBegin\n",s,
      "\\markdownRendererBlockQuoteEnd "}
  end
%    \end{macrocode}
% \par
% \begin{markdown}
%
% Define \luamdef{writer->verbatim} as a function that will transform an
% input code block `s` to the output format.
%
% \end{markdown}
%  \begin{macrocode}
  function self.verbatim(s)
    if not self.is_writing then return "" end
    s = s:gsub("\n$", "")
    local name = util.cache_verbatim(options.cacheDir, s)
    return {"\\markdownRendererInputVerbatim{",name,"}"}
  end
%    \end{macrocode}
% \par
% \begin{markdown}
%
% Define \luamdef{writer->document} as a function that will transform a
% document `d` to the output format.
%
% \end{markdown}
%  \begin{macrocode}
  function self.document(d)
    local buf = {"\\markdownRendererDocumentBegin\n", d}

    -- pop all attributes
    table.insert(buf, self.pop_attributes())

    table.insert(buf, "\\markdownRendererDocumentEnd")

    return buf
  end
%    \end{macrocode}
% \par
% \begin{markdown}
%
% Define \luamdef{writer->attributes} as a function that will transform
% input attributes `attrs` to the output format.
%
% \end{markdown}
%  \begin{macrocode}
  local seen_identifiers = {}
  local key_value_regex = "([^= ]+)%s*=%s*(.*)"
  local function normalize_attributes(attributes, auto_identifiers)
    -- normalize attributes
    local normalized_attributes = {}
    local has_explicit_identifiers = false
    local key, value
    for _, attribute in ipairs(attributes or {}) do
      if attribute:sub(1, 1) == "#" then
        table.insert(normalized_attributes, attribute)
        has_explicit_identifiers = true
        seen_identifiers[attribute:sub(2)] = true
      elseif attribute:sub(1, 1) == "." then
        table.insert(normalized_attributes, attribute)
      else
        key, value = attribute:match(key_value_regex)
        if key:lower() == "id" then
          table.insert(normalized_attributes, "#" .. value)
        elseif key:lower() == "class" then
          local classes = {}
          for class in value:gmatch("%S+") do
            table.insert(classes, class)
          end
          table.sort(classes)
          for _, class in ipairs(classes) do
            table.insert(normalized_attributes, "." .. class)
          end
        else
          table.insert(normalized_attributes, attribute)
        end
      end
    end

    -- if no explicit identifiers exist, add auto identifiers
    if not has_explicit_identifiers and auto_identifiers ~= nil then
      local seen_auto_identifiers = {}
      for _, auto_identifier in ipairs(auto_identifiers) do
        if seen_auto_identifiers[auto_identifier] == nil then
          seen_auto_identifiers[auto_identifier] = true
          if seen_identifiers[auto_identifier] == nil then
            seen_identifiers[auto_identifier] = true
            table.insert(normalized_attributes,
                         "#" .. auto_identifier)
          else
            local auto_identifier_number = 1
            while true do
              local numbered_auto_identifier = auto_identifier .. "-"
                                             .. auto_identifier_number
              if seen_identifiers[numbered_auto_identifier] == nil then
                seen_identifiers[numbered_auto_identifier] = true
                table.insert(normalized_attributes,
                             "#" .. numbered_auto_identifier)
                break
              end
              auto_identifier_number = auto_identifier_number + 1
            end
          end
        end
      end
    end

    -- sort and deduplicate normalized attributes
    table.sort(normalized_attributes)
    local seen_normalized_attributes = {}
    local deduplicated_normalized_attributes = {}
    for _, attribute in ipairs(normalized_attributes) do
      if seen_normalized_attributes[attribute] == nil then
        seen_normalized_attributes[attribute] = true
        table.insert(deduplicated_normalized_attributes, attribute)
      end
    end

    return deduplicated_normalized_attributes
  end

  function self.attributes(attributes, should_normalize_attributes)
    local normalized_attributes
    if should_normalize_attributes == false then
      normalized_attributes = attributes
    else
      normalized_attributes = normalize_attributes(attributes)
    end

    local buf = {}
    local key, value
    for _, attribute in ipairs(normalized_attributes) do
      if attribute:sub(1, 1) == "#" then
        table.insert(buf, {"\\markdownRendererAttributeIdentifier{",
                           attribute:sub(2), "}"})
      elseif attribute:sub(1, 1) == "." then
        table.insert(buf, {"\\markdownRendererAttributeClassName{",
                           attribute:sub(2), "}"})
      else
        key, value = attribute:match(key_value_regex)
        table.insert(buf, {"\\markdownRendererAttributeKeyValue{",
                           key, "}{", value, "}"})
      end
    end

    return buf
  end
%    \end{macrocode}
% \par
% \begin{markdown}
%
% Define \luamdef{writer->active\_attributes} as a stack of block-level
% attributes that are currently active. The
% \luamref{writer->active\_attributes} member variable is mutable.
%
% \end{markdown}
%  \begin{macrocode}
  self.active_attributes = {}
%    \end{macrocode}
% \par
% \begin{markdown}
%
% Define \luamdef{writer->attribute\_type\_levels} as a hash table that
% maps attribute types to the number of attributes of said type in
% \luamref{writer->active\_attributes}.
%
% \end{markdown}
%  \begin{macrocode}
  self.attribute_type_levels = {}
  setmetatable(self.attribute_type_levels,
               { __index = function() return 0 end })
%    \end{macrocode}
% \par
% \begin{markdown}
%
% Define \luamdef{writer->push\_attributes} and
% \luamdef{writer->pop\_attributes} as functions that will add a new set
% of active block-level attributes or remove the most current attributes
% from \luamref{writer->active\_attributes}.
%
% \end{markdown}
%  \begin{macrocode}
  local function apply_attributes()
    local buf = {}
    for i = 1, #self.active_attributes do
      local start_output = self.active_attributes[i][3]
      if start_output ~= nil then
        table.insert(buf, start_output)
      end
    end
    return buf
  end

  local function tear_down_attributes()
    local buf = {}
    for i = #self.active_attributes, 1, -1 do
      local end_output = self.active_attributes[i][4]
      if end_output ~= nil then
        table.insert(buf, end_output)
      end
    end
    return buf
  end
%    \end{macrocode}
% \begin{markdown}
%
% The \luamref{writer->push\_attributes} method adds `attributes`
% of type `attribute_type` to \luamref{writer->active\_attributes}. The
% `start_output` string is used to construct a rope that will be returned by
% this function, together with output produced as a result of slicing (see
% \Opt{slice}).  The `end_output` string is stored together with `attributes`
% and is used to construct the return value of the
% \luamref{writer->pop\_attributes}
% method.
%
% \end{markdown}
%  \begin{macrocode}
  function self.push_attributes(attribute_type, attributes,
                                start_output, end_output)
    local attribute_type_level = self.attribute_type_levels[attribute_type]
    self.attribute_type_levels[attribute_type] = attribute_type_level + 1

    -- index attributes in a hash table for easy lookup
    attributes = attributes or {}
    for i = 1, #attributes do
      attributes[attributes[i]] = true
    end

    local buf = {}
    -- handle slicing
    if attributes["#" .. self.slice_end_identifier] ~= nil and
       self.slice_end_type == "^" then
      if self.is_writing then
        table.insert(buf, self.undosep())
        table.insert(buf, tear_down_attributes())
      end
      self.is_writing = false
    end
    if attributes["#" .. self.slice_begin_identifier] ~= nil and
       self.slice_begin_type == "^" then
      table.insert(buf, apply_attributes())
      self.is_writing = true
    end
    if self.is_writing and start_output ~= nil then
      table.insert(buf, start_output)
    end
    table.insert(self.active_attributes,
                 {attribute_type, attributes,
                  start_output, end_output})
    return buf
  end

%    \end{macrocode}
% \begin{markdown}
%
% The \luamref{writer->pop\_attributes} method removes the most current of
% active block-level attributes from \luamref{writer->active\_attributes}
% until attributes of type `attribute_type` have been removed. The method
% returns a rope constructed from the `end_output` string specified
% in the calls of \luamref{writer->push\_attributes} that produced the most
% current attributes, and also from output produced as a result of slicing
% (see \Opt{slice}).
%
% \end{markdown}
%  \begin{macrocode}
  function self.pop_attributes(attribute_type)
    local buf = {}
    -- pop attributes until we find attributes of correct type
    -- or until no attributes remain
    local current_attribute_type = false
    while current_attribute_type ~= attribute_type and
          #self.active_attributes > 0 do
      local attributes, _, end_output
      current_attribute_type, attributes, _, end_output = table.unpack(
        self.active_attributes[#self.active_attributes])
      local attribute_type_level = self.attribute_type_levels[current_attribute_type]
      self.attribute_type_levels[current_attribute_type] = attribute_type_level - 1
      if self.is_writing and end_output ~= nil then
        table.insert(buf, end_output)
      end
      table.remove(self.active_attributes, #self.active_attributes)
      -- handle slicing
      if attributes["#" .. self.slice_end_identifier] ~= nil
         and self.slice_end_type == "$" then
        if self.is_writing then
          table.insert(buf, self.undosep())
          table.insert(buf, tear_down_attributes())
        end
        self.is_writing = false
      end
      if attributes["#" .. self.slice_begin_identifier] ~= nil and
         self.slice_begin_type == "$" then
        self.is_writing = true
        table.insert(buf, apply_attributes())
      end
    end
    return buf
  end
%    \end{macrocode}
% \begin{markdown}
%
% Create an auto identifier string by stripping and converting characters from string `s`.
%
% \end{markdown}
%  \begin{macrocode}
  local function create_auto_identifier(s)
    local buffer = {}
    local prev_space = false
    local letter_found = false
    local normalized_s = s
    if not options.unicodeNormalization or options.unicodeNormalizationForm ~= "nfc" then
      normalized_s = uni_algos.normalize.NFC(normalized_s)
    end

    for _, code in utf8.codes(normalized_s) do
      local char = utf8.char(code)

      -- Remove everything up to the first letter.
      if not letter_found then
        local is_letter = unicode.utf8.match(char, "%a")
        if is_letter then
          letter_found = true
        else
          goto continue
        end
      end

      -- Remove all non-alphanumeric characters, except underscores, hyphens, and periods.
      if not unicode.utf8.match(char, "[%w_%-%.%s]") then
        goto continue
      end

      -- Replace all spaces and newlines with hyphens.
      if unicode.utf8.match(char, "[%s\n]") then
        char = "-"
        if prev_space then
          goto continue
        else
          prev_space = true
        end
      else
        -- Convert all alphabetic characters to lowercase.
        char = unicode.utf8.lower(char)
        prev_space = false
      end

      table.insert(buffer, char)

      ::continue::
    end

    if prev_space then
      table.remove(buffer)
    end

    local identifier = #buffer == 0 and "section" or table.concat(buffer, "")
    return identifier
  end
%    \end{macrocode}
% \begin{markdown}
%
% Create an GitHub-flavored auto identifier string by stripping and converting characters from string `s`.
%
% \end{markdown}
%  \begin{macrocode}
  local function create_gfm_auto_identifier(s)
    local buffer = {}
    local prev_space = false
    local letter_found = false
    local normalized_s = s
    if not options.unicodeNormalization or options.unicodeNormalizationForm ~= "nfc" then
      normalized_s = uni_algos.normalize.NFC(normalized_s)
    end

    for _, code in utf8.codes(normalized_s) do
      local char = utf8.char(code)

      -- Remove everything up to the first non-space.
      if not letter_found then
        local is_letter = unicode.utf8.match(char, "%S")
        if is_letter then
          letter_found = true
        else
          goto continue
        end
      end

      -- Remove all non-alphanumeric characters, except underscores and hyphens.
      if not unicode.utf8.match(char, "[%w_%-%s]") then
        prev_space = false
        goto continue
      end

      -- Replace all spaces and newlines with hyphens.
      if unicode.utf8.match(char, "[%s\n]") then
        char = "-"
        if prev_space then
          goto continue
        else
          prev_space = true
        end
      else
        -- Convert all alphabetic characters to lowercase.
        char = unicode.utf8.lower(char)
        prev_space = false
      end

      table.insert(buffer, char)

      ::continue::
    end

    if prev_space then
      table.remove(buffer)
    end

    local identifier = #buffer == 0 and "section" or table.concat(buffer, "")
    return identifier
  end
%    \end{macrocode}
% \par
% \begin{markdown}
%
% Define \luamdef{writer->heading} as a function that will transform an
% input heading `s` at level `level` with attributes `attributes` to the
% output format.
%
% \end{markdown}
%  \begin{macrocode}
  self.secbegin_text = "\\markdownRendererSectionBegin\n"
  self.secend_text = "\n\\markdownRendererSectionEnd "
  function self.heading(s, level, attributes)
    local buf = {}
    local flat_text, inlines = table.unpack(s)

    -- push empty attributes for implied sections
    while self.attribute_type_levels["heading"] < level - 1 do
      table.insert(buf,
                   self.push_attributes("heading",
                                        nil,
                                        self.secbegin_text,
                                        self.secend_text))
    end

    -- pop attributes for sections that have ended
    while self.attribute_type_levels["heading"] >= level do
      table.insert(buf, self.pop_attributes("heading"))
    end

    -- construct attributes for the new section
    local auto_identifiers = {}
    if self.options.autoIdentifiers then
      table.insert(auto_identifiers, create_auto_identifier(flat_text))
    end
    if self.options.gfmAutoIdentifiers then
      table.insert(auto_identifiers, create_gfm_auto_identifier(flat_text))
    end
    local normalized_attributes = normalize_attributes(attributes, auto_identifiers)

    -- push attributes for the new section
    local start_output = {}
    local end_output = {}
    table.insert(start_output, self.secbegin_text)
    table.insert(end_output, self.secend_text)

    table.insert(buf, self.push_attributes("heading",
                                           normalized_attributes,
                                           start_output,
                                           end_output))
    assert(self.attribute_type_levels["heading"] == level)

    -- render the heading and its attributes
    if self.is_writing and #normalized_attributes > 0 then
      table.insert(buf, "\\markdownRendererHeaderAttributeContextBegin\n")
      table.insert(buf, self.attributes(normalized_attributes, false))
    end

    local cmd
    level = level + options.shiftHeadings
    if level <= 1 then
      cmd = "\\markdownRendererHeadingOne"
    elseif level == 2 then
      cmd = "\\markdownRendererHeadingTwo"
    elseif level == 3 then
      cmd = "\\markdownRendererHeadingThree"
    elseif level == 4 then
      cmd = "\\markdownRendererHeadingFour"
    elseif level == 5 then
      cmd = "\\markdownRendererHeadingFive"
    elseif level >= 6 then
      cmd = "\\markdownRendererHeadingSix"
    else
      cmd = ""
    end
    if self.is_writing then
      table.insert(buf, {cmd, "{", inlines, "}"})
    end

    if self.is_writing and #normalized_attributes > 0 then
      table.insert(buf, "\\markdownRendererHeaderAttributeContextEnd{}")
    end

    return buf
  end
%    \end{macrocode}
% \par
% \begin{markdown}
%
% Define \luamdef{writer->get_state} as a function that returns the current
% state of the writer, where the state of a writer are its mutable member
% variables.
%
% \end{markdown}
%  \begin{macrocode}
  function self.get_state()
    return {
      is_writing=self.is_writing,
      flatten_inlines=self.flatten_inlines,
      active_attributes={table.unpack(self.active_attributes)},
    }
  end
%    \end{macrocode}
% \par
% \begin{markdown}
%
% Define \luamdef{writer->set_state} as a function that restores the input
% state `s` and returns the previous state of the writer.
%
% \end{markdown}
%  \begin{macrocode}
  function self.set_state(s)
    local previous_state = self.get_state()
    for key, value in pairs(s) do
      self[key] = value
    end
    return previous_state
  end
%    \end{macrocode}
% \par
% \begin{markdown}
%
% Define \luamdef{writer->defer_call} as a function that will encapsulate the
% input function `f`, so that `f` is called with the state of the writer at the
% time of calling \luamref{writer->defer_call}.
%
% \end{markdown}
%  \begin{macrocode}
  function self.defer_call(f)
    local previous_state = self.get_state()
    return function(...)
      local state = self.set_state(previous_state)
      local return_value = f(...)
      self.set_state(state)
      return return_value
    end
  end

  return self
end
%    \end{macrocode}
% \par
% \begin{markdown}
%
%### Parsers
% The \luamdef{parsers} hash table stores \acro{peg} patterns that are
% static and can be reused between different \luamref{reader} objects.
%
% \end{markdown}
%  \begin{macrocode}
local parsers                  = {}
%    \end{macrocode}
% \par
% \begin{markdown}
%
%#### Basic Parsers
%
% \end{markdown}
%  \begin{macrocode}
parsers.percent                = P("%")
parsers.at                     = P("@")
parsers.comma                  = P(",")
parsers.asterisk               = P("*")
parsers.dash                   = P("-")
parsers.plus                   = P("+")
parsers.underscore             = P("_")
parsers.period                 = P(".")
parsers.hash                   = P("#")
parsers.dollar                 = P("$")
parsers.ampersand              = P("&")
parsers.backtick               = P("`")
parsers.less                   = P("<")
parsers.more                   = P(">")
parsers.space                  = P(" ")
parsers.squote                 = P("'")
parsers.dquote                 = P('"')
parsers.lparent                = P("(")
parsers.rparent                = P(")")
parsers.lbracket               = P("[")
parsers.rbracket               = P("]")
parsers.lbrace                 = P("{")
parsers.rbrace                 = P("}")
parsers.circumflex             = P("^")
parsers.slash                  = P("/")
parsers.equal                  = P("=")
parsers.colon                  = P(":")
parsers.semicolon              = P(";")
parsers.exclamation            = P("!")
parsers.pipe                   = P("|")
parsers.tilde                  = P("~")
parsers.backslash              = P("\\")
parsers.tab                    = P("\t")
parsers.newline                = P("\n")

parsers.digit                  = R("09")
parsers.hexdigit               = R("09","af","AF")
parsers.letter                 = R("AZ","az")
parsers.alphanumeric           = R("AZ","az","09")
parsers.keyword                = parsers.letter
                               * (parsers.alphanumeric + parsers.dash)^0

parsers.doubleasterisks        = P("**")
parsers.doubleunderscores      = P("__")
parsers.doubletildes           = P("~~")
parsers.fourspaces             = P("    ")

parsers.any                    = P(1)
parsers.succeed                = P(true)
parsers.fail                   = P(false)

parsers.internal_punctuation   = S(":;,.?")
parsers.ascii_punctuation      = S("!\"#$%&'()*+,-./:;<=>?@[\\]^_`{|}~")
%    \end{macrocode}
% \par
% \begin{markdown}
%
%### Unicode punctuation
% This section documents [the Unicode punctuation][unicode-punctuation]
% recognized by the markdown reader. The punctuation is organized in the
% \luamdef{parsers.punctuation} table according to the number of bytes occupied
% after conversion to \acro{utf}8.
%
% [unicode-punctuation]: https://spec.commonmark.org/0.31.2/#unicode-punctuation-character
%                        (CommonMark Spec, Version 0.31.2 (2024-01-28))
%
% \end{markdown}
%  \begin{macrocode}
parsers.punctuation            = {}
(function()
  local pathname = assert(kpse.find_file("UnicodeData.txt"),
    [[Could not locate file "UnicodeData.txt"]])
  local file = assert(io.open(pathname, "r"),
    [[Could not open file "UnicodeData.txt"]])
  for line in file:lines() do
    local codepoint, major_category = line:match("^(%x+);[^;]*;(%a)")
    if major_category == "P" or major_category == "S" then
      local code = unicode.utf8.char(tonumber(codepoint, 16))
      if parsers.punctuation[#code] == nil then
        parsers.punctuation[#code] = parsers.fail
      end
      local code_parser = parsers.succeed
      for i = 1, #code do
        local byte = code:sub(i, i)
        local byte_parser = S(byte)
        code_parser = code_parser
                    * byte_parser
      end
      parsers.punctuation[#code] = parsers.punctuation[#code]
                                 + code_parser
    end
  end
  assert(file:close())
end)()

parsers.escapable              = parsers.ascii_punctuation
parsers.anyescaped             = parsers.backslash / "" * parsers.escapable
                               + parsers.any

parsers.spacechar              = S("\t ")
parsers.spacing                = S(" \n\r\t")
parsers.nonspacechar           = parsers.any - parsers.spacing
parsers.optionalspace          = parsers.spacechar^0

parsers.normalchar             = parsers.any - (V("SpecialChar")
                                                + parsers.spacing)
parsers.eof                    = -parsers.any
parsers.nonindentspace         = parsers.space^-3 * - parsers.spacechar
parsers.indent                 = parsers.space^-3 * parsers.tab
                               + parsers.fourspaces / ""
parsers.linechar               = P(1 - parsers.newline)

parsers.blankline              = parsers.optionalspace
                               * parsers.newline / "\n"
parsers.blanklines             = parsers.blankline^0
parsers.skipblanklines         = (parsers.optionalspace * parsers.newline)^0
parsers.indentedline           = parsers.indent    /""
                               * C(parsers.linechar^1 * parsers.newline^-1)
parsers.optionallyindentedline = parsers.indent^-1 /""
                               * C(parsers.linechar^1 * parsers.newline^-1)
parsers.sp                     = parsers.spacing^0
parsers.spnl                   = parsers.optionalspace
                               * (parsers.newline * parsers.optionalspace)^-1
parsers.line                   = parsers.linechar^0 * parsers.newline
parsers.nonemptyline           = parsers.line - parsers.blankline
%    \end{macrocode}
%   \par
% \begin{markdown}
%
%#### Parsers Used for Indentation
%
% \end{markdown}
%  \begin{macrocode}

parsers.leader      = parsers.space^-3

%    \end{macrocode}
% \begin{markdown}
%
% Check if a trail exists and is non-empty in the indent table `indent_table`.
%
% \end{markdown}
%  \begin{macrocode}
local function has_trail(indent_table)
  return indent_table ~= nil and
    indent_table.trail ~= nil and
    next(indent_table.trail) ~= nil
end

%    \end{macrocode}
% \begin{markdown}
%
% Check if indent table `indent_table` has any indents.
%
% \end{markdown}
%  \begin{macrocode}
local function has_indents(indent_table)
  return indent_table ~= nil and
    indent_table.indents ~= nil and
    next(indent_table.indents) ~= nil
end

%    \end{macrocode}
% \begin{markdown}
%
% Add a trail `trail_info` to the indent table `indent_table`.
%
% \end{markdown}
%  \begin{macrocode}
local function add_trail(indent_table, trail_info)
  indent_table.trail = trail_info
  return indent_table
end

%    \end{macrocode}
% \begin{markdown}
%
% Remove a trail `trail_info` from the indent table `indent_table`.
%
% \end{markdown}
%  \begin{macrocode}
local function remove_trail(indent_table)
  indent_table.trail = nil
  return indent_table
end

%    \end{macrocode}
% \begin{markdown}
%
% Update the indent table `indent_table` by adding or removing a new
% indent `add`.
%
% \end{markdown}
%  \begin{macrocode}
local function update_indent_table(indent_table, new_indent, add)
  indent_table = remove_trail(indent_table)

  if not has_indents(indent_table) then
    indent_table.indents = {}
  end


  if add then
    indent_table.indents[#indent_table.indents + 1] = new_indent
  else
    if indent_table.indents[#indent_table.indents].name == new_indent.name then
      indent_table.indents[#indent_table.indents] = nil
    end
  end

  return indent_table
end

%    \end{macrocode}
% \begin{markdown}
%
% Remove an indent by its name `name`.
%
% \end{markdown}
%  \begin{macrocode}
local function remove_indent(name)
  local function remove_indent_level(s, i, indent_table) -- luacheck: ignore s i
    indent_table = update_indent_table(indent_table, {name=name}, false)
    return true, indent_table
  end

  return Cg(Cmt(Cb("indent_info"), remove_indent_level), "indent_info")
end

%    \end{macrocode}
% \begin{markdown}
%
% Process the spacing of a string of spaces and tabs `spacing` with preceding indent width from
% the start of the line `indent` and strip up to `left_strip_length` spaces. Return the remainder
% `remainder` and whether there is enough spaces to produce a code `is_code`. Return how many
% spaces were stripped, as well as if the minimum was met `is_minimum` and what remainder it
% left `minimum_remainder`.
%
% \end{markdown}
%  \begin{macrocode}
local function process_starter_spacing(indent, spacing, minimum, left_strip_length)
  left_strip_length = left_strip_length or 0

  local count = 0
  local tab_value = 4 - (indent) % 4

  local code_started, minimum_found = false, false
  local code_start, minimum_remainder = "", ""

  local left_total_stripped = 0
  local full_remainder = ""

  if spacing ~= nil then
    for i = 1, #spacing do
      local character = spacing:sub(i, i)

      if character == "\t" then
        count = count + tab_value
        tab_value = 4
      elseif character == " " then
        count = count + 1
        tab_value = 4 - (1 - tab_value) % 4
      end

      if (left_strip_length ~= 0) then
        local possible_to_strip = math.min(count, left_strip_length)
        count = count - possible_to_strip
        left_strip_length = left_strip_length - possible_to_strip
        left_total_stripped = left_total_stripped + possible_to_strip
      else
        full_remainder =  full_remainder .. character
      end

      if (minimum_found) then
        minimum_remainder = minimum_remainder .. character
      elseif (count >= minimum) then
        minimum_found = true
        minimum_remainder = minimum_remainder .. string.rep(" ", count - minimum)
      end

      if (code_started) then
        code_start = code_start .. character
      elseif (count >= minimum + 4) then
        code_started = true
        code_start = code_start .. string.rep(" ", count - (minimum + 4))
      end
    end
  end

  local remainder
  if (code_started) then
    remainder = code_start
  else
    remainder = string.rep(" ", count - minimum)
  end

  local is_minimum = count >= minimum
  return {
    is_code = code_started,
    remainder = remainder,
    left_total_stripped = left_total_stripped,
    is_minimum = is_minimum,
    minimum_remainder = minimum_remainder,
    total_length = count,
    full_remainder = full_remainder
  }
end

%    \end{macrocode}
% \begin{markdown}
%
% Count the total width of all indents in the indent table `indent_table`.
%
% \end{markdown}
%  \begin{macrocode}
local function count_indent_tab_level(indent_table)
  local count = 0
  if not has_indents(indent_table) then
    return count
  end

  for i=1, #indent_table.indents do
    count = count + indent_table.indents[i].length
  end
  return count
end

%    \end{macrocode}
% \begin{markdown}
%
% Count the total width of a delimiter `delimiter`.
%
% \end{markdown}
%  \begin{macrocode}
local function total_delimiter_length(delimiter)
  local count = 0
  if type(delimiter) == "string" then return #delimiter end
  for _, value in pairs(delimiter) do
    count = count + total_delimiter_length(value)
  end
  return count
end

%    \end{macrocode}
% \begin{markdown}
%
% Process the container starter `starter` of a type `indent_type`. Adjust the width of the indent
% if the delimiter is followed only by whitespaces `is_blank`.
%
% \end{markdown}
%  \begin{macrocode}
local function process_starter_indent(_, _, indent_table, starter, is_blank, indent_type, breakable)
  local last_trail = starter[1]
  local delimiter = starter[2]
  local raw_new_trail = starter[3]

  if indent_type == "bq" and not breakable then
    indent_table.ignore_blockquote_blank = true
  end

  if has_trail(indent_table) then
    local trail = indent_table.trail
    if trail.is_code then
      return false
    end
    last_trail = trail.remainder
  else
    local sp = process_starter_spacing(0, last_trail, 0, 0)

    if sp.is_code then
      return false
    end
    last_trail = sp.remainder
  end

  local preceding_indentation = count_indent_tab_level(indent_table) % 4
  local last_trail_length = #last_trail
  local delimiter_length = total_delimiter_length(delimiter)

  local total_indent_level = preceding_indentation + last_trail_length + delimiter_length

  local sp = {}
  if not is_blank then
    sp = process_starter_spacing(total_indent_level, raw_new_trail, 0, 1)
  end

  local del_trail_length = sp.left_total_stripped
  if is_blank then
    del_trail_length = 1
  elseif not sp.is_code then
    del_trail_length = del_trail_length + #sp.remainder
  end

  local indent_length = last_trail_length + delimiter_length + del_trail_length
  local new_indent_info = {name=indent_type, length=indent_length}

  indent_table = update_indent_table(indent_table, new_indent_info, true)
  indent_table = add_trail(indent_table, {is_code=sp.is_code, remainder=sp.remainder, total_length=sp.total_length,
                                          full_remainder=sp.full_remainder})

  return true, indent_table
end

%    \end{macrocode}
% \begin{markdown}
%
% Return the pattern corresponding with the indent name `name`.
%
% \end{markdown}
%  \begin{macrocode}
local function decode_pattern(name)
  local delimeter = parsers.succeed
  if name == "bq" then
    delimeter = parsers.more
  end

  return C(parsers.optionalspace) * C(delimeter) * C(parsers.optionalspace) * Cp()
end

%    \end{macrocode}
% \begin{markdown}
%
% Find the first blank-only indent of the indent table `indent_table` followed
% by blank-only indents.
%
% \end{markdown}
%  \begin{macrocode}
local function left_blank_starter(indent_table)
  local blank_starter_index

  if not has_indents(indent_table) then
    return
  end

  for i = #indent_table.indents,1,-1 do
    local value = indent_table.indents[i]
    if value.name == "li" then
      blank_starter_index = i
    else
      break
    end
  end

  return blank_starter_index
end

%    \end{macrocode}
% \begin{markdown}
%
% Apply the patterns decoded from the indents of the indent table `indent_table`
% iteratively starting at position `index` of the string `s`. If the `is_optional`
% mode is selected, match as many patterns as possible, else match all or fail.
% With the option `is_blank`, the parsing behaves as optional after the position
% of a blank-only indent has been surpassed.
%
% \end{markdown}
%  \begin{macrocode}
local function traverse_indent(s, i, indent_table, is_optional, is_blank, current_line_indents)
  local new_index = i

  local preceding_indentation = 0
  local current_trail = {}

  local blank_starter = left_blank_starter(indent_table)

  if current_line_indents == nil then
    current_line_indents = {}
  end

  for index = 1,#indent_table.indents do
    local value = indent_table.indents[index]
    local pattern = decode_pattern(value.name)

    -- match decoded pattern
    local new_indent_info = lpeg.match(Ct(pattern), s, new_index)
    if new_indent_info == nil then
      local blankline_end = lpeg.match(Ct(parsers.blankline * Cg(Cp(), "pos")), s, new_index)
      if is_optional or not indent_table.ignore_blockquote_blank or not blankline_end then
        return is_optional, new_index, current_trail, current_line_indents
      end

      return traverse_indent(s, tonumber(blankline_end.pos), indent_table, is_optional, is_blank, current_line_indents)
    end

    local raw_last_trail = new_indent_info[1]
    local delimiter = new_indent_info[2]
    local raw_new_trail = new_indent_info[3]
    local next_index = new_indent_info[4]

    local space_only = delimiter == ""

    -- check previous trail
    if not space_only and next(current_trail) == nil then
      local sp = process_starter_spacing(0, raw_last_trail, 0, 0)
      current_trail = {is_code=sp.is_code, remainder=sp.remainder, total_length=sp.total_length,
                       full_remainder=sp.full_remainder}
    end

    if next(current_trail) ~= nil then
      if not space_only and current_trail.is_code then
        return is_optional, new_index, current_trail, current_line_indents
      end
      if current_trail.internal_remainder ~= nil then
        raw_last_trail = current_trail.internal_remainder
      end
    end

    local raw_last_trail_length = 0
    local delimiter_length = 0

    if not space_only then
      delimiter_length = #delimiter
      raw_last_trail_length = #raw_last_trail
    end

    local total_indent_level = preceding_indentation + raw_last_trail_length + delimiter_length

    local spacing_to_process
    local minimum = 0
    local left_strip_length = 0

    if not space_only then
      spacing_to_process = raw_new_trail
      left_strip_length = 1
    else
      spacing_to_process = raw_last_trail
      minimum = value.length
    end

    local sp = process_starter_spacing(total_indent_level, spacing_to_process, minimum, left_strip_length)

    if space_only and not sp.is_minimum then
      return is_optional or (is_blank and blank_starter <= index), new_index, current_trail, current_line_indents
    end

    local indent_length = raw_last_trail_length + delimiter_length + sp.left_total_stripped

    -- update info for the next pattern
    if not space_only then
      preceding_indentation = preceding_indentation + indent_length
    else
      preceding_indentation = preceding_indentation + value.length
    end

    current_trail = {is_code=sp.is_code, remainder=sp.remainder, internal_remainder=sp.minimum_remainder,
                     total_length=sp.total_length, full_remainder=sp.full_remainder}

    current_line_indents[#current_line_indents + 1] = new_indent_info
    new_index = next_index
  end

  return true, new_index, current_trail, current_line_indents
end

%    \end{macrocode}
% \begin{markdown}
%
%  Check if a code trail is expected.
%
% \end{markdown}
%  \begin{macrocode}
local function check_trail(expect_code, is_code)
  return (expect_code and is_code) or (not expect_code and not is_code)
end

%    \end{macrocode}
% \begin{markdown}
%
% Check if the current trail of the `indent_table` would produce code if it is expected `expect_code`
% or it would not if it is not. If there is no trail, process and check the current spacing `spacing`.
%
% \end{markdown}
%  \begin{macrocode}
local function check_trail_joined(s, i, indent_table, spacing, expect_code, omit_remainder) -- luacheck: ignore s i
  local is_code
  local remainder

  if has_trail(indent_table) then
    local trail = indent_table.trail
    is_code = trail.is_code
    if is_code then
      remainder = trail.remainder
    else
      remainder = trail.full_remainder
    end
  else
    local sp = process_starter_spacing(0, spacing, 0, 0)
    is_code = sp.is_code
    if is_code then
      remainder = sp.remainder
    else
      remainder = sp.full_remainder
    end
  end

  local result = check_trail(expect_code, is_code)
  if omit_remainder then
    return result
  end
  return result, remainder
end

%    \end{macrocode}
% \begin{markdown}
%
% Check if the current trail of the `indent_table` is of length between `min` and `max`.
%
% \end{markdown}
%  \begin{macrocode}
local function check_trail_length(s, i, indent_table, spacing, min, max) -- luacheck: ignore s i
  local trail

  if has_trail(indent_table) then
    trail = indent_table.trail
  else
    trail = process_starter_spacing(0, spacing, 0, 0)
  end

  local total_length = trail.total_length
  if total_length == nil then
    return false
  end

  return min <= total_length and total_length <= max
end

%    \end{macrocode}
% \begin{markdown}
%
% Check the indentation of the continuation line, optionally with
% the mode `is_optional` selected. Check blank line exclusively with `is_blank`.
%
% \end{markdown}
%  \begin{macrocode}
local function check_continuation_indentation(s, i, indent_table, is_optional, is_blank)
  if not has_indents(indent_table) then
    return true
  end

  local passes, new_index, current_trail, current_line_indents =
    traverse_indent(s, i, indent_table, is_optional, is_blank)

  if passes then
    indent_table.current_line_indents = current_line_indents
    indent_table = add_trail(indent_table, current_trail)
    return new_index, indent_table
  end
  return false
end

%    \end{macrocode}
% \begin{markdown}
%
% Get name of the last indent from the `indent_table`.
%
% \end{markdown}
%  \begin{macrocode}
local function get_last_indent_name(indent_table)
  if has_indents(indent_table) then
    return indent_table.indents[#indent_table.indents].name
  end
end

%    \end{macrocode}
% \begin{markdown}
%
% Remove the remainder altogether if the last indent from the `indent_table`
% is blank-only.
%
% \end{markdown}
%  \begin{macrocode}
local function remove_remainder_if_blank(indent_table, remainder)
  if get_last_indent_name(indent_table) == "li" then
    return ""
  end
  return remainder
end

%    \end{macrocode}
% \begin{markdown}
%
% Take the trail `trail` or create a new one from `spacing` and compare it
% with the expected `trail_type`. On success return the index `i` and the
% remainder of the trail.
%
% \end{markdown}
%  \begin{macrocode}
local function check_trail_type(s, i, trail, spacing, trail_type) -- luacheck: ignore s i
  if trail == nil then
    trail = process_starter_spacing(0, spacing, 0, 0)
  end

  if trail_type == "non-code" then
    return check_trail(false, trail.is_code)
  end
  if trail_type == "code" then
    return check_trail(true, trail.is_code)
  end
  if trail_type == "full-code" then
    if (trail.is_code) then
      return i, trail.remainder
    end
    return i, ""
  end
  if trail_type == "full-any" then
    return i, trail.internal_remainder
  end
end

%    \end{macrocode}
% \begin{markdown}
%
% Stores or restores an `is_freezing` trail from indent table `indent_table`.
%
% \end{markdown}
%  \begin{macrocode}
local function trail_freezing(s, i, indent_table, is_freezing) -- luacheck: ignore s i
  if is_freezing then
    if indent_table.is_trail_frozen then
      indent_table.trail = indent_table.frozen_trail
    else
      indent_table.frozen_trail = indent_table.trail
      indent_table.is_trail_frozen = true
    end
  else
    indent_table.frozen_trail = nil
    indent_table.is_trail_frozen = false
  end
  return true, indent_table
end

%    \end{macrocode}
% \begin{markdown}
%
% Check the indentation of the continuation line, optionally with
% the mode `is_optional` selected. Check blank line specifically with `is_blank`.
% Additionally, also directly check the new trail with a type `trail_type`.
%
% \end{markdown}
%  \begin{macrocode}
local function check_continuation_indentation_and_trail(s, i, indent_table, is_optional, is_blank, trail_type,
                                                        reset_rem, omit_remainder)
  if not has_indents(indent_table) then
    local spacing, new_index = lpeg.match(C(parsers.spacechar^0) * Cp(), s, i)
    local result, remainder = check_trail_type(s, i, indent_table.trail, spacing, trail_type)
    if remainder == nil then
      if result then
        return new_index
      end
      return false
    end
    if result then
      return new_index, remainder
    end
    return false
  end

  local passes, new_index, current_trail = traverse_indent(s, i, indent_table, is_optional, is_blank)

  if passes then
    local spacing
    if current_trail == nil then
      local newer_spacing, newer_index = lpeg.match(C(parsers.spacechar^0) * Cp(), s, i)
      current_trail = process_starter_spacing(0, newer_spacing, 0, 0)
      new_index = newer_index
      spacing = newer_spacing
    else
      spacing = current_trail.remainder
    end
    local result, remainder = check_trail_type(s, new_index, current_trail, spacing, trail_type)
    if remainder == nil or omit_remainder then
      if result then
        return new_index
      end
      return false
    end

    if is_blank and reset_rem then
      remainder = remove_remainder_if_blank(indent_table, remainder)
    end
    if result then
      return new_index, remainder
    end
    return false
  end
  return false
end

%    \end{macrocode}
% \begin{markdown}
%
% The following patterns check whitespace indentation at the start of a block.
%
% \end{markdown}
%  \begin{macrocode}
parsers.check_trail = Cmt(Cb("indent_info") * C(parsers.spacechar^0) * Cc(false), check_trail_joined)

parsers.check_trail_no_rem = Cmt(Cb("indent_info") * C(parsers.spacechar^0) * Cc(false) * Cc(true), check_trail_joined)

parsers.check_code_trail  = Cmt(Cb("indent_info") * C(parsers.spacechar^0) * Cc(true), check_trail_joined)

parsers.check_trail_length_range  = function(min, max)
  return Cmt(Cb("indent_info") * C(parsers.spacechar^0) * Cc(min) * Cc(max), check_trail_length)
end

parsers.check_trail_length = function(n)
  return parsers.check_trail_length_range(n, n)
end

%    \end{macrocode}
% \begin{markdown}
%
% The following patterns handle trail backup, to prevent a failing pattern to modify it before
% passing it to the next.
%
% \end{markdown}
%  \begin{macrocode}
parsers.freeze_trail = Cg(Cmt(Cb("indent_info") * Cc(true), trail_freezing), "indent_info")

parsers.unfreeze_trail = Cg(Cmt(Cb("indent_info") * Cc(false), trail_freezing), "indent_info")

%    \end{macrocode}
% \begin{markdown}
%
% The following patterns check indentation in continuation lines as defined by the container start.
%
% \end{markdown}
%  \begin{macrocode}
parsers.check_minimal_indent = Cmt(Cb("indent_info") * Cc(false), check_continuation_indentation)

parsers.check_optional_indent = Cmt(Cb("indent_info") * Cc(true), check_continuation_indentation)

parsers.check_minimal_blank_indent = Cmt(Cb("indent_info") * Cc(false) * Cc(true), check_continuation_indentation)

%    \end{macrocode}
% \begin{markdown}
%
% The following patterns check indentation in continuation lines as defined by the container start.
% Additionally the subsequent trail is also directly checked.
%
% \end{markdown}
%  \begin{macrocode}

parsers.check_minimal_indent_and_trail = Cmt( Cb("indent_info")
                                            * Cc(false) * Cc(false) * Cc("non-code") * Cc(true),
                                            check_continuation_indentation_and_trail)

parsers.check_minimal_indent_and_code_trail = Cmt( Cb("indent_info")
                                                 * Cc(false) * Cc(false) * Cc("code") * Cc(false),
                                                 check_continuation_indentation_and_trail)

parsers.check_minimal_blank_indent_and_full_code_trail = Cmt( Cb("indent_info")
                                                            * Cc(false) * Cc(true) * Cc("full-code") * Cc(true),
                                                            check_continuation_indentation_and_trail)

parsers.check_minimal_indent_and_any_trail = Cmt( Cb("indent_info")
                                                * Cc(false) * Cc(false) * Cc("full-any") * Cc(true) * Cc(false),
                                                check_continuation_indentation_and_trail)

parsers.check_minimal_blank_indent_and_any_trail = Cmt( Cb("indent_info")
                                                      * Cc(false) * Cc(true) * Cc("full-any") * Cc(true) * Cc(false),
                                                       check_continuation_indentation_and_trail)

parsers.check_minimal_blank_indent_and_any_trail_no_rem = Cmt( Cb("indent_info")
                                                        * Cc(false) * Cc(true) * Cc("full-any") * Cc(true) * Cc(true),
                                                        check_continuation_indentation_and_trail)

parsers.check_optional_indent_and_any_trail = Cmt( Cb("indent_info")
                                                * Cc(true) * Cc(false) * Cc("full-any") * Cc(true) * Cc(false),
                                                check_continuation_indentation_and_trail)

parsers.check_optional_blank_indent_and_any_trail = Cmt( Cb("indent_info")
                                                      * Cc(true) * Cc(true) * Cc("full-any") * Cc(true) * Cc(false),
                                                       check_continuation_indentation_and_trail)

%    \end{macrocode}
% \begin{markdown}
%
% The following patterns specify behaviour around newlines.
%
% \end{markdown}
%  \begin{macrocode}

parsers.spnlc_noexc = parsers.optionalspace
                    * (parsers.newline * parsers.check_minimal_indent_and_any_trail)^-1

parsers.spnlc = parsers.optionalspace
              * (V("EndlineNoSub"))^-1

parsers.spnlc_sep  = parsers.optionalspace * V("EndlineNoSub")
                   + parsers.spacechar^1

parsers.only_blank = parsers.spacechar^0 * (parsers.newline + parsers.eof)

%   \end{macrocode}
% \begin{figure}
% \hspace*{-0.1\textwidth}
% \begin{minipage}{1.2\textwidth}
% \centering
% \begin{tikzpicture}[shorten >=1pt, line width=0.1mm, >={Stealth[length=2mm]}, node distance=4.5cm, on grid, auto]
% \node[state, initial by diamond, accepting] (noop) {initial};
% \node[state] (odd_backslash) [above right=of noop] {odd backslash};
% \node[state] (even_backslash) [below right=of odd_backslash] {even backslash};
% \node[state] (comment) [below=of noop] {comment};
% \node[state] (leading_spaces) [below=of even_backslash, align=center] {leading tabs\\and spaces};
% \node[state] (blank_line) [below right=of comment] {blank line};
% \path[->]
% (noop) edge [in=150, out=180, loop] node [align=center, yshift=-0.75cm] {match [$^\wedge$\textbackslash\%]\\capture \textbackslash$\!^{2k}$\meta{match}\\reset $k$} ()
%        edge [bend right=10] node [below right=-0.2cm] {match \textbackslash} (odd_backslash)
%        edge [bend left=30] node [left, align=center] {match \%\\capture \textbackslash$\!^k$\\reset $k$} (comment)
% (comment) edge [in=305, out=325, loop] node [xshift=-1.2cm] {match [$^\wedge$$\drsh$]} ()
%           edge [bend left=10] node {match $\drsh$} (leading_spaces)
% (leading_spaces) edge [loop below] node {match [\textvisiblespace$\rightleftarrows$]} ()
%                  edge [bend right=90] node [right] {match \textbackslash} (odd_backslash)
%                  edge [bend left=10] node {match \%} (comment)
%                  edge [bend right=10] node {$\epsilon$} (blank_line)
%                  edge [bend left=10] node [align=center, right=0.3cm] {match [$^\wedge$\textvisiblespace$\rightleftarrows$$\drsh$\textbackslash\%]\\capture \meta{match}} (noop)
% (blank_line) edge [loop below] node {match [\textvisiblespace$\rightleftarrows$]} ()
%              edge [bend left=90] node [align=center, below=1.2cm] {match $\drsh$\\ capture $\drsh\drsh$} (noop)
% (odd_backslash) edge [bend right=10] node [align=center, xshift=-0.3cm, yshift=0.2cm] {match \textbackslash\\increment $k$} (even_backslash)
%                 edge [bend right=10] node [align=center, above left=-0.3cm, xshift=0.1cm] {match [$^\wedge$\textbackslash]\\for \%, capture \textbackslash$\!^k$\%\\for [$^\wedge$\%], capture \textbackslash$\!^{2k+1}$\meta{match}\\reset $k$} (noop)
% (even_backslash) edge [bend left=10] node {$\epsilon$} (noop);
% \end{tikzpicture}
% \caption{A pushdown automaton that recognizes \TeX{} comments}
% \label{fig:commented_line}
% \end{minipage}
% \end{figure}
% \begin{markdown}
%
% The \luamdef{parsers.commented_line}`^1` parser recognizes the regular
% language of \TeX{} comments, see an equivalent finite automaton in Figure
% <#fig:commented_line>.
%
% \end{markdown}
%  \begin{macrocode}
parsers.commented_line_letter  = parsers.linechar
                               + parsers.newline
                               - parsers.backslash
                               - parsers.percent
parsers.commented_line         = Cg(Cc(""), "backslashes")
                               * ((#(parsers.commented_line_letter
                                    - parsers.newline)
                                  * Cb("backslashes")
                                  * Cs(parsers.commented_line_letter
                                    - parsers.newline)^1  -- initial
                                  * Cg(Cc(""), "backslashes"))
                                 + #(parsers.backslash * parsers.backslash)
                                 * Cg((parsers.backslash  -- even backslash
                                      * parsers.backslash)^1, "backslashes")
                                 + (parsers.backslash
                                   * (#parsers.percent
                                     * Cb("backslashes")
                                     / function(backslashes)
                                       return string.rep("\\", #backslashes / 2)
                                     end
                                     * C(parsers.percent)
                                     + #parsers.commented_line_letter
                                     * Cb("backslashes")
                                     * Cc("\\")
                                     * C(parsers.commented_line_letter))
                                   * Cg(Cc(""), "backslashes")))^0
                               * (#parsers.percent
                                 * Cb("backslashes")
                                 / function(backslashes)
                                   return string.rep("\\", #backslashes / 2)
                                 end
                                 * ((parsers.percent  -- comment
                                    * parsers.line
                                    * #parsers.blankline) -- blank line
                                   / "\n"
                                   + parsers.percent  -- comment
                                   * parsers.line
                                   * parsers.optionalspace)  -- leading tabs and spaces
                                 + #(parsers.newline)
                                 * Cb("backslashes")
                                 * C(parsers.newline))

parsers.chunk                  = parsers.line * (parsers.optionallyindentedline
                                                - parsers.blankline)^0

parsers.attribute_key_char     = parsers.alphanumeric + S("-_:.")
parsers.attribute_raw_char     = parsers.alphanumeric + S("-_")
parsers.attribute_key          = (parsers.attribute_key_char
                                 - parsers.dash - parsers.digit)
                               * parsers.attribute_key_char^0
parsers.attribute_value        = ( (parsers.dquote / "")
                                 * (parsers.anyescaped - parsers.dquote)^0
                                 * (parsers.dquote / ""))
                               + ( (parsers.squote / "")
                                 * (parsers.anyescaped - parsers.squote)^0
                                 * (parsers.squote / ""))
                               + ( parsers.anyescaped - parsers.dquote - parsers.rbrace
                                 - parsers.space)^0
parsers.attribute_identifier   = parsers.attribute_key_char^1
parsers.attribute_classname    = parsers.letter
                               * parsers.attribute_key_char^0
parsers.attribute_raw          = parsers.attribute_raw_char^1

parsers.attribute = (parsers.dash * Cc(".unnumbered"))
                  + C( parsers.hash
                     * parsers.attribute_identifier)
                  + C( parsers.period
                     * parsers.attribute_classname)
                  + Cs( parsers.attribute_key
                      * parsers.optionalspace * parsers.equal * parsers.optionalspace
                      * parsers.attribute_value)
parsers.attributes = parsers.lbrace
                   * parsers.optionalspace
                   * parsers.attribute
                   * (parsers.spacechar^1
                     * parsers.attribute)^0
                   * parsers.optionalspace
                   * parsers.rbrace


parsers.raw_attribute = parsers.lbrace
                      * parsers.optionalspace
                      * parsers.equal
                      * C(parsers.attribute_raw)
                      * parsers.optionalspace
                      * parsers.rbrace

-- block followed by 0 or more optionally
-- indented blocks with first line indented.
parsers.indented_blocks = function(bl)
  return Cs( bl
         * (parsers.blankline^1 * parsers.indent * -parsers.blankline * bl)^0
         * (parsers.blankline^1 + parsers.eof) )
end
%    \end{macrocode}
% \par
% \begin{markdown}
%
%#### Parsers Used for HTML Entities
%
% \end{markdown}
%  \begin{macrocode}
local function repeat_between(pattern, min, max)
  return -pattern^(max + 1) * pattern^min
end

parsers.hexentity = parsers.ampersand * parsers.hash * C(S("Xx"))
                  * C(repeat_between(parsers.hexdigit, 1, 6)) * parsers.semicolon
parsers.decentity = parsers.ampersand * parsers.hash
                  * C(repeat_between(parsers.digit, 1, 7)) * parsers.semicolon
parsers.tagentity = parsers.ampersand * C(parsers.alphanumeric^1)
                  * parsers.semicolon

parsers.html_entities = parsers.hexentity / entities.hex_entity_with_x_char
                      + parsers.decentity / entities.dec_entity
                      + parsers.tagentity / entities.char_entity
%    \end{macrocode}
% \par
% \begin{markdown}
%
%#### Parsers Used for Markdown Lists
%
% \end{markdown}
%  \begin{macrocode}
parsers.bullet = function(bullet_char, interrupting)
  local allowed_end
  if interrupting then
    allowed_end = C(parsers.spacechar^1) * #parsers.linechar
  else
    allowed_end = C(parsers.spacechar^1) + #(parsers.newline + parsers.eof)
  end
  return parsers.check_trail
        * Ct(C(bullet_char) * Cc(""))
        * allowed_end
end

local function tickbox(interior)
  return parsers.optionalspace * parsers.lbracket
       * interior * parsers.rbracket * parsers.spacechar^1
end

parsers.ticked_box = tickbox(S("xX")) * Cc(1.0)
parsers.halfticked_box = tickbox(S("./")) * Cc(0.5)
parsers.unticked_box = tickbox(parsers.spacechar^1) * Cc(0.0)

%    \end{macrocode}
% \par
% \begin{markdown}
%
%#### Parsers Used for Markdown Code Spans
%
% \end{markdown}
%  \begin{macrocode}
parsers.openticks   = Cg(parsers.backtick^1, "ticks")

local function captures_equal_length(_,i,a,b)
  return #a == #b and i
end

parsers.closeticks  = Cmt(C(parsers.backtick^1)
                          * Cb("ticks"), captures_equal_length)

parsers.intickschar = (parsers.any - S("\n\r`"))
                    + V("NoSoftLineBreakEndline")
                    + (parsers.backtick^1 - parsers.closeticks)

local function process_inticks(s)
  s = s:gsub("\n", " ")
  s = s:gsub("^ (.*) $", "%1")
  return s
end

parsers.inticks = parsers.openticks
                * C(parsers.space^0)
                * parsers.closeticks
                + parsers.openticks
                * Cs(Cs(parsers.intickschar^0) / process_inticks)
                * parsers.closeticks

%    \end{macrocode}
% \par
% \begin{markdown}
%
%#### Parsers Used for HTML
%
% \end{markdown}
%  \begin{macrocode}
-- case-insensitive match (we assume s is lowercase). must be single byte encoding
parsers.keyword_exact = function(s)
  local parser = P(0)
  for i=1,#s do
    local c = s:sub(i,i)
    local m = c .. upper(c)
    parser = parser * S(m)
  end
  return parser
end

parsers.special_block_keyword =
    parsers.keyword_exact("pre") +
    parsers.keyword_exact("script") +
    parsers.keyword_exact("style") +
    parsers.keyword_exact("textarea")

parsers.block_keyword =
    parsers.keyword_exact("address") +
    parsers.keyword_exact("article") +
    parsers.keyword_exact("aside") +
    parsers.keyword_exact("base") +
    parsers.keyword_exact("basefont") +
    parsers.keyword_exact("blockquote") +
    parsers.keyword_exact("body") +
    parsers.keyword_exact("caption") +
    parsers.keyword_exact("center") +
    parsers.keyword_exact("col") +
    parsers.keyword_exact("colgroup") +
    parsers.keyword_exact("dd") +
    parsers.keyword_exact("details") +
    parsers.keyword_exact("dialog") +
    parsers.keyword_exact("dir") +
    parsers.keyword_exact("div") +
    parsers.keyword_exact("dl") +
    parsers.keyword_exact("dt") +
    parsers.keyword_exact("fieldset") +
    parsers.keyword_exact("figcaption") +
    parsers.keyword_exact("figure") +
    parsers.keyword_exact("footer") +
    parsers.keyword_exact("form") +
    parsers.keyword_exact("frame") +
    parsers.keyword_exact("frameset") +
    parsers.keyword_exact("h1") +
    parsers.keyword_exact("h2") +
    parsers.keyword_exact("h3") +
    parsers.keyword_exact("h4") +
    parsers.keyword_exact("h5") +
    parsers.keyword_exact("h6") +
    parsers.keyword_exact("head") +
    parsers.keyword_exact("header") +
    parsers.keyword_exact("hr") +
    parsers.keyword_exact("html") +
    parsers.keyword_exact("iframe") +
    parsers.keyword_exact("legend") +
    parsers.keyword_exact("li") +
    parsers.keyword_exact("link") +
    parsers.keyword_exact("main") +
    parsers.keyword_exact("menu") +
    parsers.keyword_exact("menuitem") +
    parsers.keyword_exact("nav") +
    parsers.keyword_exact("noframes") +
    parsers.keyword_exact("ol") +
    parsers.keyword_exact("optgroup") +
    parsers.keyword_exact("option") +
    parsers.keyword_exact("p") +
    parsers.keyword_exact("param") +
    parsers.keyword_exact("section") +
    parsers.keyword_exact("source") +
    parsers.keyword_exact("summary") +
    parsers.keyword_exact("table") +
    parsers.keyword_exact("tbody") +
    parsers.keyword_exact("td") +
    parsers.keyword_exact("tfoot") +
    parsers.keyword_exact("th") +
    parsers.keyword_exact("thead") +
    parsers.keyword_exact("title") +
    parsers.keyword_exact("tr") +
    parsers.keyword_exact("track") +
    parsers.keyword_exact("ul")

-- end conditions
parsers.html_blankline_end_condition  = parsers.linechar^0
                                      * ( parsers.newline
                                        * (parsers.check_minimal_blank_indent_and_any_trail
                                          * #parsers.blankline
                                          + parsers.check_minimal_indent_and_any_trail)
                                        * parsers.linechar^1)^0
                                      * (parsers.newline^-1 / "")

local function remove_trailing_blank_lines(s)
  return s:gsub("[\n\r]+%s*$", "")
end

parsers.html_until_end = function(end_marker)
  return Cs(Cs((parsers.newline
          * (parsers.check_minimal_blank_indent_and_any_trail
            * #parsers.blankline
            + parsers.check_minimal_indent_and_any_trail)
          + parsers.linechar - end_marker)^0
          * parsers.linechar^0 * parsers.newline^-1)
         / remove_trailing_blank_lines)
end

-- attributes
parsers.html_attribute_spacing  = parsers.optionalspace
                                * V("NoSoftLineBreakEndline")
                                * parsers.optionalspace
                                + parsers.spacechar^1

parsers.html_attribute_name = (parsers.letter + parsers.colon + parsers.underscore)
                            * (parsers.alphanumeric + parsers.colon + parsers.underscore
                            + parsers.period + parsers.dash)^0

parsers.html_attribute_value  = parsers.squote
                              * (parsers.linechar - parsers.squote)^0
                              * parsers.squote
                              + parsers.dquote
                              * (parsers.linechar - parsers.dquote)^0
                              * parsers.dquote
                              + ( parsers.any - parsers.spacechar - parsers.newline
                                - parsers.dquote - parsers.squote - parsers.backtick
                                - parsers.equal - parsers.less - parsers.more)^1

parsers.html_inline_attribute_value = parsers.squote
                                    * (V("NoSoftLineBreakEndline")
                                      + parsers.any
                                      - parsers.blankline^2
                                      - parsers.squote)^0
                                    * parsers.squote
                                    + parsers.dquote
                                    * (V("NoSoftLineBreakEndline")
                                      + parsers.any
                                      - parsers.blankline^2
                                      - parsers.dquote)^0
                                    * parsers.dquote
                                    + (parsers.any - parsers.spacechar - parsers.newline
                                      - parsers.dquote - parsers.squote - parsers.backtick
                                      - parsers.equal - parsers.less - parsers.more)^1

parsers.html_attribute_value_specification  = parsers.optionalspace
                                            * parsers.equal
                                            * parsers.optionalspace
                                            * parsers.html_attribute_value

parsers.html_spnl = parsers.optionalspace
                  * (V("NoSoftLineBreakEndline") * parsers.optionalspace)^-1

parsers.html_inline_attribute_value_specification = parsers.html_spnl
                                                  * parsers.equal
                                                  * parsers.html_spnl
                                                  * parsers.html_inline_attribute_value

parsers.html_attribute  = parsers.html_attribute_spacing
                        * parsers.html_attribute_name
                        * parsers.html_inline_attribute_value_specification^-1

parsers.html_non_newline_attribute  = parsers.spacechar^1
                                    * parsers.html_attribute_name
                                    * parsers.html_attribute_value_specification^-1

parsers.nested_breaking_blank = parsers.newline
                              * parsers.check_minimal_blank_indent
                              * parsers.blankline

parsers.html_comment_start = P("<!--")

parsers.html_comment_end = P("-->")

parsers.html_comment = Cs( parsers.html_comment_start
                         * parsers.html_until_end(parsers.html_comment_end))

parsers.html_inline_comment = (parsers.html_comment_start / "")
                            * -P(">") * -P("->")
                            * Cs((V("NoSoftLineBreakEndline") + parsers.any
                                - parsers.nested_breaking_blank - parsers.html_comment_end)^0)
                            * (parsers.html_comment_end / "")

parsers.html_cdatasection_start = P("<![CDATA[")

parsers.html_cdatasection_end = P("]]>")

parsers.html_cdatasection = Cs( parsers.html_cdatasection_start
                              * parsers.html_until_end(parsers.html_cdatasection_end))

parsers.html_inline_cdatasection  = parsers.html_cdatasection_start
                                  * Cs(V("NoSoftLineBreakEndline") + parsers.any
                                      - parsers.nested_breaking_blank - parsers.html_cdatasection_end)^0
                                  * parsers.html_cdatasection_end

parsers.html_declaration_start = P("<!") * parsers.letter

parsers.html_declaration_end = P(">")

parsers.html_declaration  = Cs( parsers.html_declaration_start
                              * parsers.html_until_end(parsers.html_declaration_end))

parsers.html_inline_declaration = parsers.html_declaration_start
                                * Cs(V("NoSoftLineBreakEndline") + parsers.any
                                    - parsers.nested_breaking_blank - parsers.html_declaration_end)^0
                                * parsers.html_declaration_end

parsers.html_instruction_start = P("<?")

parsers.html_instruction_end = P("?>")

parsers.html_instruction  = Cs( parsers.html_instruction_start
                              * parsers.html_until_end(parsers.html_instruction_end))

parsers.html_inline_instruction = parsers.html_instruction_start
                                * Cs(V("NoSoftLineBreakEndline") + parsers.any
                                    - parsers.nested_breaking_blank - parsers.html_instruction_end)^0
                                * parsers.html_instruction_end

parsers.html_blankline  = parsers.newline
                        * parsers.optionalspace
                        * parsers.newline

parsers.html_tag_start = parsers.less

parsers.html_tag_closing_start  = parsers.less
                                * parsers.slash

parsers.html_tag_end  = parsers.html_spnl
                      * parsers.more

parsers.html_empty_tag_end  = parsers.html_spnl
                            * parsers.slash
                            * parsers.more

-- opening tags
parsers.html_any_open_inline_tag  = parsers.html_tag_start
                                  * parsers.keyword
                                  * parsers.html_attribute^0
                                  * parsers.html_tag_end

parsers.html_any_open_tag = parsers.html_tag_start
                          * parsers.keyword
                          * parsers.html_non_newline_attribute^0
                          * parsers.html_tag_end

parsers.html_open_tag = parsers.html_tag_start
                      * parsers.block_keyword
                      * parsers.html_attribute^0
                      * parsers.html_tag_end

parsers.html_open_special_tag = parsers.html_tag_start
                              * parsers.special_block_keyword
                              * parsers.html_attribute^0
                              * parsers.html_tag_end

-- incomplete tags
parsers.incomplete_tag_following  = parsers.spacechar
                                  + parsers.more
                                  + parsers.slash * parsers.more
                                  + #(parsers.newline + parsers.eof)

parsers.incomplete_special_tag_following  = parsers.spacechar
                                          + parsers.more
                                          + #(parsers.newline + parsers.eof)

parsers.html_incomplete_open_tag  = parsers.html_tag_start
                                  * parsers.block_keyword
                                  * parsers.incomplete_tag_following

parsers.html_incomplete_open_special_tag  = parsers.html_tag_start
                                          * parsers.special_block_keyword
                                          * parsers.incomplete_special_tag_following

parsers.html_incomplete_close_tag = parsers.html_tag_closing_start
                                  * parsers.block_keyword
                                  * parsers.incomplete_tag_following

parsers.html_incomplete_close_special_tag = parsers.html_tag_closing_start
                                          * parsers.special_block_keyword
                                          * parsers.incomplete_tag_following

-- closing tags
parsers.html_close_tag  = parsers.html_tag_closing_start
                        * parsers.block_keyword
                        * parsers.html_tag_end

parsers.html_any_close_tag  = parsers.html_tag_closing_start
                            * parsers.keyword
                            * parsers.html_tag_end

parsers.html_close_special_tag = parsers.html_tag_closing_start
                               * parsers.special_block_keyword
                               * parsers.html_tag_end

-- empty tags
parsers.html_any_empty_inline_tag = parsers.html_tag_start
                                  * parsers.keyword
                                  * parsers.html_attribute^0
                                  * parsers.html_empty_tag_end

parsers.html_any_empty_tag  = parsers.html_tag_start
                            * parsers.keyword
                            * parsers.html_non_newline_attribute^0
                            * parsers.optionalspace
                            * parsers.slash
                            * parsers.more

parsers.html_empty_tag  = parsers.html_tag_start
                        * parsers.block_keyword
                        * parsers.html_attribute^0
                        * parsers.html_empty_tag_end

parsers.html_empty_special_tag  = parsers.html_tag_start
                                * parsers.special_block_keyword
                                * parsers.html_attribute^0
                                * parsers.html_empty_tag_end

parsers.html_incomplete_blocks  = parsers.html_incomplete_open_tag
                                + parsers.html_incomplete_open_special_tag
                                + parsers.html_incomplete_close_tag

-- parse special html blocks
parsers.html_blankline_ending_special_block_opening = (parsers.html_close_special_tag
                                                      + parsers.html_empty_special_tag)
                                                    * #(parsers.optionalspace
                                                       * (parsers.newline + parsers.eof))

parsers.html_blankline_ending_special_block = parsers.html_blankline_ending_special_block_opening
                                            * parsers.html_blankline_end_condition

parsers.html_special_block_opening  = parsers.html_incomplete_open_special_tag
                                    - parsers.html_empty_special_tag

parsers.html_closing_special_block  = parsers.html_special_block_opening
                                    * parsers.html_until_end(parsers.html_close_special_tag)

parsers.html_special_block  = parsers.html_blankline_ending_special_block
                            + parsers.html_closing_special_block

-- parse html blocks
parsers.html_block_opening  = parsers.html_incomplete_open_tag
                            + parsers.html_incomplete_close_tag

parsers.html_block  = parsers.html_block_opening
                    * parsers.html_blankline_end_condition

-- parse any html blocks
parsers.html_any_block_opening  = (parsers.html_any_open_tag
                                + parsers.html_any_close_tag
                                + parsers.html_any_empty_tag)
                                * #(parsers.optionalspace * (parsers.newline + parsers.eof))

parsers.html_any_block  = parsers.html_any_block_opening
                        * parsers.html_blankline_end_condition

parsers.html_inline_comment_full  = parsers.html_comment_start
                                  * -P(">") * -P("->")
                                  * Cs((V("NoSoftLineBreakEndline") + parsers.any - P("--")
                                      - parsers.nested_breaking_blank - parsers.html_comment_end)^0)
                                  * parsers.html_comment_end

parsers.html_inline_tags  = parsers.html_inline_comment_full
                          + parsers.html_any_empty_inline_tag
                          + parsers.html_inline_instruction
                          + parsers.html_inline_cdatasection
                          + parsers.html_inline_declaration
                          + parsers.html_any_open_inline_tag
                          + parsers.html_any_close_tag

%    \end{macrocode}
% \par
% \begin{markdown}
%
%#### Parsers Used for Markdown Tags and Links
%
% \end{markdown}
%  \begin{macrocode}
parsers.urlchar = parsers.anyescaped
                - parsers.newline
                - parsers.more

parsers.auto_link_scheme_part = parsers.alphanumeric
                              + parsers.plus
                              + parsers.period
                              + parsers.dash

parsers.auto_link_scheme  = parsers.letter
                          * parsers.auto_link_scheme_part
                          * parsers.auto_link_scheme_part^-30

parsers.absolute_uri  = parsers.auto_link_scheme * parsers.colon
                      * (parsers.any - parsers.spacing - parsers.less - parsers.more)^0

parsers.printable_characters = S(".!#$%&'*+/=?^_`{|}~-")

parsers.email_address_local_part_char = parsers.alphanumeric
                                      + parsers.printable_characters

parsers.email_address_local_part = parsers.email_address_local_part_char^1

parsers.email_address_dns_label = parsers.alphanumeric
                                * (parsers.alphanumeric + parsers.dash)^-62
                                * B(parsers.alphanumeric)

parsers.email_address_domain  = parsers.email_address_dns_label
                              * (parsers.period * parsers.email_address_dns_label)^0

parsers.email_address = parsers.email_address_local_part
                      * parsers.at
                      * parsers.email_address_domain

parsers.auto_link_url = parsers.less
                      * C(parsers.absolute_uri)
                      * parsers.more

parsers.auto_link_email = parsers.less
                        * C(parsers.email_address)
                        * parsers.more

parsers.auto_link_relative_reference = parsers.less
                                    * C(parsers.urlchar^1)
                                    * parsers.more

parsers.autolink  = parsers.auto_link_url
                  + parsers.auto_link_email

-- content in balanced brackets, parentheses, or quotes:
parsers.bracketed   = P{ parsers.lbracket
                       * (( parsers.backslash / "" * parsers.rbracket
                          + parsers.any - (parsers.lbracket
                                          + parsers.rbracket
                                          + parsers.blankline^2)
                          ) + V(1))^0
                       * parsers.rbracket }

parsers.inparens    = P{ parsers.lparent
                       * ((parsers.anyescaped - (parsers.lparent
                                                + parsers.rparent
                                                + parsers.blankline^2)
                          ) + V(1))^0
                       * parsers.rparent }

parsers.squoted     = P{ parsers.squote * parsers.alphanumeric
                       * ((parsers.anyescaped - (parsers.squote
                                                + parsers.blankline^2)
                          ) + V(1))^0
                       * parsers.squote }

parsers.dquoted     = P{ parsers.dquote * parsers.alphanumeric
                       * ((parsers.anyescaped - (parsers.dquote
                                                + parsers.blankline^2)
                          ) + V(1))^0
                       * parsers.dquote }

parsers.link_text  = parsers.lbracket
                    * Cs((parsers.alphanumeric^1
                        + parsers.bracketed
                        + parsers.inticks
                        + parsers.autolink
                        + V("InlineHtml")
                        + ( parsers.backslash * parsers.backslash)
                        + ( parsers.backslash * (parsers.lbracket + parsers.rbracket)
                          + V("NoSoftLineBreakSpace")
                          + V("NoSoftLineBreakEndline")
                          + (parsers.any
                            - (parsers.newline + parsers.lbracket + parsers.rbracket + parsers.blankline^2))))^0)
                    * parsers.rbracket

parsers.link_label_body = -#(parsers.sp * parsers.rbracket)
                        * #((parsers.any - parsers.rbracket)^-999 * parsers.rbracket)
                        * Cs((parsers.alphanumeric^1
                            + parsers.inticks
                            + parsers.autolink
                            + V("InlineHtml")
                            + ( parsers.backslash * parsers.backslash)
                            + ( parsers.backslash * (parsers.lbracket + parsers.rbracket)
                              + V("NoSoftLineBreakSpace")
                              + V("NoSoftLineBreakEndline")
                              + (parsers.any
                                - (parsers.newline + parsers.lbracket + parsers.rbracket + parsers.blankline^2))))^1)

parsers.link_label  = parsers.lbracket
                    * parsers.link_label_body
                    * parsers.rbracket

parsers.inparens_url  = P{ parsers.lparent
                         * ((parsers.anyescaped - (parsers.lparent
                                                  + parsers.rparent
                                                  + parsers.spacing)
                            ) + V(1))^0
                         * parsers.rparent }

-- url for markdown links, allowing nested brackets:
parsers.url         = parsers.less * Cs((parsers.anyescaped
                                        - parsers.newline
                                        - parsers.less
                                        - parsers.more)^0)
                                   * parsers.more
                    + -parsers.less
                    * Cs((parsers.inparens_url + (parsers.anyescaped
                                                 - parsers.spacing
                                                 - parsers.lparent
                                                 - parsers.rparent))^1)

-- quoted text:
parsers.title_s     = parsers.squote
                    * Cs((parsers.html_entities
                         + V("NoSoftLineBreakSpace")
                         + V("NoSoftLineBreakEndline")
                         + (parsers.anyescaped - parsers.newline - parsers.squote - parsers.blankline^2))^0)
                    * parsers.squote

parsers.title_d     = parsers.dquote
                    * Cs((parsers.html_entities
                         + V("NoSoftLineBreakSpace")
                         + V("NoSoftLineBreakEndline")
                         + (parsers.anyescaped - parsers.newline - parsers.dquote - parsers.blankline^2))^0)
                    * parsers.dquote

parsers.title_p     = parsers.lparent
                    * Cs((parsers.html_entities
                         + V("NoSoftLineBreakSpace")
                         + V("NoSoftLineBreakEndline")
                         + (parsers.anyescaped - parsers.newline - parsers.lparent - parsers.rparent
                           - parsers.blankline^2))^0)
                    * parsers.rparent

parsers.title       = parsers.title_d + parsers.title_s + parsers.title_p

parsers.optionaltitle
                    = parsers.spnlc * parsers.title * parsers.spacechar^0
                    + Cc("")

%    \end{macrocode}
% \par
% \begin{markdown}
%
%#### Helpers for Links and Link Reference Definitions
%
% \end{markdown}
%  \begin{macrocode}
-- parse a reference definition:  [foo]: /bar "title"
parsers.define_reference_parser = (parsers.check_trail / "") * parsers.link_label * parsers.colon
                                * parsers.spnlc * parsers.url
                                * ( parsers.spnlc_sep * parsers.title * parsers.only_blank
                                  + Cc("") * parsers.only_blank)
%    \end{macrocode}
% \par
% \begin{markdown}
%
%#### Inline Elements
%
% \end{markdown}
%  \begin{macrocode}
parsers.Inline         = V("Inline")

-- parse many p between starter and ender
parsers.between = function(p, starter, ender)
  local ender2 = B(parsers.nonspacechar) * ender
  return (starter * #parsers.nonspacechar * Ct(p * (p - ender2)^0) * ender2)
end

%    \end{macrocode}
% \par
% \begin{markdown}
%
%#### Block Elements
%
% \end{markdown}
%  \begin{macrocode}
parsers.lineof = function(c)
    return (parsers.check_trail_no_rem * (P(c) * parsers.optionalspace)^3
           * (parsers.newline + parsers.eof))
end

parsers.thematic_break_lines = parsers.lineof(parsers.asterisk)
                             + parsers.lineof(parsers.dash)
                             + parsers.lineof(parsers.underscore)
%    \end{macrocode}
% \par
% \begin{markdown}
%
%#### Headings
%
% \end{markdown}
%  \begin{macrocode}
-- parse Atx heading start and return level
parsers.heading_start = #parsers.hash * C(parsers.hash^-6)
                      * -parsers.hash / length

-- parse setext header ending and return level
parsers.heading_level = parsers.nonindentspace * parsers.equal^1 * parsers.optionalspace * #parsers.newline * Cc(1)
                      + parsers.nonindentspace * parsers.dash^1 * parsers.optionalspace * #parsers.newline * Cc(2)

local function strip_atx_end(s)
  return s:gsub("%s+#*%s*\n$","")
end

parsers.atx_heading = parsers.check_trail_no_rem
                    * Cg(parsers.heading_start, "level")
                    * (C( parsers.optionalspace
                        * parsers.hash^0
                        * parsers.optionalspace
                        * parsers.newline)
                      + parsers.spacechar^1
                      * C(parsers.line))
%    \end{macrocode}
% \par
% \begin{markdown}
%
%### Markdown Reader {#markdown-reader}
%
% This section documents the \luamref{reader} object, which implements the
% routines for parsing the markdown input. The object corresponds to the
% markdown reader object that was located in the
% `lunamark/reader/markdown.lua` file in the Lunamark Lua module.
%
% The \luamdef{reader.new} method creates and returns a new \TeX{} reader
% object associated with the Lua interface options (see Section
% <#sec:lua-options>) `options` and with a writer object `writer`. When
% `options` are unspecified, it is assumed that an empty table was passed to
% the method.
%
% The objects produced by the \luamref{reader.new} method expose instance methods
% and variables of their own. As a convention, I will refer to these
% \meta{member}s as `reader->`\meta{member}.
%
% \end{markdown}
%  \begin{macrocode}
M.reader = {}
function M.reader.new(writer, options)
  local self = {}
%    \end{macrocode}
% \par
% \begin{markdown}
%
% Make the `writer` and `options` parameters available as
% \luamdef{reader->writer} and \luamdef{reader->options}, respectively, so
% that they are accessible from extensions.
%
% \end{markdown}
%  \begin{macrocode}
  self.writer = writer
  self.options = options
%    \end{macrocode}
% \par
% \begin{markdown}
%
% Create a \luamdef{reader->parsers} hash table that stores \acro{peg} patterns
% that depend on the received `options`. Make \luamref{reader->parsers} inherit
% from the global \luamref{parsers} table.
%
% \end{markdown}
%  \begin{macrocode}
  self.parsers = {}
  (function(parsers)
    setmetatable(self.parsers, {
      __index = function (_, key)
        return parsers[key]
      end
    })
  end)(parsers)
%    \end{macrocode}
% \begin{markdown}
%
% Make \luamref{reader->parsers} available as a local `parsers` variable that
% will shadow the global \luamref{parsers} table and will make
% \luamref{reader->parsers} easier to type in the rest of the reader code.
%
% \end{markdown}
%  \begin{macrocode}
  local parsers = self.parsers
%    \end{macrocode}
% \par
% \begin{markdown}
%
%#### Top-Level Helper Functions
% Define \luamdef{reader->normalize_tag} as a function that normalizes a
% markdown reference tag by lowercasing it, and by collapsing any adjacent
% whitespace characters.
%
% \end{markdown}
%  \begin{macrocode}
  function self.normalize_tag(tag)
    tag = util.rope_to_string(tag)
    tag = tag:gsub("[ \n\r\t]+", " ")
    tag = tag:gsub("^ ", ""):gsub(" $", "")
    tag = uni_algos.case.casefold(tag, true, false)
    return tag
  end
%    \end{macrocode}
% \par
% \begin{markdown}
%
% Define \luamdef{iterlines} as a function that iterates over the lines of
% the input string `s`, transforms them using an input function `f`, and
% reassembles them into a new string, which it returns.
%
% \end{markdown}
%  \begin{macrocode}
  local function iterlines(s, f)
    local rope = lpeg.match(Ct((parsers.line / f)^1), s)
    return util.rope_to_string(rope)
  end
%    \end{macrocode}
% \par
% \begin{markdown}
%
% Define \luamdef{expandtabs} either as an identity function, when the
% \Opt{preserveTabs} Lua interface option is enabled, or to a function that
% expands tabs into spaces otherwise.
%
% \end{markdown}
%  \begin{macrocode}
  if options.preserveTabs then
    self.expandtabs = function(s) return s end
  else
    self.expandtabs = function(s)
                        if s:find("\t") then
                          return iterlines(s, util.expand_tabs_in_line)
                        else
                          return s
                        end
                      end
  end
%    \end{macrocode}
% \par
% \begin{markdown}
%
%#### High-Level Parser Functions
%
% Create a \luamdef{reader->parser_functions} hash table that stores high-level
% parser functions. Define \luamdef{reader->create_parser} as a function that
% will create a high-level parser function \luamdef{reader->parser_functions.name},
% that matches input using grammar `grammar`. If `toplevel` is true, the input
% is expected to come straight from the user, not from a recursive call, and
% will be preprocessed.
%
% \end{markdown}
%  \begin{macrocode}
  self.parser_functions = {}
  self.create_parser = function(name, grammar, toplevel)
    self.parser_functions[name] = function(str)
%    \end{macrocode}
% \par
% \begin{markdown}
%
% If the parser function is top-level and the \Opt{stripIndent} Lua option is
% enabled, we will first expand tabs in the input string `str` into spaces
% and then we will count the minimum indent across all lines, skipping
% blank lines. Next, we will remove the minimum indent from all lines.
%
% \end{markdown}
%  \begin{macrocode}
      if toplevel and options.stripIndent then
          local min_prefix_length, min_prefix = nil, ''
          str = iterlines(str, function(line)
              if lpeg.match(parsers.nonemptyline, line) == nil then
                  return line
              end
              line = util.expand_tabs_in_line(line)
              local prefix = lpeg.match(C(parsers.optionalspace), line)
              local prefix_length = #prefix
              local is_shorter = min_prefix_length == nil
              is_shorter = is_shorter or prefix_length < min_prefix_length
              if is_shorter then
                  min_prefix_length, min_prefix = prefix_length, prefix
              end
              return line
          end)
          str = str:gsub('^' .. min_prefix, '')
      end
%    \end{macrocode}
% \par
% \begin{markdown}
%
% If the parser is top-level and the \Opt{texComments} or \Opt{hybrid} Lua
% options are enabled, we will strip all plain \TeX{} comments from the input
% string `str` together with the trailing newline characters.
%
% \end{markdown}
%  \begin{macrocode}
      if toplevel and (options.texComments or options.hybrid) then
        str = lpeg.match(Ct(parsers.commented_line^1), str)
        str = util.rope_to_string(str)
      end
      local res = lpeg.match(grammar(), str)
      if res == nil then
        return writer.error(format("%s failed on:\n%s", name, str:sub(1,20)))
      else
        return res
      end
    end
  end

  self.create_parser("parse_blocks",
                     function()
                       return parsers.blocks
                     end, true)

  self.create_parser("parse_blocks_nested",
                     function()
                       return parsers.blocks_nested
                     end, false)

  self.create_parser("parse_inlines",
                     function()
                       return parsers.inlines
                     end, false)

  self.create_parser("parse_inlines_no_inline_note",
                     function()
                       return parsers.inlines_no_inline_note
                     end, false)

  self.create_parser("parse_inlines_no_html",
                     function()
                       return parsers.inlines_no_html
                     end, false)

  self.create_parser("parse_inlines_nbsp",
                     function()
                       return parsers.inlines_nbsp
                     end, false)
  self.create_parser("parse_inlines_no_link_or_emphasis",
                    function()
                      return parsers.inlines_no_link_or_emphasis
                    end, false)
%    \end{macrocode}
% \par
% \begin{markdown}
%
%#### Parsers Used for Indentation (local)
%
% The following patterns represent basic building blocks of indented content.
%
% \end{markdown}
%  \begin{macrocode}
  parsers.minimally_indented_blankline = parsers.check_minimal_indent * (parsers.blankline / "")

  parsers.minimally_indented_block = parsers.check_minimal_indent * V("Block")

  parsers.minimally_indented_block_or_paragraph = parsers.check_minimal_indent * V("BlockOrParagraph")

  parsers.minimally_indented_paragraph = parsers.check_minimal_indent * V("Paragraph")

  parsers.minimally_indented_plain = parsers.check_minimal_indent * V("Plain")

  parsers.minimally_indented_par_or_plain = parsers.minimally_indented_paragraph
                                          + parsers.minimally_indented_plain

  parsers.minimally_indented_par_or_plain_no_blank  = parsers.minimally_indented_par_or_plain
                                                    - parsers.minimally_indented_blankline

  parsers.minimally_indented_ref = parsers.check_minimal_indent * V("Reference")

  parsers.minimally_indented_blank = parsers.check_minimal_indent * V("Blank")

  parsers.conditionally_indented_blankline = parsers.check_minimal_blank_indent * (parsers.blankline / "")

  parsers.minimally_indented_ref_or_block = parsers.minimally_indented_ref
                                          + parsers.minimally_indented_block
                                          - parsers.minimally_indented_blankline

  parsers.minimally_indented_ref_or_block_or_par  = parsers.minimally_indented_ref
                                                  + parsers.minimally_indented_block_or_paragraph
                                                  - parsers.minimally_indented_blankline

%    \end{macrocode}
% \begin{markdown}
%
% The following pattern parses the properly indented content that follows the initial container start.
%
% \end{markdown}
%  \begin{macrocode}

  parsers.separator_loop = function(separated_block, paragraph, block_separator, paragraph_separator)
    return  separated_block
          + block_separator
            * paragraph
            * separated_block
          + paragraph_separator
          * paragraph
  end

  parsers.create_loop_body_pair = function(separated_block, paragraph, block_separator, paragraph_separator)
    return {
      block = parsers.separator_loop(separated_block, paragraph, block_separator, block_separator),
      par = parsers.separator_loop(separated_block, paragraph, block_separator, paragraph_separator)
    }
  end

  parsers.block_sep_group = function(blank)
    return  blank^0 * parsers.eof
          + ( blank^2 / writer.paragraphsep
            + blank^0 / writer.interblocksep
            )
  end

  parsers.par_sep_group = function(blank)
    return  blank^0 * parsers.eof
          + blank^0 / writer.paragraphsep
  end

  parsers.sep_group_no_output = function(blank)
    return  blank^0 * parsers.eof
          + blank^0
  end

  parsers.content_blank = parsers.minimally_indented_blankline

  parsers.ref_or_block_separated  = parsers.sep_group_no_output(parsers.content_blank)
                                  * ( parsers.minimally_indented_ref
                                    - parsers.content_blank)
                                  + parsers.block_sep_group(parsers.content_blank)
                                  * ( parsers.minimally_indented_block
                                    - parsers.content_blank)

  parsers.loop_body_pair  =
    parsers.create_loop_body_pair(parsers.ref_or_block_separated,
                                  parsers.minimally_indented_par_or_plain_no_blank,
                                  parsers.block_sep_group(parsers.content_blank),
                                  parsers.par_sep_group(parsers.content_blank))

  parsers.content_loop  = ( V("Block")
                          * parsers.loop_body_pair.block^0
                          + (V("Paragraph") + V("Plain"))
                          * parsers.ref_or_block_separated
                          * parsers.loop_body_pair.block^0
                          + (V("Paragraph") + V("Plain"))
                          * parsers.loop_body_pair.par^0)
                        * parsers.content_blank^0

  parsers.indented_content = function()
    return  Ct( (V("Reference") + (parsers.blankline / ""))
              * parsers.content_blank^0
              * parsers.check_minimal_indent
              * parsers.content_loop
              + (V("Reference") + (parsers.blankline / ""))
              * parsers.content_blank^0
              + parsers.content_loop)
  end

  parsers.add_indent = function(pattern, name, breakable)
    return  Cg(Cmt( Cb("indent_info")
                  * Ct(pattern)
                  * (#parsers.linechar * Cc(false) + Cc(true)) -- check if starter is blank
                  * Cc(name)
                  * Cc(breakable),
              process_starter_indent), "indent_info")
  end

%    \end{macrocode}
% \par
% \begin{markdown}
%
%#### Parsers Used for Markdown Lists (local)
%
% \end{markdown}
%  \begin{macrocode}
  if options.hashEnumerators then
    parsers.dig = parsers.digit + parsers.hash
  else
    parsers.dig = parsers.digit
  end

  parsers.enumerator = function(delimiter_type, interrupting)
    local delimiter_range
    local allowed_end
    if interrupting then
      delimiter_range = P("1")
      allowed_end = C(parsers.spacechar^1) * #parsers.linechar
    else
      delimiter_range = parsers.dig * parsers.dig^-8
      allowed_end = C(parsers.spacechar^1) + #(parsers.newline + parsers.eof)
    end

    return parsers.check_trail
             * Ct(C(delimiter_range) * C(delimiter_type))
             * allowed_end
  end

  parsers.starter = parsers.bullet(parsers.dash)
                  + parsers.bullet(parsers.asterisk)
                  + parsers.bullet(parsers.plus)
                  + parsers.enumerator(parsers.period)
                  + parsers.enumerator(parsers.rparent)

%    \end{macrocode}
% \par
% \begin{markdown}
%
%#### Parsers Used for Blockquotes (local)
%
% \end{markdown}
%  \begin{macrocode}
  parsers.blockquote_start = parsers.check_trail * C(parsers.more) * C(parsers.spacechar^0)

  parsers.blockquote_body = parsers.add_indent(parsers.blockquote_start, "bq", true)
                          * parsers.indented_content()
                          * remove_indent("bq")

  if not options.breakableBlockquotes then
    parsers.blockquote_body = parsers.add_indent(parsers.blockquote_start, "bq", false)
                            * parsers.indented_content()
                            * remove_indent("bq")
  end
%    \end{macrocode}
% \par
% \begin{markdown}
%
%#### Helpers for Emphasis and Strong Emphasis (local)
%
% Parse the content of a table `content_part` with links, images and emphasis disabled.
%
% \end{markdown}
%  \begin{macrocode}
  local function parse_content_part(content_part)
    local rope = util.rope_to_string(content_part)
    local parsed = self.parser_functions.parse_inlines_no_link_or_emphasis(rope)
    parsed.indent_info = nil
    return parsed
  end

%    \end{macrocode}
% \begin{markdown}
%
% Collect the content between the `opening_index` and `closing_index` in the delimiter table `t`.
%
% \end{markdown}
%  \begin{macrocode}
  local function collect_emphasis_content(t, opening_index, closing_index)
    local content = {}

    local content_part = {}
    for i = opening_index, closing_index do
      local value = t[i]

      if value.rendered ~= nil then
        content[#content + 1] = parse_content_part(content_part)
        content_part = {}
        content[#content + 1] = value.rendered
        value.rendered = nil
      else
        if value.type == "delimiter" and value.element == "emphasis" then
          if value.is_active then
            content_part[#content_part + 1] = string.rep(value.character, value.current_count)
          end
        else
          content_part[#content_part + 1] = value.content
        end
        value.content = ''
        value.is_active = false
      end
    end

    if next(content_part) ~= nil then
      content[#content + 1] = parse_content_part(content_part)
    end

    return content
  end

%    \end{macrocode}
% \begin{markdown}
%
% Render content between the `opening_index` and `closing_index` in the delimiter table `t`
% as emphasis.
%
% \end{markdown}
%  \begin{macrocode}
  local function fill_emph(t, opening_index, closing_index)
    local content = collect_emphasis_content(t, opening_index + 1, closing_index - 1)
    t[opening_index + 1].is_active = true
    t[opening_index + 1].rendered = writer.emphasis(content)
  end

%    \end{macrocode}
% \begin{markdown}
%
% Render content between the `opening_index` and `closing_index` in the delimiter table `t`
% as strong emphasis.
%
% \end{markdown}
%  \begin{macrocode}
  local function fill_strong(t, opening_index, closing_index)
    local content = collect_emphasis_content(t, opening_index + 1, closing_index - 1)
    t[opening_index + 1].is_active = true
    t[opening_index + 1].rendered = writer.strong(content)
  end

%    \end{macrocode}
% \begin{markdown}
%
% Check whether the opening delimiter `opening_delimiter` and closing delimiter `closing_delimiter` break
% rule three together.
%
% \end{markdown}
%  \begin{macrocode}
  local function breaks_three_rule(opening_delimiter, closing_delimiter)
    return (opening_delimiter.is_closing or closing_delimiter.is_opening) and
      ((opening_delimiter.original_count + closing_delimiter.original_count) % 3 == 0) and
      (opening_delimiter.original_count % 3 ~= 0 or closing_delimiter.original_count % 3 ~= 0)
  end

%    \end{macrocode}
% \begin{markdown}
%
% Look for the first potential emphasis opener in the delimiter table `t` in the range from
% `bottom_index` to `latest_index` that has the same character `character` as the closing
% delimiter `closing_delimiter`.
%
% \end{markdown}
%  \begin{macrocode}
  local function find_emphasis_opener(t, bottom_index, latest_index, character, closing_delimiter)
    for i = latest_index, bottom_index, -1 do
      local value = t[i]
      if value.is_active and
         value.is_opening and
         value.type == "delimiter" and
         value.element == "emphasis" and
         (value.character == character) and
         (value.current_count > 0) then
        if not breaks_three_rule(value, closing_delimiter) then
          return i
        end
      end
    end
  end

%    \end{macrocode}
% \begin{markdown}
%
% Iterate over the delimiters in the delimiter table `t`, producing emphasis or strong emphasis macros.
%
% \end{markdown}
%  \begin{macrocode}
  local function process_emphasis(t, opening_index, closing_index)
    for i = opening_index, closing_index do
      local value = t[i]
      if value.type == "delimiter" and value.element == "emphasis" then
          local delimiter_length = string.len(value.content)
          value.character = string.sub(value.content, 1, 1)
          value.current_count = delimiter_length
          value.original_count = delimiter_length
      end
    end

    local openers_bottom = {
      ['*'] = {
        [true] = {opening_index, opening_index, opening_index},
        [false] = {opening_index, opening_index, opening_index}
      },
      ['_'] = {
        [true] = {opening_index, opening_index, opening_index},
        [false] = {opening_index, opening_index, opening_index}
      }
    }

    local current_position = opening_index
    local max_position = closing_index

    while current_position <= max_position do
      local value = t[current_position]

      if value.type ~= "delimiter" or
        value.element ~= "emphasis" or
        not value.is_active or
        not value.is_closing or
        (value.current_count <= 0) then
        current_position = current_position + 1
        goto continue
      end

      local character = value.character
      local is_opening = value.is_opening
      local closing_length_modulo_three = value.original_count % 3

      local current_openers_bottom = openers_bottom[character][is_opening][closing_length_modulo_three + 1]

      local opener_position = find_emphasis_opener(t, current_openers_bottom, current_position - 1, character, value)

      if (opener_position == nil) then
        openers_bottom[character][is_opening][closing_length_modulo_three + 1] = current_position
        current_position = current_position + 1
        goto continue
      end

      local opening_delimiter = t[opener_position]

      local current_opening_count = opening_delimiter.current_count
      local current_closing_count = t[current_position].current_count

      if (current_opening_count >= 2) and (current_closing_count >= 2) then
        opening_delimiter.current_count = current_opening_count - 2
        t[current_position].current_count = current_closing_count - 2
        fill_strong(t, opener_position, current_position)
      else
        opening_delimiter.current_count = current_opening_count - 1
        t[current_position].current_count = current_closing_count - 1
        fill_emph(t, opener_position, current_position)
      end

      ::continue::
    end
  end

  local cont = lpeg.R("\128\191") -- continuation byte

%    \end{macrocode}
% \begin{markdown}
%
% Match a UTF-8 character of byte length `n`.
%
% \end{markdown}
%  \begin{macrocode}
  local function utf8_by_byte_count(n)
    if (n == 1) then
      return lpeg.R("\0\127")
    end
    if (n == 2) then
      return lpeg.R("\194\223") * cont
    end
    if (n == 3) then
      return lpeg.R("\224\239") * cont * cont
    end
    if (n == 4) then
      return lpeg.R("\240\244") * cont * cont * cont
    end
  end
%    \end{macrocode}
% \begin{markdown}
%
% Check if a there is a character of a type `chartype` between the start position `start_pos`
% and end position `end_pos` in a string `s` relative to current index `i`.
%
% \end{markdown}
%  \begin{macrocode}
  local function check_unicode_type(s, i, start_pos, end_pos, chartype)
    local c
    local char_length
    for pos = start_pos, end_pos, 1 do
      if (start_pos < 0) then
        char_length = -pos
      else
        char_length = pos + 1
      end

      if (chartype == "punctuation") then
        if lpeg.match(parsers.punctuation[char_length], s, i+pos) then
          return i
        end
      else
        c = lpeg.match({ C(utf8_by_byte_count(char_length)) },s,i+pos)
        if (c ~= nil) and (unicode.utf8.match(c, chartype)) then
          return i
        end
      end
    end
  end

  local function check_preceding_unicode_punctuation(s, i)
    return check_unicode_type(s, i, -4, -1, "punctuation")
  end

  local function check_preceding_unicode_whitespace(s, i)
    return check_unicode_type(s, i, -4, -1, "%s")
  end

  local function check_following_unicode_punctuation(s, i)
    return check_unicode_type(s, i, 0, 3, "punctuation")
  end

  local function check_following_unicode_whitespace(s, i)
    return check_unicode_type(s, i, 0, 3, "%s")
  end

  parsers.unicode_preceding_punctuation = B(parsers.escapable)
                                        + Cmt(parsers.succeed, check_preceding_unicode_punctuation)

  parsers.unicode_preceding_whitespace = Cmt(parsers.succeed, check_preceding_unicode_whitespace)

  parsers.unicode_following_punctuation = #parsers.escapable
                                        + Cmt(parsers.succeed, check_following_unicode_punctuation)

  parsers.unicode_following_whitespace = Cmt(parsers.succeed, check_following_unicode_whitespace)

  parsers.delimiter_run = function(character)
    return  (B(parsers.backslash * character) + -B(character))
          * character^1
          * -#character
  end

  parsers.left_flanking_delimiter_run = function(character)
    return  (B( parsers.any)
              * (parsers.unicode_preceding_punctuation + parsers.unicode_preceding_whitespace)
             + -B(parsers.any))
            * parsers.delimiter_run(character)
            * parsers.unicode_following_punctuation
            + parsers.delimiter_run(character)
            * -#(parsers.unicode_following_punctuation + parsers.unicode_following_whitespace
                + parsers.eof)
  end

  parsers.right_flanking_delimiter_run = function(character)
    return  parsers.unicode_preceding_punctuation
          * parsers.delimiter_run(character)
          * (parsers.unicode_following_punctuation + parsers.unicode_following_whitespace
            + parsers.eof)
          + (B(parsers.any)
            * -(parsers.unicode_preceding_punctuation + parsers.unicode_preceding_whitespace))
          * parsers.delimiter_run(character)
  end

  if options.underscores then
    parsers.emph_start = parsers.left_flanking_delimiter_run(parsers.asterisk)
                      + (-#parsers.right_flanking_delimiter_run(parsers.underscore)
                          + (parsers.unicode_preceding_punctuation
                            * #parsers.right_flanking_delimiter_run(parsers.underscore)))
                      * parsers.left_flanking_delimiter_run(parsers.underscore)

    parsers.emph_end  = parsers.right_flanking_delimiter_run(parsers.asterisk)
                      + (-#parsers.left_flanking_delimiter_run(parsers.underscore)
                        + #(parsers.left_flanking_delimiter_run(parsers.underscore)
                          * parsers.unicode_following_punctuation))
                      * parsers.right_flanking_delimiter_run(parsers.underscore)
  else
    parsers.emph_start = parsers.left_flanking_delimiter_run(parsers.asterisk)

    parsers.emph_end  = parsers.right_flanking_delimiter_run(parsers.asterisk)
  end

  parsers.emph_capturing_open_and_close = #parsers.emph_start * #parsers.emph_end
                                        * Ct( Cg(Cc("delimiter"), "type")
                                            * Cg(Cc("emphasis"), "element")
                                            * Cg(C(parsers.emph_start), "content")
                                            * Cg(Cc(true), "is_opening")
                                            * Cg(Cc(true), "is_closing"))

  parsers.emph_capturing_open = Ct( Cg(Cc("delimiter"), "type")
                                  * Cg(Cc("emphasis"), "element")
                                  * Cg(C(parsers.emph_start), "content")
                                  * Cg(Cc(true), "is_opening")
                                  * Cg(Cc(false), "is_closing"))

  parsers.emph_capturing_close = Ct( Cg(Cc("delimiter"), "type")
                                   * Cg(Cc("emphasis"), "element")
                                   * Cg(C(parsers.emph_end), "content")
                                   * Cg(Cc(false), "is_opening")
                                   * Cg(Cc(true), "is_closing"))

  parsers.emph_open_or_close  = parsers.emph_capturing_open_and_close
                              + parsers.emph_capturing_open
                              + parsers.emph_capturing_close

  parsers.emph_open = parsers.emph_capturing_open_and_close
                    + parsers.emph_capturing_open

  parsers.emph_close  = parsers.emph_capturing_open_and_close
                      + parsers.emph_capturing_close

%    \end{macrocode}
% \par
% \begin{markdown}
%
%#### Helpers for Links and Link Reference Definitions (local)
%
% \end{markdown}
%  \begin{macrocode}
  -- List of references defined in the document
  local references

  -- List of note references defined in the document
  parsers.rawnotes = {}

%    \end{macrocode}
% \par
% \begin{markdown}
%
% The \luamdef{reader->register_link} method registers
% a link reference, where `tag` is the link label, `url`
% is the link destination, `title` is the optional link
% title, and `attributes` are the optional attributes.
%
% \end{markdown}
%  \begin{macrocode}
  function self.register_link(_, tag, url, title,
                              attributes)
    local normalized_tag = self.normalize_tag(tag)
      if references[normalized_tag] == nil then
        references[normalized_tag] = {
          url = url,
          title = title,
          attributes = attributes
        }
      end
    return ""
  end

%    \end{macrocode}
% \par
% \begin{markdown}
%
% The \luamdef{reader->lookup_reference} method looks up a
% reference with link label `tag`.
%
% \end{markdown}
%  \begin{macrocode}
  function self.lookup_reference(tag)
    return references[self.normalize_tag(tag)]
  end

%    \end{macrocode}
% \par
% \begin{markdown}
%
% The \luamdef{reader->lookup_note_reference} method looks up a
% note reference with label `tag`.
%
% \end{markdown}
%  \begin{macrocode}
  function self.lookup_note_reference(tag)
    return parsers.rawnotes[self.normalize_tag(tag)]
  end

  parsers.title_s_direct_ref  = parsers.squote
                              * Cs((parsers.html_entities
                                  + (parsers.anyescaped - parsers.squote - parsers.blankline^2))^0)
                              * parsers.squote

  parsers.title_d_direct_ref  = parsers.dquote
                              * Cs((parsers.html_entities
                                  + (parsers.anyescaped - parsers.dquote - parsers.blankline^2))^0)
                              * parsers.dquote

  parsers.title_p_direct_ref  = parsers.lparent
                              * Cs((parsers.html_entities
                                  + (parsers.anyescaped - parsers.lparent - parsers.rparent - parsers.blankline^2))^0)
                              * parsers.rparent

  parsers.title_direct_ref  = parsers.title_s_direct_ref
                            + parsers.title_d_direct_ref
                            + parsers.title_p_direct_ref

  parsers.inline_direct_ref_inside  = parsers.lparent * parsers.spnl
                                    * Cg(parsers.url + Cc(""), "url")
                                    * parsers.spnl
                                    * Cg(parsers.title_direct_ref + Cc(""), "title")
                                    * parsers.spnl * parsers.rparent

  parsers.inline_direct_ref = parsers.lparent * parsers.spnlc
                            * Cg(parsers.url + Cc(""), "url")
                            * parsers.spnlc
                            * Cg(parsers.title + Cc(""), "title")
                            * parsers.spnlc * parsers.rparent

  parsers.empty_link  = parsers.lbracket
                      * parsers.rbracket

  parsers.inline_link = parsers.link_text
                      * parsers.inline_direct_ref

  parsers.full_link = parsers.link_text
                    * parsers.link_label

  parsers.shortcut_link = parsers.link_label
                        * -(parsers.empty_link + parsers.link_label)

  parsers.collapsed_link  = parsers.link_label
                          * parsers.empty_link

  parsers.image_opening = #(parsers.exclamation * parsers.inline_link)
                        * Cg(Cc("inline"), "link_type")
                        + #(parsers.exclamation * parsers.full_link)
                        * Cg(Cc("full"), "link_type")
                        + #(parsers.exclamation * parsers.collapsed_link)
                        * Cg(Cc("collapsed"), "link_type")
                        + #(parsers.exclamation * parsers.shortcut_link)
                        * Cg(Cc("shortcut"), "link_type")
                        + #(parsers.exclamation * parsers.empty_link)
                        * Cg(Cc("empty"), "link_type")

  parsers.link_opening  = #parsers.inline_link
                        * Cg(Cc("inline"), "link_type")
                        + #parsers.full_link
                        * Cg(Cc("full"), "link_type")
                        + #parsers.collapsed_link
                        * Cg(Cc("collapsed"), "link_type")
                        + #parsers.shortcut_link
                        * Cg(Cc("shortcut"), "link_type")
                        + #parsers.empty_link
                        * Cg(Cc("empty_link"), "link_type")
                        + #parsers.link_text
                        * Cg(Cc("link_text"), "link_type")

  parsers.note_opening  = #(parsers.circumflex * parsers.link_text)
                        * Cg(Cc("note_inline"), "link_type")

  parsers.raw_note_opening  = #( parsers.lbracket
                               * parsers.circumflex
                               * parsers.link_label_body
                               * parsers.rbracket)
                            * Cg(Cc("raw_note"), "link_type")

  local inline_note_element = Cg(Cc("note"), "element")
                            * parsers.note_opening
                            * Cg(parsers.circumflex * parsers.lbracket, "content")

  local image_element = Cg(Cc("image"), "element")
                      * parsers.image_opening
                      * Cg(parsers.exclamation * parsers.lbracket, "content")

  local note_element  = Cg(Cc("note"), "element")
                      * parsers.raw_note_opening
                      * Cg(parsers.lbracket * parsers.circumflex, "content")

  local link_element  = Cg(Cc("link"), "element")
                      * parsers.link_opening
                      * Cg(parsers.lbracket, "content")

  local opening_elements = parsers.fail

  if options.inlineNotes then
    opening_elements = opening_elements + inline_note_element
  end

  opening_elements = opening_elements + image_element

  if options.notes then
    opening_elements = opening_elements + note_element
  end

  opening_elements = opening_elements + link_element

  parsers.link_image_opening  = Ct( Cg(Cc("delimiter"), "type")
                                  * Cg(Cc(true), "is_opening")
                                  * Cg(Cc(false), "is_closing")
                                  * opening_elements)

  parsers.link_image_closing  = Ct( Cg(Cc("delimiter"), "type")
                                  * Cg(Cc("link"), "element")
                                  * Cg(Cc(false), "is_opening")
                                  * Cg(Cc(true), "is_closing")
                                  * ( Cg(Cc(true), "is_direct")
                                    * Cg(parsers.rbracket * #parsers.inline_direct_ref, "content")
                                    + Cg(Cc(false), "is_direct")
                                    * Cg(parsers.rbracket, "content")))

  parsers.link_image_open_or_close  = parsers.link_image_opening
                                    + parsers.link_image_closing

  if options.html then
    parsers.link_emph_precedence  = parsers.inticks
                                  + parsers.autolink
                                  + parsers.html_inline_tags
  else
    parsers.link_emph_precedence  = parsers.inticks
                                  + parsers.autolink
  end

  parsers.link_and_emph_endline = parsers.newline
                                * ((parsers.check_minimal_indent
                                  * -V("EndlineExceptions")
                                  + parsers.check_optional_indent
                                  * -V("EndlineExceptions")
                                  * -parsers.starter) / "")
                                * parsers.spacechar^0 / "\n"

  parsers.link_and_emph_content = Ct( Cg(Cc("content"), "type")
                                    * Cg(Cs(( parsers.link_emph_precedence
                                            + parsers.backslash * parsers.any
                                            + parsers.link_and_emph_endline
                                            + (parsers.linechar
                                              - parsers.blankline^2
                                              - parsers.link_image_open_or_close
                                              - parsers.emph_open_or_close))^0), "content"))

  parsers.link_and_emph_table = (parsers.link_image_opening + parsers.emph_open)
                              * parsers.link_and_emph_content
                              * ((parsers.link_image_open_or_close + parsers.emph_open_or_close)
                                * parsers.link_and_emph_content)^1

%    \end{macrocode}
% \begin{markdown}
%
% Collect the content between the `opening_index` and `closing_index` in the delimiter table `t`.
%
% \end{markdown}
%  \begin{macrocode}
  local function collect_link_content(t, opening_index, closing_index)
    local content = {}
    for i = opening_index, closing_index do
      content[#content + 1] = t[i].content
    end
    return util.rope_to_string(content)
  end

%    \end{macrocode}
% \begin{markdown}
%
% Look for the closest potential link opener in the delimiter table `t` in the range from
% `bottom_index` to `latest_index`.
%
% \end{markdown}
%  \begin{macrocode}
  local function find_link_opener(t, bottom_index, latest_index)
    for i = latest_index, bottom_index, -1 do
      local value = t[i]
      if value.type == "delimiter" and
         value.is_opening and
         (value.element == "link" or value.element == "image" or value.element == "note")
         and not value.removed then
        if value.is_active then
          return i
        end
        value.removed = true
        return nil
      end
    end
  end

%    \end{macrocode}
% \begin{markdown}
%
% Find the position of a delimiter that closes a full link after an an index `latest_index`
% in the delimiter table `t`.
%
% \end{markdown}
%  \begin{macrocode}
  local function find_next_link_closing_index(t, latest_index)
    for i = latest_index, #t do
      local value = t[i]
      if value.is_closing and
         value.element == "link" and
         not value.removed then
        return i
      end
    end
  end

%    \end{macrocode}
% \begin{markdown}
%
% Disable all preceding opening link delimiters by marking them inactive with the `is_active` property
% to prevent links within links. Images within links are allowed.
%
% \end{markdown}
%  \begin{macrocode}
  local function disable_previous_link_openers(t, opening_index)
    if t[opening_index].element == "image" then
      return
    end

    for i = opening_index, 1, -1 do
      local value = t[i]
      if value.is_active and
         value.type == "delimiter" and
         value.is_opening and
         value.element == "link" then
        value.is_active = false
      end
    end
  end

%    \end{macrocode}
% \begin{markdown}
%
% Disable the delimiters between the `opening_index` and `closing_index` in the delimiter table `t`
% by marking them inactive with the `is_active` property.
%
% \end{markdown}
%  \begin{macrocode}
  local function disable_range(t, opening_index, closing_index)
    for i = opening_index, closing_index do
      local value = t[i]
      if value.is_active then
        value.is_active = false
        if value.type == "delimiter" then
          value.removed = true
        end
      end
    end
  end

%    \end{macrocode}
% \begin{markdown}
%
% Clear the parsed content between the `opening_index` and `closing_index` in the delimiter table `t`.
%
% \end{markdown}
%  \begin{macrocode}
  local function delete_parsed_content_in_range(t, opening_index, closing_index)
    for i = opening_index, closing_index do
      t[i].rendered = nil
    end
  end

%    \end{macrocode}
% \begin{markdown}
%
% Clear the content between the `opening_index` and `closing_index` in the delimiter table `t`.
%
% \end{markdown}
%  \begin{macrocode}
  local function empty_content_in_range(t, opening_index, closing_index)
    for i = opening_index, closing_index do
      t[i].content = ''
    end
  end

%    \end{macrocode}
% \begin{markdown}
%
% Join the attributes from the link reference definition `reference_attributes` with the link's own
% attributes `own_attributes`.
%
% \end{markdown}
%  \begin{macrocode}
  local function join_attributes(reference_attributes, own_attributes)
    local merged_attributes = {}
    for _, attribute in ipairs(reference_attributes or {}) do
      table.insert(merged_attributes, attribute)
    end
    for _, attribute in ipairs(own_attributes or {}) do
      table.insert(merged_attributes, attribute)
    end
    if next(merged_attributes) == nil then
      merged_attributes = nil
    end
    return merged_attributes
  end

%    \end{macrocode}
% \begin{markdown}
%
% Parse content between two delimiters in the delimiter table `t`. Produce the respective link and image
% macros.
%
% \end{markdown}
%  \begin{macrocode}
  local function render_link_or_image(t, opening_index, closing_index, content_end_index, reference)
    process_emphasis(t, opening_index, content_end_index)
    local mapped = collect_emphasis_content(t, opening_index + 1, content_end_index - 1)

    local rendered = {}
    if (t[opening_index].element == "link") then
      rendered = writer.link(mapped, reference.url, reference.title, reference.attributes)
    end

    if (t[opening_index].element == "image") then
      rendered = writer.image(mapped, reference.url, reference.title, reference.attributes)
    end

    if (t[opening_index].element == "note") then
      if (t[opening_index].link_type == "note_inline") then
        rendered = writer.note(mapped)
      end
      if (t[opening_index].link_type == "raw_note") then
        rendered = writer.note(reference)
      end
    end

    t[opening_index].rendered = rendered
    delete_parsed_content_in_range(t, opening_index + 1, closing_index)
    empty_content_in_range(t, opening_index, closing_index)
    disable_previous_link_openers(t, opening_index)
    disable_range(t, opening_index, closing_index)
  end

%    \end{macrocode}
% \begin{markdown}
%
% Match the link destination of an inline link at index `closing_index` in table `t`
% when `match_reference` is true. Additionally, match attributes when the option
% \Opt{linkAttributes} is enabled.
%
% \end{markdown}
%  \begin{macrocode}
  local function resolve_inline_following_content(t, closing_index, match_reference, match_link_attributes)
    local content = ""
    for i = closing_index + 1, #t do
      content = content .. t[i].content
    end

    local matching_content = parsers.succeed

    if match_reference then
      matching_content = matching_content * parsers.inline_direct_ref_inside
    end

    if match_link_attributes then
      matching_content = matching_content * Cg(Ct(parsers.attributes^-1), "attributes")
    end

    local matched = lpeg.match(Ct(matching_content * Cg(Cp(), "end_position")), content)

    local matched_count = matched.end_position - 1
    for i = closing_index + 1, #t do
      local value = t[i]

      local chars_left = matched_count
      matched_count = matched_count - #value.content

      if matched_count <= 0 then
        value.content = value.content:sub(chars_left + 1)
        break
      end

      value.content = ''
      value.is_active = false
    end

    local attributes = matched.attributes
    if attributes == nil or next(attributes) == nil then
      attributes = nil
    end

    return {
      url = matched.url or "",
      title = matched.title or "",
      attributes = attributes
    }
  end

%    \end{macrocode}
% \begin{markdown}
%
% Resolve an inline link `[a](b "c")` from the delimiters at `opening_index` and `closing_index`
% within a delimiter table `t`. Here, compared to other types of links, no reference definition is needed.
%
% \end{markdown}
%  \begin{macrocode}
  local function resolve_inline_link(t, opening_index, closing_index)
    local inline_content = resolve_inline_following_content(t, closing_index, true, t.match_link_attributes)
    render_link_or_image(t, opening_index, closing_index, closing_index, inline_content)
  end

%    \end{macrocode}
% \begin{markdown}
%
% Resolve an inline note `^[a]` from the delimiters at `opening_index` and `closing_index`
% within a delimiter table `t`.
%
% \end{markdown}
%  \begin{macrocode}
  local function resolve_note_inline_link(t, opening_index, closing_index)
    local inline_content = resolve_inline_following_content(t, closing_index, false, false)
    render_link_or_image(t, opening_index, closing_index, closing_index, inline_content)
  end

%    \end{macrocode}
% \begin{markdown}
%
% Resolve a shortcut link `[a]` from the delimiters at `opening_index` and `closing_index` within a delimiter table `t`.
% Continue if a tag `a` is not found in the references.
%
% \end{markdown}
%  \begin{macrocode}
  local function resolve_shortcut_link(t, opening_index, closing_index)
    local content = collect_link_content(t, opening_index + 1, closing_index - 1)
    local r = self.lookup_reference(content)

    if r then
      local inline_content = resolve_inline_following_content(t, closing_index, false, t.match_link_attributes)
      r.attributes = join_attributes(r.attributes, inline_content.attributes)
      render_link_or_image(t, opening_index, closing_index, closing_index, r)
    end
  end

%    \end{macrocode}
% \begin{markdown}
%
% Resolve a note `[^a]` from the delimiters at `opening_index` and `closing_index` within a delimiter table `t`.
% Continue if a tag `a` is not found in the rawnotes.
%
% \end{markdown}
%  \begin{macrocode}
  local function resolve_raw_note_link(t, opening_index, closing_index)
    local content = collect_link_content(t, opening_index + 1, closing_index - 1)
    local r = self.lookup_note_reference(content)

    if r then
      local parsed_ref = self.parser_functions.parse_blocks_nested(r)
      render_link_or_image(t, opening_index, closing_index, closing_index, parsed_ref)
    end
  end

%    \end{macrocode}
% \begin{markdown}
%
% Resolve a full link `[a][b]` from the delimiters at `opening_index` and `closing_index` within a delimiter table `t`.
% Continue if a tag `b` is not found in the references.
%
% \end{markdown}
%  \begin{macrocode}
  local function resolve_full_link(t, opening_index, closing_index)
    local next_link_closing_index = find_next_link_closing_index(t, closing_index + 4)
    local next_link_content = collect_link_content(t, closing_index + 3, next_link_closing_index - 1)
    local r = self.lookup_reference(next_link_content)

    if r then
      local inline_content = resolve_inline_following_content(t, next_link_closing_index, false,
                                                              t.match_link_attributes)
      r.attributes = join_attributes(r.attributes, inline_content.attributes)
      render_link_or_image(t, opening_index, next_link_closing_index, closing_index, r)
    end
  end

%    \end{macrocode}
% \begin{markdown}
%
% Resolve a collapsed link `[a][]` from the delimiters at `opening_index` and `closing_index`
% within a delimiter table `t`.
% Continue if a tag `a` is not found in the references.
%
% \end{markdown}
%  \begin{macrocode}
  local function resolve_collapsed_link(t, opening_index, closing_index)
    local next_link_closing_index = find_next_link_closing_index(t, closing_index + 4)
    local content = collect_link_content(t, opening_index + 1, closing_index - 1)
    local r = self.lookup_reference(content)

    if r then
      local inline_content = resolve_inline_following_content(t, closing_index, false, t.match_link_attributes)
      r.attributes = join_attributes(r.attributes, inline_content.attributes)
      render_link_or_image(t, opening_index, next_link_closing_index, closing_index, r)
    end
  end

%    \end{macrocode}
% \begin{markdown}
%
% Parse a table of link and emphasis delimiters `t`.
% First, iterate over the link delimiters and produce either link or image macros.
% Then run `process_emphasis` over the entire delimiter table, resolving emphasis and strong
% emphasis and parsing any content outside of closed delimiters.
%
% \end{markdown}
%  \begin{macrocode}
  local function process_links_and_emphasis(t)
    for _,value in ipairs(t) do
      value.is_active = true
    end

    for i,value in ipairs(t) do
      if not value.is_closing
          or value.type ~= "delimiter"
          or not (value.element == "link" or value.element == "image" or value.element == "note")
          or value.removed then
        goto continue
      end

      local opener_position = find_link_opener(t, 1, i - 1)
      if (opener_position == nil) then
        goto continue
      end

      local opening_delimiter = t[opener_position]
      opening_delimiter.removed = true

      local link_type = opening_delimiter.link_type

      if (link_type == "inline") then
        resolve_inline_link(t, opener_position, i)
      end
      if (link_type == "shortcut") then
        resolve_shortcut_link(t, opener_position, i)
      end
      if (link_type == "full") then
        resolve_full_link(t, opener_position, i)
      end
      if (link_type == "collapsed") then
        resolve_collapsed_link(t, opener_position, i)
      end
      if (link_type == "note_inline") then
        resolve_note_inline_link(t, opener_position, i)
      end
      if (link_type == "raw_note") then
        resolve_raw_note_link(t, opener_position, i)
      end

      ::continue::
    end

    t[#t].content = t[#t].content:gsub("%s*$","")

    process_emphasis(t, 1, #t)
    local final_result = collect_emphasis_content(t, 1, #t)
    return final_result
  end

  function self.defer_link_and_emphasis_processing(delimiter_table)
    return writer.defer_call(function()
      return process_links_and_emphasis(delimiter_table)
    end)
  end

%    \end{macrocode}
% \par
% \begin{markdown}
%
%#### Inline Elements (local)
%
% \end{markdown}
%  \begin{macrocode}
  parsers.Str      = (parsers.normalchar * (parsers.normalchar + parsers.at)^0)
                   / writer.string

  parsers.Symbol   = (parsers.backtick^1 + V("SpecialChar"))
                   / writer.string

  parsers.Ellipsis = P("...") / writer.ellipsis

  parsers.Smart    = parsers.Ellipsis

  parsers.Code     = parsers.inticks / writer.code

  if options.blankBeforeBlockquote then
    parsers.bqstart = parsers.fail
  else
    parsers.bqstart = parsers.blockquote_start
  end

  if options.blankBeforeHeading then
    parsers.headerstart = parsers.fail
  else
    parsers.headerstart = parsers.atx_heading
  end

  if options.blankBeforeList then
    parsers.interrupting_bullets = parsers.fail
    parsers.interrupting_enumerators = parsers.fail
  else
    parsers.interrupting_bullets  = parsers.bullet(parsers.dash, true)
                                  + parsers.bullet(parsers.asterisk, true)
                                  + parsers.bullet(parsers.plus, true)

    parsers.interrupting_enumerators  = parsers.enumerator(parsers.period, true)
                                      + parsers.enumerator(parsers.rparent, true)
  end

  if options.html then
    parsers.html_interrupting = parsers.check_trail
                              * ( parsers.html_incomplete_open_tag
                                + parsers.html_incomplete_close_tag
                                + parsers.html_incomplete_open_special_tag
                                + parsers.html_comment_start
                                + parsers.html_cdatasection_start
                                + parsers.html_declaration_start
                                + parsers.html_instruction_start
                                - parsers.html_close_special_tag
                                - parsers.html_empty_special_tag)
  else
    parsers.html_interrupting = parsers.fail
  end

  parsers.EndlineExceptions
                     = parsers.blankline -- paragraph break
                     + parsers.eof       -- end of document
                     + parsers.bqstart
                     + parsers.thematic_break_lines
                     + parsers.interrupting_bullets
                     + parsers.interrupting_enumerators
                     + parsers.headerstart
                     + parsers.html_interrupting

  parsers.NoSoftLineBreakEndlineExceptions = parsers.EndlineExceptions

  parsers.endline = parsers.newline
                  * (parsers.check_minimal_indent
                    * -V("EndlineExceptions")
                    + parsers.check_optional_indent
                    * -V("EndlineExceptions")
                    * -parsers.starter)
                  * parsers.spacechar^0

  parsers.Endline = parsers.endline
                  / writer.soft_line_break

  parsers.EndlineNoSub = parsers.endline

  parsers.NoSoftLineBreakEndline
                        = parsers.newline
                        * (parsers.check_minimal_indent
                          * -V("NoSoftLineBreakEndlineExceptions")
                          + parsers.check_optional_indent
                          * -V("NoSoftLineBreakEndlineExceptions")
                          * -parsers.starter)
                        * parsers.spacechar^0
                        / writer.space

  parsers.EndlineBreak = parsers.backslash * parsers.Endline
                                           / writer.hard_line_break

  parsers.OptionalIndent
                     = parsers.spacechar^1 / writer.space

  parsers.Space      = parsers.spacechar^2 * parsers.Endline
                                           / writer.hard_line_break
                     + parsers.spacechar^1 * parsers.Endline^-1 * parsers.eof / self.expandtabs
                     + parsers.spacechar^1 * parsers.Endline
                                           / writer.soft_line_break
                     + parsers.spacechar^1 * -parsers.newline / self.expandtabs

  parsers.NoSoftLineBreakSpace
                     = parsers.spacechar^2 * parsers.Endline
                                           / writer.hard_line_break
                     + parsers.spacechar^1 * parsers.Endline^-1 * parsers.eof / self.expandtabs
                     + parsers.spacechar^1 * parsers.Endline
                                           / writer.soft_line_break
                     + parsers.spacechar^1 * -parsers.newline / self.expandtabs

  parsers.NonbreakingEndline
                    = parsers.endline
                    / writer.soft_line_break

  parsers.NonbreakingSpace
                  = parsers.spacechar^2 * parsers.Endline
                                        / writer.hard_line_break
                  + parsers.spacechar^1 * parsers.Endline^-1 * parsers.eof / ""
                  + parsers.spacechar^1 * parsers.Endline
                                        * parsers.optionalspace
                                        / writer.soft_line_break
                  + parsers.spacechar^1 * parsers.optionalspace
                                        / writer.nbsp

%    \end{macrocode}
% \par
% \begin{markdown}
%
% The \luamdef{reader->auto_link_url} method produces an
% autolink to a URL or a relative reference in the output
% format, where `url` is the link destination and
% `attributes` are the optional attributes.
%
% \end{markdown}
%  \begin{macrocode}
function self.auto_link_url(url, attributes)
  return writer.link(writer.escape(url),
                     url, nil, attributes)
end
%    \end{macrocode}
% \par
% \begin{markdown}
%
% The \luamdef{reader->auto_link_email} method produces an
% autolink to an e-mail in the output format, where `email` is the email
% address destination and `attributes` are the optional attributes.
%
% \end{markdown}
%  \begin{macrocode}
function self.auto_link_email(email, attributes)
  return writer.link(writer.escape(email),
                     "mailto:"..email,
                     nil, attributes)
end

  parsers.AutoLinkUrl = parsers.auto_link_url
                      / self.auto_link_url

  parsers.AutoLinkEmail
                      = parsers.auto_link_email
                      / self.auto_link_email

  parsers.AutoLinkRelativeReference
                      = parsers.auto_link_relative_reference
                      / self.auto_link_url

  parsers.LinkAndEmph = Ct(parsers.link_and_emph_table)
                      / self.defer_link_and_emphasis_processing

  parsers.EscapedChar   = parsers.backslash * C(parsers.escapable) / writer.string

  parsers.InlineHtml    = Cs(parsers.html_inline_comment) / writer.inline_html_comment
                        + Cs(parsers.html_any_empty_inline_tag
                            + parsers.html_inline_instruction
                            + parsers.html_inline_cdatasection
                            + parsers.html_inline_declaration
                            + parsers.html_any_open_inline_tag
                            + parsers.html_any_close_tag)
                          / writer.inline_html_tag

  parsers.HtmlEntity    = parsers.html_entities / writer.string
%    \end{macrocode}
% \par
% \begin{markdown}
%
%#### Block Elements (local)
%
% \end{markdown}
%  \begin{macrocode}
  parsers.DisplayHtml = Cs(parsers.check_trail
                          * ( parsers.html_comment
                            + parsers.html_special_block
                            + parsers.html_block
                            + parsers.html_any_block
                            + parsers.html_instruction
                            + parsers.html_cdatasection
                            + parsers.html_declaration))
                        / writer.block_html_element

  parsers.indented_non_blank_line = parsers.indentedline - parsers.blankline

  parsers.Verbatim  = Cs(
                        parsers.check_code_trail
                      * (parsers.line - parsers.blankline)
                      * ((parsers.check_minimal_blank_indent_and_full_code_trail * parsers.blankline)^0
                        * ((parsers.check_minimal_indent / "") * parsers.check_code_trail
                           * (parsers.line - parsers.blankline))^1)^0
                      ) / self.expandtabs / writer.verbatim

  parsers.Blockquote   = parsers.blockquote_body
                       / writer.blockquote

  parsers.ThematicBreak = parsers.thematic_break_lines
                        / writer.thematic_break

  parsers.Reference    = parsers.define_reference_parser
                       / self.register_link

  parsers.Paragraph   = parsers.freeze_trail
                      * (Ct((parsers.Inline)^1)
                      * (parsers.newline + parsers.eof)
                      * parsers.unfreeze_trail
                      / writer.paragraph)

  parsers.Plain        = parsers.nonindentspace * Ct(parsers.Inline^1)
                       / writer.plain
%    \end{macrocode}
% \par
% \begin{markdown}
%
%#### Lists (local)
%
% \end{markdown}
%  \begin{macrocode}

  if options.taskLists then
    parsers.tickbox = ( parsers.ticked_box
                      + parsers.halfticked_box
                      + parsers.unticked_box
                      ) / writer.tickbox
  else
     parsers.tickbox = parsers.fail
  end

  parsers.list_blank = parsers.conditionally_indented_blankline

  parsers.ref_or_block_list_separated = parsers.sep_group_no_output(parsers.list_blank)
                                      * parsers.minimally_indented_ref
                                      + parsers.block_sep_group(parsers.list_blank)
                                      * parsers.minimally_indented_block

  parsers.ref_or_block_non_separated  = parsers.minimally_indented_ref
                                      + (parsers.succeed / writer.interblocksep)
                                      * parsers.minimally_indented_block
                                      - parsers.minimally_indented_blankline

  parsers.tight_list_loop_body_pair  =
    parsers.create_loop_body_pair(parsers.ref_or_block_non_separated,
                                  parsers.minimally_indented_par_or_plain_no_blank,
                                  (parsers.succeed / writer.interblocksep),
                                  (parsers.succeed / writer.paragraphsep))

  parsers.loose_list_loop_body_pair  =
    parsers.create_loop_body_pair(parsers.ref_or_block_list_separated,
                                  parsers.minimally_indented_par_or_plain,
                                  parsers.block_sep_group(parsers.list_blank),
                                  parsers.par_sep_group(parsers.list_blank))

  parsers.tight_list_content_loop = V("Block")
                                  * parsers.tight_list_loop_body_pair.block^0
                                  + (V("Paragraph") + V("Plain"))
                                  * parsers.ref_or_block_non_separated
                                  * parsers.tight_list_loop_body_pair.block^0
                                  +  (V("Paragraph") + V("Plain"))
                                  * parsers.tight_list_loop_body_pair.par^0

  parsers.loose_list_content_loop = V("Block")
                                  * parsers.loose_list_loop_body_pair.block^0
                                  + (V("Paragraph") + V("Plain"))
                                  * parsers.ref_or_block_list_separated
                                  * parsers.loose_list_loop_body_pair.block^0
                                  + (V("Paragraph") + V("Plain"))
                                  * parsers.loose_list_loop_body_pair.par^0

  parsers.list_item_tightness_condition = -#( parsers.list_blank^0
                                            * parsers.minimally_indented_ref_or_block_or_par)
                                        * remove_indent("li")
                                        + remove_indent("li")
                                        * parsers.fail

  parsers.indented_content_tight  = Ct( (parsers.blankline / "")
                                      * #parsers.list_blank
                                      * remove_indent("li")
                                      + ( (V("Reference") + (parsers.blankline / ""))
                                        * parsers.check_minimal_indent
                                        * parsers.tight_list_content_loop
                                        + (V("Reference") + (parsers.blankline / ""))
                                        + (parsers.tickbox^-1 / writer.escape)
                                        * parsers.tight_list_content_loop
                                        )
                                      * parsers.list_item_tightness_condition
                                  )

  parsers.indented_content_loose  = Ct( (parsers.blankline / "")
                                      * #parsers.list_blank
                                      + ( (V("Reference") + (parsers.blankline / ""))
                                        * parsers.check_minimal_indent
                                        * parsers.loose_list_content_loop
                                        + (V("Reference") + (parsers.blankline / ""))
                                        + (parsers.tickbox^-1 / writer.escape)
                                        * parsers.loose_list_content_loop
                                        )
                                  )

  parsers.TightListItem = function(starter)
    return  -parsers.ThematicBreak
            * parsers.add_indent(starter, "li")
            * parsers.indented_content_tight
  end

  parsers.LooseListItem = function(starter)
    return  -parsers.ThematicBreak
            * parsers.add_indent(starter, "li")
            * parsers.indented_content_loose
            * remove_indent("li")
  end

  parsers.BulletListOfType = function(bullet_type)
    local bullet = parsers.bullet(bullet_type)
    return  ( Ct( parsers.TightListItem(bullet)
                * ( (parsers.check_minimal_indent / "")
                  * parsers.TightListItem(bullet)
                  )^0
              )
              * Cc(true)
              * -#( (parsers.list_blank^0 / "")
                  * parsers.check_minimal_indent
                  * (bullet - parsers.ThematicBreak)
              )
              + Ct( parsers.LooseListItem(bullet)
                  * ( (parsers.list_blank^0 / "")
                    * (parsers.check_minimal_indent / "")
                    * parsers.LooseListItem(bullet)
                    )^0
              )
              * Cc(false)
            ) / writer.bulletlist
  end

  parsers.BulletList = parsers.BulletListOfType(parsers.dash)
                     + parsers.BulletListOfType(parsers.asterisk)
                     + parsers.BulletListOfType(parsers.plus)

  local function ordered_list(items,tight,starter)
    local startnum = starter[2][1]
    if options.startNumber then
      startnum = tonumber(startnum) or 1  -- fallback for '#'
      if startnum ~= nil then
        startnum = math.floor(startnum)
      end
    else
      startnum = nil
    end
    return writer.orderedlist(items,tight,startnum)
  end

  parsers.OrderedListOfType = function(delimiter_type)
    local enumerator = parsers.enumerator(delimiter_type)
    return  Cg(enumerator, "listtype")
          * (Ct( parsers.TightListItem(Cb("listtype"))
               * ((parsers.check_minimal_indent / "") * parsers.TightListItem(enumerator))^0)
          * Cc(true)
          * -#((parsers.list_blank^0 / "")
              * parsers.check_minimal_indent * enumerator)
          + Ct( parsers.LooseListItem(Cb("listtype"))
              * ((parsers.list_blank^0 / "")
                * (parsers.check_minimal_indent / "") * parsers.LooseListItem(enumerator))^0)
          * Cc(false)
          ) * Ct(Cb("listtype")) / ordered_list
  end

  parsers.OrderedList = parsers.OrderedListOfType(parsers.period)
                      + parsers.OrderedListOfType(parsers.rparent)
%    \end{macrocode}
% \par
% \begin{markdown}
%
%#### Blank (local)
%
% \end{markdown}
%  \begin{macrocode}
  parsers.Blank        = parsers.blankline / ""
                       + V("Reference")
%    \end{macrocode}
% \par
% \begin{markdown}
%
%#### Headings (local)
%
% \end{markdown}
%  \begin{macrocode}
  function parsers.parse_heading_text(s)
    local inlines = self.parser_functions.parse_inlines(s)
    local flatten_inlines = self.writer.flatten_inlines
    self.writer.flatten_inlines = true
    local flat_text = self.parser_functions.parse_inlines(s)
    flat_text = util.rope_to_string(flat_text)
    self.writer.flatten_inlines = flatten_inlines
    return {flat_text, inlines}
  end

  -- parse atx header
  parsers.AtxHeading = parsers.check_trail_no_rem
                     * Cg(parsers.heading_start, "level")
                     * ((C( parsers.optionalspace
                          * parsers.hash^0
                          * parsers.optionalspace
                          * parsers.newline)
                        + parsers.spacechar^1
                        * C(parsers.line))
                       / strip_atx_end
                       / parsers.parse_heading_text)
                     * Cb("level")
                     / writer.heading

  parsers.heading_line  = parsers.linechar^1
                        - parsers.thematic_break_lines

  parsers.heading_text = parsers.heading_line
                       * ((V("Endline") / "\n") * (parsers.heading_line - parsers.heading_level))^0
                       * parsers.newline^-1

  parsers.SetextHeading = parsers.freeze_trail * parsers.check_trail_no_rem
                        * #(parsers.heading_text
                           * parsers.check_minimal_indent * parsers.check_trail * parsers.heading_level)
                        * Cs(parsers.heading_text)
                        / parsers.parse_heading_text
                        * parsers.check_minimal_indent_and_trail * parsers.heading_level
                        * parsers.newline
                        * parsers.unfreeze_trail
                        / writer.heading

  parsers.Heading = parsers.AtxHeading + parsers.SetextHeading
%    \end{macrocode}
% \par
% \begin{markdown}
%
%#### Syntax Specification
% Define \luamdef{reader->finalize_grammar} as a function that constructs the
% \acro{peg} grammar of markdown, applies syntax extensions `extensions` and
% returns a conversion function that takes a markdown string and turns it into
% a plain \TeX{} output.
%
% \end{markdown}
%  \begin{macrocode}
  function self.finalize_grammar(extensions)
%    \end{macrocode}
% \begin{markdown}
%
% Create a local writable copy of the global read-only
% \luamref{walkable_syntax} hash table. This table can be used by user-defined
% syntax extensions to insert new \acro{peg} patterns into existing rules
% of the \acro{peg} grammar of markdown using
% the \luamref{reader->insert_pattern} method. Furthermore, built-in syntax
% extensions can use this table to override existing rules using the
% \luamref{reader->update_rule} method.
%
% \end{markdown}
%  \begin{macrocode}
    local walkable_syntax = (function(global_walkable_syntax)
      local local_walkable_syntax = {}
      for lhs, rule in pairs(global_walkable_syntax) do
        local_walkable_syntax[lhs] = util.table_copy(rule)
      end
      return local_walkable_syntax
    end)(walkable_syntax)
%    \end{macrocode}
% \par
% \begin{markdown}
%
% The \luamref{reader->insert_pattern} method adds a pattern to
% \luamref{walkable_syntax}`[`*left-hand side terminal symbol*`]` before,
% instead of, or after a right-hand-side terminal symbol.
%
% \end{markdown}
%  \begin{macrocode}
    local current_extension_name = nil
    self.insert_pattern = function(selector, pattern, pattern_name)
      assert(pattern_name == nil or type(pattern_name) == "string")
      local _, _, lhs, pos, rhs = selector:find("^(%a+)%s+([%a%s]+%a+)%s+(%a+)$")
      assert(lhs ~= nil,
        [[Expected selector in form "LHS (before|after|instead of) RHS", not "]]
        .. selector .. [["]])
      assert(walkable_syntax[lhs] ~= nil,
        [[Rule ]] .. lhs .. [[ -> ... does not exist in markdown grammar]])
      assert(pos == "before" or pos == "after" or pos == "instead of",
        [[Expected positional specifier "before", "after", or "instead of", not "]]
        .. pos .. [["]])
      local rule = walkable_syntax[lhs]
      local index = nil
      for current_index, current_rhs in ipairs(rule) do
        if type(current_rhs) == "string" and current_rhs == rhs then
          index = current_index
          if pos == "after" then
            index = index + 1
          end
          break
        end
      end
      assert(index ~= nil,
        [[Rule ]] .. lhs .. [[ -> ]] .. rhs
          .. [[ does not exist in markdown grammar]])
      local accountable_pattern
      if current_extension_name then
        accountable_pattern = { pattern, current_extension_name, pattern_name }
      else
        assert(type(pattern) == "string",
          [[reader->insert_pattern() was called outside an extension with ]]
          .. [[a PEG pattern instead of a rule name]])
        accountable_pattern = pattern
      end
      if pos == "instead of" then
        rule[index] = accountable_pattern
      else
        table.insert(rule, index, accountable_pattern)
      end
    end
%    \end{macrocode}
% \par
% \begin{markdown}
%
% Create a local \luamdef{syntax} hash table that stores those rules of the
% \acro{peg} grammar of markdown that can't be represented as an ordered choice
% of terminal symbols.
%
% \end{markdown}
%  \begin{macrocode}
    local syntax =
      { "Blocks",

        Blocks                = V("InitializeState")
                              * V("ExpectedJekyllData")
                              * V("Blank")^0
%    \end{macrocode}
% \par
% \begin{markdown}
%
% Only create interblock separators between pairs of blocks that are not
% both paragraphs. Between a pair of paragraphs, any number of blank
% lines will always produce a paragraph separator.
%
% \end{markdown}
%  \begin{macrocode}
                              * ( V("Block")
                                * ( V("Blank")^0 * parsers.eof
                                  + ( V("Blank")^2 / writer.paragraphsep
                                    + V("Blank")^0 / writer.interblocksep
                                    )
                                  )
                                + ( V("Paragraph") + V("Plain") )
                                * ( V("Blank")^0 * parsers.eof
                                  + ( V("Blank")^2 / writer.paragraphsep
                                    + V("Blank")^0 / writer.interblocksep
                                    )
                                  )
                                * V("Block")
                                * ( V("Blank")^0 * parsers.eof
                                  + ( V("Blank")^2 / writer.paragraphsep
                                    + V("Blank")^0 / writer.interblocksep
                                    )
                                  )
                                + ( V("Paragraph") + V("Plain") )
                                * ( V("Blank")^0 * parsers.eof
                                  + V("Blank")^0 / writer.paragraphsep
                                  )
                                )^0,

        ExpectedJekyllData    = parsers.succeed,

        Blank                 = parsers.Blank,
        Reference             = parsers.Reference,

        Blockquote            = parsers.Blockquote,
        Verbatim              = parsers.Verbatim,
        ThematicBreak         = parsers.ThematicBreak,
        BulletList            = parsers.BulletList,
        OrderedList           = parsers.OrderedList,
        DisplayHtml           = parsers.DisplayHtml,
        Heading               = parsers.Heading,
        Paragraph             = parsers.Paragraph,
        Plain                 = parsers.Plain,

        EndlineExceptions     = parsers.EndlineExceptions,
        NoSoftLineBreakEndlineExceptions
                              = parsers.NoSoftLineBreakEndlineExceptions,

        Str                   = parsers.Str,
        Space                 = parsers.Space,
        NoSoftLineBreakSpace  = parsers.NoSoftLineBreakSpace,
        OptionalIndent        = parsers.OptionalIndent,
        Endline               = parsers.Endline,
        EndlineNoSub          = parsers.EndlineNoSub,
        NoSoftLineBreakEndline
                              = parsers.NoSoftLineBreakEndline,
        EndlineBreak          = parsers.EndlineBreak,
        LinkAndEmph           = parsers.LinkAndEmph,
        Code                  = parsers.Code,
        AutoLinkUrl           = parsers.AutoLinkUrl,
        AutoLinkEmail         = parsers.AutoLinkEmail,
        AutoLinkRelativeReference
                              = parsers.AutoLinkRelativeReference,
        InlineHtml            = parsers.InlineHtml,
        HtmlEntity            = parsers.HtmlEntity,
        EscapedChar           = parsers.EscapedChar,
        Smart                 = parsers.Smart,
        Symbol                = parsers.Symbol,
        SpecialChar           = parsers.fail,
        InitializeState       = parsers.succeed,
      }
%    \end{macrocode}
% \par
% \begin{markdown}
%
% Define \luamref{reader->update_rule} as a function that receives two
% arguments: a left-hand side terminal symbol and a function that accepts
% the current \acro{peg} pattern in \luamref{walkable_syntax}`[`left-hand side
% terminal symbol`]` if defined or `nil` otherwise and returns a
% \acro{peg} pattern that will (re)define \luamref{walkable_syntax}`[`left-hand
% side terminal symbol`]`.
%
% \end{markdown}
%  \begin{macrocode}
    self.update_rule = function(rule_name, get_pattern)
      assert(current_extension_name ~= nil)
      assert(syntax[rule_name] ~= nil,
        [[Rule ]] .. rule_name .. [[ -> ... does not exist in markdown grammar]])
      local previous_pattern
      local extension_name
      if walkable_syntax[rule_name] then
        local previous_accountable_pattern = walkable_syntax[rule_name][1]
        previous_pattern = previous_accountable_pattern[1]
        extension_name = previous_accountable_pattern[2] .. ", " .. current_extension_name
      else
        previous_pattern = nil
        extension_name = current_extension_name
      end
      local pattern
%    \end{macrocode}
% \par
% \begin{markdown}
%
% Instead of a function, a \acro{peg} pattern `pattern` may also be
% supplied with roughly the same effect as supplying the following
% function, which will define \luamref{walkable_syntax}`[`left-hand
% side terminal symbol`]` unless it has been previously defined.
%
% ``` lua
% function(previous_pattern)
%   assert(previous_pattern == nil)
%   return pattern
% end
% ```
%
% \end{markdown}
%  \begin{macrocode}
      if type(get_pattern) == "function" then
        pattern = get_pattern(previous_pattern)
      else
        assert(previous_pattern == nil,
               [[Rule ]] .. rule_name ..
               [[ has already been updated by ]] .. extension_name)
        pattern = get_pattern
      end
      local accountable_pattern = { pattern, extension_name, rule_name }
      walkable_syntax[rule_name] = { accountable_pattern }
    end
%    \end{macrocode}
% \par
% \begin{markdown}
%
% Define a hash table of all characters with special meaning and add method
% \luamref{reader->add_special_character} that extends the hash table and
% updates the \acro{peg} grammar of markdown.
%
% \end{markdown}
%  \begin{macrocode}
    local special_characters = {}
    self.add_special_character = function(c)
      table.insert(special_characters, c)
      syntax.SpecialChar = S(table.concat(special_characters, ""))
    end

    self.add_special_character("*")
    self.add_special_character("[")
    self.add_special_character("]")
    self.add_special_character("<")
    self.add_special_character("!")
    self.add_special_character("\\")
%    \end{macrocode}
% \par
% \begin{markdown}
%
% Add method \luamdef{reader->initialize_named_group} that defines named groups
% with a default capture value.
%
% \end{markdown}
%  \begin{macrocode}
    self.initialize_named_group = function(name, value)
      local pattern = Ct("")
      if value ~= nil then
        pattern = pattern / value
      end
      syntax.InitializeState = syntax.InitializeState
                             * Cg(pattern, name)
    end
%    \end{macrocode}
% \par
% \begin{markdown}
%
% Add a named group for indentation.
%
% \end{markdown}
%  \begin{macrocode}
    self.initialize_named_group("indent_info")
%    \end{macrocode}
% \par
% \begin{markdown}
%
% Apply syntax extensions.
%
% \end{markdown}
%  \begin{macrocode}
    for _, extension in ipairs(extensions) do
      current_extension_name = extension.name
      extension.extend_writer(writer)
      extension.extend_reader(self)
    end
    current_extension_name = nil
%    \end{macrocode}
% \par
% \begin{markdown}
%
% If the \Opt{debugExtensions} option is enabled, serialize
% \luamref{walkable_syntax} to a \acro{JSON} for debugging purposes.
%
% \end{markdown}
%  \begin{macrocode}
    if options.debugExtensions then
      local sorted_lhs = {}
      for lhs, _ in pairs(walkable_syntax) do
        table.insert(sorted_lhs, lhs)
      end
      table.sort(sorted_lhs)

      local output_lines = {"{"}
      for lhs_index, lhs in ipairs(sorted_lhs) do
        local encoded_lhs = util.encode_json_string(lhs)
        table.insert(output_lines, [[    ]] ..encoded_lhs .. [[: []])
        local rule = walkable_syntax[lhs]
        for rhs_index, rhs in ipairs(rule) do
          local human_readable_rhs
          if type(rhs) == "string" then
            human_readable_rhs = rhs
          else
            local pattern_name
            if rhs[3] then
              pattern_name = rhs[3]
            else
              pattern_name = "Anonymous Pattern"
            end
            local extension_name = rhs[2]
            human_readable_rhs = pattern_name .. [[ (]] .. extension_name .. [[)]]
          end
          local encoded_rhs = util.encode_json_string(human_readable_rhs)
          local output_line = [[        ]] .. encoded_rhs
          if rhs_index < #rule then
            output_line = output_line .. ","
          end
          table.insert(output_lines, output_line)
        end
        local output_line = "    ]"
        if lhs_index < #sorted_lhs then
          output_line = output_line .. ","
        end
        table.insert(output_lines, output_line)
      end
      table.insert(output_lines, "}")

      local output = table.concat(output_lines, "\n")
      local output_filename = options.debugExtensionsFileName
      local output_file = assert(io.open(output_filename, "w"),
        [[Could not open file "]] .. output_filename .. [[" for writing]])
      assert(output_file:write(output))
      assert(output_file:close())
    end
%    \end{macrocode}
% \par
% \begin{markdown}
%
% Materialize \luamref{walkable_syntax} and merge it into \luamref{syntax} to
% produce the complete \acro{peg} grammar of markdown. Whenever a rule exists
% in both \luamref{walkable_syntax} and \luamref{syntax}, the rule from
% \luamref{walkable_syntax} overrides the rule from \luamref{syntax}.
%
% \end{markdown}
%  \begin{macrocode}
    for lhs, rule in pairs(walkable_syntax) do
      syntax[lhs] = parsers.fail
      for _, rhs in ipairs(rule) do
        local pattern
%    \end{macrocode}
% \begin{markdown}
%
% Although the interface of the \luamref{reader->insert_pattern} method does
% not document this (see Section <#sec:lua-user-extensions>), we allow the
% \luamref{reader->insert_pattern} and \luamref{reader->update_rule}
% methods to insert not just \acro{peg} patterns, but also rule names that
% reference the \acro{peg} grammar of Markdown.
%
% \end{markdown}
%  \begin{macrocode}
        if type(rhs) == "string" then
          pattern = V(rhs)
        else
          pattern = rhs[1]
          if type(pattern) == "string" then
            pattern = V(pattern)
          end
        end
        syntax[lhs] = syntax[lhs] + pattern
      end
    end
%    \end{macrocode}
% \par
% \begin{markdown}
%
% Finalize the parser by reacting to options and by producing special parsers
% for difficult edge cases such as blocks nested in definition lists or
% inline content nested in link, note, and image labels.
%
% \end{markdown}
%  \begin{macrocode}
    if options.underscores then
      self.add_special_character("_")
    end

    if not options.codeSpans then
      syntax.Code = parsers.fail
    else
      self.add_special_character("`")
    end

    if not options.html then
      syntax.DisplayHtml = parsers.fail
      syntax.InlineHtml = parsers.fail
      syntax.HtmlEntity  = parsers.fail
    else
      self.add_special_character("&")
    end

    if options.preserveTabs then
      options.stripIndent = false
    end

    if not options.smartEllipses then
      syntax.Smart = parsers.fail
    else
      self.add_special_character(".")
    end

    if not options.relativeReferences then
      syntax.AutoLinkRelativeReference = parsers.fail
    end

    if options.contentLevel == "inline" then
      syntax[1] = "Inlines"
      syntax.Inlines = V("InitializeState")
                      * parsers.Inline^0
                      * ( parsers.spacing^0
                        * parsers.eof / "")
      syntax.Space = parsers.Space + parsers.blankline / writer.space
    end

    local blocks_nested_t = util.table_copy(syntax)
    blocks_nested_t.ExpectedJekyllData = parsers.succeed
    parsers.blocks_nested = Ct(blocks_nested_t)

    parsers.blocks = Ct(syntax)

    local inlines_t = util.table_copy(syntax)
    inlines_t[1] = "Inlines"
    inlines_t.Inlines = V("InitializeState")
                      * parsers.Inline^0
                      * ( parsers.spacing^0
                        * parsers.eof / "")
    parsers.inlines = Ct(inlines_t)

    local inlines_no_inline_note_t = util.table_copy(inlines_t)
    inlines_no_inline_note_t.InlineNote = parsers.fail
    parsers.inlines_no_inline_note = Ct(inlines_no_inline_note_t)

    local inlines_no_html_t = util.table_copy(inlines_t)
    inlines_no_html_t.DisplayHtml = parsers.fail
    inlines_no_html_t.InlineHtml = parsers.fail
    inlines_no_html_t.HtmlEntity = parsers.fail
    parsers.inlines_no_html = Ct(inlines_no_html_t)

    local inlines_nbsp_t = util.table_copy(inlines_t)
    inlines_nbsp_t.Endline = parsers.NonbreakingEndline
    inlines_nbsp_t.Space = parsers.NonbreakingSpace
    parsers.inlines_nbsp = Ct(inlines_nbsp_t)

    local inlines_no_link_or_emphasis_t = util.table_copy(inlines_t)
    inlines_no_link_or_emphasis_t.LinkAndEmph = parsers.fail
    inlines_no_link_or_emphasis_t.EndlineExceptions = parsers.EndlineExceptions - parsers.eof
    parsers.inlines_no_link_or_emphasis = Ct(inlines_no_link_or_emphasis_t)
%    \end{macrocode}
% \par
% \begin{markdown}
%
% Return a function that converts markdown string `input` into a plain \TeX{}
% output and returns it..
%
% \end{markdown}
%  \begin{macrocode}
    return function(input)
%    \end{macrocode}
% \begin{markdown}
% Unicode-normalize the input.
% \end{markdown}
%  \begin{macrocode}
      if options.unicodeNormalization then
        local form = options.unicodeNormalizationForm
        if form == "nfc" then
          input = uni_algos.normalize.NFC(input)
        elseif form == "nfd" then
          input = uni_algos.normalize.NFD(input)
        elseif form == "nfkc" then
          input = uni_algos.normalize.NFKC(input)
        elseif form == "nfkd" then
          input = uni_algos.normalize.NFKD(input)
        else
          return writer.error(format("Unknown normalization form %s", form))
        end
      end
%    \end{macrocode}
% \begin{markdown}
% Since the Lua converter expects \acro{unix} line endings, normalize the
% input. Also add a line ending at the end of the file in case the input file
% has none.
% \end{markdown}
%  \begin{macrocode}
      input = input:gsub("\r\n?", "\n")
      if input:sub(-1) ~= "\n" then
        input = input .. "\n"
      end
%    \end{macrocode}
% \begin{markdown}
% When determining the name of the cache file, create salt for the hashing
% function out of the package version and the passed options recognized by the
% Lua interface (see Section <#sec:lua-options>). The \Opt{cacheDir} option
% is disregarded.
% \end{markdown}
%  \begin{macrocode}
      references = {}
      local opt_string = {}
      for k, _ in pairs(defaultOptions) do
        local v = options[k]
        if type(v) == "table" then
          for _, i in ipairs(v) do
            opt_string[#opt_string+1] = k .. "=" .. tostring(i)
          end
        elseif k ~= "cacheDir" then
          opt_string[#opt_string+1] = k .. "=" .. tostring(v)
        end
      end
      table.sort(opt_string)
      local salt = table.concat(opt_string, ",") .. "," .. metadata.version
      local output
      local function convert(input)
        local document = self.parser_functions.parse_blocks(input)
        local output = util.rope_to_string(writer.document(document))
%    \end{macrocode}
% \begin{markdown}
% Remove block element / paragraph separators immediately followed by the
% output of \luamref{writer->undosep}, possibly interleaved by section ends.
% Then, remove any leftover output of \luamref{writer->undosep}.
% \end{markdown}
%  \begin{macrocode}
        local undosep_start, undosep_end
        local potential_secend_start, secend_start
        local potential_sep_start, sep_start
        while true do
          -- find a `writer->undosep`
          undosep_start, undosep_end = output:find(writer.undosep_text, 1, true)
          if undosep_start == nil then break end
          -- skip any preceding section ends
          secend_start = undosep_start
          while true do
            potential_secend_start = secend_start - #writer.secend_text
            if potential_secend_start < 1
               or output:sub(potential_secend_start, secend_start - 1) ~= writer.secend_text then
              break
            end
            secend_start = potential_secend_start
          end
          -- find an immediately preceding block element / paragraph separator
          sep_start = secend_start
          potential_sep_start = sep_start - #writer.interblocksep_text
          if potential_sep_start >= 1
             and output:sub(potential_sep_start, sep_start - 1) == writer.interblocksep_text then
            sep_start = potential_sep_start
          else
            potential_sep_start = sep_start - #writer.paragraphsep_text
            if potential_sep_start >= 1
               and output:sub(potential_sep_start, sep_start - 1) == writer.paragraphsep_text then
              sep_start = potential_sep_start
            end
          end
          -- remove `writer->undosep` and immediately preceding block element / paragraph separator
          output = output:sub(1, sep_start - 1)
                .. output:sub(secend_start, undosep_start - 1)
                .. output:sub(undosep_end + 1)
        end
        return output
      end
%    \end{macrocode}
% \begin{markdown}
% If we cache markdown documents, produce the cache file and transform its
% filename to plain \TeX{} output via the \luamref{writer->pack} method.
% \end{markdown}
%  \begin{macrocode}
      if options.eagerCache or options.finalizeCache then
        local name = util.cache(options.cacheDir, input, salt, convert,
                                ".md" .. writer.suffix)
        output = writer.pack(name)
%    \end{macrocode}
% \begin{markdown}
% Otherwise, return the result of the conversion directly.
% \end{markdown}
%  \begin{macrocode}
      else
        output = convert(input)
      end
%    \end{macrocode}
% \begin{markdown}
% If the \Opt{finalizeCache} option is enabled, populate the frozen cache in
% the file \Opt{frozenCacheFileName} with an entry for markdown document
% number \Opt{frozenCacheCounter}.
% \end{markdown}
%  \begin{macrocode}
      if options.finalizeCache then
        local file, mode
        if options.frozenCacheCounter > 0 then
          mode = "a"
        else
          mode = "w"
        end
        file = assert(io.open(options.frozenCacheFileName, mode),
          [[Could not open file "]] .. options.frozenCacheFileName
          .. [[" for writing]])
        assert(file:write([[\expandafter\global\expandafter\def\csname ]]
          .. [[markdownFrozenCache]] .. options.frozenCacheCounter
          .. [[\endcsname{]] .. output .. [[}]] .. "\n"))
        assert(file:close())
      end
      return output
    end
  end
  return self
end
%    \end{macrocode}
% \begin{markdown}
%
%### Built-In Syntax Extensions {#lua-built-in-extensions}
%
% Create \luamref{extensions} hash table that contains built-in syntax
% extensions. Syntax extensions are functions that produce objects with two
% methods: `extend_writer` and `extend_reader`. The `extend_writer` object
% takes a \luamref{writer} object as the only parameter and mutates it.
% Similarly, `extend_reader` takes a \luamref{reader} object as the only
% parameter and mutates it.
%
% \end{markdown}
%  \begin{macrocode}
M.extensions = {}
%    \end{macrocode}
% \begin{markdown}
%
%#### Bracketed Spans
%
% The \luamdef{extensions.bracketed_spans} function implements the Pandoc
% bracketed span syntax extension.
%
% \end{markdown}
%  \begin{macrocode}
M.extensions.bracketed_spans = function()
  return {
    name = "built-in bracketed_spans syntax extension",
    extend_writer = function(self)
%    \end{macrocode}
% \par
% \begin{markdown}
%
% Define \luamdef{writer->span} as a function that will transform an input
% bracketed span `s` with attributes `attr` to the output format.
%
% \end{markdown}
%  \begin{macrocode}
      function self.span(s, attr)
        if self.flatten_inlines then return s end
        return {"\\markdownRendererBracketedSpanAttributeContextBegin",
                self.attributes(attr),
                s,
                "\\markdownRendererBracketedSpanAttributeContextEnd{}"}
      end
    end, extend_reader = function(self)
      local parsers = self.parsers
      local writer = self.writer

      local span_label  = parsers.lbracket
                          * (Cs((parsers.alphanumeric^1
                              + parsers.inticks
                              + parsers.autolink
                              + V("InlineHtml")
                              + ( parsers.backslash * parsers.backslash)
                              + ( parsers.backslash * (parsers.lbracket + parsers.rbracket)
                                + V("Space") + V("Endline")
                                + (parsers.any
                                  - (parsers.newline + parsers.lbracket + parsers.rbracket
                                    + parsers.blankline^2))))^1)
                          / self.parser_functions.parse_inlines)
                          * parsers.rbracket

      local Span = span_label
                 * Ct(parsers.attributes)
                 / writer.span

      self.insert_pattern("Inline before LinkAndEmph",
                          Span, "Span")
    end
  }
end
%    \end{macrocode}
% \begin{markdown}
%
%#### Citations
%
% The \luamdef{extensions.citations} function implements the Pandoc citation
% syntax extension. When the `citation_nbsps` parameter is enabled, the syntax
% extension will replace regular spaces with non-breaking spaces inside the
% prenotes and postnotes of citations.
%
% \end{markdown}
%  \begin{macrocode}
M.extensions.citations = function(citation_nbsps)
  return {
    name = "built-in citations syntax extension",
    extend_writer = function(self)
%    \end{macrocode}
% \par
% \begin{markdown}
%
% Define \luamdef{writer->citations} as a function that will transform an
% input array of citations `cites` to the output format. If `text_cites`
% is enabled, the citations should be rendered in-text, when applicable.
% The `cites` array contains tables with the following keys and values:
% \begin{itemize}
%   \item`suppress_author` -- If the value of the key is true, then the
%     author of the work should be omitted in the citation, when applicable.
%   \item`prenote` -- The value of the key is either `nil` or a rope
%     that should be inserted before the citation.
%   \item`postnote` -- The value of the key is either `nil` or a rope
%     that should be inserted after the citation.
%   \item`name` -- The value of this key is the citation name.
% \end{itemize}
%
% \end{markdown}
%  \begin{macrocode}
      function self.citations(text_cites, cites)
        local buffer = {}
        if self.flatten_inlines then
          for _,cite in ipairs(cites) do
            if cite.prenote then
              table.insert(buffer, {cite.prenote, " "})
            end
            table.insert(buffer, cite.name)
            if cite.postnote then
              table.insert(buffer, {" ", cite.postnote})
            end
          end
        else
          table.insert(buffer, {"\\markdownRenderer", text_cites and "TextCite" or "Cite",
            "{", #cites, "}"})
          for _,cite in ipairs(cites) do
            table.insert(buffer, {cite.suppress_author and "-" or "+", "{",
              cite.prenote or "", "}{", cite.postnote or "", "}{", cite.name, "}"})
          end
        end
        return buffer
      end
    end, extend_reader = function(self)
      local parsers = self.parsers
      local writer = self.writer

      local citation_chars
                    = parsers.alphanumeric
                    + S("#$%&-+<>~/_")

      local citation_name
                    = Cs(parsers.dash^-1) * parsers.at
                    * Cs(citation_chars
                        * (((citation_chars + parsers.internal_punctuation
                            - parsers.comma - parsers.semicolon)
                           * -#((parsers.internal_punctuation - parsers.comma
                                - parsers.semicolon)^0
                               * -(citation_chars + parsers.internal_punctuation
                                  - parsers.comma - parsers.semicolon)))^0
                          * citation_chars)^-1)

      local citation_body_prenote
                    = Cs((parsers.alphanumeric^1
                         + parsers.bracketed
                         + parsers.inticks
                         + parsers.autolink
                         + V("InlineHtml")
                         + V("Space") + V("Endline")
                         + (parsers.anyescaped
                           - (parsers.newline + parsers.rbracket + parsers.blankline^2))
                         - (parsers.spnl * parsers.dash^-1 * parsers.at))^1)

      local citation_body_postnote
                    = Cs((parsers.alphanumeric^1
                         + parsers.bracketed
                         + parsers.inticks
                         + parsers.autolink
                         + V("InlineHtml")
                         + V("Space") + V("Endline")
                         + (parsers.anyescaped
                           - (parsers.newline + parsers.rbracket + parsers.semicolon
                             + parsers.blankline^2))
                         - (parsers.spnl * parsers.rbracket))^1)

      local citation_body_chunk
                    = ( citation_body_prenote
                      * parsers.spnlc_sep
                      + Cc("")
                      * parsers.spnlc
                    )
                    * citation_name
                    * (parsers.internal_punctuation - parsers.semicolon)^-1
                    * ( parsers.spnlc / function(_) return end
                      * citation_body_postnote
                      + Cc("")
                      * parsers.spnlc
                    )

      local citation_body
                    = citation_body_chunk
                    * ( parsers.semicolon
                      * parsers.spnlc
                      * citation_body_chunk
                    )^0

      local citation_headless_body_postnote
                    = Cs((parsers.alphanumeric^1
                         + parsers.bracketed
                         + parsers.inticks
                         + parsers.autolink
                         + V("InlineHtml")
                         + V("Space") + V("Endline")
                         + (parsers.anyescaped
                           - (parsers.newline + parsers.rbracket + parsers.at
                             + parsers.semicolon + parsers.blankline^2))
                         - (parsers.spnl * parsers.rbracket))^0)

      local citation_headless_body
                    = citation_headless_body_postnote
                    * ( parsers.semicolon
                      * parsers.spnlc
                      * citation_body_chunk
                    )^0

      local citations
                    = function(text_cites, raw_cites)
          local function normalize(str)
              if str == "" then
                  str = nil
              else
                  str = (citation_nbsps and
                    self.parser_functions.parse_inlines_nbsp or
                    self.parser_functions.parse_inlines)(str)
              end
              return str
          end

          local cites = {}
          for i = 1,#raw_cites,4 do
              cites[#cites+1] = {
                  prenote = normalize(raw_cites[i]),
                  suppress_author = raw_cites[i+1] == "-",
                  name = writer.identifier(raw_cites[i+2]),
                  postnote = normalize(raw_cites[i+3]),
              }
          end
          return writer.citations(text_cites, cites)
      end

      local TextCitations
                    = Ct((parsers.spnlc
                    * Cc("")
                    * citation_name
                    * ((parsers.spnlc
                        * parsers.lbracket
                        * citation_headless_body
                        * parsers.rbracket) + Cc("")))^1)
                    / function(raw_cites)
                        return citations(true, raw_cites)
                      end

      local ParenthesizedCitations
                    = Ct((parsers.spnlc
                    * parsers.lbracket
                    * citation_body
                    * parsers.rbracket)^1)
                    / function(raw_cites)
                        return citations(false, raw_cites)
                      end

      local Citations = TextCitations + ParenthesizedCitations

      self.insert_pattern("Inline before LinkAndEmph",
                          Citations, "Citations")

      self.add_special_character("@")
      self.add_special_character("-")
    end
  }
end
%    \end{macrocode}
% \begin{markdown}
%
%#### Content Blocks
%
% The \luamdef{extensions.content_blocks} function implements the iA\\,Writer
% content blocks syntax extension. The `language_map` parameter specifies
% the filename of the \acro{JSON} file that maps filename extensions to
% programming language names.
%
% \end{markdown}
%  \begin{macrocode}
M.extensions.content_blocks = function(language_map)
%    \end{macrocode}
% \par
% \begin{markdown}
%
% The \luamdef{languages_json} table maps programming language filename
% extensions to fence infostrings. All `language_map` files located by the
% \pkg{kpathsea} library are loaded into a chain of tables.
% \luamref{languages_json} corresponds to the first table and is chained with
% the rest via Lua metatables.
%
% \end{markdown}
%  \begin{macrocode}
  local languages_json = (function()
    local base, prev, curr
    for _, pathname in ipairs{kpse.lookup(language_map, { all=true })} do
      local file = io.open(pathname, "r")
      if not file then goto continue end
      local input = assert(file:read("*a"))
      assert(file:close())
      local json = input:gsub('("[^\n]-"):','[%1]=')
      curr = load("_ENV = {}; return "..json)()
      if type(curr) == "table" then
        if base == nil then
          base = curr
        else
          setmetatable(prev, { __index = curr })
        end
        prev = curr
      end
      ::continue::
    end
    return base or {}
  end)()

  return {
    name = "built-in content_blocks syntax extension",
    extend_writer = function(self)
%    \end{macrocode}
% \par
% \begin{markdown}
%
% Define \luamdef{writer->contentblock} as a function that will transform an
% input iA\\,Writer content block to the output format, where `src`
% corresponds to the \acro{uri} prefix, `suf` to the \acro{uri} extension,
% `type` to the type of the content block (`localfile` or `onlineimage`),
% and `tit` to the title of the content block.
%
% \end{markdown}
%  \begin{macrocode}
      function self.contentblock(src,suf,type,tit)
        if not self.is_writing then return "" end
        src = src.."."..suf
        suf = suf:lower()
        if type == "onlineimage" then
          return {"\\markdownRendererContentBlockOnlineImage{",suf,"}",
                                 "{",self.string(src),"}",
                                 "{",self.uri(src),"}",
                                 "{",self.string(tit or ""),"}"}
        elseif languages_json[suf] then
          return {"\\markdownRendererContentBlockCode{",suf,"}",
                                 "{",self.string(languages_json[suf]),"}",
                                 "{",self.string(src),"}",
                                 "{",self.uri(src),"}",
                                 "{",self.string(tit or ""),"}"}
        else
          return {"\\markdownRendererContentBlock{",suf,"}",
                                 "{",self.string(src),"}",
                                 "{",self.uri(src),"}",
                                 "{",self.string(tit or ""),"}"}
        end
      end
    end, extend_reader = function(self)
      local parsers = self.parsers
      local writer = self.writer

      local contentblock_tail
                    = parsers.optionaltitle
                    * (parsers.newline + parsers.eof)

      -- case insensitive online image suffix:
      local onlineimagesuffix
                    = (function(...)
                        local parser = nil
                        for _, suffix in ipairs({...}) do
                          local pattern=nil
                          for i=1,#suffix do
                            local char=suffix:sub(i,i)
                            char = S(char:lower()..char:upper())
                            if pattern == nil then
                              pattern = char
                            else
                              pattern = pattern * char
                            end
                          end
                          if parser == nil then
                            parser = pattern
                          else
                            parser = parser + pattern
                          end
                        end
                        return parser
                      end)("png", "jpg", "jpeg", "gif", "tif", "tiff")

      -- online image url for iA Writer content blocks with mandatory suffix,
      -- allowing nested brackets:
      local onlineimageurl
                    = (parsers.less
                      * Cs((parsers.anyescaped
                           - parsers.more
                           - parsers.spacing
                           - #(parsers.period
                              * onlineimagesuffix
                              * parsers.more
                              * contentblock_tail))^0)
                      * parsers.period
                      * Cs(onlineimagesuffix)
                      * parsers.more
                      + (Cs((parsers.inparens
                            + (parsers.anyescaped
                              - parsers.spacing
                              - parsers.rparent
                              - #(parsers.period
                                 * onlineimagesuffix
                                 * contentblock_tail)))^0)
                        * parsers.period
                        * Cs(onlineimagesuffix))
                      ) * Cc("onlineimage")

      -- filename for iA Writer content blocks with mandatory suffix:
      local localfilepath
                    = parsers.slash
                    * Cs((parsers.anyescaped
                         - parsers.tab
                         - parsers.newline
                         - #(parsers.period
                            * parsers.alphanumeric^1
                            * contentblock_tail))^1)
                    * parsers.period
                    * Cs(parsers.alphanumeric^1)
                    * Cc("localfile")

      local ContentBlock
                    = parsers.check_trail_no_rem
                    * (localfilepath + onlineimageurl)
                    * contentblock_tail
                    / writer.contentblock

      self.insert_pattern("Block before Blockquote",
                          ContentBlock, "ContentBlock")
    end
  }
end
%    \end{macrocode}
% \begin{markdown}
%
%#### Definition Lists
%
% The \luamdef{extensions.definition_lists} function implements the Pandoc
% definition list syntax extension. If the `tight_lists` parameter is `true`,
% tight lists will produce special right item renderers.
%
% \end{markdown}
%  \begin{macrocode}
M.extensions.definition_lists = function(tight_lists)
  return {
    name = "built-in definition_lists syntax extension",
    extend_writer = function(self)
%    \end{macrocode}
% \par
% \begin{markdown}
%
% Define \luamdef{writer->definitionlist} as a function that will transform an
% input definition list to the output format, where `items` is an array of
% tables, each of the form `{ term = t, definitions = defs }`, where `t`
% is a term and `defs` is an array of definitions. `tight` specifies,
% whether the list is tight or not.
%
% \end{markdown}
%  \begin{macrocode}
      local function dlitem(term, defs)
        local retVal = {"\\markdownRendererDlItem{",term,"}"}
        for _, def in ipairs(defs) do
          retVal[#retVal+1] = {"\\markdownRendererDlDefinitionBegin ",def,
                               "\\markdownRendererDlDefinitionEnd "}
        end
        retVal[#retVal+1] = "\\markdownRendererDlItemEnd "
        return retVal
      end

      function self.definitionlist(items,tight)
        if not self.is_writing then return "" end
        local buffer = {}
        for _,item in ipairs(items) do
          buffer[#buffer + 1] = dlitem(item.term, item.definitions)
        end
        if tight and tight_lists then
          return {"\\markdownRendererDlBeginTight\n", buffer,
            "\n\\markdownRendererDlEndTight"}
        else
          return {"\\markdownRendererDlBegin\n", buffer,
            "\n\\markdownRendererDlEnd"}
        end
      end
    end, extend_reader = function(self)
      local parsers = self.parsers
      local writer = self.writer

      local defstartchar = S("~:")

      local defstart  = parsers.check_trail_length(0) * defstartchar * #parsers.spacing
                                      * (parsers.tab + parsers.space^-3)
                      + parsers.check_trail_length(1) * defstartchar * #parsers.spacing
                                       * (parsers.tab + parsers.space^-2)
                      + parsers.check_trail_length(2) * defstartchar * #parsers.spacing
                                       * (parsers.tab + parsers.space^-1)
                      + parsers.check_trail_length(3) * defstartchar * #parsers.spacing

      local indented_line = (parsers.check_minimal_indent / "") * parsers.check_code_trail * parsers.line

      local blank = parsers.check_minimal_blank_indent_and_any_trail * parsers.optionalspace * parsers.newline

      local dlchunk = Cs(parsers.line * (indented_line - blank)^0)

      local indented_blocks = function(bl)
        return Cs( bl
              * (blank^1 * (parsers.check_minimal_indent / "")
                * parsers.check_code_trail * -parsers.blankline * bl)^0
              * (blank^1 + parsers.eof))
      end

      local function definition_list_item(term, defs, _)
        return { term = self.parser_functions.parse_inlines(term),
                 definitions = defs }
      end

      local DefinitionListItemLoose
                    = C(parsers.line) * blank^0
                    * Ct((parsers.check_minimal_indent * (defstart
                         * indented_blocks(dlchunk)
                         / self.parser_functions.parse_blocks_nested))^1)
                    * Cc(false) / definition_list_item

      local DefinitionListItemTight
                    = C(parsers.line)
                    * Ct((parsers.check_minimal_indent * (defstart * dlchunk
                         / self.parser_functions.parse_blocks_nested))^1)
                    * Cc(true) / definition_list_item

      local DefinitionList
                    = ( Ct(DefinitionListItemLoose^1) * Cc(false)
                      + Ct(DefinitionListItemTight^1)
                      * (blank^0
                        * -DefinitionListItemLoose * Cc(true))
                      ) / writer.definitionlist

      self.insert_pattern("Block after Heading",
                          DefinitionList, "DefinitionList")
    end
  }
end
%    \end{macrocode}
% \begin{markdown}
%
%#### Fancy Lists
%
% The \luamdef{extensions.fancy_lists} function implements the Pandoc fancy
% list syntax extension.
%
% \end{markdown}
%  \begin{macrocode}
M.extensions.fancy_lists = function()
  return {
    name = "built-in fancy_lists syntax extension",
    extend_writer = function(self)
      local options = self.options

%    \end{macrocode}
% \par
% \begin{markdown}
%
% Define \luamdef{writer->fancylist} as a function that will transform an
% input ordered list to the output format, where:
%
%- `items` is an array of the list items,
%- `tight` specifies, whether the list is tight or not,
%- `startnum` is the number of the first list item,
%- `numstyle` is the style of the list item labels from among the following:
%    - `Decimal` -- decimal arabic numbers,
%    - `LowerRoman` -- lower roman numbers,
%    - `UpperRoman` -- upper roman numbers,
%    - `LowerAlpha` -- lower ASCII alphabetic characters, and
%    - `UpperAlpha` -- upper ASCII alphabetic characters, and
%- `numdelim` is the style of delimiters between list item labels and
%  texts from among the following:
%    - `Default` -- default style,
%    - `OneParen` -- parentheses, and
%    - `Period` -- periods.
%
% \end{markdown}
%  \begin{macrocode}
      function self.fancylist(items,tight,startnum,numstyle,numdelim)
        if not self.is_writing then return "" end
        local buffer = {}
        local num = startnum
        for _,item in ipairs(items) do
          if item ~= "" then
            buffer[#buffer + 1] = self.fancyitem(item,num)
          end
          if num ~= nil and item ~= "" then
            num = num + 1
          end
        end
        local contents = util.intersperse(buffer,"\n")
        if tight and options.tightLists then
          return {"\\markdownRendererFancyOlBeginTight{",
                  numstyle,"}{",numdelim,"}",contents,
                  "\n\\markdownRendererFancyOlEndTight "}
        else
          return {"\\markdownRendererFancyOlBegin{",
                  numstyle,"}{",numdelim,"}",contents,
                  "\n\\markdownRendererFancyOlEnd "}
        end
      end
%    \end{macrocode}
% \begin{markdown}
%
% Define \luamdef{writer->fancyitem} as a function that will transform an
% input fancy ordered list item to the output format, where `s` is the text of
% the list item. If the optional parameter `num` is present, it is the number
% of the list item.
%
% \end{markdown}
%  \begin{macrocode}
      function self.fancyitem(s,num)
        if num ~= nil then
          return {"\\markdownRendererFancyOlItemWithNumber{",num,"}",s,
                  "\\markdownRendererFancyOlItemEnd "}
        else
          return {"\\markdownRendererFancyOlItem ",s,"\\markdownRendererFancyOlItemEnd "}
        end
      end
    end, extend_reader = function(self)
      local parsers = self.parsers
      local options = self.options
      local writer = self.writer

      local function combine_markers_and_delims(markers, delims)
        local markers_table = {}
        for _,marker in ipairs(markers) do
          local start_marker
          local continuation_marker
          if type(marker) == "table" then
            start_marker = marker[1]
            continuation_marker = marker[2]
          else
            start_marker = marker
            continuation_marker = marker
          end
          for _,delim in ipairs(delims) do
            table.insert(markers_table, {start_marker, continuation_marker, delim})
          end
        end
        return markers_table
      end

      local function join_table_with_func(func, markers_table)
        local pattern = func(table.unpack(markers_table[1]))
        for i = 2, #markers_table do
          pattern = pattern + func(table.unpack(markers_table[i]))
        end
        return pattern
      end

      local lowercase_letter_marker = R("az")
      local uppercase_letter_marker = R("AZ")

      local roman_marker = function(chars)
        local m, d, c = P(chars[1]), P(chars[2]), P(chars[3])
        local l, x, v, i = P(chars[4]), P(chars[5]), P(chars[6]), P(chars[7])
        return  m^-3
                * (c*m + c*d + d^-1 * c^-3)
                * (x*c + x*l + l^-1 * x^-3)
                * (i*x + i*v + v^-1 * i^-3)
      end

      local lowercase_roman_marker  = roman_marker({"m", "d", "c", "l", "x", "v", "i"})
      local uppercase_roman_marker  = roman_marker({"M", "D", "C", "L", "X", "V", "I"})

      local lowercase_opening_roman_marker  = P("i")
      local uppercase_opening_roman_marker  = P("I")

      local digit_marker = parsers.dig * parsers.dig^-8

      local markers = {
        {lowercase_opening_roman_marker, lowercase_roman_marker},
        {uppercase_opening_roman_marker, uppercase_roman_marker},
        lowercase_letter_marker,
        uppercase_letter_marker,
        lowercase_roman_marker,
        uppercase_roman_marker,
        digit_marker
      }

      local delims = {
        parsers.period,
        parsers.rparent
      }

      local markers_table = combine_markers_and_delims(markers, delims)

      local function enumerator(start_marker, _, delimiter_type, interrupting)
        local delimiter_range
        local allowed_end
        if interrupting then
          delimiter_range = P("1")
          allowed_end = C(parsers.spacechar^1) * #parsers.linechar
        else
          delimiter_range = start_marker
          allowed_end = C(parsers.spacechar^1) + #(parsers.newline + parsers.eof)
        end

        return parsers.check_trail
                * Ct(C(delimiter_range) * C(delimiter_type))
                * allowed_end
      end

      local starter = join_table_with_func(enumerator, markers_table)

      local TightListItem = function(starter)
        return  parsers.add_indent(starter, "li")
                * parsers.indented_content_tight
      end

      local LooseListItem = function(starter)
        return  parsers.add_indent(starter, "li")
                * parsers.indented_content_loose
                * remove_indent("li")
      end

      local function roman2number(roman)
        local romans = { ["M"] = 1000, ["D"] = 500, ["C"] = 100, ["L"] = 50, ["X"] = 10, ["V"] = 5, ["I"] = 1 }
        local numeral = 0

        local i = 1
        local len = string.len(roman)
        while i < len do
          local z1, z2 = romans[ string.sub(roman, i, i) ], romans[ string.sub(roman, i+1, i+1) ]
          if z1 < z2 then
              numeral = numeral + (z2 - z1)
              i = i + 2
          else
              numeral = numeral + z1
              i = i + 1
          end
        end
        if i <= len then numeral = numeral + romans[ string.sub(roman,i,i) ] end
        return numeral
      end

      local function sniffstyle(numstr, delimend)
        local numdelim
        if delimend == ")" then
          numdelim = "OneParen"
        elseif delimend == "." then
          numdelim = "Period"
        else
          numdelim = "Default"
        end

        local num
        num = numstr:match("^([I])$")
        if num then
          return roman2number(num), "UpperRoman", numdelim
        end
        num = numstr:match("^([i])$")
        if num then
          return roman2number(string.upper(num)), "LowerRoman", numdelim
        end
        num = numstr:match("^([A-Z])$")
        if num then
          return string.byte(num) - string.byte("A") + 1, "UpperAlpha", numdelim
        end
        num = numstr:match("^([a-z])$")
        if num then
          return string.byte(num) - string.byte("a") + 1, "LowerAlpha", numdelim
        end
        num = numstr:match("^([IVXLCDM]+)")
        if num then
          return roman2number(num), "UpperRoman", numdelim
        end
        num = numstr:match("^([ivxlcdm]+)")
        if num then
          return roman2number(string.upper(num)), "LowerRoman", numdelim
        end
        return math.floor(tonumber(numstr) or 1), "Decimal", numdelim
      end

      local function fancylist(items,tight,start)
        local startnum, numstyle, numdelim = sniffstyle(start[2][1], start[2][2])
        return writer.fancylist(items,tight,
                                options.startNumber and startnum or 1,
                                numstyle or "Decimal",
                                numdelim or "Default")
      end

      local FancyListOfType = function(start_marker, continuation_marker, delimiter_type)
        local enumerator_start = enumerator(start_marker, continuation_marker, delimiter_type)
        local enumerator_cont = enumerator(continuation_marker, continuation_marker, delimiter_type)
        return Cg(enumerator_start, "listtype")
             * (Ct( TightListItem(Cb("listtype"))
                 * ((parsers.check_minimal_indent / "") * TightListItem(enumerator_cont))^0)
             * Cc(true)
             * -#((parsers.conditionally_indented_blankline^0 / "")
                 * parsers.check_minimal_indent * enumerator_cont)
             + Ct( LooseListItem(Cb("listtype"))
                 * ((parsers.conditionally_indented_blankline^0 / "")
                   * (parsers.check_minimal_indent / "") * LooseListItem(enumerator_cont))^0)
             * Cc(false)
             ) * Ct(Cb("listtype")) / fancylist
      end

      local FancyList = join_table_with_func(FancyListOfType, markers_table)

      local Endline   = parsers.newline
                      * (parsers.check_minimal_indent
                        * -parsers.EndlineExceptions
                        + parsers.check_optional_indent
                        * -parsers.EndlineExceptions
                        * -starter)
                      * parsers.spacechar^0
                      / writer.soft_line_break

      self.update_rule("OrderedList", FancyList)
      self.update_rule("Endline", Endline)
    end
  }
end
%    \end{macrocode}
% \begin{markdown}
%
%#### Fenced Code
%
% The \luamdef{extensions.fenced_code} function implements the commonmark
% fenced code block syntax extension. When the `blank_before_code_fence`
% parameter is `true`, the syntax extension requires a blank line between a
% paragraph and the following fenced code block.
%
% When the `allow_attributes` option is `true`, the syntax extension permits
% attributes following the infostring. When the `allow_raw_blocks` option is
% `true`, the syntax extension permits the specification of raw blocks using
% the Pandoc raw attribute syntax extension.
%
% \end{markdown}
%  \begin{macrocode}
M.extensions.fenced_code = function(blank_before_code_fence,
                                    allow_attributes,
                                    allow_raw_blocks)
  return {
    name = "built-in fenced_code syntax extension",
    extend_writer = function(self)
      local options = self.options

%    \end{macrocode}
% \par
% \begin{markdown}
%
% Define \luamdef{writer->fencedCode} as a function that will transform an
% input fenced code block `s` with the infostring `i` and optional attributes
% `attr` to the output format.
%
% \end{markdown}
%  \begin{macrocode}
      function self.fencedCode(s, i, attr)
        if not self.is_writing then return "" end
        s = s:gsub("\n$", "")
        local buf = {}
        if attr ~= nil then
          table.insert(buf, {"\\markdownRendererFencedCodeAttributeContextBegin",
                             self.attributes(attr)})
        end
        local name = util.cache_verbatim(options.cacheDir, s)
        table.insert(buf, {"\\markdownRendererInputFencedCode{",
                           name,"}{",self.string(i),"}{",self.infostring(i),"}"})
        if attr ~= nil then
          table.insert(buf, "\\markdownRendererFencedCodeAttributeContextEnd{}")
        end
        return buf
      end

%    \end{macrocode}
% \par
% \begin{markdown}
%
% Define \luamdef{writer->rawBlock} as a function that will transform an
% input raw block `s` with the raw attribute `attr` to the output format.
%
% \end{markdown}
%  \begin{macrocode}
      if allow_raw_blocks then
        function self.rawBlock(s, attr)
          if not self.is_writing then return "" end
          s = s:gsub("\n$", "")
          local name = util.cache_verbatim(options.cacheDir, s)
          return {"\\markdownRendererInputRawBlock{",
                  name,"}{", self.string(attr),"}"}
        end
      end
    end, extend_reader = function(self)
      local parsers = self.parsers
      local writer = self.writer

      local function captures_geq_length(_,i,a,b)
        return #a >= #b and i
      end

      local function strip_enclosing_whitespaces(str)
        return str:gsub("^%s*(.-)%s*$", "%1")
      end

      local tilde_infostring = Cs(Cs((V("HtmlEntity")
                                     + parsers.anyescaped
                                     - parsers.newline)^0)
                                 / strip_enclosing_whitespaces)

      local backtick_infostring = Cs(Cs((V("HtmlEntity")
                                        + (-#(parsers.backslash * parsers.backtick) * parsers.anyescaped)
                                         - parsers.newline
                                         - parsers.backtick)^0)
                                    / strip_enclosing_whitespaces)

      local fenceindent

      local function has_trail(indent_table)
        return indent_table ~= nil and
          indent_table.trail ~= nil and
          next(indent_table.trail) ~= nil
      end

      local function has_indents(indent_table)
        return indent_table ~= nil and
          indent_table.indents ~= nil and
          next(indent_table.indents) ~= nil
      end

      local function get_last_indent_name(indent_table)
        if has_indents(indent_table) then
          return indent_table.indents[#indent_table.indents].name
        end
      end

      local function count_fenced_start_indent(_, _, indent_table, trail)
        local last_indent_name = get_last_indent_name(indent_table)
        fenceindent = 0
        if last_indent_name ~= "li" then
          fenceindent = #trail
        end
        return true
      end

      local fencehead      = function(char, infostring)
        return               Cmt(Cb("indent_info") * parsers.check_trail, count_fenced_start_indent)
                           * Cg(char^3, "fencelength")
                           * parsers.optionalspace
                           * infostring
                           * (parsers.newline + parsers.eof)
      end

      local fencetail      = function(char)
        return               parsers.check_trail_no_rem
                           * Cmt(C(char^3) * Cb("fencelength"), captures_geq_length)
                           * parsers.optionalspace * (parsers.newline + parsers.eof)
                           + parsers.eof
      end

      local function process_fenced_line(s, i, indent_table, line_content, is_blank) -- luacheck: ignore s i
        local remainder = ""
        if has_trail(indent_table) then
          remainder = indent_table.trail.internal_remainder
        end

        if is_blank and get_last_indent_name(indent_table) == "li" then
          remainder = ""
        end

        local str = remainder .. line_content
        local index = 1
        local remaining = fenceindent

        while true do
          local c = str:sub(index, index)
          if c == " " and remaining > 0 then
            remaining = remaining - 1
            index = index + 1
          elseif c == "\t" and remaining > 3 then
            remaining = remaining - 4
            index = index + 1
          else
            break
          end
        end

        return true, str:sub(index)
      end

      local fencedline = function(char)
        return Cmt(Cb("indent_info") * C(parsers.line - fencetail(char)) * Cc(false), process_fenced_line)
      end

      local blankfencedline = Cmt(Cb("indent_info") * C(parsers.blankline) * Cc(true), process_fenced_line)

      local TildeFencedCode
             = fencehead(parsers.tilde, tilde_infostring)
             * Cs(((parsers.check_minimal_blank_indent / "") * blankfencedline
                   + (parsers.check_minimal_indent / "") * fencedline(parsers.tilde))^0)
             * ((parsers.check_minimal_indent / "") * fencetail(parsers.tilde) + parsers.succeed)

      local BacktickFencedCode
             = fencehead(parsers.backtick, backtick_infostring)
             * Cs(((parsers.check_minimal_blank_indent / "") * blankfencedline
                   + (parsers.check_minimal_indent / "") * fencedline(parsers.backtick))^0)
             * ((parsers.check_minimal_indent / "") * fencetail(parsers.backtick) + parsers.succeed)

      local infostring_with_attributes
                        = Ct(C((parsers.linechar
                              - ( parsers.optionalspace
                                * parsers.attributes))^0)
                            * parsers.optionalspace
                            * Ct(parsers.attributes))

      local FencedCode
               = ((TildeFencedCode + BacktickFencedCode)
               / function(infostring, code)
                   local expanded_code = self.expandtabs(code)

                   if allow_raw_blocks then
                     local raw_attr = lpeg.match(parsers.raw_attribute,
                                                 infostring)
                     if raw_attr then
                       return writer.rawBlock(expanded_code, raw_attr)
                     end
                   end

                   local attr = nil
                   if allow_attributes then
                     local match = lpeg.match(infostring_with_attributes,
                                              infostring)
                     if match then
                       infostring, attr = table.unpack(match)
                     end
                   end
                   return writer.fencedCode(expanded_code, infostring, attr)
                 end)

      self.insert_pattern("Block after Verbatim",
                          FencedCode, "FencedCode")

      local fencestart
      if blank_before_code_fence then
        fencestart = parsers.fail
      else
        fencestart = fencehead(parsers.backtick, backtick_infostring)
                   + fencehead(parsers.tilde, tilde_infostring)
      end

      self.update_rule("EndlineExceptions", function(previous_pattern)
        if previous_pattern == nil then
          previous_pattern = parsers.EndlineExceptions
        end
        return previous_pattern + fencestart
      end)

      self.add_special_character("`")
      self.add_special_character("~")
    end
  }
end
%    \end{macrocode}
% \begin{markdown}
%
%#### Fenced Divs
%
% The \luamdef{extensions.fenced_divs} function implements the Pandoc fenced
% div syntax extension. When the `blank_before_div_fence` parameter is `true`,
% the syntax extension requires a blank line between a paragraph and the
% following fenced code block.
%
% \end{markdown}
%  \begin{macrocode}
M.extensions.fenced_divs = function(blank_before_div_fence)
  return {
    name = "built-in fenced_divs syntax extension",
    extend_writer = function(self)
%    \end{macrocode}
% \par
% \begin{markdown}
%
% Define \luamdef{writer->div_begin} as a function that will transform the
% beginning of an input fenced div with with attributes `attributes` to the
% output format.
%
% \end{markdown}
%  \begin{macrocode}
      function self.div_begin(attributes)
        local start_output = {"\\markdownRendererFencedDivAttributeContextBegin\n",
                              self.attributes(attributes)}
        local end_output = {"\\markdownRendererFencedDivAttributeContextEnd{}"}
        return self.push_attributes("div", attributes, start_output, end_output)
      end
%    \end{macrocode}
% \par
% \begin{markdown}
%
% Define \luamdef{writer->div_end} as a function that will produce the end of a
% fenced div in the output format.
%
% \end{markdown}
%  \begin{macrocode}
      function self.div_end()
        return self.pop_attributes("div")
      end
    end, extend_reader = function(self)
      local parsers = self.parsers
      local writer = self.writer
%    \end{macrocode}
% \par
% \begin{markdown}
%
% Define basic patterns for matching the opening and the closing tag of a div.
%
% \end{markdown}
%  \begin{macrocode}
      local fenced_div_infostring
                             = C((parsers.linechar
                                - ( parsers.spacechar^1
                                  * parsers.colon^1))^1)

      local fenced_div_begin = parsers.nonindentspace
                             * parsers.colon^3
                             * parsers.optionalspace
                             * fenced_div_infostring
                             * ( parsers.spacechar^1
                               * parsers.colon^1)^0
                             * parsers.optionalspace
                             * (parsers.newline + parsers.eof)

      local fenced_div_end = parsers.nonindentspace
                           * parsers.colon^3
                           * parsers.optionalspace
                           * (parsers.newline + parsers.eof)
%    \end{macrocode}
% \par
% \begin{markdown}
%
% Initialize a named group named `fenced_div_level` for tracking how deep
% we are nested in divs and the named group `fenced_div_num_opening_indents`
% for tracking the indent of the starting div fence. The former named group
% is immutable and should roll back properly when we fail to match a fenced
% div. The latter is mutable and may contain items from unsuccessful matches
% on top. However, we always know how many items at the head of the latter we
% can trust by consulting the former.
%
% \end{markdown}
%  \begin{macrocode}
      self.initialize_named_group("fenced_div_level", "0")
      self.initialize_named_group("fenced_div_num_opening_indents")

      local function increment_div_level()
        local function push_indent_table(s, i, indent_table, -- luacheck: ignore s i
                                         fenced_div_num_opening_indents, fenced_div_level)
          fenced_div_level = tonumber(fenced_div_level) + 1
          local num_opening_indents = 0
          if indent_table.indents ~= nil then
            num_opening_indents = #indent_table.indents
          end
          fenced_div_num_opening_indents[fenced_div_level] = num_opening_indents
          return true, fenced_div_num_opening_indents
        end

        local function increment_level(s, i, fenced_div_level) -- luacheck: ignore s i
          fenced_div_level = tonumber(fenced_div_level) + 1
          return true, tostring(fenced_div_level)
        end

        return Cg( Cmt( Cb("indent_info")
                      * Cb("fenced_div_num_opening_indents")
                      * Cb("fenced_div_level"), push_indent_table)
                 , "fenced_div_num_opening_indents")
             * Cg( Cmt( Cb("fenced_div_level"), increment_level)
                 , "fenced_div_level")
      end

      local function decrement_div_level()
        local function pop_indent_table(s, i, fenced_div_indent_table, fenced_div_level) -- luacheck: ignore s i
          fenced_div_level = tonumber(fenced_div_level)
          fenced_div_indent_table[fenced_div_level] = nil
          return true, tostring(fenced_div_level - 1)
        end

        return Cg( Cmt( Cb("fenced_div_num_opening_indents")
                      * Cb("fenced_div_level"), pop_indent_table)
                 , "fenced_div_level")
      end


      local non_fenced_div_block  = parsers.check_minimal_indent * V("Block")
                                  - parsers.check_minimal_indent_and_trail * fenced_div_end

      local non_fenced_div_paragraph  = parsers.check_minimal_indent * V("Paragraph")
                                      - parsers.check_minimal_indent_and_trail * fenced_div_end

      local blank = parsers.minimally_indented_blank

      local block_separated  = parsers.block_sep_group(blank)
                             * non_fenced_div_block

      local loop_body_pair  = parsers.create_loop_body_pair(block_separated,
                                                            non_fenced_div_paragraph,
                                                            parsers.block_sep_group(blank),
                                                            parsers.par_sep_group(blank))

      local content_loop  = ( non_fenced_div_block
                            * loop_body_pair.block^0
                            + non_fenced_div_paragraph
                            * block_separated
                            * loop_body_pair.block^0
                            + non_fenced_div_paragraph
                            * loop_body_pair.par^0)
                          * blank^0

      local FencedDiv = fenced_div_begin
                      / function (infostring)
                          local attr = lpeg.match(Ct(parsers.attributes), infostring)
                          if attr == nil then
                            attr = {"." .. infostring}
                          end
                          return attr
                        end
                      / writer.div_begin
                      * increment_div_level()
                      * parsers.skipblanklines
                      * Ct(content_loop)
                      * parsers.minimally_indented_blank^0
                      * parsers.check_minimal_indent_and_trail * fenced_div_end
                      * decrement_div_level()
                      * (Cc("") / writer.div_end)

      self.insert_pattern("Block after Verbatim",
                          FencedDiv, "FencedDiv")

      self.add_special_character(":")

%    \end{macrocode}
% \par
% \begin{markdown}
%
% If the `blank_before_div_fence` parameter is `false`, we will have the
% closing div at the beginning of a line break the current paragraph if
% we are currently nested in a div and the indentation matches the opening
% div fence.
%
% \end{markdown}
%  \begin{macrocode}
      local function is_inside_div()
        local function check_div_level(s, i, fenced_div_level) -- luacheck: ignore s i
          fenced_div_level = tonumber(fenced_div_level)
          return fenced_div_level > 0
        end

        return Cmt(Cb("fenced_div_level"), check_div_level)
      end

      local function check_indent()
        local function compare_indent(s, i, indent_table, -- luacheck: ignore s i
                                      fenced_div_num_opening_indents, fenced_div_level)
          fenced_div_level = tonumber(fenced_div_level)
          local num_current_indents = (indent_table.current_line_indents ~= nil and
                                      #indent_table.current_line_indents) or 0
          local num_opening_indents = fenced_div_num_opening_indents[fenced_div_level]
          return num_current_indents == num_opening_indents
        end

        return Cmt( Cb("indent_info")
                  * Cb("fenced_div_num_opening_indents")
                  * Cb("fenced_div_level"), compare_indent)
      end

      local fencestart = is_inside_div()
                       * fenced_div_end
                       * check_indent()

      if not blank_before_div_fence then
        self.update_rule("EndlineExceptions", function(previous_pattern)
          if previous_pattern == nil then
            previous_pattern = parsers.EndlineExceptions
          end
          return previous_pattern + fencestart
        end)
      end
    end
  }
end
%    \end{macrocode}
% \begin{markdown}
%
%#### Header Attributes
%
% The \luamdef{extensions.header_attributes} function implements the Pandoc
% header attribute syntax extension.
%
% \end{markdown}
%  \begin{macrocode}
M.extensions.header_attributes = function()
  return {
    name = "built-in header_attributes syntax extension",
    extend_writer = function()
    end, extend_reader = function(self)
      local parsers = self.parsers
      local writer = self.writer

      local function strip_atx_end(s)
        return s:gsub("%s+#*%s*$","")
      end

      local AtxHeading = Cg(parsers.heading_start, "level")
                       * parsers.optionalspace
                       * (C(((parsers.linechar
                              - (parsers.attributes
                                * parsers.optionalspace
                                * parsers.newline))
                             * (parsers.linechar
                               - parsers.lbrace)^0)^1)
                             / strip_atx_end
                             / parsers.parse_heading_text)
                       * Cg(Ct(parsers.newline
                              + (parsers.attributes
                                * parsers.optionalspace
                                * parsers.newline)), "attributes")
                       * Cb("level")
                       * Cb("attributes")
                       / writer.heading

      local function strip_trailing_spaces(s)
        return s:gsub("%s*$","")
      end

      local heading_line  = (parsers.linechar
                            - (parsers.attributes
                              * parsers.optionalspace
                              * parsers.newline))^1
                          - parsers.thematic_break_lines

      local heading_text  = heading_line
                          * ((V("Endline") / "\n") * (heading_line - parsers.heading_level))^0
                          * parsers.newline^-1

      local SetextHeading  = parsers.freeze_trail * parsers.check_trail_no_rem
                           * #(heading_text
                              * (parsers.attributes
                                * parsers.optionalspace
                                * parsers.newline)^-1
                              * parsers.check_minimal_indent * parsers.check_trail * parsers.heading_level)
                           * Cs(heading_text) / strip_trailing_spaces
                           / parsers.parse_heading_text
                           * Cg(Ct((parsers.attributes
                                 * parsers.optionalspace
                                 * parsers.newline)^-1), "attributes")
                           * parsers.check_minimal_indent_and_trail * parsers.heading_level
                           * Cb("attributes")
                           * parsers.newline
                           * parsers.unfreeze_trail
                           / writer.heading

      local Heading = AtxHeading + SetextHeading
      self.update_rule("Heading", Heading)
    end
  }
end
%    \end{macrocode}
% \begin{markdown}
%
%#### Inline Code Attributes
%
% The \luamdef{extensions.inline_code_attributes} function implements the
% Pandoc inline code attribute syntax extension.
%
% \end{markdown}
%  \begin{macrocode}
M.extensions.inline_code_attributes = function()
  return {
    name = "built-in inline_code_attributes syntax extension",
    extend_writer = function()
    end, extend_reader = function(self)
      local writer = self.writer

      local CodeWithAttributes = parsers.inticks
                               * Ct(parsers.attributes)
                               / writer.code

      self.insert_pattern("Inline before Code",
                          CodeWithAttributes,
                          "CodeWithAttributes")
    end
  }
end
%    \end{macrocode}
% \begin{markdown}
%
%#### Line Blocks
%
% The \luamdef{extensions.line_blocks} function implements the Pandoc line block
% syntax extension.
%
% \end{markdown}
%  \begin{macrocode}
M.extensions.line_blocks = function()
  return {
    name = "built-in line_blocks syntax extension",
    extend_writer = function(self)
%    \end{macrocode}
% \par
% \begin{markdown}
%
% Define \luamdef{writer->lineblock} as a function that will transform
% a line block consisted of `lines` to the output format, with
% all but the last newline rendered as a line break.
%
% \end{markdown}
%  \begin{macrocode}
      function self.lineblock(lines)
        if not self.is_writing then return "" end
        local buffer = {}
        for i = 1, #lines - 1 do
          buffer[#buffer + 1] = { lines[i], self.hard_line_break }
        end
        buffer[#buffer + 1] = lines[#lines]

        return {"\\markdownRendererLineBlockBegin\n"
                  ,buffer,
                  "\n\\markdownRendererLineBlockEnd "}
      end
    end, extend_reader = function(self)
      local parsers = self.parsers
      local writer = self.writer

      local LineBlock = Ct(
                        (Cs(
                          ( (parsers.pipe * parsers.space)/""
                          * ((parsers.space)/entities.char_entity("nbsp"))^0
                          * parsers.linechar^0 * (parsers.newline/""))
                          * (-parsers.pipe
                            * (parsers.space^1/" ")
                            * parsers.linechar^1
                            * (parsers.newline/"")
                            )^0
                          * (parsers.blankline/"")^0
                        ) / self.parser_functions.parse_inlines)^1) / writer.lineblock

      self.insert_pattern("Block after Blockquote",
                           LineBlock, "LineBlock")
    end
  }
end
%    \end{macrocode}
% \begin{markdown}
%
%#### Marked text
%
% The \luamdef{extensions.mark} function implements the Pandoc mark syntax
% extension.
%
% \end{markdown}
%  \begin{macrocode}
M.extensions.mark = function()
  return {
    name = "built-in mark syntax extension",
    extend_writer = function(self)
%    \end{macrocode}
% \par
% \begin{markdown}
%
% Define \luamdef{writer->mark} as a function that will transform an input
% marked text `s` to the output format.
%
% \end{markdown}
%  \begin{macrocode}
      function self.mark(s)
        if self.flatten_inlines then return s end
        return {"\\markdownRendererMark{", s, "}"}
      end
    end, extend_reader = function(self)
      local parsers = self.parsers
      local writer = self.writer

      local doubleequals = P("==")

      local Mark = parsers.between(V("Inline"), doubleequals, doubleequals)
                 / function (inlines) return writer.mark(inlines) end

      self.add_special_character("=")
      self.insert_pattern("Inline before LinkAndEmph",
                          Mark, "Mark")
    end
  }
end
%    \end{macrocode}
% \begin{markdown}
%
%#### Link Attributes
%
% The \luamdef{extensions.link_attributes} function implements the Pandoc
% link attribute syntax extension.
%
% \end{markdown}
%  \begin{macrocode}
M.extensions.link_attributes = function()
  return {
    name = "built-in link_attributes syntax extension",
    extend_writer = function()
    end, extend_reader = function(self)
      local parsers = self.parsers
      local options = self.options

%    \end{macrocode}
% \begin{markdown}
%
% The following patterns define link reference definitions with attributes.
%
% \end{markdown}
%  \begin{macrocode}
      local define_reference_parser = (parsers.check_trail / "") * parsers.link_label * parsers.colon
                                    * parsers.spnlc * parsers.url
                                    * ( parsers.spnlc_sep * parsers.title * (parsers.spnlc * Ct(parsers.attributes))
                                      * parsers.only_blank
                                      + parsers.spnlc_sep * parsers.title * parsers.only_blank
                                      + Cc("") * (parsers.spnlc * Ct(parsers.attributes)) * parsers.only_blank
                                      + Cc("") * parsers.only_blank)

      local ReferenceWithAttributes = define_reference_parser
                                    / self.register_link

      self.update_rule("Reference", ReferenceWithAttributes)

%    \end{macrocode}
% \begin{markdown}
%
% The following patterns define direct and indirect links with attributes.
%
% \end{markdown}
%  \begin{macrocode}

      local LinkWithAttributesAndEmph = Ct(parsers.link_and_emph_table * Cg(Cc(true), "match_link_attributes"))
                                      / self.defer_link_and_emphasis_processing

      self.update_rule("LinkAndEmph", LinkWithAttributesAndEmph)

%    \end{macrocode}
% \begin{markdown}
%
% The following patterns define autolinks with attributes.
%
% \end{markdown}
%  \begin{macrocode}
      local AutoLinkUrlWithAttributes
                      = parsers.auto_link_url
                      * Ct(parsers.attributes)
                      / self.auto_link_url

      self.insert_pattern("Inline before AutoLinkUrl",
                          AutoLinkUrlWithAttributes,
                          "AutoLinkUrlWithAttributes")

      local AutoLinkEmailWithAttributes
                      = parsers.auto_link_email
                      * Ct(parsers.attributes)
                      / self.auto_link_email

      self.insert_pattern("Inline before AutoLinkEmail",
                          AutoLinkEmailWithAttributes,
                          "AutoLinkEmailWithAttributes")

      if options.relativeReferences then

        local AutoLinkRelativeReferenceWithAttributes
                        = parsers.auto_link_relative_reference
                        * Ct(parsers.attributes)
                        / self.auto_link_url

        self.insert_pattern(
          "Inline before AutoLinkRelativeReference",
          AutoLinkRelativeReferenceWithAttributes,
          "AutoLinkRelativeReferenceWithAttributes")

      end

    end
  }
end
%    \end{macrocode}
% \begin{markdown}
%
%#### Notes
%
% The \luamdef{extensions.notes} function implements the Pandoc note
% and inline note syntax extensions. When the `note` parameter is
% `true`, the Pandoc note syntax extension will be enabled.  When the
% `inline_notes` parameter is `true`, the Pandoc inline note syntax
% extension will be enabled.
%
% \end{markdown}
%  \begin{macrocode}
M.extensions.notes = function(notes, inline_notes)
  assert(notes or inline_notes)
  return {
    name = "built-in notes syntax extension",
    extend_writer = function(self)
%    \end{macrocode}
% \par
% \begin{markdown}
%
% Define \luamdef{writer->note} as a function that will transform an
% input note `s` to the output format.
%
% \end{markdown}
%  \begin{macrocode}
      function self.note(s)
        if self.flatten_inlines then return "" end
        return {"\\markdownRendererNote{",s,"}"}
      end
    end, extend_reader = function(self)
      local parsers = self.parsers
      local writer = self.writer

      local rawnotes = parsers.rawnotes

      if inline_notes then
        local InlineNote
                    = parsers.circumflex
                    * (parsers.link_label / self.parser_functions.parse_inlines_no_inline_note)
                    / writer.note

        self.insert_pattern("Inline after LinkAndEmph",
                            InlineNote, "InlineNote")
      end
      if notes then
        local function strip_first_char(s)
          return s:sub(2)
        end

        local RawNoteRef
                      = #(parsers.lbracket * parsers.circumflex)
                      * parsers.link_label / strip_first_char

        -- like indirect_link
        local function lookup_note(ref)
          return writer.defer_call(function()
            local found = rawnotes[self.normalize_tag(ref)]
            if found then
              return writer.note(
                self.parser_functions.parse_blocks_nested(found))
            else
              return {"[",
                self.parser_functions.parse_inlines("^" .. ref), "]"}
            end
          end)
        end

        local function register_note(ref,rawnote)
          local normalized_tag = self.normalize_tag(ref)
          if rawnotes[normalized_tag] == nil then
            rawnotes[normalized_tag] = rawnote
          end
          return ""
        end

        local NoteRef = RawNoteRef / lookup_note

        local optionally_indented_line = parsers.check_optional_indent_and_any_trail * parsers.line

        local blank = parsers.check_optional_blank_indent_and_any_trail * parsers.optionalspace * parsers.newline

        local chunk = Cs(parsers.line * (optionally_indented_line - blank)^0)

        local indented_blocks = function(bl)
          return Cs( bl
                * (blank^1 * (parsers.check_optional_indent / "")
                  * parsers.check_code_trail * -parsers.blankline * bl)^0)
        end

        local NoteBlock
                    = parsers.check_trail_no_rem * RawNoteRef * parsers.colon
                    * parsers.spnlc * indented_blocks(chunk)
                    / register_note

        local Reference = NoteBlock + parsers.Reference

        self.update_rule("Reference", Reference)
        self.insert_pattern("Inline before LinkAndEmph",
                            NoteRef, "NoteRef")
      end

      self.add_special_character("^")
    end
  }
end
%    \end{macrocode}
% \begin{markdown}
%
%#### Pipe Tables
%
% The \luamdef{extensions.pipe_table} function implements the \acro{PHP}
% Markdown table syntax extension (also known as pipe tables in Pandoc). When
% the `table_captions` parameter is `true`, the function also implements the
% Pandoc table caption syntax extension for table captions. When the
% `table_attributes` parameter is also `true`, the function also
% allows attributes to be attached to the (possibly empty) table captions.
%
% \end{markdown}
%  \begin{macrocode}
M.extensions.pipe_tables = function(table_captions, table_attributes)

  local function make_pipe_table_rectangular(rows)
    local num_columns = #rows[2]
    local rectangular_rows = {}
    for i = 1, #rows do
      local row = rows[i]
      local rectangular_row = {}
      for j = 1, num_columns do
        rectangular_row[j] = row[j] or ""
      end
      table.insert(rectangular_rows, rectangular_row)
    end
    return rectangular_rows
  end

  local function pipe_table_row(allow_empty_first_column
                               , nonempty_column
                               , column_separator
                               , column)
    local row_beginning
    if allow_empty_first_column then
      row_beginning = -- empty first column
                      #(parsers.spacechar^4
                       * column_separator)
                    * parsers.optionalspace
                    * column
                    * parsers.optionalspace
                    -- non-empty first column
                    + parsers.nonindentspace
                    * nonempty_column^-1
                    * parsers.optionalspace
    else
      row_beginning = parsers.nonindentspace
                    * nonempty_column^-1
                    * parsers.optionalspace
    end

    return Ct(row_beginning
             * (-- single column with no leading pipes
                #(column_separator
                 * parsers.optionalspace
                 * parsers.newline)
               * column_separator
               * parsers.optionalspace
               -- single column with leading pipes or
               -- more than a single column
               + (column_separator
                 * parsers.optionalspace
                 * column
                 * parsers.optionalspace)^1
               * (column_separator
                 * parsers.optionalspace)^-1))
  end

  return {
    name = "built-in pipe_tables syntax extension",
    extend_writer = function(self)
%    \end{macrocode}
% \par
% \begin{markdown}
%
% Define \luamdef{writer->table} as a function that will transform an input
% table to the output format, where `rows` is a sequence of columns and a
% column is a sequence of cell texts.
%
% \end{markdown}
%  \begin{macrocode}
      function self.table(rows, caption, attributes)
        if not self.is_writing then return "" end
        local buffer = {}
        if attributes ~= nil then
          table.insert(buffer,
                       "\\markdownRendererTableAttributeContextBegin\n")
          table.insert(buffer, self.attributes(attributes))
        end
        table.insert(buffer,
                     {"\\markdownRendererTable{",
                      caption or "", "}{", #rows - 1, "}{",
                      #rows[1], "}"})
        local temp = rows[2] -- put alignments on the first row
        rows[2] = rows[1]
        rows[1] = temp
        for i, row in ipairs(rows) do
          table.insert(buffer, "{")
          for _, column in ipairs(row) do
            if i > 1 then -- do not use braces for alignments
              table.insert(buffer, "{")
            end
            table.insert(buffer, column)
            if i > 1 then
              table.insert(buffer, "}")
            end
          end
          table.insert(buffer, "}")
        end
        if attributes ~= nil then
          table.insert(buffer,
                       "\\markdownRendererTableAttributeContextEnd{}")
        end
        return buffer
      end
    end, extend_reader = function(self)
      local parsers = self.parsers
      local writer = self.writer

      local table_hline_separator = parsers.pipe + parsers.plus

      local table_hline_column = (parsers.dash
                                 - #(parsers.dash
                                    * (parsers.spacechar
                                      + table_hline_separator
                                      + parsers.newline)))^1
                               * (parsers.colon * Cc("r")
                                 + parsers.dash * Cc("d"))
                               + parsers.colon
                               * (parsers.dash
                                 - #(parsers.dash
                                    * (parsers.spacechar
                                      + table_hline_separator
                                      + parsers.newline)))^1
                               * (parsers.colon * Cc("c")
                                 + parsers.dash * Cc("l"))

      local table_hline = pipe_table_row(false
                                        , table_hline_column
                                        , table_hline_separator
                                        , table_hline_column)

      local table_caption_beginning = (parsers.check_minimal_blank_indent_and_any_trail_no_rem
                                      * parsers.optionalspace * parsers.newline)^0
                                    * parsers.check_minimal_indent_and_trail
                                    * (P("Table")^-1 * parsers.colon)
                                    * parsers.optionalspace

      local function strip_trailing_spaces(s)
        return s:gsub("%s*$","")
      end

      local table_row = pipe_table_row(true
                                      , (C((parsers.linechar - parsers.pipe)^1)
                                        / strip_trailing_spaces
                                        / self.parser_functions.parse_inlines)
                                      , parsers.pipe
                                      , (C((parsers.linechar - parsers.pipe)^0)
                                        / strip_trailing_spaces
                                        / self.parser_functions.parse_inlines))

      local table_caption
      if table_captions then
        table_caption = #table_caption_beginning
                      * table_caption_beginning
        if table_attributes then
          table_caption = table_caption
                        * (C(((( parsers.linechar
                               - (parsers.attributes
                                 * parsers.optionalspace
                                 * parsers.newline
                                 * -#( parsers.optionalspace
                                     * parsers.linechar)))
                              + ( parsers.newline
                                * #( parsers.optionalspace
                                   * parsers.linechar)
                                * C(parsers.optionalspace) / writer.space))
                             * (parsers.linechar
                               - parsers.lbrace)^0)^1)
                             / self.parser_functions.parse_inlines)
                        * (parsers.newline
                          + ( Ct(parsers.attributes)
                            * parsers.optionalspace
                            * parsers.newline))
        else
          table_caption = table_caption
                        * C(( parsers.linechar
                            + ( parsers.newline
                              * #( parsers.optionalspace
                                 * parsers.linechar)
                              * C(parsers.optionalspace) / writer.space))^1)
                        / self.parser_functions.parse_inlines
                        * parsers.newline
        end
      else
        table_caption = parsers.fail
      end

      local PipeTable = Ct(table_row * parsers.newline * (parsers.check_minimal_indent_and_trail / {})
                        * table_hline * parsers.newline
                        * ((parsers.check_minimal_indent / {}) * table_row * parsers.newline)^0)
                      / make_pipe_table_rectangular
                      * table_caption^-1
                      / writer.table

      self.insert_pattern("Block after Blockquote",
                          PipeTable, "PipeTable")
    end
  }
end
%    \end{macrocode}
% \begin{markdown}
%
%#### Raw Attributes
%
% The \luamdef{extensions.raw_inline} function implements the Pandoc
% raw attribute syntax extension for inline code spans.
%
% \end{markdown}
%  \begin{macrocode}
M.extensions.raw_inline = function()
  return {
    name = "built-in raw_inline syntax extension",
    extend_writer = function(self)
      local options = self.options

%    \end{macrocode}
% \par
% \begin{markdown}
%
% Define \luamdef{writer->rawInline} as a function that will transform an
% input inline raw span `s` with the raw attribute `attr` to the output format.
%
% \end{markdown}
%  \begin{macrocode}
      function self.rawInline(s, attr)
        if not self.is_writing then return "" end
        if self.flatten_inlines then return s end
        local name = util.cache_verbatim(options.cacheDir, s)
        return {"\\markdownRendererInputRawInline{",
                name,"}{", self.string(attr),"}"}
      end
    end, extend_reader = function(self)
      local writer = self.writer

      local RawInline = parsers.inticks
                      * parsers.raw_attribute
                      / writer.rawInline

      self.insert_pattern("Inline before Code",
                          RawInline, "RawInline")
    end
  }
end
%    \end{macrocode}
% \begin{markdown}
%
%#### Strike-Through
%
% The \luamdef{extensions.strike_through} function implements the Pandoc
% strike-through syntax extension.
%
% \end{markdown}
%  \begin{macrocode}
M.extensions.strike_through = function()
  return {
    name = "built-in strike_through syntax extension",
    extend_writer = function(self)
%    \end{macrocode}
% \par
% \begin{markdown}
%
% Define \luamdef{writer->strike_through} as a function that will transform
% a strike-through span `s` of input text to the output format.
%
% \end{markdown}
%  \begin{macrocode}
      function self.strike_through(s)
        if self.flatten_inlines then return s end
        return {"\\markdownRendererStrikeThrough{",s,"}"}
      end
    end, extend_reader = function(self)
      local parsers = self.parsers
      local writer = self.writer

      local StrikeThrough = (
        parsers.between(parsers.Inline, parsers.doubletildes,
                        parsers.doubletildes)
      ) / writer.strike_through

      self.insert_pattern("Inline after LinkAndEmph",
                          StrikeThrough, "StrikeThrough")

      self.add_special_character("~")
    end
  }
end
%    \end{macrocode}
% \begin{markdown}
%
%#### Subscripts
%
% The \luamdef{extensions.subscripts} function implements the Pandoc
% subscript syntax extension.
%
% \end{markdown}
%  \begin{macrocode}
M.extensions.subscripts = function()
  return {
    name = "built-in subscripts syntax extension",
    extend_writer = function(self)
%    \end{macrocode}
% \par
% \begin{markdown}
%
% Define \luamdef{writer->subscript} as a function that will transform
% a subscript span `s` of input text to the output format.
%
% \end{markdown}
%  \begin{macrocode}
      function self.subscript(s)
        if self.flatten_inlines then return s end
        return {"\\markdownRendererSubscript{",s,"}"}
      end
    end, extend_reader = function(self)
      local parsers = self.parsers
      local writer = self.writer

      local Subscript = (
        parsers.between(parsers.Str, parsers.tilde, parsers.tilde)
      ) / writer.subscript

      self.insert_pattern("Inline after LinkAndEmph",
                          Subscript, "Subscript")

      self.add_special_character("~")
    end
  }
end
%    \end{macrocode}
% \begin{markdown}
%
%#### Superscripts
%
% The \luamdef{extensions.superscripts} function implements the Pandoc
% superscript syntax extension.
%
% \end{markdown}
%  \begin{macrocode}
M.extensions.superscripts = function()
  return {
    name = "built-in superscripts syntax extension",
    extend_writer = function(self)
%    \end{macrocode}
% \par
% \begin{markdown}
%
% Define \luamdef{writer->superscript} as a function that will transform
% a superscript span `s` of input text to the output format.
%
% \end{markdown}
%  \begin{macrocode}
      function self.superscript(s)
        if self.flatten_inlines then return s end
        return {"\\markdownRendererSuperscript{",s,"}"}
      end
    end, extend_reader = function(self)
      local parsers = self.parsers
      local writer = self.writer

      local Superscript = (
        parsers.between(parsers.Str, parsers.circumflex, parsers.circumflex)
      ) / writer.superscript

      self.insert_pattern("Inline after LinkAndEmph",
                          Superscript, "Superscript")

      self.add_special_character("^")
    end
  }
end
%    \end{macrocode}
% \begin{markdown}
%
%#### \TeX{} Math
%
% The \luamdef{extensions.tex_math} function implements the Pandoc math
% syntax extensions.
%
% \end{markdown}
%  \begin{macrocode}
M.extensions.tex_math = function(tex_math_dollars,
                                 tex_math_single_backslash,
                                 tex_math_double_backslash)
  return {
    name = "built-in tex_math syntax extension",
    extend_writer = function(self)
%    \end{macrocode}
% \par
% \begin{markdown}
%
% Define \luamdef{writer->display_math} as a function that will transform
% a math span `s` of input text to the output format.
%
% \end{markdown}
%  \begin{macrocode}
      function self.display_math(s)
        if self.flatten_inlines then return s end
        return {"\\markdownRendererDisplayMath{",self.math(s),"}"}
      end
%    \end{macrocode}
% \begin{markdown}
%
% Define \luamdef{writer->inline_math} as a function that will transform
% a math span `s` of input text to the output format.
%
% \end{markdown}
%  \begin{macrocode}
      function self.inline_math(s)
        if self.flatten_inlines then return s end
        return {"\\markdownRendererInlineMath{",self.math(s),"}"}
      end
    end, extend_reader = function(self)
      local parsers = self.parsers
      local writer = self.writer

      local function between(p, starter, ender)
        return (starter * Cs(p * (p - ender)^0) * ender)
      end

      local function strip_preceding_whitespaces(str)
        return str:gsub("^%s*(.-)$", "%1")
      end

      local allowed_before_closing = B( parsers.backslash * parsers.any
                                      + parsers.any * (parsers.any - parsers.backslash))

      local allowed_before_closing_no_space = B( parsers.backslash * parsers.any
                                               + parsers.any * (parsers.nonspacechar - parsers.backslash))

%    \end{macrocode}
% \begin{markdown}
%
% The following patterns implement the Pandoc dollar math syntax extension.
%
% \end{markdown}
%  \begin{macrocode}
      local dollar_math_content = (parsers.newline * (parsers.check_optional_indent / "")
                                + parsers.backslash^-1
                                * parsers.linechar)
                                - parsers.blankline^2
                                - parsers.dollar

      local inline_math_opening_dollars = parsers.dollar
                                        * #(parsers.nonspacechar)

      local inline_math_closing_dollars = allowed_before_closing_no_space
                                        * parsers.dollar
                                        * -#(parsers.digit)

      local inline_math_dollars = between(Cs( dollar_math_content),
                                          inline_math_opening_dollars,
                                          inline_math_closing_dollars)

      local display_math_opening_dollars  = parsers.dollar
                                          * parsers.dollar

      local display_math_closing_dollars  = parsers.dollar
                                          * parsers.dollar

      local display_math_dollars = between(Cs( dollar_math_content),
                                           display_math_opening_dollars,
                                           display_math_closing_dollars)
%    \end{macrocode}
% \begin{markdown}
%
% The following patterns implement the Pandoc single and double
% backslash math syntax extensions.
%
% \end{markdown}
%  \begin{macrocode}
      local backslash_math_content  = (parsers.newline * (parsers.check_optional_indent / "")
                                    + parsers.linechar)
                                    - parsers.blankline^2
%    \end{macrocode}
% \begin{markdown}
%
% The following patterns implement the Pandoc double backslash math
% syntax extension.
%
% \end{markdown}
%  \begin{macrocode}
      local inline_math_opening_double  = parsers.backslash
                                        * parsers.backslash
                                        * parsers.lparent

      local inline_math_closing_double  = allowed_before_closing
                                        * parsers.spacechar^0
                                        * parsers.backslash
                                        * parsers.backslash
                                        * parsers.rparent

      local inline_math_double  = between(Cs( backslash_math_content),
                                          inline_math_opening_double,
                                          inline_math_closing_double)
                                / strip_preceding_whitespaces

      local display_math_opening_double = parsers.backslash
                                        * parsers.backslash
                                        * parsers.lbracket

      local display_math_closing_double = allowed_before_closing
                                        * parsers.spacechar^0
                                        * parsers.backslash
                                        * parsers.backslash
                                        * parsers.rbracket

      local display_math_double = between(Cs( backslash_math_content),
                                          display_math_opening_double,
                                          display_math_closing_double)
                                / strip_preceding_whitespaces
%    \end{macrocode}
% \begin{markdown}
%
% The following patterns implement the Pandoc single backslash math
% syntax extension.
%
% \end{markdown}
%  \begin{macrocode}
      local inline_math_opening_single  = parsers.backslash
                                        * parsers.lparent

      local inline_math_closing_single  = allowed_before_closing
                                        * parsers.spacechar^0
                                        * parsers.backslash
                                        * parsers.rparent

      local inline_math_single  = between(Cs( backslash_math_content),
                                          inline_math_opening_single,
                                          inline_math_closing_single)
                                / strip_preceding_whitespaces

      local display_math_opening_single = parsers.backslash
                                        * parsers.lbracket

      local display_math_closing_single = allowed_before_closing
                                        * parsers.spacechar^0
                                        * parsers.backslash
                                        * parsers.rbracket

      local display_math_single = between(Cs( backslash_math_content),
                                          display_math_opening_single,
                                          display_math_closing_single)
                                / strip_preceding_whitespaces

      local display_math = parsers.fail

      local inline_math = parsers.fail

      if tex_math_dollars then
        display_math = display_math + display_math_dollars
        inline_math = inline_math + inline_math_dollars
      end

      if tex_math_double_backslash then
        display_math = display_math + display_math_double
        inline_math = inline_math + inline_math_double
      end

      if tex_math_single_backslash then
        display_math = display_math + display_math_single
        inline_math = inline_math + inline_math_single
      end

      local TexMath = display_math / writer.display_math
                    + inline_math / writer.inline_math

      self.insert_pattern("Inline after LinkAndEmph",
                          TexMath, "TexMath")

      if tex_math_dollars then
        self.add_special_character("$")
      end

      if tex_math_single_backslash or tex_math_double_backslash then
        self.add_special_character("\\")
        self.add_special_character("[")
        self.add_special_character("]")
        self.add_special_character(")")
        self.add_special_character("(")
      end
    end
  }
end
%    \end{macrocode}
% \begin{markdown}
%
%#### YAML Metadata
%
% The \luamdef{extensions.jekyll_data} function implements the Pandoc
% \acro{yaml} metadata block syntax extension. When the
% `expect_jekyll_data` parameter is `true`, then a markdown document
% may begin directly with \acro{yaml} metadata and may contain nothing
% but \acro{yaml} metadata. When both `expect_jekyll_data` and
% `ensure_jekyll_data` parameters are `true`, then a a markdown document must
% begin directly with \acro{yaml} metadata and must contain nothing but
% \acro{yaml} metadata.
%
% \end{markdown}
%  \begin{macrocode}
M.extensions.jekyll_data = function(expect_jekyll_data, ensure_jekyll_data)
  return {
    name = "built-in jekyll_data syntax extension",
    extend_writer = function(self)
%    \end{macrocode}
% \par
% \begin{markdown}
%
% Define \luamdef{writer->jekyllData} as a function that will transform an
% input \acro{yaml} table `d` to the output format. The table is the value for
% the key `p` in the parent table; if `p` is nil, then the table has no parent.
% All scalar keys and values encountered in the table will be cast to a string
% following \acro{yaml} serialization rules. String values will also be
% transformed using the function `t` for the typographic output format used by
% the \mref{markdownRendererJekyllDataTypographicString} macro.
%
% \end{markdown}
%  \begin{macrocode}
      function self.jekyllData(d, t, p)
        if not self.is_writing then return "" end

        local buf = {}

        local keys = {}
        for k, _ in pairs(d) do
          table.insert(keys, k)
        end
%    \end{macrocode}
% \begin{markdown}
%
% For reproducibility, sort the keys. For mixed string-and-numeric keys, sort
% numeric keys before string keys.
%
% \end{markdown}
%  \begin{macrocode}
        table.sort(keys, function(first, second)
          if type(first) ~= type(second) then
            return type(first) < type(second)
          else
            return first < second
          end
        end)

        if not p then
          table.insert(buf, "\\markdownRendererJekyllDataBegin")
        end

        local is_sequence = false
        if #d > 0 and #d == #keys then
          for i=1, #d do
            if d[i] == nil then
              goto not_a_sequence
            end
          end
          is_sequence = true
        end
        ::not_a_sequence::

        if is_sequence then
            table.insert(buf, "\\markdownRendererJekyllDataSequenceBegin{")
            table.insert(buf, self.identifier(p or "null"))
            table.insert(buf, "}{")
            table.insert(buf, #keys)
            table.insert(buf, "}")
        else
            table.insert(buf, "\\markdownRendererJekyllDataMappingBegin{")
            table.insert(buf, self.identifier(p or "null"))
            table.insert(buf, "}{")
            table.insert(buf, #keys)
            table.insert(buf, "}")
        end

        for _, k in ipairs(keys) do
          local v = d[k]
          local typ = type(v)
          k = tostring(k or "null")
          if typ == "table" and next(v) ~= nil then
            table.insert(
              buf,
              self.jekyllData(v, t, k)
            )
          else
            k = self.identifier(k)
            v = tostring(v)
            if typ == "boolean" then
              table.insert(buf, "\\markdownRendererJekyllDataBoolean{")
              table.insert(buf, k)
              table.insert(buf, "}{")
              table.insert(buf, v)
              table.insert(buf, "}")
            elseif typ == "number" then
              table.insert(buf, "\\markdownRendererJekyllDataNumber{")
              table.insert(buf, k)
              table.insert(buf, "}{")
              table.insert(buf, v)
              table.insert(buf, "}")
            elseif typ == "string" then
              table.insert(buf, "\\markdownRendererJekyllDataProgrammaticString{")
              table.insert(buf, k)
              table.insert(buf, "}{")
              table.insert(buf, self.identifier(v))
              table.insert(buf, "}")
              table.insert(buf, "\\markdownRendererJekyllDataTypographicString{")
              table.insert(buf, k)
              table.insert(buf, "}{")
              table.insert(buf, t(v))
              table.insert(buf, "}")
            elseif typ == "table" then
              table.insert(buf, "\\markdownRendererJekyllDataEmpty{")
              table.insert(buf, k)
              table.insert(buf, "}")
            else
              local error = self.error(format("Unexpected type %s for value of " ..
                                              "YAML key %s", typ, k))
              table.insert(buf, error)
            end
          end
        end

        if is_sequence then
          table.insert(buf, "\\markdownRendererJekyllDataSequenceEnd")
        else
          table.insert(buf, "\\markdownRendererJekyllDataMappingEnd")
        end

        if not p then
          table.insert(buf, "\\markdownRendererJekyllDataEnd")
        end

        return buf
      end
    end, extend_reader = function(self)
      local parsers = self.parsers
      local writer = self.writer

      local JekyllData
                    = Cmt( C((parsers.line - P("---") - P("..."))^0)
                         , function(s, i, text) -- luacheck: ignore s i
                             local data
                             local ran_ok, _ = pcall(function()
                               -- TODO: Replace with `require("tinyyaml")` in TeX Live 2023
                               local tinyyaml = require("markdown-tinyyaml")
                               data = tinyyaml.parse(text, {timestamps=false})
                             end)
                             if ran_ok and data ~= nil then
                               return true, writer.jekyllData(data, function(s)
                                 return self.parser_functions.parse_blocks_nested(s)
                               end, nil)
                             else
                               return false
                             end
                           end
                         )

      local UnexpectedJekyllData
                    = P("---")
                    * parsers.blankline / 0
                    * #(-parsers.blankline)  -- if followed by blank, it's thematic break
                    * JekyllData
                    * (P("---") + P("..."))

      local ExpectedJekyllData
                    = ( P("---")
                      * parsers.blankline / 0
                      * #(-parsers.blankline)  -- if followed by blank, it's thematic break
                      )^-1
                    * JekyllData
                    * (P("---") + P("..."))^-1

      if ensure_jekyll_data then
        ExpectedJekyllData = ExpectedJekyllData
                           * parsers.eof
      else
        ExpectedJekyllData = ( ExpectedJekyllData
                             * (V("Blank")^0 / writer.interblocksep)
                             )^-1
      end

      self.insert_pattern("Block before Blockquote",
                          UnexpectedJekyllData, "UnexpectedJekyllData")
      if expect_jekyll_data then
        self.update_rule("ExpectedJekyllData", ExpectedJekyllData)
      end
    end
  }
end
%    \end{macrocode}
% \begin{markdown}
%
%### Conversion from Markdown to Plain \TeX{}
%
% The \luamref{new} function returns a conversion function that takes a
% markdown string and turns it into a plain \TeX{} output. See Section
% <#sec:lua-conversion>.
%
% \end{markdown}
%  \begin{macrocode}
function M.new(options)
%    \end{macrocode}
% \par
% \begin{markdown}
%
% Make the `options` table inherit from the \luamref{defaultOptions} table.
%
% \end{markdown}
%  \begin{macrocode}
  options = options or {}
  setmetatable(options, { __index = function (_, key)
    return defaultOptions[key] end })
%    \end{macrocode}
% \par
% \begin{markdown}
%
% If the singleton cache contains a conversion function for the same `options`,
% reuse it.
%
% \end{markdown}
%  \begin{macrocode}
  if options.singletonCache and singletonCache.convert then
    for k, v in pairs(defaultOptions) do
      if type(v) == "table" then
        for i = 1, math.max(#singletonCache.options[k], #options[k]) do
          if singletonCache.options[k][i] ~= options[k][i] then
            goto miss
          end
        end
      elseif singletonCache.options[k] ~= options[k] then
        goto miss
      end
    end
    return singletonCache.convert
  end
  ::miss::
%    \end{macrocode}
% \par
% \begin{markdown}
%
% Apply built-in syntax extensions based on `options`.
%
% \end{markdown}
%  \begin{macrocode}
  local extensions = {}

  if options.bracketedSpans then
    local bracketed_spans_extension = M.extensions.bracketed_spans()
    table.insert(extensions, bracketed_spans_extension)
  end

  if options.contentBlocks then
    local content_blocks_extension = M.extensions.content_blocks(
      options.contentBlocksLanguageMap)
    table.insert(extensions, content_blocks_extension)
  end

  if options.definitionLists then
    local definition_lists_extension = M.extensions.definition_lists(
      options.tightLists)
    table.insert(extensions, definition_lists_extension)
  end

  if options.fencedCode then
    local fenced_code_extension = M.extensions.fenced_code(
      options.blankBeforeCodeFence,
      options.fencedCodeAttributes,
      options.rawAttribute)
    table.insert(extensions, fenced_code_extension)
  end

  if options.fencedDivs then
    local fenced_div_extension = M.extensions.fenced_divs(
      options.blankBeforeDivFence)
    table.insert(extensions, fenced_div_extension)
  end

  if options.headerAttributes then
    local header_attributes_extension = M.extensions.header_attributes()
    table.insert(extensions, header_attributes_extension)
  end

  if options.inlineCodeAttributes then
    local inline_code_attributes_extension =
      M.extensions.inline_code_attributes()
    table.insert(extensions, inline_code_attributes_extension)
  end

  if options.jekyllData then
    local jekyll_data_extension = M.extensions.jekyll_data(
      options.expectJekyllData, options.ensureJekyllData)
    table.insert(extensions, jekyll_data_extension)
  end

  if options.linkAttributes then
    local link_attributes_extension =
      M.extensions.link_attributes()
    table.insert(extensions, link_attributes_extension)
  end

  if options.lineBlocks then
    local line_block_extension = M.extensions.line_blocks()
    table.insert(extensions, line_block_extension)
  end

  if options.mark then
    local mark_extension = M.extensions.mark()
    table.insert(extensions, mark_extension)
  end

  if options.pipeTables then
    local pipe_tables_extension = M.extensions.pipe_tables(
      options.tableCaptions, options.tableAttributes)
    table.insert(extensions, pipe_tables_extension)
  end

  if options.rawAttribute then
    local raw_inline_extension = M.extensions.raw_inline()
    table.insert(extensions, raw_inline_extension)
  end

  if options.strikeThrough then
    local strike_through_extension = M.extensions.strike_through()
    table.insert(extensions, strike_through_extension)
  end

  if options.subscripts then
    local subscript_extension = M.extensions.subscripts()
    table.insert(extensions, subscript_extension)
  end

  if options.superscripts then
    local superscript_extension = M.extensions.superscripts()
    table.insert(extensions, superscript_extension)
  end

  if options.texMathDollars or
     options.texMathSingleBackslash or
     options.texMathDoubleBackslash then
    local tex_math_extension = M.extensions.tex_math(
      options.texMathDollars,
      options.texMathSingleBackslash,
      options.texMathDoubleBackslash)
    table.insert(extensions, tex_math_extension)
  end

  if options.notes or options.inlineNotes then
    local notes_extension = M.extensions.notes(
      options.notes, options.inlineNotes)
    table.insert(extensions, notes_extension)
  end

  if options.citations then
    local citations_extension = M.extensions.citations(options.citationNbsps)
    table.insert(extensions, citations_extension)
  end

  if options.fancyLists then
    local fancy_lists_extension = M.extensions.fancy_lists()
    table.insert(extensions, fancy_lists_extension)
  end
%    \end{macrocode}
% \par
% \begin{markdown}
%
% Apply user-defined syntax extensions based on `options.extensions`.
%
% \end{markdown}
%  \begin{macrocode}
  for _, user_extension_filename in ipairs(options.extensions) do
    local user_extension = (function(filename)
%    \end{macrocode}
% \begin{markdown}
%
% First, load and compile the contents of the user-defined syntax extension.
%
% \end{markdown}
%  \begin{macrocode}
      local pathname = assert(kpse.find_file(filename),
        [[Could not locate user-defined syntax extension "]]
        .. filename)
      local input_file = assert(io.open(pathname, "r"),
        [[Could not open user-defined syntax extension "]]
        .. pathname .. [[" for reading]])
      local input = assert(input_file:read("*a"))
      assert(input_file:close())
      local user_extension, err = load([[
        local sandbox = {}
        setmetatable(sandbox, {__index = _G})
        _ENV = sandbox
      ]] .. input)()
      assert(user_extension,
        [[Failed to compile user-defined syntax extension "]]
        .. pathname .. [[": ]] .. (err or [[]]))
%    \end{macrocode}
% \begin{markdown}
%
% Then, validate the user-defined syntax extension.
%
% \end{markdown}
%  \begin{macrocode}
      assert(user_extension.api_version ~= nil,
        [[User-defined syntax extension "]] .. pathname
        .. [[" does not specify mandatory field "api_version"]])
      assert(type(user_extension.api_version) == "number",
        [[User-defined syntax extension "]] .. pathname
        .. [[" specifies field "api_version" of type "]]
        .. type(user_extension.api_version)
        .. [[" but "number" was expected]])
      assert(user_extension.api_version > 0
         and user_extension.api_version <= metadata.user_extension_api_version,
        [[User-defined syntax extension "]] .. pathname
        .. [[" uses syntax extension API version "]]
        .. user_extension.api_version .. [[ but markdown.lua ]]
        .. metadata.version .. [[ uses API version ]]
        .. metadata.user_extension_api_version
        .. [[, which is incompatible]])

      assert(user_extension.grammar_version ~= nil,
        [[User-defined syntax extension "]] .. pathname
        .. [[" does not specify mandatory field "grammar_version"]])
      assert(type(user_extension.grammar_version) == "number",
        [[User-defined syntax extension "]] .. pathname
        .. [[" specifies field "grammar_version" of type "]]
        .. type(user_extension.grammar_version)
        .. [[" but "number" was expected]])
      assert(user_extension.grammar_version == metadata.grammar_version,
        [[User-defined syntax extension "]] .. pathname
        .. [[" uses grammar version "]] .. user_extension.grammar_version
        .. [[ but markdown.lua ]] .. metadata.version
        .. [[ uses grammar version ]] .. metadata.grammar_version
        .. [[, which is incompatible]])

      assert(user_extension.finalize_grammar ~= nil,
        [[User-defined syntax extension "]] .. pathname
        .. [[" does not specify mandatory "finalize_grammar" field]])
      assert(type(user_extension.finalize_grammar) == "function",
        [[User-defined syntax extension "]] .. pathname
        .. [[" specifies field "finalize_grammar" of type "]]
        .. type(user_extension.finalize_grammar)
        .. [[" but "function" was expected]])
%    \end{macrocode}
% \begin{markdown}
%
% Finally, cast the user-defined syntax extension to the internal format
% of user extensions used by the Markdown package (see Section
% <#sec:lua-built-in-extensions>.)
%
% \end{markdown}
%  \begin{macrocode}
      local extension = {
        name = [[user-defined "]] .. pathname .. [[" syntax extension]],
        extend_reader = user_extension.finalize_grammar,
        extend_writer = function() end,
      }
      return extension
    end)(user_extension_filename)
    table.insert(extensions, user_extension)
  end
%    \end{macrocode}
% \par
% \begin{markdown}
%
% Produce a conversion function from markdown to plain \TeX.
%
% \end{markdown}
%  \begin{macrocode}
  local writer = M.writer.new(options)
  local reader = M.reader.new(writer, options)
  local convert = reader.finalize_grammar(extensions)
%    \end{macrocode}
% \par
% \begin{markdown}
%
% Force garbage collection to reclaim memory for temporary
% objects created in \luamref{writer.new}, \luamref{reader.new},
% and \luamref{reader->finalize_grammar}.
%
% \end{markdown}
%  \begin{macrocode}
  collectgarbage("collect")
%    \end{macrocode}
% \par
% \begin{markdown}
%
% Update the singleton cache.
%
% \end{markdown}
%  \begin{macrocode}
  if options.singletonCache then
    local singletonCacheOptions = {}
    for k, v in pairs(options) do
      singletonCacheOptions[k] = v
    end
    setmetatable(singletonCacheOptions,
      { __index = function (_, key)
        return defaultOptions[key] end })
    singletonCache.options = singletonCacheOptions
    singletonCache.convert = convert
  end
%    \end{macrocode}
% \par
% \begin{markdown}
%
% Return the conversion function from markdown to plain \TeX.
%
% \end{markdown}
%  \begin{macrocode}
  return convert
end

return M
%    \end{macrocode}
% \iffalse
%</lua>
%<*lua-cli>
% \fi
% \par
% \begin{markdown}
%
%### Command-Line Implementation {#lua-cli-implementation}
%
% The command-line implementation provides the actual conversion routine for
% the command-line interface described in Section <#sec:lua-cli-interface>.
%
% \end{markdown}
%  \begin{macrocode}

local input
if input_filename then
  local input_file = assert(io.open(input_filename, "r"),
    [[Could not open file "]] .. input_filename .. [[" for reading]])
  input = assert(input_file:read("*a"))
  assert(input_file:close())
else
  input = assert(io.read("*a"))
end

%    \end{macrocode}
% \begin{markdown}
% First, ensure that the `options.cacheDir` directory exists.
% \end{markdown}
%  \begin{macrocode}
local lfs = require("lfs")
if options.cacheDir and not lfs.isdir(options.cacheDir) then
  assert(lfs.mkdir(options["cacheDir"]))
end
%    \end{macrocode}
% \begin{markdown}
%
% If \pkg{Kpathsea} has not been loaded before or if Lua\TeX{} has not yet
% been initialized, configure \pkg{Kpathsea} on top of loading it.
%
% \end{markdown}
%  \begin{macrocode}
local kpse
(function()
  local should_initialize = package.loaded.kpse == nil
                       or tex.initialize ~= nil
  kpse = require("kpse")
  if should_initialize then
    kpse.set_program_name("luatex")
  end
end)()
local md = require("markdown")
%    \end{macrocode}
% \begin{markdown}
% Since we are loading the rest of the Lua implementation dynamically,
% check that both the `markdown` module and the command line implementation
% are the same version.
% \end{markdown}
%  \begin{macrocode}
if metadata.version ~= md.metadata.version then
  warn("markdown-cli.lua " .. metadata.version .. " used with " ..
       "markdown.lua " .. md.metadata.version .. ".")
end
local convert = md.new(options)
local output = convert(input)

if output_filename then
  local output_file = assert(io.open(output_filename, "w"),
    [[Could not open file "]] .. output_filename .. [[" for writing]])
  assert(output_file:write(output))
  assert(output_file:close())
else
  assert(io.write(output))
end
%    \end{macrocode}
% \begin{markdown}
% Remove the `options.cacheDir` directory if it is empty.
% \end{markdown}
%  \begin{macrocode}
if options.cacheDir then
  lfs.rmdir(options["cacheDir"])
end
%    \end{macrocode}
% \iffalse
%</lua-cli>
%<*tex>
% \fi
% \par
% \begin{markdown}
%
% Plain \TeX{} Implementation {#teximplementation}
%-----------------------------
%
% The plain \TeX{} implementation provides macros for the interfacing between
% \TeX{} and Lua and for the buffering of input text. These macros are then
% used to implement the macros for the conversion from markdown to plain \TeX{}
% exposed by the plain \TeX{} interface (see Section <#sec:texinterface>).
%
%### Logging Facilities {#tex-interface-logging}
%
% \end{markdown}
%  \begin{macrocode}
\ExplSyntaxOn
\cs_if_free:NT
  \markdownInfo
  {
    \cs_new:Npn
      \markdownInfo #1
      {
        \msg_info:nne
          { markdown }
          { generic-message }
          { #1 }
      }
  }
\cs_if_free:NT
  \markdownWarning
  {
    \cs_new:Npn
      \markdownWarning #1
      {
        \msg_warning:nne
          { markdown }
          { generic-message }
          { #1 }
      }
  }
\cs_if_free:NT
  \markdownError
  {
    \cs_new:Npn
      \markdownError #1 #2
      {
        \msg_error:nnee
          { markdown }
          { generic-message-with-help-text }
          { #1 }
          { #2 }
      }
  }
\msg_new:nnn
  { markdown }
  { generic-message }
  { #1 }
\msg_new:nnnn
  { markdown }
  { generic-message-with-help-text }
  { #1 }
  { #2 }
\cs_generate_variant:Nn
  \msg_info:nnn
  { nne }
\cs_generate_variant:Nn
  \msg_warning:nnn
  { nne }
\cs_generate_variant:Nn
  \msg_error:nnnn
  { nnee }
\ExplSyntaxOff
%    \end{macrocode}
% \par
% \begin{markdown}
%
%### Themes {#themes-implementation}
%
% This section implements the theme-loading mechanism and the built-in themes
% provided with the Markdown package. Furthermore, this section also implements
% the built-in plain \TeX{} themes provided with the Markdown package.
%
% \end{markdown}
%  \begin{macrocode}
\ExplSyntaxOn
\prop_new:N \g_@@_plain_tex_loaded_themes_linenos_prop
\cs_new:Nn
  \@@_plain_tex_load_theme:nn
  {
    \prop_get:NnNTF
      \g_@@_plain_tex_loaded_themes_linenos_prop
      { #1 }
      \l_tmpa_tl
      {
        \msg_warning:nnnV
          { markdown }
          { repeatedly-loaded-plain-tex-theme }
          { #1 }
          \l_tmpa_tl
      }
      {
        \msg_info:nnn
          { markdown }
          { loading-plain-tex-theme }
          { #1 }
        \prop_gput:Nnx
          \g_@@_plain_tex_loaded_themes_linenos_prop
          { #1 }
          { \tex_the:D \tex_inputlineno:D }
        \file_input:n
          { markdown theme #2 }
      }
  }
\msg_new:nnn
  { markdown }
  { loading-plain-tex-theme }
  { Loading~plain~TeX~Markdown~theme~#1 }
\msg_new:nnn
  { markdown }
  { repeatedly-loaded-plain-tex-theme }
  {
    Plain~TeX~Markdown~theme~#1~was~previously~
    loaded~on~line~#2,~not~loading~it~again
  }
\cs_generate_variant:Nn
  \prop_gput:Nnn
  { Nnx }
\cs_gset_eq:NN
  \@@_load_theme:nn
  \@@_plain_tex_load_theme:nn
\cs_generate_variant:Nn
  \@@_load_theme:nn
  { nV }
%    \end{macrocode}
% \begin{markdown}
%
% Developers can use the \mref{markdownLoadPlainTeXTheme} macro to load a
% corresponding plain \TeX{} theme from within themes for higher-level \TeX{}
% formats such as \LaTeX{} and \Hologo{ConTeXt}.
%
% \end{markdown}
%  \begin{macrocode}
\cs_new:Npn
  \markdownLoadPlainTeXTheme
  {
%    \end{macrocode}
% \begin{markdown}
%
% First, we extract the name of the current theme from the
% \mref{g_@@_current_theme_tl} macro.
%
% \end{markdown}
%  \begin{macrocode}
    \tl_set:NV
      \l_tmpa_tl
      \g_@@_current_theme_tl
    \tl_reverse:N
      \l_tmpa_tl
    \tl_set:Ne
      \l_tmpb_tl
      {
        \tl_tail:V
          \l_tmpa_tl
      }
    \tl_reverse:N
      \l_tmpb_tl
%    \end{macrocode}
% \begin{markdown}
%
% Next, we munge the theme name.
%
% \end{markdown}
%  \begin{macrocode}
    \str_set:NV
      \l_tmpa_str
      \l_tmpb_tl
    \str_replace_all:Nnn
      \l_tmpa_str
      { / }
      { _ }
%    \end{macrocode}
% \begin{markdown}
%
% Finally, we load the plain \TeX{} theme.
%
% \end{markdown}
%  \begin{macrocode}
    \@@_plain_tex_load_theme:VV
      \l_tmpb_tl
      \l_tmpa_str
  }
\cs_generate_variant:Nn
  \tl_set:Nn
  { Ne }
\cs_generate_variant:Nn
  \@@_plain_tex_load_theme:nn
  { VV }
\ExplSyntaxOff
%    \end{macrocode}
% \iffalse
%</tex>
%<*themes-witiko-tilde>
% \fi
% \par
% \begin{markdown}
%
% The `witiko/tilde` theme redefines the tilde token renderer prototype,
% so that it expands to a non-breaking space:
%
% \end{markdown}
%  \begin{macrocode}
\markdownSetup {
  rendererPrototypes = {
    tilde = {~},
  },
}
%    \end{macrocode}
% \iffalse
%</themes-witiko-tilde>
%<*themes-witiko-markdown-defaults-tex>
% \fi
% \begin{markdown}
%
% The `witiko/markdown/defaults` plain \TeX{} theme provides default
% definitions for token renderer prototypes. See Section
% <#sec:tex-token-renderer-prototypes> for the actual definitions.
%
%### Token Renderer Prototypes {#tex-token-renderer-prototypes}
%
% The following definitions should be considered placeholder.
%
% \end{markdown}
%  \begin{macrocode}
\def\markdownRendererInterblockSeparatorPrototype{\par}%
\def\markdownRendererParagraphSeparatorPrototype{%
  \markdownRendererInterblockSeparator}%
\def\markdownRendererHardLineBreakPrototype{\hfil\break}%
\def\markdownRendererSoftLineBreakPrototype{ }%
\let\markdownRendererEllipsisPrototype\dots
\def\markdownRendererNbspPrototype{~}%
\def\markdownRendererLeftBracePrototype{\char`\{}%
\def\markdownRendererRightBracePrototype{\char`\}}%
\def\markdownRendererDollarSignPrototype{\char`$}%
\def\markdownRendererPercentSignPrototype{\char`\%}%
\def\markdownRendererAmpersandPrototype{\&}%
\def\markdownRendererUnderscorePrototype{\char`_}%
\def\markdownRendererHashPrototype{\char`\#}%
\def\markdownRendererCircumflexPrototype{\char`^}%
\def\markdownRendererBackslashPrototype{\char`\\}%
\def\markdownRendererTildePrototype{\char`~}%
\def\markdownRendererPipePrototype{|}%
\def\markdownRendererCodeSpanPrototype#1{{\tt#1}}%
\def\markdownRendererLinkPrototype#1#2#3#4{#2}%
\def\markdownRendererContentBlockPrototype#1#2#3#4{%
  \markdownInput{#3}}%
\def\markdownRendererContentBlockOnlineImagePrototype{%
  \markdownRendererImage}%
\def\markdownRendererContentBlockCodePrototype#1#2#3#4#5{%
  \markdownRendererInputFencedCode{#3}{#2}{#2}}%
\def\markdownRendererImagePrototype#1#2#3#4{#2}%
\def\markdownRendererUlBeginPrototype{}%
\def\markdownRendererUlBeginTightPrototype{}%
\def\markdownRendererUlItemPrototype{}%
\def\markdownRendererUlItemEndPrototype{}%
\def\markdownRendererUlEndPrototype{}%
\def\markdownRendererUlEndTightPrototype{}%
\def\markdownRendererOlBeginPrototype{}%
\def\markdownRendererOlBeginTightPrototype{}%
\def\markdownRendererFancyOlBeginPrototype#1#2{\markdownRendererOlBegin}%
\def\markdownRendererFancyOlBeginTightPrototype#1#2{\markdownRendererOlBeginTight}%
\def\markdownRendererOlItemPrototype{}%
\def\markdownRendererOlItemWithNumberPrototype#1{}%
\def\markdownRendererOlItemEndPrototype{}%
\def\markdownRendererFancyOlItemPrototype{\markdownRendererOlItem}%
\def\markdownRendererFancyOlItemWithNumberPrototype{\markdownRendererOlItemWithNumber}%
\def\markdownRendererFancyOlItemEndPrototype{}%
\def\markdownRendererOlEndPrototype{}%
\def\markdownRendererOlEndTightPrototype{}%
\def\markdownRendererFancyOlEndPrototype{\markdownRendererOlEnd}%
\def\markdownRendererFancyOlEndTightPrototype{\markdownRendererOlEndTight}%
\def\markdownRendererDlBeginPrototype{}%
\def\markdownRendererDlBeginTightPrototype{}%
\def\markdownRendererDlItemPrototype#1{#1}%
\def\markdownRendererDlItemEndPrototype{}%
\def\markdownRendererDlDefinitionBeginPrototype{}%
\def\markdownRendererDlDefinitionEndPrototype{\par}%
\def\markdownRendererDlEndPrototype{}%
\def\markdownRendererDlEndTightPrototype{}%
\def\markdownRendererEmphasisPrototype#1{{\it#1}}%
\def\markdownRendererStrongEmphasisPrototype#1{{\bf#1}}%
\def\markdownRendererBlockQuoteBeginPrototype{\begingroup\it}%
\def\markdownRendererBlockQuoteEndPrototype{\endgroup\par}%
\def\markdownRendererLineBlockBeginPrototype{\begingroup\parindent=0pt}%
\def\markdownRendererLineBlockEndPrototype{\endgroup}%
\def\markdownRendererInputVerbatimPrototype#1{%
  \par{\tt\input#1\relax{}}\par}%
\def\markdownRendererInputFencedCodePrototype#1#2#3{%
  \markdownRendererInputVerbatim{#1}}%
\def\markdownRendererHeadingOnePrototype#1{#1}%
\def\markdownRendererHeadingTwoPrototype#1{#1}%
\def\markdownRendererHeadingThreePrototype#1{#1}%
\def\markdownRendererHeadingFourPrototype#1{#1}%
\def\markdownRendererHeadingFivePrototype#1{#1}%
\def\markdownRendererHeadingSixPrototype#1{#1}%
\def\markdownRendererThematicBreakPrototype{}%
\def\markdownRendererNotePrototype#1{#1}%
\def\markdownRendererCitePrototype#1{}%
\def\markdownRendererTextCitePrototype#1{}%
\def\markdownRendererTickedBoxPrototype{[X]}%
\def\markdownRendererHalfTickedBoxPrototype{[/]}%
\def\markdownRendererUntickedBoxPrototype{[ ]}%
\def\markdownRendererStrikeThroughPrototype#1{#1}%
\def\markdownRendererSuperscriptPrototype#1{#1}%
\def\markdownRendererSubscriptPrototype#1{#1}%
\def\markdownRendererDisplayMathPrototype#1{$$#1$$}%
\def\markdownRendererInlineMathPrototype#1{$#1$}%
\ExplSyntaxOn
\cs_gset:Npn
  \markdownRendererHeaderAttributeContextBeginPrototype
  {
    \group_begin:
    \color_group_begin:
  }
\cs_gset:Npn
  \markdownRendererHeaderAttributeContextEndPrototype
  {
    \color_group_end:
    \group_end:
  }
\cs_gset_eq:NN
  \markdownRendererBracketedSpanAttributeContextBeginPrototype
  \markdownRendererHeaderAttributeContextBeginPrototype
\cs_gset_eq:NN
  \markdownRendererBracketedSpanAttributeContextEndPrototype
  \markdownRendererHeaderAttributeContextEndPrototype
\cs_gset_eq:NN
  \markdownRendererFencedDivAttributeContextBeginPrototype
  \markdownRendererHeaderAttributeContextBeginPrototype
\cs_gset_eq:NN
  \markdownRendererFencedDivAttributeContextEndPrototype
  \markdownRendererHeaderAttributeContextEndPrototype
\cs_gset_eq:NN
  \markdownRendererFencedCodeAttributeContextBeginPrototype
  \markdownRendererHeaderAttributeContextBeginPrototype
\cs_gset_eq:NN
  \markdownRendererFencedCodeAttributeContextEndPrototype
  \markdownRendererHeaderAttributeContextEndPrototype
\cs_gset:Npn
  \markdownRendererReplacementCharacterPrototype
  { \codepoint_str_generate:n { fffd } }
\ExplSyntaxOff
\def\markdownRendererSectionBeginPrototype{}%
\def\markdownRendererSectionEndPrototype{}%
\let\markdownRendererWarningPrototype\markdownWarning
\let\markdownRendererErrorPrototype\markdownError
%    \end{macrocode}
% \par
% \begin{markdown}
%
%#### Raw Attributes
%
% In the raw block and inline raw span renderer prototypes, execute the content
% with TeX when the raw attribute is `tex`, display the content as markdown when
% the raw attribute is `md`, and ignore the content otherwise.
%
% \end{markdown}
%  \begin{macrocode}
\ExplSyntaxOn
\cs_new:Nn
  \@@_plain_tex_default_input_raw_inline_renderer_prototype:nn
  {
    \str_case:nn
      { #2 }
      {
        { md  } { \markdownInput{#1}  }
        { tex } { \markdownEscape{#1} \unskip }
      }
  }
\cs_new:Nn
  \@@_plain_tex_default_input_raw_block_renderer_prototype:nn
  {
    \str_case:nn
      { #2 }
      {
        { md  } { \markdownInput{#1}  }
        { tex } { \markdownEscape{#1} }
      }
  }
\cs_gset:Npn
  \markdownRendererInputRawInlinePrototype#1#2
  {
    \@@_plain_tex_default_input_raw_inline_renderer_prototype:nn
      { #1 }
      { #2 }
  }
\cs_gset:Npn
  \markdownRendererInputRawBlockPrototype#1#2
  {
    \@@_plain_tex_default_input_raw_block_renderer_prototype:nn
      { #1 }
      { #2 }
  }
\ExplSyntaxOff
%    \end{macrocode}
% \par
% \begin{markdown}
%
%#### YAML Metadata Renderer Prototypes
%
% To keep track of the current type of structure we inhabit when we are
% traversing a \acro{yaml} document, we will maintain the
% \mdef{g_\@\@_jekyll_data_datatypes_seq} stack. At every step of the traversal,
% the stack will contain one of the following constants at any position $p$:
%
% \mdef{c_\@\@_jekyll_data_sequence_tl}
%
%:    The currently traversed branch of the \acro{yaml} document contains a sequence
%     at depth $p$.
%
% \mdef{c_\@\@_jekyll_data_mapping_tl}
%
%:    The currently traversed branch of the \acro{yaml} document contains a mapping
%     at depth $p$.
%
% \mdef{c_\@\@_jekyll_data_scalar_tl}
%
%:    The currently traversed branch of the \acro{yaml} document contains a scalar
%     value at depth $p$.
%
% \end{markdown}
%  \begin{macrocode}
\ExplSyntaxOn
\seq_new:N   \g_@@_jekyll_data_datatypes_seq
\tl_const:Nn \c_@@_jekyll_data_sequence_tl   { sequence }
\tl_const:Nn \c_@@_jekyll_data_mapping_tl    { mapping  }
\tl_const:Nn \c_@@_jekyll_data_scalar_tl     { scalar   }
%    \end{macrocode}
% \par
% \begin{markdown}
%
% To keep track of our current place when we are traversing a \acro{yaml}
% document, we will maintain the
% \mdef{g_\@\@_jekyll_data_wildcard_absolute_address_seq} stack of keys using
% the \mdef{markdown_jekyll_data_push_address_segment:n} macro.
%
% \end{markdown}
%  \begin{macrocode}
\seq_new:N \g_@@_jekyll_data_wildcard_absolute_address_seq
\cs_new:Nn \markdown_jekyll_data_push_address_segment:n
  {
    \seq_if_empty:NF
      \g_@@_jekyll_data_datatypes_seq
      {
        \seq_get_right:NN
          \g_@@_jekyll_data_datatypes_seq
          \l_tmpa_tl
%    \end{macrocode}
% \begin{markdown}
%
% If we are currently in a sequence, we will put an asterisk (`*`) instead of
% a key into \mref{g_\@\@_jekyll_data_wildcard_absolute_address_seq} to make
% it represent a *wildcard*. Keeping a wildcard instead of a precise address
% makes it easy for the users to react to *any* item of a sequence regardless
% of how many there are, which can often be useful.
%
% \end{markdown}
%  \begin{macrocode}
      \str_if_eq:NNTF
        \l_tmpa_tl
        \c_@@_jekyll_data_sequence_tl
        {
          \seq_put_right:Nn
            \g_@@_jekyll_data_wildcard_absolute_address_seq
            { *  }
        }
        {
          \seq_put_right:Nn
            \g_@@_jekyll_data_wildcard_absolute_address_seq
            { #1 }
        }
      }
  }
%    \end{macrocode}
% \par
% \begin{markdown}
%
% Out of \mref{g_\@\@_jekyll_data_wildcard_absolute_address_seq}, we will
% construct the following two token lists:
%
% \mdef{g_\@\@_jekyll_data_wildcard_absolute_address_tl}
%
%:    An *absolute wildcard*: The wildcard from the root of the document
%     prefixed with a slash (`/`) with individual keys and asterisks also
%     delimited by slashes. Allows the users to react to complex
%     context-sensitive structures with ease.
%
%:    For example, the `name` key in the following \acro{yaml} document
%     would correspond to the `/*/person/name` absolute wildcard:
%     ``` yaml
%     [{person: {name: Elon, surname: Musk}}]
%     ```
%
% \mdef{g_\@\@_jekyll_data_wildcard_relative_address_tl}
%
%:    A *relative wildcard*: The rightmost segment of the wildcard. Allows the
%     users to react to simple context-free structures.
%
%:    For example, the `name` key in the following \acro{yaml} document
%     would correspond to the `name` relative wildcard:
%     ``` yaml
%     [{person: {name: Elon, surname: Musk}}]
%     ```
%
% We will construct \mref{g_\@\@_jekyll_data_wildcard_absolute_address_tl}
% using the \mdef{markdown_jekyll_data_concatenate_address:NN} macro and
% we will construct both token lists using the
% \mdef{markdown_jekyll_data_update_address_tls:} macro.
%
% \end{markdown}
%  \begin{macrocode}
\tl_new:N  \g_@@_jekyll_data_wildcard_absolute_address_tl
\tl_new:N  \g_@@_jekyll_data_wildcard_relative_address_tl
\cs_new:Nn \markdown_jekyll_data_concatenate_address:NN
  {
    \seq_pop_left:NN #1 \l_tmpa_tl
    \tl_set:Nx #2 { / \seq_use:Nn #1 { / } }
    \seq_put_left:NV #1 \l_tmpa_tl
  }
\cs_new:Nn \markdown_jekyll_data_update_address_tls:
  {
    \markdown_jekyll_data_concatenate_address:NN
      \g_@@_jekyll_data_wildcard_absolute_address_seq
      \g_@@_jekyll_data_wildcard_absolute_address_tl
    \seq_get_right:NN
      \g_@@_jekyll_data_wildcard_absolute_address_seq
      \g_@@_jekyll_data_wildcard_relative_address_tl
  }
%    \end{macrocode}
% \par
% \begin{markdown}
%
% To make sure that the stacks and token lists stay in sync, we will use the
% \mdef{markdown_jekyll_data_push:nN} and \mdef{markdown_jekyll_data_pop:}
% macros.
%
% \end{markdown}
%  \begin{macrocode}
\cs_new:Nn \markdown_jekyll_data_push:nN
  {
    \markdown_jekyll_data_push_address_segment:n
      { #1 }
    \seq_put_right:NV
     \g_@@_jekyll_data_datatypes_seq
     #2
    \markdown_jekyll_data_update_address_tls:
  }
\cs_new:Nn \markdown_jekyll_data_pop:
  {
    \seq_pop_right:NN
      \g_@@_jekyll_data_wildcard_absolute_address_seq
      \l_tmpa_tl
    \seq_pop_right:NN
      \g_@@_jekyll_data_datatypes_seq
      \l_tmpa_tl
    \markdown_jekyll_data_update_address_tls:
  }
%    \end{macrocode}
% \par
% \begin{markdown}
%
% To set a single key--value, we will use the
% \mdef{markdown_jekyll_data_set_keyval:Nn} macro, ignoring unknown keys.
% To set key--values for both absolute and relative wildcards, we will use the
% \mdef{markdown_jekyll_data_set_keyvals:nn} macro.
%
% \end{markdown}
%  \begin{macrocode}
\cs_new:Nn \markdown_jekyll_data_set_keyval:nn
  {
    \keys_set_known:nn
      { markdown/jekyllData }
      { { #1 } = { #2 } }
  }
\cs_generate_variant:Nn
  \markdown_jekyll_data_set_keyval:nn
  { Vn }
\cs_new:Nn \markdown_jekyll_data_set_keyvals:nn
  {
    \markdown_jekyll_data_push:nN
      { #1 }
      \c_@@_jekyll_data_scalar_tl
    \markdown_jekyll_data_set_keyval:Vn
      \g_@@_jekyll_data_wildcard_absolute_address_tl
      { #2 }
    \markdown_jekyll_data_set_keyval:Vn
      \g_@@_jekyll_data_wildcard_relative_address_tl
      { #2 }
    \markdown_jekyll_data_pop:
  }
%    \end{macrocode}
% \par
% \begin{markdown}
%
% Finally, we will register our macros as token renderer prototypes
% to be able to react to the traversal of a \acro{yaml} document.
%
% \end{markdown}
%  \begin{macrocode}
\def\markdownRendererJekyllDataSequenceBeginPrototype#1#2{
  \markdown_jekyll_data_push:nN
    { #1 }
    \c_@@_jekyll_data_sequence_tl
}
\def\markdownRendererJekyllDataMappingBeginPrototype#1#2{
  \markdown_jekyll_data_push:nN
    { #1 }
    \c_@@_jekyll_data_mapping_tl
}
\def\markdownRendererJekyllDataSequenceEndPrototype{
  \markdown_jekyll_data_pop:
}
\def\markdownRendererJekyllDataMappingEndPrototype{
  \markdown_jekyll_data_pop:
}
\def\markdownRendererJekyllDataBooleanPrototype#1#2{
  \markdown_jekyll_data_set_keyvals:nn
    { #1 }
    { #2 }
}
\def\markdownRendererJekyllDataEmptyPrototype#1{}
\def\markdownRendererJekyllDataNumberPrototype#1#2{
  \markdown_jekyll_data_set_keyvals:nn
    { #1 }
    { #2 }
}
%    \end{macrocode}
% \par
% \begin{markdown}
%
% We will process all string scalar values assuming that they may contain
% markdown markup and are intended for typesetting.
%
% \end{markdown}
%  \begin{macrocode}
\def\markdownRendererJekyllDataProgrammaticStringPrototype#1#2{}
\def\markdownRendererJekyllDataTypographicStringPrototype#1#2{
  \markdown_jekyll_data_set_keyvals:nn
    { #1 }
    { #2 }
}
\ExplSyntaxOff
%    \end{macrocode}
% \iffalse
%</themes-witiko-markdown-defaults-tex>
%<*tex>
% \fi
% \begin{markdown}
%
% If plain \TeX{} is the top layer, we load the `witiko/markdown/defaults`
% plain \TeX{} theme with the default definitions for token renderer
% prototypes unless the option `noDefaults` has been enabled (see Section
% <#sec:plain>).
%
% \end{markdown}
%  \begin{macrocode}
\ExplSyntaxOn
\str_if_eq:VVT
  \c_@@_top_layer_tl
  \c_@@_option_layer_plain_tex_tl
  {
    \ExplSyntaxOff
    \@@_if_option:nF
      { noDefaults }
      {
        \@@_setup:n
          {theme = witiko/markdown/defaults}
      }
    \ExplSyntaxOn
  }
\ExplSyntaxOff
%    \end{macrocode}
% \begin{markdown}
%
%### Lua Snippets
% After the \mdef{markdownPrepareLuaOptions} macro has been fully expanded,
% the \mdef{markdownLuaOptions} macro will expands to a Lua table that
% contains the plain \TeX{} options (see Section <#sec:tex-options>) in a
% format recognized by Lua (see Section <#sec:lua-options>).
%
% \end{markdown}
%  \begin{macrocode}
\ExplSyntaxOn
\tl_new:N \g_@@_formatted_lua_options_tl
\cs_new:Nn \@@_format_lua_options:
  {
    \tl_gclear:N
      \g_@@_formatted_lua_options_tl
    \seq_map_function:NN
      \g_@@_lua_options_seq
      \@@_format_lua_option:n
  }
\cs_new:Nn \@@_format_lua_option:n
  {
    \@@_typecheck_option:n
      { #1 }
    \@@_get_option_type:nN
      { #1 }
      \l_tmpa_tl
    \bool_case_true:nF
      {
        {
          \str_if_eq_p:VV
            \l_tmpa_tl
            \c_@@_option_type_boolean_tl ||
          \str_if_eq_p:VV
            \l_tmpa_tl
            \c_@@_option_type_number_tl ||
          \str_if_eq_p:VV
            \l_tmpa_tl
            \c_@@_option_type_counter_tl
        }
          {
            \@@_get_option_value:nN
              { #1 }
              \l_tmpa_tl
            \tl_gput_right:Nx
              \g_@@_formatted_lua_options_tl
              { #1~=~  \l_tmpa_tl   ,~ }
          }
        {
          \str_if_eq_p:VV
            \l_tmpa_tl
            \c_@@_option_type_clist_tl
        }
          {
            \@@_get_option_value:nN
              { #1 }
              \l_tmpa_tl
            \tl_gput_right:Nx
              \g_@@_formatted_lua_options_tl
              { #1~=~\c_left_brace_str }
            \clist_map_inline:Vn
              \l_tmpa_tl
              {
                \tl_gput_right:Nx
                  \g_@@_formatted_lua_options_tl
                  { "##1" ,~ }
              }
            \tl_gput_right:Nx
              \g_@@_formatted_lua_options_tl
              { \c_right_brace_str ,~ }
          }
      }
      {
        \@@_get_option_value:nN
          { #1 }
          \l_tmpa_tl
        \tl_gput_right:Nx
          \g_@@_formatted_lua_options_tl
          { #1~=~ " \l_tmpa_tl " ,~ }
      }
  }
\cs_generate_variant:Nn
  \clist_map_inline:nn
  { Vn }
\let\markdownPrepareLuaOptions=\@@_format_lua_options:
\def\markdownLuaOptions{{ \g_@@_formatted_lua_options_tl }}
\ExplSyntaxOff
%    \end{macrocode}
% \par
% \begin{markdown}
%
% The \mdef{markdownPrepare} macro contains the Lua code that is executed prior
% to any conversion from markdown to plain \TeX{}. It exposes the
% `convert` function for the use by any further Lua code.
%
% \end{markdown}
%  \begin{macrocode}
\def\markdownPrepare{%
%    \end{macrocode}
% \begin{markdown}
% First, ensure that the \Opt{cacheDir} directory exists.
% \end{markdown}
%  \begin{macrocode}
  local lfs = require("lfs")
  local cacheDir = "\markdownOptionCacheDir"
  if not lfs.isdir(cacheDir) then
    assert(lfs.mkdir(cacheDir))
  end
%    \end{macrocode}
% \begin{markdown}
% Next, load the `markdown` module and create a converter function using
% the plain \TeX{} options, which were serialized to a Lua table via the
% \mref{markdownLuaOptions} macro.
% \end{markdown}
%  \begin{macrocode}
  local md = require("markdown")
  local convert = md.new(\markdownLuaOptions)
}%
%    \end{macrocode}
% \par
% \begin{markdown}
%
% The \mdef{markdownCleanup} macro contains the Lua code that is executed
% after any conversion from markdown to plain \TeX{}.
%
% \end{markdown}
%  \begin{macrocode}
\def\markdownCleanup{%
%    \end{macrocode}
% \begin{markdown}
% Remove the `options.cacheDir` directory if it is empty.
% \end{markdown}
%  \begin{macrocode}
  lfs.rmdir(cacheDir)
}%
%    \end{macrocode}
% \par
% \begin{markdown}
%
%### Buffering Block-Level Markdown Input {#buffering-block}
%
% The macros \mdef{markdownInputFileStream} and \mdef{markdownOutputFileStream}
% contain the number of the input and output file streams that will be used for
% the IO operations of the package.
%
% \end{markdown}
%  \begin{macrocode}
\csname newread\endcsname\markdownInputFileStream
\csname newwrite\endcsname\markdownOutputFileStream
%    \end{macrocode}
% \par
% \begin{markdown}
%
% The \mdef{markdownReadAndConvertTab} macro contains the tab character literal.
%
% \end{markdown}
%  \begin{macrocode}
\begingroup
  \catcode`\^^I=12%
  \gdef\markdownReadAndConvertTab{^^I}%
\endgroup
%    \end{macrocode}
% \par
% \begin{markdown}
%
% The \mref{markdownReadAndConvert} macro is largely a rewrite of the
% \Hologo{LaTeX2e} `\filecontents` macro to plain \TeX{}.
%
% \end{markdown}
%  \begin{macrocode}
\begingroup
%    \end{macrocode}
% \begin{markdown}
% Make the newline and tab characters active and swap the character codes of
% the backslash symbol (`\`) and the pipe symbol (`|`), so that we can use the
% backslash as an ordinary character inside the macro definition.  Likewise,
% swap the character codes of the percent sign (`\%`) and the ampersand (`@`),
% so that we can remove percent signs from the beginning of lines when
% \Opt{stripPercentSigns} is enabled.
% \end{markdown}
%  \begin{macrocode}
  \catcode`\^^M=13%
  \catcode`\^^I=13%
  \catcode`|=0%
  \catcode`\\=12%
  |catcode`@=14%
  |catcode`|%=12@
  |gdef|markdownReadAndConvert#1#2{@
    |begingroup@
%    \end{macrocode}
% \begin{markdown}
% If we are not reading markdown documents from the frozen cache,
% open the \Opt{inputTempFileName} file for writing.
% \end{markdown}
%  \begin{macrocode}
    |markdownIfOption{frozenCache}{}{@
      |immediate|openout|markdownOutputFileStream@
        |markdownOptionInputTempFileName|relax@
      |markdownInfo{@
        Buffering block-level markdown input into the temporary @
        input file "|markdownOptionInputTempFileName" and scanning @
        for the closing token sequence "#1"}@
    }@
%    \end{macrocode}
% \begin{markdown}
% Locally change the category of the special plain \TeX{} characters to
% *other* in order to prevent unwanted interpretation of the input.  Change
% also the category of the space character, so that we can retrieve it
% unaltered.
% \end{markdown}
%  \begin{macrocode}
    |def|do##1{|catcode`##1=12}|dospecials@
    |catcode`| =12@
    |markdownMakeOther@
%    \end{macrocode}
% \begin{markdown}
% The \mdef{markdownReadAndConvertStripPercentSigns} macro will process the
% individual lines of output, stipping away leading percent signs (`\%`) when
% \Opt{stripPercentSigns} is enabled.
% Notice the use of the comments (`@`) to ensure that the entire macro is at
% a single line and therefore no (active) newline symbols
% (`^^M`) are produced.
% \end{markdown}
%  \begin{macrocode}
    |def|markdownReadAndConvertStripPercentSign##1{@
      |markdownIfOption{stripPercentSigns}{@
        |if##1%@
          |expandafter|expandafter|expandafter@
            |markdownReadAndConvertProcessLine@
        |else@
          |expandafter|expandafter|expandafter@
            |markdownReadAndConvertProcessLine@
            |expandafter|expandafter|expandafter##1@
        |fi@
      }{@
        |expandafter@
          |markdownReadAndConvertProcessLine@
          |expandafter##1@
      }@
    }@
%    \end{macrocode}
% \begin{markdown}
% The \mdef{markdownReadAndConvertProcessLine} macro will process the individual
% lines of output.
% Notice the use of the comments (`@`) to ensure that the entire macro is at
% a single line and therefore no (active) newline symbols
% (`^^M`) are produced.
% \end{markdown}
%  \begin{macrocode}
    |def|markdownReadAndConvertProcessLine##1#1##2#1##3|relax{@
%    \end{macrocode}
% \begin{markdown}
% If we are not reading markdown documents from the frozen cache and the ending
% token sequence does not appear in the line, store the line in the
% \Opt{inputTempFileName} file.
% If we are reading markdown documents from the frozen cache and the
% ending token sequence does not appear in the line, gobble the line.
% \end{markdown}
%  \begin{macrocode}
      |ifx|relax##3|relax@
        |markdownIfOption{frozenCache}{}{@
          |immediate|write|markdownOutputFileStream{##1}@
        }@
      |else@
%    \end{macrocode}
% \begin{markdown}
% When the ending token sequence appears in the line, make the next newline
% character close the \Opt{inputTempFileName} file, return the
% character categories back to the former state, convert the
% \Opt{inputTempFileName} file from markdown to plain \TeX{},
% `\input` the result of the conversion, and expand the ending control
% sequence.
% \end{markdown}
%  \begin{macrocode}
        |def^^M{@
          |markdownInfo{The ending token sequence was found}@
          |markdownIfOption{frozenCache}{}{@
            |immediate|closeout|markdownOutputFileStream@
          }@
          |endgroup@
          |markdownInput{@
            |markdownOptionOutputDir@
            /|markdownOptionInputTempFileName@
          }@
          #2}@
      |fi@
%    \end{macrocode}
% \begin{markdown}
% Repeat with the next line.
% \end{markdown}
%  \begin{macrocode}
      ^^M}@
%    \end{macrocode}
% \begin{markdown}
% Make the tab character active at expansion time and make it expand to a
% literal tab character.
% \end{markdown}
%  \begin{macrocode}
    |catcode`|^^I=13@
    |def^^I{|markdownReadAndConvertTab}@
%    \end{macrocode}
% \begin{markdown}
% Make the newline character active at expansion time and make it consume the
% rest of the line on expansion. Throw away the rest of the first line and
% pass the second line to the \mref{markdownReadAndConvertProcessLine} macro.
% \end{markdown}
%  \begin{macrocode}
    |catcode`|^^M=13@
    |def^^M##1^^M{@
      |def^^M####1^^M{@
        |markdownReadAndConvertStripPercentSign####1#1#1|relax}@
      ^^M}@
    ^^M}@
%    \end{macrocode}
% \begin{markdown}
% Reset the character categories back to the former state.
% \end{markdown}
%  \begin{macrocode}
|endgroup
%    \end{macrocode}
% \par
% \begin{markdown}
%
% Use the \pkg{lt3luabridge} library to define the \mdef{markdownLuaExecute}
% macro, which takes in a Lua scripts and expands to the standard output
% produced by its execution.
%
% \end{markdown}
%  \begin{macrocode}
\ExplSyntaxOn
\cs_new:Npn
  \markdownLuaExecute
  #1
  {
    \int_compare:nNnT
      { \g_luabridge_method_int }
      =
      { \c_luabridge_method_shell_int }
      {
        \sys_if_shell_unrestricted:F
          {
            \sys_if_shell:TF
              {
                \msg_error:nn
                  { markdown }
                  { restricted-shell-access }
              }
              {
                \msg_error:nn
                  { markdown }
                  { disabled-shell-access }
              }
          }
      }
    \str_gset:NV
      \g_luabridge_output_dirname_str
      \markdownOptionOutputDir
    \luabridge_now:e
      { #1 }
  }
\cs_generate_variant:Nn
  \msg_new:nnnn
  { nnnV }
\tl_set:Nn
  \l_tmpa_tl
  {
    You~may~need~to~run~TeX~with~the~--shell-escape~or~the~
    --enable-write18~flag,~or~write~shell_escape=t~in~the~
    texmf.cnf~file.
  }
\msg_new:nnnV
  { markdown }
  { restricted-shell-access }
  { Shell~escape~is~restricted }
  \l_tmpa_tl
\msg_new:nnnV
  { markdown }
  { disabled-shell-access }
  { Shell~escape~is~disabled }
  \l_tmpa_tl
\ExplSyntaxOff
%    \end{macrocode}
% \begin{markdown}
%
%### Buffering Inline Markdown Input {#buffering-inline}
%
% This section describes the implementation of the macro \mref{markinline}.
%
% \end{markdown}
%  \begin{macrocode}
\ExplSyntaxOn
\tl_new:N
  \g_@@_after_markinline_tl
\tl_gset:Nn
  \g_@@_after_markinline_tl
  { \unskip }
\cs_new:Npn
  \markinline
  {
%    \end{macrocode}
% \begin{markdown}
%
% Locally change the category of the special plain \TeX{} characters
% to *other* in order to prevent unwanted interpretation of the input
% markdown text as \TeX{} code.
%
% \end{markdown}
%  \begin{macrocode}
    \group_begin:
    \cctab_select:N
      \c_other_cctab
%    \end{macrocode}
% \begin{markdown}
%
% Unless we are reading markdown documents from the frozen cache,
% open the file \Opt{inputTempFileName} for writing.
%
% \end{markdown}
%  \begin{macrocode}
    \@@_if_option:nF
      { frozenCache }
      {
        \immediate
          \openout
          \markdownOutputFileStream
          \markdownOptionInputTempFileName
          \relax
        \msg_info:nne
          { markdown }
          { buffering-markinline }
          { \markdownOptionInputTempFileName }
      }
%    \end{macrocode}
% \begin{markdown}
%
% Peek ahead and extract the inline markdown text.
%
% \end{markdown}
%  \begin{macrocode}
    \peek_regex_replace_once:nnF
      { { (.*?) } }
      {
%    \end{macrocode}
% \begin{markdown}
%
% Unless we are reading markdown documents from the frozen cache,
% store the text in the file \Opt{inputTempFileName} and close it.
%
% \end{markdown}
%  \begin{macrocode}
        \c { @@_if_option:nF }
          \cB { frozenCache \cE }
          \cB {
            \c { immediate }
              \c { write }
              \c { markdownOutputFileStream }
              \cB { \1 \cE }
            \c { immediate }
              \c { closeout }
              \c { markdownOutputFileStream }
          \cE }
%    \end{macrocode}
% \begin{markdown}
%
% Reset the category codes and `\input` the result of the conversion.
%
% \end{markdown}
%  \begin{macrocode}
        \c { group_end: }
        \c { group_begin: }
        \c { @@_setup:n }
          \cB { contentLevel = inline \cE }
        \c { markdownInput }
          \cB {
            \c { markdownOptionOutputDir } /
            \c { markdownOptionInputTempFileName }
          \cE }
        \c { group_end: }
        \c { tl_use:N }
          \c { g_@@_after_markinline_tl }
      }
      {
        \msg_error:nn
          { markdown }
          { markinline-peek-failure }
        \group_end:
        \tl_use:N
          \g_@@_after_markinline_tl
      }
  }
\msg_new:nnn
  { markdown }
  { buffering-markinline }
  { Buffering~inline~markdown~input~into~the~temporary~input~file~"#1". }
\msg_new:nnnn
  { markdown }
  { markinline-peek-failure }
  { Use~of~\iow_char:N \\ markinline~doesn't~match~its~definition }
  { The~macro~should~be~followed~by~inline~markdown~text~in~curly~braces }
\ExplSyntaxOff
%    \end{macrocode}
% \begin{markdown}
%
%### Typesetting Markdown
%
% The \mref{markdownInput} macro uses an implementation of the
% \mref{markdownLuaExecute} macro to convert the contents of the file whose
% filename it has received as its single argument from markdown to plain
% \TeX{}.
%
% \end{markdown}
%  \begin{macrocode}
\ExplSyntaxOn
\cs_new:Npn
  \markdownInput
  #1
  {
%    \end{macrocode}
% \begin{markdown}
%
% If the file does not exist in the current directory, we will search for it in
% the directories specified in \mref{l_file_search_path_seq}. On \LaTeX, this
% also includes the directories specified in \mref{input@path}.
%
% \end{markdown}
%  \begin{macrocode}
    \file_get_full_name:nNTF
      { #1 }
      \l_tmpa_tl
      {
        \exp_args:NV
          \markdownInputRaw
          \l_tmpa_tl
      }
      {
        \msg_error:nnnV
          { markdown }
          { markdown-file-does-not-exist }
          { #1 }
      }
  }
\msg_new:nnn
  { markdown }
  { markdown-file-does-not-exist }
  {
    Markdown~file~#1~does~not~exist
  }
\ExplSyntaxOff
\begingroup
%    \end{macrocode}
% \begin{markdown}
% Swap the category code of the backslash symbol and the pipe symbol, so that
% we may use the backslash symbol freely inside the Lua code. Furthermore,
% use the ampersand symbol to specify parameters.
% \end{markdown}
%  \begin{macrocode}
  \catcode`|=0%
  \catcode`\\=12%
  \catcode`|&=6%
  |gdef|markdownInputRaw#1{%
%    \end{macrocode}
% \begin{markdown}
% Change the category code of the percent sign (`\%`) to other, so that a user
% of the \Opt{hybrid} Lua option or a malevolent actor can't produce TeX
% comments in the plain TeX output of the Markdown package.
% \end{markdown}
%  \begin{macrocode}
    |begingroup
    |catcode`|%=12
%    \end{macrocode}
% \begin{markdown}
% Furthermore, also change the category code of the hash sign (`#`) to other,
% so that it's safe to tokenize the plain TeX output without mistaking hash
% signs with TeX's parameter numbers.
% \end{markdown}
%  \begin{macrocode}
    |catcode`|#=12
%    \end{macrocode}
% \begin{markdown}
% If we are reading from the frozen cache, input it, expand the corresponding
% `\markdownFrozenCache`\meta{number} macro, and increment
% \Opt{frozenCacheCounter}.
% \end{markdown}
%  \begin{macrocode}
    |markdownIfOption{frozenCache}{%
      |ifnum|markdownOptionFrozenCacheCounter=0|relax
        |markdownInfo{Reading frozen cache from
          "|markdownOptionFrozenCacheFileName"}%
        |input|markdownOptionFrozenCacheFileName|relax
      |fi
      |markdownInfo{Including markdown document number
        "|the|markdownOptionFrozenCacheCounter" from frozen cache}%
      |csname markdownFrozenCache|the|markdownOptionFrozenCacheCounter|endcsname
      |global|advance|markdownOptionFrozenCacheCounter by 1|relax
    }{%
      |markdownInfo{Including markdown document "&1"}%
%    \end{macrocode}
% \begin{markdown}
% Attempt to open the markdown document to record it in the `.log` and
% `.fls` files. This allows external programs such as \LaTeX Mk to track
% changes to the markdown document.
% \end{markdown}
%  \begin{macrocode}
      |openin|markdownInputFileStream&1
      |closein|markdownInputFileStream
      |markdownPrepareLuaOptions
      |markdownLuaExecute{%
        |markdownPrepare
        local file = assert(io.open("&1", "r"),
          [[Could not open file "&1" for reading]])
        local input = assert(file:read("*a"))
        assert(file:close())
        print(convert(input))
        |markdownCleanup}%
%    \end{macrocode}
% \begin{markdown}
% If we are finalizing the frozen cache, increment \Opt{frozenCacheCounter}.
% \end{markdown}
%  \begin{macrocode}
      |markdownIfOption{finalizeCache}{%
        |global|advance|markdownOptionFrozenCacheCounter by 1|relax}{}%
    }%
    |endgroup
  }%
|endgroup
%    \end{macrocode}
% \par
% \begin{markdown}
% The \mref{markdownEscape} macro resets the category codes of the percent sign
% and the hash sign back to comment and parameter, respectively, before using
% the `\input` built-in of \TeX{} to execute a \TeX{} document in the middle of
% a markdown document fragment.
% \end{markdown}
%  \begin{macrocode}
\gdef\markdownEscape#1{%
  \catcode`\%=14\relax
  \catcode`\#=6\relax
  \input #1\relax
  \catcode`\%=12\relax
  \catcode`\#=12\relax
}%
%    \end{macrocode}
% \iffalse
%</tex>
%<*latex>
% \fi
% \par
% \begin{markdown}
%
% \LaTeX{} Implementation {#lateximplementation}
%-------------------------
%
% The \LaTeX{} implementation makes use of the fact that, apart from some subtle
% differences, \LaTeX{} implements the majority of the plain \TeX{}
% format~[@latex17, Section 9]. As a consequence, we can directly reuse the
% existing plain \TeX{} implementation.
%
% \end{markdown}
%  \begin{macrocode}
\def\markdownVersionSpace{ }%
\ProvidesPackage{markdown}[\markdownLastModified\markdownVersionSpace v%
  \markdownVersion\markdownVersionSpace markdown renderer]%
%    \end{macrocode}
% \par
% \begin{markdown}
%
%### Typesetting Markdown
% The \mdef{markinlinePlainTeX} macro is used to store the original plain
% \TeX{} implementation of the \mref{markinline} macro. The \mref{markinline}
% macro is then redefined to accept an optional argument with options
% recognized by the \LaTeX{} interface (see Section <#sec:latex-options>).
%
% \end{markdown}
%  \begin{macrocode}
\ExplSyntaxOn
\cs_gset_eq:NN
  \markinlinePlainTeX
  \markinline
\cs_gset:Npn
  \markinline
  {
    \peek_regex_replace_once:nn
      { ( \[ (.*?) \] ) ? }
      {
%    \end{macrocode}
% \par
% \begin{markdown}
%
% Apply the options locally.
%
% \end{markdown}
%  \begin{macrocode}
        \c { group_begin: }
        \c { @@_setup:n }
          \cB { \2 \cE }
        \c { tl_put_right:Nn }
          \c { g_@@_after_markinline_tl }
          \cB { \c { group_end: } \cE }
        \c { markinlinePlainTeX }
      }
  }
\ExplSyntaxOff
%    \end{macrocode}
% \par
% \begin{markdown}
%
% The \mdef{markdownInputPlainTeX} macro is used to store the original plain
% \TeX{} implementation of the \mref{markdownInput} macro. The \mref{markdownInput}
% macro is then redefined to accept an optional argument with options recognized by
% the \LaTeX{} interface (see Section <#sec:latex-options>).
%
% \end{markdown}
%  \begin{macrocode}
\let\markdownInputPlainTeX\markdownInput
\renewcommand\markdownInput[2][]{%
  \begingroup
    \markdownSetup{#1}%
    \markdownInputPlainTeX{#2}%
  \endgroup}%
%    \end{macrocode}
% \par
% \begin{markdown}
%
% The \envmref{markdown}, and \envmref{markdown*} \LaTeX{} environments are
% implemented using the \mref{markdownReadAndConvert} macro.
%
% \end{markdown}
%  \begin{macrocode}
\ExplSyntaxOn
\renewenvironment
  { markdown }
  {
%    \end{macrocode}
% \markdownBegin
%
% In our implementation of the \envmref{markdown} \LaTeX{} environment, we
% want to distinguish between the following two cases:
% ``` tex
% \begin{markdown} [smartEllipses]   \begin{markdown}
% \% This is an optional argument ^     [smartEllipses]
% \% ...                              \% ^ This is link
% \end{markdown}                     \end{markdown}
% ```````
%
% Therefore, we cannot use the built-in \LaTeX{} support for environments
% with optional arguments or packages such as \pkg{xparse}. Instead, we
% must read the optional argument manually and prevent reading past the
% end of a line.
%
% To prevent reading past the end of a line when looking for the optional
% argument of the \envmref{markdown} \LaTeX{} environment and accidentally
% tokenizing markdown text, we change the category code of carriage return
% (`\r`, ASCII character 13 in decimal) from 5 (end of line).
%
% While any category code other than 5 (end of line) would work, we switch to
% the category 13 (active), which is also used by the
% \mref{markdownReadAndConvert} macro. This is necessary if we read until the
% end of a line, because then the carriage return character will be
% produced by \TeX{} via the `\endlinechar` plain \TeX{} macro and it needs
% to have the correct category code, so that \mref{markdownReadAndConvert}
% processes it correctly.
%
% \markdownEnd
%  \begin{macrocode}
    \group_begin:
    \char_set_catcode_active:n { 13 }
%    \end{macrocode}
% \begin{markdown}
%
% To prevent doubling the hash signs (`#`, ASCII code 35 in decimal), we switch
% its category from 6 (parameter) to 12 (letter).
%
% \end{markdown}
%  \begin{macrocode}
    \char_set_catcode_letter:n { 35 }
%    \end{macrocode}
% \begin{markdown}
%
% After we have matched the opening `[` that begins the optional argument,
% we accept carriage returns as well.
%
% \end{markdown}
%  \begin{macrocode}
    \peek_regex_replace_once:nnF
      { \ *\[\r*([^]]*)\][^\r]* }
      {
%    \end{macrocode}
% \begin{markdown}
%
% After we have matched the optional argument, we switch back the category
% code of carriage returns and hash signs and we retokenize the content. This
% will cause single new lines to produce a space token and multiple new lines
% to produce `\par` tokens. Furthermore, this will cause hash signs followed
% by a number to be recognized as parameter numbers, which is necessary when
% we use the optional argument to redefine token renderers and token renderer
% prototypes.
%
% \end{markdown}
%  \begin{macrocode}
        \c { group_end: }
        \c { tl_set_rescan:Nnn } \c { l_tmpa_tl } { } { \1 }
%    \end{macrocode}
% \begin{markdown}
%
% Then, we pass the retokenized content to the \mref{markdownSetup} macro.
%
% \end{markdown}
%  \begin{macrocode}
        \c { @@_setup:V } \c { l_tmpa_tl }
%    \end{macrocode}
% \begin{markdown}
%
% Finally, regardless of whether or not we have matched the optional argument,
% we let the \mref{markdownReadAndConvert} macro process the rest of the
% \LaTeX{} environment.
%
% \end{markdown}
%  \begin{macrocode}
        \c { markdownReadAndConvert@markdown } { }
      }
      {
        \group_end:
        \markdownReadAndConvert@markdown { }
      }
  }
  { \markdownEnd }
\renewenvironment
  { markdown* }
  [ 1 ]
  {
    \msg_warning:nnn
      { markdown }
      { latex-markdown-star-deprecated }
      { #1 }
    \@@_setup:n
      { #1 }
    \markdownReadAndConvert@markdown *
  }
  { \markdownEnd }
\msg_new:nnn
  { markdown }
  { latex-markdown-star-deprecated }
  {
    The~markdown*~LaTeX~environment~has~been~deprecated~and~will~
    be~removed~in~the~next~major~version~of~the~Markdown~package.
  }
\ExplSyntaxOff
\begingroup
%    \end{macrocode}
% \begin{markdown}
% Locally swap the category code of the backslash symbol with the pipe symbol,
% and of the left (`{`) and right brace (`}`) with the less-than (`<`)
% and greater-than (`>`) signs. This is required in order that all the
% special symbols that appear in the first argument of the
% `markdownReadAndConvert` macro have the category code *other*.
% \end{markdown}
%  \begin{macrocode}
  \catcode`\|=0\catcode`\<=1\catcode`\>=2%
  \catcode`\\=12|catcode`|{=12|catcode`|}=12%
  |gdef|markdownReadAndConvert@markdown#1<%
    |markdownReadAndConvert<\end{markdown#1}>%
                           <|end<markdown#1>>>%
|endgroup
%    \end{macrocode}
% \par
% \begin{markdown}
%
%### Options
% The supplied package options are processed using the \mref{markdownSetup} macro.
%
% \end{markdown}
%  \begin{macrocode}
\DeclareOption*{%
  \expandafter\markdownSetup\expandafter{\CurrentOption}}%
\ProcessOptions\relax
%    \end{macrocode}
% \begin{markdown}
%
%### Themes {#latex-themes-implementation}
%
% This section overrides the plain \TeX{} implementation of the theme-loading
% mechanism from Section <#sec:themes-implementation>. Furthermore, this section
% also implements the built-in \LaTeX{} themes provided with the Markdown package.
%
% \end{markdown}
%  \begin{macrocode}
\ExplSyntaxOn
\cs_gset:Nn
  \@@_load_theme:nn
  {
%    \end{macrocode}
% \par
% \begin{markdown}
%
% If the Markdown package has already been loaded, determine whether
% a file named `markdowntheme`\meta{munged theme name}`.sty` exists
% and whether we are still in the preamble.
%
% \end{markdown}
%  \begin{macrocode}
    \ifmarkdownLaTeXLoaded
      \ifx\@onlypreamble\@notprerr
%    \end{macrocode}
% \begin{markdown}
%
% If both conditions are true does, end with an error, since we cannot load
% \LaTeX{} themes after the preamble. Otherwise, try loading a plain \TeX{}
% theme instead.
%
% \end{markdown}
%  \begin{macrocode}
        \file_if_exist:nTF
          { markdown theme #2.sty }
          {
            \msg_error:nnn
              { markdown }
              { latex-theme-after-preamble }
              { #1 }
          }
          {
            \@@_plain_tex_load_theme:nn
              { #1 }
              { #2 }
          }
      \else
%    \end{macrocode}
% \begin{markdown}
%
% If the Markdown package has already been loaded but we are still in the
% preamble, load a \LaTeX{} theme if it exists or load a plain \TeX{} theme
% otherwise.
%
% \end{markdown}
%  \begin{macrocode}
        \file_if_exist:nTF
          { markdown theme #2.sty }
          {
            \msg_info:nnn
              { markdown }
              { loading-latex-theme }
              { #1 }
            \RequirePackage
              { markdown theme #2 }
          }
          {
            \@@_plain_tex_load_theme:nn
              { #1 }
              { #2 }
          }
      \fi
    \else
%    \end{macrocode}
% \begin{markdown}
%
% If the Markdown package has not yet been loaded, postpone the loading until
% the Markdown package has finished loading.
%
% \end{markdown}
%  \begin{macrocode}
      \msg_info:nnn
        { markdown }
        { theme-loading-postponed }
        { #1 }
      \AtEndOfPackage
        {
          \@@_load_theme:nn
            { #1 }
            { #2 }
        }
    \fi
  }
\msg_new:nnn
  { markdown }
  { theme-loading-postponed }
  {
    Postponing~loading~Markdown~theme~#1~until~
    Markdown~package~has~finished~loading
  }
\msg_new:nnn
  { markdown }
  { loading-latex-theme }
  { Loading~LaTeX~Markdown~theme~#1 }
\cs_generate_variant:Nn
  \msg_new:nnnn
  { nnVV }
\tl_set:Nn
  \l_tmpa_tl
  { Cannot~load~LaTeX~Markdown~theme~#1~after~ }
\tl_put_right:NV
  \l_tmpa_tl
  \c_backslash_str
\tl_put_right:Nn
  \l_tmpa_tl
  { begin{document} }
\tl_set:Nn
  \l_tmpb_tl
  { Load~Markdown~theme~#1~before~ }
\tl_put_right:NV
  \l_tmpb_tl
  \c_backslash_str
\tl_put_right:Nn
  \l_tmpb_tl
  { begin{document} }
\msg_new:nnVV
  { markdown }
  { latex-theme-after-preamble }
  \l_tmpa_tl
  \l_tmpb_tl
\ExplSyntaxOff
%    \end{macrocode}
% \begin{markdown}
%
% The `witiko/dot` theme enables the \Opt{fencedCode} Lua option:
%
% \end{markdown}
% \iffalse
%</latex>
%<*themes-witiko-dot>
% \fi
%  \begin{macrocode}
\markdownSetup{fencedCode}%
%    \end{macrocode}
% \begin{markdown}
%
% We load the \pkg{ifthen} and \pkg{grffile} packages, see also
% Section <#sec:latex-prerequisites>:
%
% \end{markdown}
%  \begin{macrocode}
\RequirePackage{ifthen,grffile}
%    \end{macrocode}
% \begin{markdown}
%
% We store the previous definition of the fenced code token renderer prototype:
%
% \end{markdown}
%  \begin{macrocode}
\let\markdown@witiko@dot@oldRendererInputFencedCodePrototype
  \markdownRendererInputFencedCodePrototype
%    \end{macrocode}
% \begin{markdown}
%
% If the infostring starts with `dot …`, we redefine the fenced code block
% token renderer prototype, so that it typesets the code block via Graphviz
% tools if and only if the \Opt{frozenCache} plain \TeX{} option is
% disabled and the code block has not been previously typeset:
%
% \end{markdown}
%  \begin{macrocode}
\renewcommand\markdownRendererInputFencedCodePrototype[3]{%
  \def\next##1 ##2\relax{%
    \ifthenelse{\equal{##1}{dot}}{%
      \markdownIfOption{frozenCache}{}{%
        \immediate\write18{%
          if ! test -e #1.pdf.source || ! diff #1 #1.pdf.source;
          then
            dot -Tpdf -o #1.pdf #1;
            cp #1 #1.pdf.source;
          fi}}%
%    \end{macrocode}
% \begin{markdown}
%
% We include the typeset image using the image token renderer:
%
% \end{markdown}
%  \begin{macrocode}
      \markdownRendererImage{Graphviz image}{#1.pdf}{#1.pdf}{##2}%
%    \end{macrocode}
% \begin{markdown}
%
% If the infostring does not start with `dot …`, we use the previous definition
% of the fenced code token renderer prototype:
%
% \end{markdown}
%  \begin{macrocode}
    }{%
      \markdown@witiko@dot@oldRendererInputFencedCodePrototype{#1}{#2}{#3}%
    }%
  }%
  \next#2 \relax}%
%    \end{macrocode}
% \iffalse
%</themes-witiko-dot>
%<*themes-witiko-graphicx-http>
% \fi
% \par
% \begin{markdown}
%
% The `witiko/graphicx/http` theme stores the previous definition of the image
% token renderer prototype:
%
% \end{markdown}
%  \begin{macrocode}
\let\markdown@witiko@graphicx@http@oldRendererImagePrototype
  \markdownRendererImagePrototype
%    \end{macrocode}
% \begin{markdown}
%
% We load the \pkg{catchfile} and \pkg{grffile} packages, see also
% Section <#sec:latex-prerequisites>:
%
% \end{markdown}
%  \begin{macrocode}
\RequirePackage{catchfile,grffile}
%    \end{macrocode}
% \begin{markdown}
%
% We define the \mdef{markdown@witiko@graphicx@http@counter} counter to enumerate
% the images for caching and the \mdef{markdown@witiko@graphicx@http@filename}
% command, which will store the pathname of the file containing the pathname
% of the downloaded image file.
%
% \end{markdown}
%  \begin{macrocode}
\newcount\markdown@witiko@graphicx@http@counter
\markdown@witiko@graphicx@http@counter=0
\newcommand\markdown@witiko@graphicx@http@filename{%
  \markdownOptionCacheDir/witiko_graphicx_http%
  .\the\markdown@witiko@graphicx@http@counter}%
%    \end{macrocode}
% \begin{markdown}
%
% We define the \mdef{markdown@witiko@graphicx@http@download} command, which will
% receive two arguments that correspond to the URL of the online image and to
% the pathname, where the online image should be downloaded. The command will
% produce a shell command that tries to downloads the online image to the
% pathname.
%
% \end{markdown}
%  \begin{macrocode}
\newcommand\markdown@witiko@graphicx@http@download[2]{%
  wget -O #2 #1 || curl --location -o #2 #1 || rm -f #2}
%    \end{macrocode}
% \begin{markdown}
%
% We locally swap the category code of the percentage sign with the line feed
% control character, so that we can use percentage signs in the shell code:
%
% \end{markdown}
%  \begin{macrocode}
\begingroup
\catcode`\%=12
\catcode`\^^A=14
%    \end{macrocode}
% \begin{markdown}
%
% We redefine the image token renderer prototype, so that it tries to download
% an online image.
%
% \end{markdown}
%  \begin{macrocode}
\global\def\markdownRendererImagePrototype#1#2#3#4{^^A
  \begingroup
    \edef\filename{\markdown@witiko@graphicx@http@filename}^^A
%    \end{macrocode}
% \begin{markdown}
%
% The image will be downloaded only if the image URL has the http or https
% protocols and the \Opt{frozenCache} plain \TeX{} option is disabled:
%
% \end{markdown}
%  \begin{macrocode}
    \markdownIfOption{frozenCache}{}{^^A
      \immediate\write18{^^A
        mkdir -p "\markdownOptionCacheDir";
        if printf '%s' "#3" | grep -q -E '^https?:';
        then
%    \end{macrocode}
% \begin{markdown}
%
% The image will be downloaded to the pathname \Opt{cacheDir}<!--
% -->`/`\meta{the MD5 digest of the image URL}`.`\meta{the suffix of the
% image URL}:
%
% \end{markdown}
%  \begin{macrocode}
          OUTPUT_PREFIX="\markdownOptionCacheDir";
          OUTPUT_BODY="$(printf '%s' '#3' | md5sum | cut -d' ' -f1)";
          OUTPUT_SUFFIX="$(printf '%s' '#3' | sed 's/.*[.]//')";
          OUTPUT="$OUTPUT_PREFIX/$OUTPUT_BODY.$OUTPUT_SUFFIX";
%    \end{macrocode}
% \begin{markdown}
%
% The image will be downloaded only if it has not already been downloaded:
%
% \end{markdown}
%  \begin{macrocode}
          if ! [ -e "$OUTPUT" ];
          then
            \markdown@witiko@graphicx@http@download{'#3'}{"$OUTPUT"};
            printf '%s' "$OUTPUT" > "\filename";
          fi;
%    \end{macrocode}
% \begin{markdown}
%
% If the image does not have the http or https protocols or the image has
% already been downloaded, the URL will be stored as-is:
%
% \end{markdown}
%  \begin{macrocode}
        else
          printf '%s' '#3' > "\filename";
        fi}}^^A
%    \end{macrocode}
% \begin{markdown}
%
% We load the pathname of the downloaded image and we typeset the image using
% the previous definition of the image renderer prototype:
%
% \end{markdown}
%  \begin{macrocode}
    \CatchFileDef{\filename}{\filename}{\endlinechar=-1}^^A
    \markdown@witiko@graphicx@http@oldRendererImagePrototype^^A
      {#1}{#2}{\filename}{#4}^^A
  \endgroup
  \global\advance\markdown@witiko@graphicx@http@counter by 1\relax}^^A
\endgroup
%    \end{macrocode}
% \iffalse
%</themes-witiko-graphicx-http>
%<*themes-witiko-markdown-defaults-latex>
% \fi
% \par
% \begin{markdown}
%
% The `witiko/markdown/defaults` \LaTeX{} theme provides default definitions
% for token renderer prototypes. First, the \LaTeX{} theme loads the plain
% \TeX{} theme with the default definitions for plain \TeX{}:
%
% \end{markdown}
%  \begin{macrocode}
\markdownLoadPlainTeXTheme
%    \end{macrocode}
% \par
% \begin{markdown}
%
% Next, the \LaTeX{} theme overrides some of the plain \TeX{} definitions.
% See Section <#sec:latex-token-renderer-prototypes> for the actual
% definitions.
%
%### Token Renderer Prototypes {#latex-token-renderer-prototypes}
%
% The following configuration should be considered placeholder. If the option
% `plain` has been enabled (see Section <#sec:plain>), none of the definitions
% will take effect.
% \end{markdown}
%  \begin{macrocode}
\markdownIfOption{plain}{\iffalse}{\iftrue}
%    \end{macrocode}
% \par
% \begin{markdown}%
% If either the \Opt{tightLists} or the \Opt{fancyLists} Lua option is enabled
% and the current document class is not \pkg{beamer}, then load the
% \pkg{paralist} package.
%
% \end{markdown}
%  \begin{macrocode}
\@ifclassloaded{beamer}{}{%
  \markdownIfOption{tightLists}{\RequirePackage{paralist}}{}%
  \markdownIfOption{fancyLists}{\RequirePackage{paralist}}{}%
}
%    \end{macrocode}
% \par
% \begin{markdown}
%
% If we loaded the \pkg{paralist} package, define the respective renderer
% prototypes to make use of the capabilities of the package. Otherwise,
% define the renderer prototypes to fall back on the corresponding renderers
% for the non-tight lists.
%
% \end{markdown}
%  \begin{macrocode}
\ExplSyntaxOn
\@ifpackageloaded{paralist}{
  \tl_new:N
    \l_@@_latex_fancy_list_item_label_number_style_tl
  \tl_new:N
    \l_@@_latex_fancy_list_item_label_delimiter_style_tl
  \cs_new:Nn
    \@@_latex_fancy_list_item_label_number:nn
    {
      \str_case:nn
        { #1 }
        {
          { Decimal } { #2 }
          { LowerRoman } { \int_to_roman:n { #2 } }
          { UpperRoman } { \int_to_Roman:n { #2 } }
          { LowerAlpha } { \int_to_alph:n { #2 } }
          { UpperAlpha } { \int_to_Alph:n { #2 } }
        }
    }
  \cs_new:Nn
    \@@_latex_fancy_list_item_label_delimiter:n
    {
      \str_case:nn
        { #1 }
        {
          { Default } { . }
          { OneParen } { ) }
          { Period } { . }
        }
    }
  \cs_new:Nn
    \@@_latex_fancy_list_item_label:nnn
    {
      \@@_latex_fancy_list_item_label_number:nn
        { #1 }
        { #3 }
      \@@_latex_fancy_list_item_label_delimiter:n
        { #2 }
    }
  \cs_new:Nn
    \@@_latex_paralist_style:nn
    {
      \str_case:nn
        { #1 }
        {
          { Decimal } { 1 }
          { LowerRoman } { i }
          { UpperRoman } { I }
          { LowerAlpha } { a }
          { UpperAlpha } { A }
        }
      \@@_latex_fancy_list_item_label_delimiter:n
        { #2 }
    }
  \markdownSetup{rendererPrototypes={
%    \end{macrocode}
% \par
% \begin{markdown}
%
% Make tight bullet lists a little less compact by adding extra vertical space
% above and below them.
%
% \end{markdown}
%  \begin{macrocode}
    ulBeginTight = {%
      \group_begin:
      \pltopsep=\topsep
      \plpartopsep=\partopsep
      \begin{compactitem}
    },
    ulEndTight = {
      \end{compactitem}
      \group_end:
    },
    fancyOlBegin = {
      \group_begin:
      \tl_set:Nn
        \l_@@_latex_fancy_list_item_label_number_style_tl
        { #1 }
      \tl_set:Nn
        \l_@@_latex_fancy_list_item_label_delimiter_style_tl
        { #2 }
      \@@_if_option:nTF
        { startNumber }
        {
          \tl_set:Nn
            \l_tmpa_tl
            { \begin{enumerate} }
        }
        {
          \tl_set:Nn
            \l_tmpa_tl
            { \begin{enumerate}[ }
          \tl_put_right:Nx
            \l_tmpa_tl
            { \@@_latex_paralist_style:nn { #1 } { #2 } }
          \tl_put_right:Nn
            \l_tmpa_tl
            { ] }
        }
      \tl_use:N
        \l_tmpa_tl
    },
    fancyOlEnd = {
      \end{enumerate}
      \group_end:
    },
%    \end{macrocode}
% \par
% \begin{markdown}
%
% Make tight ordered lists a little less compact by adding extra vertical
% space above and below them.
%
% \end{markdown}
%  \begin{macrocode}
    olBeginTight = {%
      \group_begin:
      \plpartopsep=\partopsep
      \pltopsep=\topsep
      \begin{compactenum}
    },
    olEndTight = {
      \end{compactenum}
      \group_end:
    },
    fancyOlBeginTight = {
      \group_begin:
      \tl_set:Nn
        \l_@@_latex_fancy_list_item_label_number_style_tl
        { #1 }
      \tl_set:Nn
        \l_@@_latex_fancy_list_item_label_delimiter_style_tl
        { #2 }
      \@@_if_option:nTF
        { startNumber }
        {
          \tl_set:Nn
            \l_tmpa_tl
            { \begin{compactenum} }
        }
        {
          \tl_set:Nn
            \l_tmpa_tl
            { \begin{compactenum}[ }
          \tl_put_right:Nx
            \l_tmpa_tl
            { \@@_latex_paralist_style:nn { #1 } { #2 } }
          \tl_put_right:Nn
            \l_tmpa_tl
            { ] }
        }
      \tl_put_left:Nn
        \l_tmpa_tl
        {
          \plpartopsep=\partopsep
          \pltopsep=\topsep
        }
      \tl_use:N
        \l_tmpa_tl
    },
    fancyOlEndTight = {
      \end{compactenum}
      \group_end:
    },
    fancyOlItemWithNumber = {
      \item
        [
          \@@_latex_fancy_list_item_label:VVn
            \l_@@_latex_fancy_list_item_label_number_style_tl
            \l_@@_latex_fancy_list_item_label_delimiter_style_tl
            { #1 }
        ]
    },
%    \end{macrocode}
% \par
% \begin{markdown}
%
% Make tight definition lists a little less compact by adding extra
% vertical space above and below them.
%
% \end{markdown}
%  \begin{macrocode}
    dlBeginTight = {
      \group_begin:
      \plpartopsep=\partopsep
      \pltopsep=\topsep
      \begin{compactdesc}
    },
    dlEndTight = {
      \end{compactdesc}
      \group_end:
    }}}
  \cs_generate_variant:Nn
    \@@_latex_fancy_list_item_label:nnn
    { VVn }
}{
  \markdownSetup{rendererPrototypes={
    ulBeginTight = {\markdownRendererUlBegin},
    ulEndTight = {\markdownRendererUlEnd},
    fancyOlBegin = {\markdownRendererOlBegin},
    fancyOlEnd = {\markdownRendererOlEnd},
    olBeginTight = {\markdownRendererOlBegin},
    olEndTight = {\markdownRendererOlEnd},
    fancyOlBeginTight = {\markdownRendererOlBegin},
    fancyOlEndTight = {\markdownRendererOlEnd},
    dlBeginTight = {\markdownRendererDlBegin},
    dlEndTight = {\markdownRendererDlEnd}}}
}
\ExplSyntaxOff
\RequirePackage{amsmath}
%    \end{macrocode}
% \par
% \begin{markdown}
%
% Unless the \pkg{unicode-math} package has been loaded, load the \pkg{amssymb}
% package with symbols to be used for tickboxes.
%
% \end{markdown}
%  \begin{macrocode}
\@ifpackageloaded{unicode-math}{
  \markdownSetup{rendererPrototypes={
    untickedBox = {$\mdlgwhtsquare$},
  }}
}{
  \RequirePackage{amssymb}
  \markdownSetup{rendererPrototypes={
    untickedBox = {$\square$},
  }}
}
\RequirePackage{csvsimple}
\RequirePackage{fancyvrb}
\RequirePackage{graphicx}
\markdownSetup{rendererPrototypes={
  hardLineBreak = {\\},
  leftBrace = {\textbraceleft},
  rightBrace = {\textbraceright},
  dollarSign = {\textdollar},
  underscore = {\textunderscore},
  circumflex = {\textasciicircum},
  backslash = {\textbackslash},
  tilde = {\textasciitilde},
  pipe = {\textbar},
%    \end{macrocode}
% \par
% \begin{markdown}
%
% We can capitalize on the fact that the expansion of renderers is performed by
% \TeX{} during the typesetting. Therefore, even if we don't know whether a
% span of text is part of math formula or not when we are parsing markdown,%
% ^[This property may actually be undecidable. Suppose a span of text is a part
% of a macro definition.  Then, whether the span of text is part of a math
% formula or not depends on where the macro is later used, which may easily
% be *both* inside and outside a math formula.] we can reliably detect math
% mode inside the renderer.
%
% Here, we will redefine the code span renderer prototype to typeset upright
% text in math formulae and typewriter text outside math formulae.
%
% \end{markdown}
%  \begin{macrocode}
  codeSpan = {%
    \ifmmode
      \text{#1}%
    \else
      \texttt{#1}%
    \fi
  }}}
\ExplSyntaxOn
\markdownSetup{
  rendererPrototypes = {
    contentBlock = {
      \str_case:nnF
        { #1 }
        {
          { csv }
            {
              \begin{table}
                \begin{center}
                  \csvautotabular{#3}
                \end{center}
                \tl_if_empty:nF
                  { #4 }
                  { \caption{#4} }
              \end{table}
            }
          { tex } { \markdownEscape{#3} }
        }
        { \markdownInput{#3} }
    },
  },
}
\ExplSyntaxOff
\markdownSetup{rendererPrototypes={
  image = {%
    \begin{figure}%
      \begin{center}%
        \includegraphics[alt={#1}]{#3}%
      \end{center}%
      \ifx\empty#4\empty\else
        \caption{#4}%
      \fi
    \end{figure}},
  ulBegin = {\begin{itemize}},
  ulEnd = {\end{itemize}},
  olBegin = {\begin{enumerate}},
  olItem = {\item{}},
  olItemWithNumber = {\item[#1.]},
  olEnd = {\end{enumerate}},
  dlBegin = {\begin{description}},
  dlItem = {\item[#1]},
  dlEnd = {\end{description}},
  emphasis = {\emph{#1}},
  tickedBox = {$\boxtimes$},
  halfTickedBox = {$\boxdot$}}}
%    \end{macrocode}
% \par
% \begin{markdown}
%
% If \acro{HTML} identifiers appear after a heading, we make them
% produce `\label` macros.
%
% \end{markdown}
%  \begin{macrocode}
\ExplSyntaxOn
\seq_new:N
  \l_@@_header_identifiers_seq
\markdownSetup
  {
    rendererPrototypes = {
      headerAttributeContextBegin = {
        \markdownSetup
          {
            rendererPrototypes = {
              attributeIdentifier = {
                \seq_put_right:Nn
                  \l_@@_header_identifiers_seq
                  { ##1 }
              },
            },
          }
      },
      headerAttributeContextEnd = {
        \seq_map_inline:Nn
          \l_@@_header_identifiers_seq
          { \label { ##1 } }
        \seq_clear:N
          \l_@@_header_identifiers_seq
      },
    },
  }
%    \end{macrocode}
% \begin{markdown}
%
% If the `unnumbered` \acro{HTML} class (or the `{-}` shorthand) appears after
% a heading the heading and all its subheadings will be unnumbered.
%
% \end{markdown}
%  \begin{macrocode}
\bool_new:N
  \l_@@_header_unnumbered_bool
\markdownSetup
  {
    rendererPrototypes = {
      headerAttributeContextBegin += {
        \markdownSetup
          {
            rendererPrototypes = {
              attributeClassName = {
                \bool_if:nT
                  {
                    \str_if_eq_p:nn
                      { ##1 }
                      { unnumbered } &&
                    ! \l_@@_header_unnumbered_bool
                  }
                  {
                    \group_begin:
                    \bool_set_true:N
                      \l_@@_header_unnumbered_bool
                    \c@secnumdepth = 0
                    \markdownSetup
                      {
                        rendererPrototypes = {
                          sectionBegin = {
                            \group_begin:
                          },
                          sectionEnd = {
                            \group_end:
                          },
                        },
                      }
                  }
              },
            },
          }
      },
    },
  }
\ExplSyntaxOff
\markdownSetup{rendererPrototypes={
  superscript = {\textsuperscript{#1}},
  subscript = {\textsubscript{#1}},
  blockQuoteBegin = {\begin{quotation}},
  blockQuoteEnd = {\end{quotation}},
  inputVerbatim = {\VerbatimInput{#1}},
  thematicBreak = {\noindent\rule[0.5ex]{\linewidth}{1pt}},
  note = {\footnote{#1}}}}
%    \end{macrocode}
% \par
% \begin{markdown}
%
%#### Fenced Code
% When no infostring has been specified, default to the indented code block
% renderer.
%
% \end{markdown}
%  \begin{macrocode}
\RequirePackage{ltxcmds}
\ExplSyntaxOn
\cs_gset:Npn
  \markdownRendererInputFencedCodePrototype#1#2#3
  {
    \tl_if_empty:nTF
      { #2 }
      { \markdownRendererInputVerbatim{#1} }
%    \end{macrocode}
% \begin{markdown}
%
% Otherwise, extract the first word of the infostring and treat it as the name
% of the programming language in which the code block is written.
%
% \end{markdown}
%  \begin{macrocode}
      {
        \regex_extract_once:nnN
          { \w* }
          { #2 }
          \l_tmpa_seq
        \seq_pop_left:NN
          \l_tmpa_seq
          \l_tmpa_tl
%    \end{macrocode}
% \par
% \begin{markdown}
%
% When the \pkg{minted} package is loaded, use it for syntax highlighting.
%
% \end{markdown}
%  \begin{macrocode}
        \ltx@ifpackageloaded
          { minted }
          {
            \catcode`\#=6\relax
            \exp_args:NV
              \inputminted
              \l_tmpa_tl
              { #1 }
            \catcode`\#=12\relax
          }
          {
%    \end{macrocode}
% \par
% \begin{markdown}
%
% When the \pkg{listings} package is loaded, use it for syntax highlighting.
%
% \end{markdown}
%  \begin{macrocode}
            \ltx@ifpackageloaded
              { listings }
              { \lstinputlisting[language=\l_tmpa_tl]{#1} }
%    \end{macrocode}
% \par
% \begin{markdown}
%
% When neither the \pkg{listings} package nor the \pkg{minted} package is
% loaded, act as though no infostring were given.
%
% \end{markdown}
%  \begin{macrocode}
              { \markdownRendererInputFencedCode{#1}{}{} }
          }
      }
  }
\ExplSyntaxOff
%    \end{macrocode}
% \par
% \begin{markdown}
%
% Support the nesting of strong emphasis.
%
% \end{markdown}
%  \begin{macrocode}
\ExplSyntaxOn
\def\markdownLATEXStrongEmphasis#1{%
  \str_if_in:NnTF
    \f@series
    { b }
    { \textnormal{#1} }
    { \textbf{#1} }
}
\ExplSyntaxOff
\markdownSetup{rendererPrototypes={strongEmphasis={%
  \protect\markdownLATEXStrongEmphasis{#1}}}}
%    \end{macrocode}
% \par
% \begin{markdown}
%
% Support \LaTeX{} document classes that do not provide chapters.
%
% \end{markdown}
%  \begin{macrocode}
\@ifundefined{chapter}{%
  \markdownSetup{rendererPrototypes = {
    headingOne = {\section{#1}},
    headingTwo = {\subsection{#1}},
    headingThree = {\subsubsection{#1}},
    headingFour = {\paragraph{#1}},
    headingFive = {\subparagraph{#1}}}}
}{%
  \markdownSetup{rendererPrototypes = {
    headingOne = {\chapter{#1}},
    headingTwo = {\section{#1}},
    headingThree = {\subsection{#1}},
    headingFour = {\subsubsection{#1}},
    headingFive = {\paragraph{#1}},
    headingSix = {\subparagraph{#1}}}}
}%
%    \end{macrocode}
% \par
% \begin{markdown}
%
%#### Tickboxes
% If the \Opt{taskLists} option is enabled, we will hide bullets in unordered
% list items with tickboxes.
%
% \end{markdown}
%  \begin{macrocode}
\markdownSetup{
  rendererPrototypes = {
    ulItem = {%
      \futurelet\markdownLaTeXCheckbox\markdownLaTeXUlItem
    },
  },
}
\def\markdownLaTeXUlItem{%
  \ifx\markdownLaTeXCheckbox\markdownRendererTickedBox
    \item[\markdownLaTeXCheckbox]%
    \expandafter\@gobble
  \else
    \ifx\markdownLaTeXCheckbox\markdownRendererHalfTickedBox
      \item[\markdownLaTeXCheckbox]%
      \expandafter\expandafter\expandafter\@gobble
    \else
      \ifx\markdownLaTeXCheckbox\markdownRendererUntickedBox
        \item[\markdownLaTeXCheckbox]%
        \expandafter\expandafter\expandafter\expandafter
          \expandafter\expandafter\expandafter\@gobble
      \else
        \item{}%
      \fi
    \fi
  \fi
}
%    \end{macrocode}
% \par
% \begin{markdown}
%
%#### HTML elements
% If the \Opt{html} option is enabled and we are using [\TeX{}4ht][1], we will
% pass HTML elements to the output HTML document unchanged.
%
%  [1]: https://tug.org/tex4ht/
%
% \end{markdown}
%  \begin{macrocode}
\@ifundefined{HCode}{}{
  \markdownSetup{
    rendererPrototypes = {
      inlineHtmlTag = {%
        \ifvmode
          \IgnorePar
          \EndP
        \fi
        \HCode{#1}%
      },
      inputBlockHtmlElement = {%
        \ifvmode
          \IgnorePar
        \fi
        \EndP
        \special{t4ht*<#1}%
        \par
        \ShowPar
      },
    },
  }
}
%    \end{macrocode}
% \par
% \begin{markdown}
%
%#### Citations
% Here is a basic implementation for citations that uses the \LaTeX{} `\cite`
% macro. There are also implementations that use the \pkg{natbib} `\citep`,
% and `\citet` macros, and the Bib\LaTeX{} `\autocites` and `\textcites`
% macros. These implementations will be used, when the respective packages are
% loaded.
%
% \end{markdown}
%  \begin{macrocode}
\newcount\markdownLaTeXCitationsCounter

% Basic implementation
\RequirePackage{gobble}
\def\markdownLaTeXBasicCitations#1#2#3#4#5#6{%
  \advance\markdownLaTeXCitationsCounter by 1\relax
  \ifx\relax#4\relax
    \ifx\relax#5\relax
      \ifnum\markdownLaTeXCitationsCounter>\markdownLaTeXCitationsTotal\relax
        \cite{#1#2#6}%  Without prenotes and postnotes, just accumulate cites
        \expandafter\expandafter\expandafter
        \expandafter\expandafter\expandafter\expandafter
        \@gobblethree
      \fi
    \else%  Before a postnote (#5), dump the accumulator
      \ifx\relax#1\relax\else
        \cite{#1}%
      \fi
      \cite[#5]{#6}%
      \ifnum\markdownLaTeXCitationsCounter>\markdownLaTeXCitationsTotal\relax
      \else
        \expandafter\expandafter\expandafter
        \expandafter\expandafter\expandafter\expandafter
        \expandafter\expandafter\expandafter
        \expandafter\expandafter\expandafter\expandafter
        \markdownLaTeXBasicCitations
      \fi
      \expandafter\expandafter\expandafter
      \expandafter\expandafter\expandafter\expandafter{%
      \expandafter\expandafter\expandafter
      \expandafter\expandafter\expandafter\expandafter}%
      \expandafter\expandafter\expandafter
      \expandafter\expandafter\expandafter\expandafter{%
      \expandafter\expandafter\expandafter
      \expandafter\expandafter\expandafter\expandafter}%
      \expandafter\expandafter\expandafter
      \@gobblethree
    \fi
  \else%  Before a prenote (#4), dump the accumulator
    \ifx\relax#1\relax\else
      \cite{#1}%
    \fi
    \ifnum\markdownLaTeXCitationsCounter>1\relax
      \space  % Insert a space before the prenote in later citations
    \fi
    #4~\expandafter\cite\ifx\relax#5\relax{#6}\else[#5]{#6}\fi
    \ifnum\markdownLaTeXCitationsCounter>\markdownLaTeXCitationsTotal\relax
    \else
      \expandafter\expandafter\expandafter
      \expandafter\expandafter\expandafter\expandafter
      \markdownLaTeXBasicCitations
    \fi
    \expandafter\expandafter\expandafter{%
    \expandafter\expandafter\expandafter}%
    \expandafter\expandafter\expandafter{%
    \expandafter\expandafter\expandafter}%
    \expandafter
    \@gobblethree
  \fi\markdownLaTeXBasicCitations{#1#2#6},}
\let\markdownLaTeXBasicTextCitations\markdownLaTeXBasicCitations

% Natbib implementation
\def\markdownLaTeXNatbibCitations#1#2#3#4#5{%
  \advance\markdownLaTeXCitationsCounter by 1\relax
  \ifx\relax#3\relax
    \ifx\relax#4\relax
      \ifnum\markdownLaTeXCitationsCounter>\markdownLaTeXCitationsTotal\relax
        \citep{#1,#5}%  Without prenotes and postnotes, just accumulate cites
        \expandafter\expandafter\expandafter
        \expandafter\expandafter\expandafter\expandafter
        \@gobbletwo
      \fi
    \else%  Before a postnote (#4), dump the accumulator
      \ifx\relax#1\relax\else
        \citep{#1}%
      \fi
      \citep[][#4]{#5}%
      \ifnum\markdownLaTeXCitationsCounter>\markdownLaTeXCitationsTotal\relax
      \else
        \expandafter\expandafter\expandafter
        \expandafter\expandafter\expandafter\expandafter
        \expandafter\expandafter\expandafter
        \expandafter\expandafter\expandafter\expandafter
        \markdownLaTeXNatbibCitations
      \fi
      \expandafter\expandafter\expandafter
      \expandafter\expandafter\expandafter\expandafter{%
      \expandafter\expandafter\expandafter
      \expandafter\expandafter\expandafter\expandafter}%
      \expandafter\expandafter\expandafter
      \@gobbletwo
    \fi
  \else%  Before a prenote (#3), dump the accumulator
    \ifx\relax#1\relax\relax\else
      \citep{#1}%
    \fi
    \citep[#3][#4]{#5}%
    \ifnum\markdownLaTeXCitationsCounter>\markdownLaTeXCitationsTotal\relax
    \else
      \expandafter\expandafter\expandafter
      \expandafter\expandafter\expandafter\expandafter
      \markdownLaTeXNatbibCitations
    \fi
    \expandafter\expandafter\expandafter{%
    \expandafter\expandafter\expandafter}%
    \expandafter
    \@gobbletwo
  \fi\markdownLaTeXNatbibCitations{#1,#5}}
\def\markdownLaTeXNatbibTextCitations#1#2#3#4#5{%
  \advance\markdownLaTeXCitationsCounter by 1\relax
  \ifx\relax#3\relax
    \ifx\relax#4\relax
      \ifnum\markdownLaTeXCitationsCounter>\markdownLaTeXCitationsTotal\relax
        \citet{#1,#5}%  Without prenotes and postnotes, just accumulate cites
        \expandafter\expandafter\expandafter
        \expandafter\expandafter\expandafter\expandafter
        \@gobbletwo
      \fi
    \else%  After a prenote or a postnote, dump the accumulator
      \ifx\relax#1\relax\else
        \citet{#1}%
      \fi
      , \citet[#3][#4]{#5}%
      \ifnum\markdownLaTeXCitationsCounter<\markdownLaTeXCitationsTotal\relax
        ,
      \else
        \ifnum\markdownLaTeXCitationsCounter=\markdownLaTeXCitationsTotal\relax
          ,
        \fi
      \fi
      \expandafter\expandafter\expandafter
      \expandafter\expandafter\expandafter\expandafter
      \markdownLaTeXNatbibTextCitations
      \expandafter\expandafter\expandafter
      \expandafter\expandafter\expandafter\expandafter{%
      \expandafter\expandafter\expandafter
      \expandafter\expandafter\expandafter\expandafter}%
      \expandafter\expandafter\expandafter
      \@gobbletwo
    \fi
  \else%  After a prenote or a postnote, dump the accumulator
    \ifx\relax#1\relax\relax\else
      \citet{#1}%
    \fi
    , \citet[#3][#4]{#5}%
    \ifnum\markdownLaTeXCitationsCounter<\markdownLaTeXCitationsTotal\relax
      ,
    \else
      \ifnum\markdownLaTeXCitationsCounter=\markdownLaTeXCitationsTotal\relax
        ,
      \fi
    \fi
    \expandafter\expandafter\expandafter
    \markdownLaTeXNatbibTextCitations
    \expandafter\expandafter\expandafter{%
    \expandafter\expandafter\expandafter}%
    \expandafter
    \@gobbletwo
  \fi\markdownLaTeXNatbibTextCitations{#1,#5}}

% BibLaTeX implementation
\def\markdownLaTeXBibLaTeXCitations#1#2#3#4#5{%
  \advance\markdownLaTeXCitationsCounter by 1\relax
  \ifnum\markdownLaTeXCitationsCounter>\markdownLaTeXCitationsTotal\relax
    \autocites#1[#3][#4]{#5}%
    \expandafter\@gobbletwo
  \fi\markdownLaTeXBibLaTeXCitations{#1[#3][#4]{#5}}}
\def\markdownLaTeXBibLaTeXTextCitations#1#2#3#4#5{%
  \advance\markdownLaTeXCitationsCounter by 1\relax
  \ifnum\markdownLaTeXCitationsCounter>\markdownLaTeXCitationsTotal\relax
    \textcites#1[#3][#4]{#5}%
    \expandafter\@gobbletwo
  \fi\markdownLaTeXBibLaTeXTextCitations{#1[#3][#4]{#5}}}

\markdownSetup{rendererPrototypes = {
  cite = {%
    \markdownLaTeXCitationsCounter=1%
    \def\markdownLaTeXCitationsTotal{#1}%
    \@ifundefined{autocites}{%
      \@ifundefined{citep}{%
        \expandafter\expandafter\expandafter
        \markdownLaTeXBasicCitations
        \expandafter\expandafter\expandafter{%
        \expandafter\expandafter\expandafter}%
        \expandafter\expandafter\expandafter{%
        \expandafter\expandafter\expandafter}%
      }{%
        \expandafter\expandafter\expandafter
        \markdownLaTeXNatbibCitations
        \expandafter\expandafter\expandafter{%
        \expandafter\expandafter\expandafter}%
      }%
    }{%
      \expandafter\expandafter\expandafter
      \markdownLaTeXBibLaTeXCitations
      \expandafter{\expandafter}%
    }},
  textCite = {%
    \markdownLaTeXCitationsCounter=1%
    \def\markdownLaTeXCitationsTotal{#1}%
    \@ifundefined{autocites}{%
      \@ifundefined{citep}{%
        \expandafter\expandafter\expandafter
        \markdownLaTeXBasicTextCitations
        \expandafter\expandafter\expandafter{%
        \expandafter\expandafter\expandafter}%
        \expandafter\expandafter\expandafter{%
        \expandafter\expandafter\expandafter}%
      }{%
        \expandafter\expandafter\expandafter
        \markdownLaTeXNatbibTextCitations
        \expandafter\expandafter\expandafter{%
        \expandafter\expandafter\expandafter}%
      }%
    }{%
      \expandafter\expandafter\expandafter
      \markdownLaTeXBibLaTeXTextCitations
      \expandafter{\expandafter}%
    }}}}
%    \end{macrocode}
% \par
% \begin{markdown}
%
%#### Links
% Here is an implementation for hypertext links and relative references.
%
% \end{markdown}
%  \begin{macrocode}
\RequirePackage{url}
\RequirePackage{expl3}
\ExplSyntaxOn
\def\markdownRendererLinkPrototype#1#2#3#4{
  \tl_set:Nn \l_tmpa_tl { #1 }
  \tl_set:Nn \l_tmpb_tl { #2 }
  \bool_set:Nn
    \l_tmpa_bool
    {
      \tl_if_eq_p:NN
        \l_tmpa_tl
        \l_tmpb_tl
    }
  \tl_set:Nn \l_tmpa_tl { #4 }
  \bool_set:Nn
    \l_tmpb_bool
    {
      \tl_if_empty_p:N
        \l_tmpa_tl
    }
%    \end{macrocode}
% \begin{markdown}
% If the label and the fully-escaped URI are equivalent and the title is
% empty, assume that the link is an autolink. Otherwise, assume that the
% link is either direct or indirect.
% \end{markdown}
%  \begin{macrocode}
  \bool_if:nTF
    {
      \l_tmpa_bool && \l_tmpb_bool
    }
    {
      \markdownLaTeXRendererAutolink { #2 } { #3 }
    }{
      \markdownLaTeXRendererDirectOrIndirectLink { #1 } { #2 } { #3 } { #4 }
    }
}
\def\markdownLaTeXRendererAutolink#1#2{%
%    \end{macrocode}
% \begin{markdown}
% If the URL begins with a hash sign, then we assume that it is a relative
% reference. Otherwise, we assume that it is an absolute URL.
% \end{markdown}
%  \begin{macrocode}
  \tl_set:Nn
    \l_tmpa_tl
    { #2 }
  \tl_trim_spaces:N
    \l_tmpa_tl
  \tl_set:Nx
    \l_tmpb_tl
    {
      \tl_range:Nnn
        \l_tmpa_tl
        { 1 }
        { 1 }
    }
  \str_if_eq:NNTF
    \l_tmpb_tl
    \c_hash_str
    {
      \tl_set:Nx
        \l_tmpb_tl
        {
          \tl_range:Nnn
            \l_tmpa_tl
            { 2 }
            { -1 }
        }
      \exp_args:NV
        \ref
        \l_tmpb_tl
    }{
      \url { #2 }
    }
}
\ExplSyntaxOff
\def\markdownLaTeXRendererDirectOrIndirectLink#1#2#3#4{%
  #1\footnote{\ifx\empty#4\empty\else#4: \fi\url{#3}}}
%    \end{macrocode}
% \par
% \begin{markdown}
%
%#### Tables
% Here is a basic implementation of tables. If the \pkg{booktabs} package is
% loaded, then it is used to produce horizontal lines.
%
% \end{markdown}
%  \begin{macrocode}
\newcount\markdownLaTeXRowCounter
\newcount\markdownLaTeXRowTotal
\newcount\markdownLaTeXColumnCounter
\newcount\markdownLaTeXColumnTotal
\newtoks\markdownLaTeXTable
\newtoks\markdownLaTeXTableAlignment
\newtoks\markdownLaTeXTableEnd
\AtBeginDocument{%
  \@ifpackageloaded{booktabs}{%
    \def\markdownLaTeXTopRule{\toprule}%
    \def\markdownLaTeXMidRule{\midrule}%
    \def\markdownLaTeXBottomRule{\bottomrule}%
  }{%
    \def\markdownLaTeXTopRule{\hline}%
    \def\markdownLaTeXMidRule{\hline}%
    \def\markdownLaTeXBottomRule{\hline}%
  }%
}
\markdownSetup{rendererPrototypes={
  table = {%
    \markdownLaTeXTable={}%
    \markdownLaTeXTableAlignment={}%
    \markdownLaTeXTableEnd={%
      \markdownLaTeXBottomRule
      \end{tabular}}%
    \ifx\empty#1\empty\else
      \addto@hook\markdownLaTeXTable{%
        \begin{table}
        \centering}%
      \addto@hook\markdownLaTeXTableEnd{%
        \caption{#1}
        \end{table}}%
    \fi
    \addto@hook\markdownLaTeXTable{\begin{tabular}}%
    \markdownLaTeXRowCounter=0%
    \markdownLaTeXRowTotal=#2%
    \markdownLaTeXColumnTotal=#3%
    \markdownLaTeXRenderTableRow
  }
}}
\def\markdownLaTeXRenderTableRow#1{%
  \markdownLaTeXColumnCounter=0%
  \ifnum\markdownLaTeXRowCounter=0\relax
    \markdownLaTeXReadAlignments#1%
    \markdownLaTeXTable=\expandafter\expandafter\expandafter{%
      \expandafter\the\expandafter\markdownLaTeXTable\expandafter{%
        \the\markdownLaTeXTableAlignment}}%
    \addto@hook\markdownLaTeXTable{\markdownLaTeXTopRule}%
  \else
    \markdownLaTeXRenderTableCell#1%
  \fi
  \ifnum\markdownLaTeXRowCounter=1\relax
    \addto@hook\markdownLaTeXTable\markdownLaTeXMidRule
  \fi
  \advance\markdownLaTeXRowCounter by 1\relax
  \ifnum\markdownLaTeXRowCounter>\markdownLaTeXRowTotal\relax
    \the\markdownLaTeXTable
    \the\markdownLaTeXTableEnd
    \expandafter\@gobble
  \fi\markdownLaTeXRenderTableRow}
\def\markdownLaTeXReadAlignments#1{%
  \advance\markdownLaTeXColumnCounter by 1\relax
  \if#1d%
    \addto@hook\markdownLaTeXTableAlignment{l}%
  \else
    \addto@hook\markdownLaTeXTableAlignment{#1}%
  \fi
  \ifnum\markdownLaTeXColumnCounter<\markdownLaTeXColumnTotal\relax\else
    \expandafter\@gobble
  \fi\markdownLaTeXReadAlignments}
\def\markdownLaTeXRenderTableCell#1{%
  \advance\markdownLaTeXColumnCounter by 1\relax
  \ifnum\markdownLaTeXColumnCounter<\markdownLaTeXColumnTotal\relax
    \addto@hook\markdownLaTeXTable{#1&}%
  \else
    \addto@hook\markdownLaTeXTable{#1\\}%
    \expandafter\@gobble
  \fi\markdownLaTeXRenderTableCell}
%    \end{macrocode}
% \par
% \begin{markdown}
%
%#### Line Blocks
% Here is a basic implementation of line blocks. If the \pkg{verse} package is
% loaded, then it is used to produce the verses.
%
% \end{markdown}
%  \begin{macrocode}

\markdownIfOption{lineBlocks}{%
  \RequirePackage{verse}
  \markdownSetup{rendererPrototypes={
    lineBlockBegin = {%
      \begingroup
        \def\markdownRendererHardLineBreak{\\}%
        \begin{verse}%
    },
    lineBlockEnd = {%
        \end{verse}%
      \endgroup
    },
  }}
}{}

%    \end{macrocode}
% \par
% \begin{markdown}
%
%#### YAML Metadata {#latex-yaml-metadata}
%
% The default setup of \acro{yaml} metadata will invoke the `\title`,
% `\author`, and `\date` macros when scalar values for keys that
% correspond to the `title`, `author`, and `date` relative wildcards are
% encountered, respectively.
%
% \end{markdown}
%  \begin{macrocode}
\ExplSyntaxOn
\keys_define:nn
  { markdown/jekyllData }
  {
    author  .code:n = { \author{#1} },
    date    .code:n = { \date{#1}   },
    title   .code:n = { \title{#1}  },
  }
%    \end{macrocode}
% \begin{markdown}
%
% To complement the default setup of our key--values, we will use
% the `\maketitle` macro to typeset the title page of a document
% at the end of \acro{yaml} metadata. If we are in the preamble, we will wait
% macro until after the beginning of the document. Otherwise, we will use
% the `\maketitle` macro straight away.
%
% \end{markdown}
%  \begin{macrocode}
\markdownSetup{
  rendererPrototypes = {
    jekyllDataEnd = {
      \AddToHook{begindocument/end}{\maketitle}
    },
  },
}
\ExplSyntaxOff
%    \end{macrocode}
% \par
% \begin{markdown}
%
%#### Strike-Through
% If the \Opt{strikeThrough} option is enabled, we will load the
% \pkg{soulutf8} package and use it to implement strike-throughs.
%
% \end{markdown}
%  \begin{macrocode}
\markdownIfOption{strikeThrough}{%
  \RequirePackage{soulutf8}%
  \markdownSetup{
    rendererPrototypes = {
      strikeThrough = {%
        \st{#1}%
      },
    }
  }
}{}
%    \end{macrocode}
% \par
% \begin{markdown}
%
%#### Marked Text
% If the \Opt{mark} option is enabled, we will load the \pkg{soulutf8} package
% and use it to implement marked text.
%
% \end{markdown}
%  \begin{macrocode}
\markdownIfOption{mark}{%
  \RequirePackage{soulutf8}%
  \markdownSetup{
    rendererPrototypes = {
      mark = {%
        \hl{#1}%
      },
    }
  }
}{}
%    \end{macrocode}
% \par
% \begin{markdown}
%
%#### Image Attributes
%
% If the \Opt{linkAttributes} option is enabled, we will load
% the \pkg{graphicx} package. Furthermore, in image attribute contexts,
% we will make attributes in the form \meta{key}`=`\meta{value} set the
% corresponding keys of the \pkg{graphicx} package to the corresponding
% values.
%
% \end{markdown}
%  \begin{macrocode}
\ExplSyntaxOn
\@@_if_option:nT
  { linkAttributes }
  {
    \RequirePackage{graphicx}
    \markdownSetup{
      rendererPrototypes = {
        imageAttributeContextBegin = {
          \group_begin:
          \markdownSetup{
            rendererPrototypes = {
              attributeKeyValue = {
                \setkeys
                  { Gin }
                  { { ##1 } = { ##2 } }
              },
            },
          }
        },
        imageAttributeContextEnd = {
          \group_end:
        },
      },
    }
  }
\ExplSyntaxOff
%    \end{macrocode}
% \par
% \begin{markdown}
%
%#### Raw Attributes
%
% In the raw block and inline raw span renderer prototypes, default to the
% plain TeX renderer prototypes, translating raw attribute `latex` to `tex`.
%
% \end{markdown}
%  \begin{macrocode}
\ExplSyntaxOn
\cs_gset:Npn
  \markdownRendererInputRawInlinePrototype#1#2
  {
    \str_case:nnF
      { #2 }
      {
        { latex }
          {
            \@@_plain_tex_default_input_raw_inline_renderer_prototype:nn
              { #1 }
              { tex }
          }
      }
      {
        \@@_plain_tex_default_input_raw_inline_renderer_prototype:nn
          { #1 }
          { #2 }
      }
  }
\cs_gset:Npn
  \markdownRendererInputRawBlockPrototype#1#2
  {
    \str_case:nnF
      { #2 }
      {
        { latex }
          {
            \@@_plain_tex_default_input_raw_block_renderer_prototype:nn
              { #1 }
              { tex }
          }
      }
      {
        \@@_plain_tex_default_input_raw_block_renderer_prototype:nn
          { #1 }
          { #2 }
      }
  }
\ExplSyntaxOff
\fi % Closes `\markdownIfOption{plain}{\iffalse}{\iftrue}`
%    \end{macrocode}
% \iffalse
%</themes-witiko-markdown-defaults-latex>
%<*latex>
% \fi
% \par
% \begin{markdown}
%
%### Miscellanea
% When buffering user input, we should disable the bytes with the high bit set,
% since these are made active by the \pkg{inputenc} package. We will do this by
% redefining the \mref{markdownMakeOther} macro accordingly. The code is courtesy
% of Scott Pakin, the creator of the \pkg{filecontents} package.
%
% \end{markdown}
%  \begin{macrocode}
\newcommand\markdownMakeOther{%
  \count0=128\relax
  \loop
    \catcode\count0=11\relax
    \advance\count0 by 1\relax
  \ifnum\count0<256\repeat}%
%    \end{macrocode}
% \iffalse
%</latex>
%<*context>
% \fi
% \par
% \begin{markdown}
%
% \Hologo{ConTeXt} Implementation {#contextimplementation}
%---------------------------------
%
% The \Hologo{ConTeXt} implementation makes use of the fact that, apart from
% some subtle differences, the Mark II and Mark IV \Hologo{ConTeXt} formats
% *seem* to implement (the documentation is scarce) the majority of the
% plain \TeX{} format required by the plain \TeX{} implementation.  As a
% consequence, we can directly reuse the existing plain \TeX{} implementation
% after supplying the missing plain \TeX{} macros.
%
% When buffering user input, we should disable the bytes with the high bit set,
% since these are made active by the `\enableregime` macro. We will do this
% by redefining the \mref{markdownMakeOther} macro accordingly. The code is
% courtesy of Scott Pakin, the creator of the \pkg{filecontents} \LaTeX{}
% package.
%
% \end{markdown}
%  \begin{macrocode}
\def\markdownMakeOther{%
  \count0=128\relax
  \loop
    \catcode\count0=11\relax
    \advance\count0 by 1\relax
  \ifnum\count0<256\repeat
%    \end{macrocode}
% \par
% \begin{markdown}
%
% On top of that, make the pipe character (`|`) inactive during the scanning.
% This is necessary, since the character is active in \Hologo{ConTeXt}.
%
% \end{markdown}
%  \begin{macrocode}
  \catcode`|=12}%
%    \end{macrocode}
% \par
% \begin{markdown}
%
%### Typesetting Markdown
% The \mref{inputmarkdown} macro is defined to accept an optional argument with
% options recognized by the \Hologo{ConTeXt} interface (see Section
% <#sec:context-options>).
%
% \end{markdown}
%  \begin{macrocode}
\long\def\inputmarkdown{%
  \dosingleempty
  \doinputmarkdown}%
\long\def\doinputmarkdown[#1]#2{%
  \begingroup
    \iffirstargument
      \setupmarkdown[#1]%
    \fi
    \markdownInput{#2}%
  \endgroup}%
%    \end{macrocode}
% \par
% \begin{markdown}
%
% The \mref{startmarkdown} and \mref{stopmarkdown} macros are implemented using the
% \mref{markdownReadAndConvert} macro.
%
% In Knuth's \TeX, trailing spaces are removed very early on when a line is
% being put to the input buffer.~[@knuth86b, sec. 31].  According to
% @eijkhout92 [sec. 2.2], this is because ``these spaces are hard to see in
% an editor''. At the moment, there is no option to suppress this behavior in
% (Lua)\TeX, but \Hologo{ConTeXt} MkIV funnels all input through its own input
% handler. This makes it possible to suppress the removal of trailing spaces
% in \Hologo{ConTeXt} MkIV and therefore to insert hard line breaks into
% markdown text.
%
% \end{markdown}
%  \begin{macrocode}
\startluacode
  document.markdown_buffering = false
  local function preserve_trailing_spaces(line)
    if document.markdown_buffering then
      line = line:gsub("[ \t][ \t]$", "\t\t")
    end
    return line
  end
  resolvers.installinputlinehandler(preserve_trailing_spaces)
\stopluacode
\begingroup
  \catcode`\|=0%
  \catcode`\\=12%
  |gdef|startmarkdown{%
    |ctxlua{document.markdown_buffering = true}%
    |markdownReadAndConvert{\stopmarkdown}%
                           {|stopmarkdown}}%
  |gdef|stopmarkdown{%
    |ctxlua{document.markdown_buffering = false}%
    |markdownEnd}%
|endgroup
%    \end{macrocode}
% \par
% \begin{markdown}
%
%### Themes {#context-themes-implementation}
%
% This section overrides the plain \TeX{} implementation of the theme-loading
% mechanism from Section <#sec:themes-implementation>. Furthermore, this section
% also implements the built-in \Hologo{ConTeXt} themes provided with the
% Markdown package.
%
% \end{markdown}
%  \begin{macrocode}
\ExplSyntaxOn
\cs_gset:Nn
  \@@_load_theme:nn
  {
%    \end{macrocode}
% \par
% \begin{markdown}
%
% Determine whether a file named `t-markdowntheme`\meta{munged theme
% name}`.tex` exists. If it does, load it. Otherwise, try loading a plain
% \TeX{} theme instead.
%
% \end{markdown}
%  \begin{macrocode}
    \file_if_exist:nTF
      { t - markdown theme #2.tex }
      {
        \msg_info:nnn
          { markdown }
          { loading-context-theme }
          { #1 }
        \usemodule
          [ t ]
          [ markdown theme #2 ]
      }
      {
        \@@_plain_tex_load_theme:nn
          { #1 }
          { #2 }
      }
  }
\msg_new:nnn
  { markdown }
  { loading-context-theme }
  { Loading~ConTeXt~Markdown~theme~#1 }
\ExplSyntaxOff
%    \end{macrocode}
% \iffalse
%</context>
%<*themes-witiko-markdown-defaults-context>
% \fi
% \par
% \begin{markdown}
%
% The `witiko/markdown/defaults` \Hologo{ConTeXt} theme provides default
% definitions for token renderer prototypes. First, the \Hologo{ConTeXt} theme
% loads the plain \TeX{} theme with the default definitions for plain \TeX{}:
%
% \end{markdown}
%  \begin{macrocode}
\markdownLoadPlainTeXTheme
%    \end{macrocode}
% \par
% \begin{markdown}
%
% Next, the \Hologo{ConTeXt} theme overrides some of the plain \TeX{} definitions.
% See Section <#sec:context-token-renderer-prototypes> for the actual
% definitions.
%
%### Token Renderer Prototypes {#context-token-renderer-prototypes}
%
% The following configuration should be considered placeholder. If the option
% `plain` has been enabled (see Section <#sec:plain>), none of the definitions
% will take effect.
%
% \end{markdown}
%  \begin{macrocode}
\markdownIfOption{plain}{\iffalse}{\iftrue}
\def\markdownRendererHardLineBreakPrototype{\blank}%
\def\markdownRendererLeftBracePrototype{\textbraceleft}%
\def\markdownRendererRightBracePrototype{\textbraceright}%
\def\markdownRendererDollarSignPrototype{\textdollar}%
\def\markdownRendererPercentSignPrototype{\percent}%
\def\markdownRendererUnderscorePrototype{\textunderscore}%
\def\markdownRendererCircumflexPrototype{\textcircumflex}%
\def\markdownRendererBackslashPrototype{\textbackslash}%
\def\markdownRendererTildePrototype{\textasciitilde}%
\def\markdownRendererPipePrototype{\char`|}%
\def\markdownRendererLinkPrototype#1#2#3#4{%
  \useURL[#1][#3][][#4]#1\footnote[#1]{\ifx\empty#4\empty\else#4:
  \fi\tt<\hyphenatedurl{#3}>}}%
\usemodule[database]
\defineseparatedlist
  [MarkdownConTeXtCSV]
  [separator={,},
   before=\bTABLE,after=\eTABLE,
   first=\bTR,last=\eTR,
   left=\bTD,right=\eTD]
\def\markdownConTeXtCSV{csv}
\def\markdownRendererContentBlockPrototype#1#2#3#4{%
  \def\markdownConTeXtCSV@arg{#1}%
  \ifx\markdownConTeXtCSV@arg\markdownConTeXtCSV
    \placetable[][tab:#1]{#4}{%
      \processseparatedfile[MarkdownConTeXtCSV][#3]}%
  \else
    \markdownInput{#3}%
  \fi}%
\def\markdownRendererImagePrototype#1#2#3#4{%
  \placefigure[][]{#4}{\externalfigure[#3]}}%
\def\markdownRendererUlBeginPrototype{\startitemize}%
\def\markdownRendererUlBeginTightPrototype{\startitemize[packed]}%
\def\markdownRendererUlItemPrototype{\item}%
\def\markdownRendererUlEndPrototype{\stopitemize}%
\def\markdownRendererUlEndTightPrototype{\stopitemize}%
\def\markdownRendererOlBeginPrototype{\startitemize[n]}%
\def\markdownRendererOlBeginTightPrototype{\startitemize[packed,n]}%
\def\markdownRendererOlItemPrototype{\item}%
\def\markdownRendererOlItemWithNumberPrototype#1{\sym{#1.}}%
\def\markdownRendererOlEndPrototype{\stopitemize}%
\def\markdownRendererOlEndTightPrototype{\stopitemize}%
\definedescription
  [MarkdownConTeXtDlItemPrototype]
  [location=hanging,
   margin=standard,
   headstyle=bold]%
\definestartstop
  [MarkdownConTeXtDlPrototype]
  [before=\blank,
   after=\blank]%
\definestartstop
  [MarkdownConTeXtDlTightPrototype]
  [before=\blank\startpacked,
   after=\stoppacked\blank]%
\def\markdownRendererDlBeginPrototype{%
  \startMarkdownConTeXtDlPrototype}%
\def\markdownRendererDlBeginTightPrototype{%
  \startMarkdownConTeXtDlTightPrototype}%
\def\markdownRendererDlItemPrototype#1{%
  \startMarkdownConTeXtDlItemPrototype{#1}}%
\def\markdownRendererDlItemEndPrototype{%
  \stopMarkdownConTeXtDlItemPrototype}%
\def\markdownRendererDlEndPrototype{%
  \stopMarkdownConTeXtDlPrototype}%
\def\markdownRendererDlEndTightPrototype{%
  \stopMarkdownConTeXtDlTightPrototype}%
\def\markdownRendererEmphasisPrototype#1{{\em#1}}%
\def\markdownRendererStrongEmphasisPrototype#1{{\bf#1}}%
\def\markdownRendererBlockQuoteBeginPrototype{\startquotation}%
\def\markdownRendererBlockQuoteEndPrototype{\stopquotation}%
\def\markdownRendererLineBlockBeginPrototype{%
  \begingroup
    \def\markdownRendererHardLineBreak{
    }%
    \startlines
}%
\def\markdownRendererLineBlockEndPrototype{%
    \stoplines
  \endgroup
}%
\def\markdownRendererInputVerbatimPrototype#1{\typefile{#1}}%
%    \end{macrocode}
% \par
% \begin{markdown}
%
%#### Fenced Code
% When no infostring has been specified, default to the indented code block
% renderer.
%
% \end{markdown}
%  \begin{macrocode}
\ExplSyntaxOn
\cs_gset:Npn
  \markdownRendererInputFencedCodePrototype#1#2#3
  {
    \tl_if_empty:nTF
      { #2 }
      { \markdownRendererInputVerbatim{#1} }
%    \end{macrocode}
% \par
% \begin{markdown}
%
% Otherwise, extract the first word of the infostring and treat it as the name
% of the programming language in which the code block is written.
% This name is then used in the \Hologo{ConTeXt} `\definetyping` macro, which
% allows the user to set up code highlighting mapping as follows:
% ````` tex
% % Map the `TEX` syntax highlighter to the `latex` infostring.
% \definetyping [latex]
% \setuptyping  [latex] [option=TEX]
%
% \starttext
%   \startmarkdown
% ~~~ latex
% \documentclass{article}
% \begin{document}
%   Hello world!
% \end{document}
% ~~~
%   \stopmarkdown
% \stoptext
% `````````
%
% \end{markdown}
%  \begin{macrocode}
      {
        \regex_extract_once:nnN
          { \w* }
          { #2 }
          \l_tmpa_seq
        \seq_pop_left:NN
          \l_tmpa_seq
          \l_tmpa_tl
        \typefile[\l_tmpa_tl][]{#1}
      }
  }
\ExplSyntaxOff
\def\markdownRendererHeadingOnePrototype#1{\chapter{#1}}%
\def\markdownRendererHeadingTwoPrototype#1{\section{#1}}%
\def\markdownRendererHeadingThreePrototype#1{\subsection{#1}}%
\def\markdownRendererHeadingFourPrototype#1{\subsubsection{#1}}%
\def\markdownRendererHeadingFivePrototype#1{\subsubsubsection{#1}}%
\def\markdownRendererHeadingSixPrototype#1{\subsubsubsubsection{#1}}%
\def\markdownRendererThematicBreakPrototype{%
  \blackrule[height=1pt, width=\hsize]}%
\def\markdownRendererNotePrototype#1{\footnote{#1}}%
\def\markdownRendererTickedBoxPrototype{$\boxtimes$}
\def\markdownRendererHalfTickedBoxPrototype{$\boxdot$}
\def\markdownRendererUntickedBoxPrototype{$\square$}
\def\markdownRendererStrikeThroughPrototype#1{\overstrikes{#1}}
\def\markdownRendererSuperscriptPrototype#1{\high{#1}}
\def\markdownRendererSubscriptPrototype#1{\low{#1}}
\def\markdownRendererDisplayMathPrototype#1{\startformula#1\stopformula}%
%    \end{macrocode}
% \par
% \begin{markdown}
%
%#### Tables
% There is a basic implementation of tables.
%
% \end{markdown}
%  \begin{macrocode}
\newcount\markdownConTeXtRowCounter
\newcount\markdownConTeXtRowTotal
\newcount\markdownConTeXtColumnCounter
\newcount\markdownConTeXtColumnTotal
\newtoks\markdownConTeXtTable
\newtoks\markdownConTeXtTableFloat
\def\markdownRendererTablePrototype#1#2#3{%
  \markdownConTeXtTable={}%
  \ifx\empty#1\empty
    \markdownConTeXtTableFloat={%
      \the\markdownConTeXtTable}%
  \else
    \markdownConTeXtTableFloat={%
      \placetable{#1}{\the\markdownConTeXtTable}}%
  \fi
  \begingroup
  \setupTABLE[r][each][topframe=off, bottomframe=off, leftframe=off, rightframe=off]
  \setupTABLE[c][each][topframe=off, bottomframe=off, leftframe=off, rightframe=off]
  \setupTABLE[r][1][topframe=on, bottomframe=on]
  \setupTABLE[r][#1][bottomframe=on]
  \markdownConTeXtRowCounter=0%
  \markdownConTeXtRowTotal=#2%
  \markdownConTeXtColumnTotal=#3%
  \markdownConTeXtRenderTableRow}
\def\markdownConTeXtRenderTableRow#1{%
  \markdownConTeXtColumnCounter=0%
  \ifnum\markdownConTeXtRowCounter=0\relax
    \markdownConTeXtReadAlignments#1%
    \markdownConTeXtTable={\bTABLE}%
  \else
    \markdownConTeXtTable=\expandafter{%
      \the\markdownConTeXtTable\bTR}%
    \markdownConTeXtRenderTableCell#1%
    \markdownConTeXtTable=\expandafter{%
      \the\markdownConTeXtTable\eTR}%
  \fi
  \advance\markdownConTeXtRowCounter by 1\relax
  \ifnum\markdownConTeXtRowCounter>\markdownConTeXtRowTotal\relax
    \markdownConTeXtTable=\expandafter{%
      \the\markdownConTeXtTable\eTABLE}%
    \the\markdownConTeXtTableFloat
    \endgroup
    \expandafter\gobbleoneargument
  \fi\markdownConTeXtRenderTableRow}
\def\markdownConTeXtReadAlignments#1{%
  \advance\markdownConTeXtColumnCounter by 1\relax
  \if#1d%
    \setupTABLE[c][\the\markdownConTeXtColumnCounter][align=right]
  \fi\if#1l%
    \setupTABLE[c][\the\markdownConTeXtColumnCounter][align=right]
  \fi\if#1c%
    \setupTABLE[c][\the\markdownConTeXtColumnCounter][align=middle]
  \fi\if#1r%
    \setupTABLE[c][\the\markdownConTeXtColumnCounter][align=left]
  \fi
  \ifnum\markdownConTeXtColumnCounter<\markdownConTeXtColumnTotal\relax\else
    \expandafter\gobbleoneargument
  \fi\markdownConTeXtReadAlignments}
\def\markdownConTeXtRenderTableCell#1{%
  \advance\markdownConTeXtColumnCounter by 1\relax
  \markdownConTeXtTable=\expandafter{%
    \the\markdownConTeXtTable\bTD#1\eTD}%
  \ifnum\markdownConTeXtColumnCounter<\markdownConTeXtColumnTotal\relax\else
    \expandafter\gobbleoneargument
  \fi\markdownConTeXtRenderTableCell}
%    \end{macrocode}
% \par
% \begin{markdown}
%
%#### Raw Attributes
%
% In the raw block and inline raw span renderer prototypes, default to the
% plain TeX renderer prototypes, translating raw attribute `context` to `tex`.
%
% \end{markdown}
%  \begin{macrocode}
\ExplSyntaxOn
\cs_gset:Npn
  \markdownRendererInputRawInlinePrototype#1#2
  {
    \str_case:nnF
      { #2 }
      {
        { latex }
          {
            \@@_plain_tex_default_input_raw_inline_renderer_prototype:nn
              { #1 }
              { context }
          }
      }
      {
        \@@_plain_tex_default_input_raw_inline_renderer_prototype:nn
          { #1 }
          { #2 }
      }
  }
\cs_gset:Npn
  \markdownRendererInputRawBlockPrototype#1#2
  {
    \str_case:nnF
      { #2 }
      {
        { context }
          {
            \@@_plain_tex_default_input_raw_block_renderer_prototype:nn
              { #1 }
              { tex }
          }
      }
      {
        \@@_plain_tex_default_input_raw_block_renderer_prototype:nn
          { #1 }
          { #2 }
      }
  }
\cs_gset_eq:NN
  \markdownRendererInputRawBlockPrototype
  \markdownRendererInputRawInlinePrototype
\fi % Closes `\markdownIfOption{plain}{\iffalse}{\iftrue}`
\ExplSyntaxOff
\stopmodule
\protect
%    \end{macrocode}
% \iffalse
%</themes-witiko-markdown-defaults-context>
%<*context>
% \fi
% \par
% \begin{markdown}
%
% At the end of the \Hologo{ConTeXt} module, we load the
% `witiko/markdown/defaults` \Hologo{ConTeXt} theme with the default
% definitions for token renderer prototypes unless the option `noDefaults`
% has been enabled (see Section <#sec:plain>).
%
% \end{markdown}
%  \begin{macrocode}
\markdownIfOption{noDefaults}{}{
  \setupmarkdown[theme=witiko/markdown/defaults]
}
\stopmodule
\protect
%    \end{macrocode}
% \iffalse
%</context>
% \fi
